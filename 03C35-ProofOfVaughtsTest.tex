\documentclass[12pt]{article}
\usepackage{pmmeta}
\pmcanonicalname{ProofOfVaughtsTest}
\pmcreated{2013-03-22 13:00:44}
\pmmodified{2013-03-22 13:00:44}
\pmowner{Evandar}{27}
\pmmodifier{Evandar}{27}
\pmtitle{proof of Vaught's test}
\pmrecord{4}{33395}
\pmprivacy{1}
\pmauthor{Evandar}{27}
\pmtype{Proof}
\pmcomment{trigger rebuild}
\pmclassification{msc}{03C35}

\endmetadata

% this is the default PlanetMath preamble.  as your knowledge
% of TeX increases, you will probably want to edit this, but
% it should be fine as is for beginners.

% almost certainly you want these
\usepackage{amssymb}
\usepackage{amsmath}
\usepackage{amsfonts}

% used for TeXing text within eps files
%\usepackage{psfrag}
% need this for including graphics (\includegraphics)
%\usepackage{graphicx}
% for neatly defining theorems and propositions
%\usepackage{amsthm}
% making logically defined graphics
%%%\usepackage{xypic} 

% there are many more packages, add them here as you need them

\newtheorem{theorem}{Theorem}[section]
\newtheorem{lemma}[theorem]{Lemma}
\newtheorem{proposition}[theorem]{Proposition}
\newtheorem{corollary}[theorem]{Corollary}

\newenvironment{proof}[1][Proof]{\begin{trivlist}
\item[\hskip \labelsep {\bfseries #1}]}{\end{trivlist}}
\newenvironment{definition}[1][Definition]{\begin{trivlist}
\item[\hskip \labelsep {\bfseries #1}]}{\end{trivlist}}
\newenvironment{example}[1][Example]{\begin{trivlist}
\item[\hskip \labelsep {\bfseries #1}]}{\end{trivlist}}
\newenvironment{remark}[1][Remark]{\begin{trivlist}
\item[\hskip \labelsep {\bfseries #1}]}{\end{trivlist}}
\begin{document}
Let $\varphi$ be an $L$-sentence, and let $\mathcal{A}$ be the unique model of S of cardinality $\kappa$.  Suppose $\mathcal{A}\vDash\varphi$.  Then if $\mathcal{B}$ is any model of $S$ then by the \PMlinkname{upward}{UpwardLowenheimSkolemTheorem} and downward Lowenheim-Skolem theorems, there is a model $\mathcal{C}$ of $S$ which is elementarily equivalent to $\mathcal{B}$ such that $|\mathcal{C}|=\kappa$.  Then $\mathcal{C}$ is isomorphic to $\mathcal{A}$, and so $\mathcal{C}\vDash\varphi$, and $\mathcal{B}\vDash\varphi$.  So $\mathcal{B}\vDash\varphi$ for all models $\mathcal{B}$ of $S$, so $S\vDash\varphi$.

Similarly, if $\mathcal{A}\vDash\lnot\varphi$ then $S\vDash\lnot\varphi$.  So $S$ is \PMlinkname{complete}{Complete6}.$\square$
%%%%%
%%%%%
\end{document}
