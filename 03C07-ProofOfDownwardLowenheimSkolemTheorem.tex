\documentclass[12pt]{article}
\usepackage{pmmeta}
\pmcanonicalname{ProofOfDownwardLowenheimSkolemTheorem}
\pmcreated{2013-03-22 18:18:59}
\pmmodified{2013-03-22 18:18:59}
\pmowner{GodelsTheorem}{21277}
\pmmodifier{GodelsTheorem}{21277}
\pmtitle{proof of downward Lowenheim-Skolem theorem}
\pmrecord{14}{40942}
\pmprivacy{1}
\pmauthor{GodelsTheorem}{21277}
\pmtype{Proof}
\pmcomment{trigger rebuild}
\pmclassification{msc}{03C07}

% this is the default PlanetMath preamble.  as your knowledge
% of TeX increases, you will probably want to edit this, but
% it should be fine as is for beginners.

% almost certainly you want these
\usepackage{amssymb}
\usepackage{amsmath}
\usepackage{amsfonts}

% used for TeXing text within eps files
%\usepackage{psfrag}
% need this for including graphics (\includegraphics)
%\usepackage{graphicx}
% for neatly defining theorems and propositions
%\usepackage{amsthm}
% making logically defined graphics
%%%\usepackage{xypic}

% there are many more packages, add them here as you need them

% define commands here

\begin{document}
We present here a proof of the Downward Lowenheim Skolem Theorem. The idea is to construct a submodel that meets the requirements of the DLS Theorem: we take $K$ and close it under a procedure of choosing appropriate witnesses for the existential formulas satisfied by $\mathcal{A}$. Choosing the appropriate witnesses is done with the help of the so-called  \emph{Skolem functions} and thus rests upon the \emph{choice function}.

\emph{Proof}: First of all we introduce a usefull tool for the proof.
Lemma: (\emph{Tarski's Lemma}) If  $\mathcal{A}$ and  $\mathcal{B}$ are $L$-structures with the domain of $\mathcal{A}$ being a subset of the domain of $\mathcal{B}$  then $\mathcal{A}$ is an elementary substructure of $\mathcal{B}$ if for every $L$-formula $\phi(\emph{x},y)$ with $a \in A$ and $b \in B$, we have
$\mathcal{B}\models \phi(a,b)$ <-> For some $a' \in A$ we have $\mathcal{A}\models \phi(a,a')$ 

\emph{Proof}. Let's supposes the biconditionnal holds, then we need to show that $\mathcal{A}$ is a substructure of $\mathcal{B}$. This means that we need to show that every formula that is true in $\mathcal{A}$ is true in $\mathcal{B}$. The proof is straighforward with an induction on the complexity of formulas (connectives, negation, quantifiers).  
                       

Now, fix a point $p \in {dom}(\mathcal{A})$. For each $L$-formula $\phi(\emph{x},y)$, define the \emph{Skolem function} of $\phi$, $g_\phi : A^n \to A$ (with $A={dom}(\mathcal{A})$) by:

       $p \in A^n \to$ some $p' \in A$ such that $\mathcal{A}\models \phi(p,p')$ if such a $p'$ exists and $p$ otherwise.
The set of \emph{Skolem functions} has a cardinality equal to that of $L$.

The purpose here is to construct a model $\mathcal{B}$ whose domain $B$ is closed under the skolem functions. This means that the domain of $\mathcal{A}$ contains all the witnesses we've appropriately choosen. If we take an existential formula $\exists y \phi(\emph{x},y)$ and $b \in B^k$ and if we apply the Skolem function to $b$ we will have a witness for $\exists x \phi(\emph{b},x)$. In other words, this means that $\mathcal{A}\models \exists x \phi(b,x) \to \mathcal{A}\models \phi(b,g_\phi(b))$. By construction $g_\phi(b)$ is in $B$ and thus $\mathcal{B}$ meets the requirements of Tarski's Lemma. We can find an elementary substructure of  $\mathcal{A}$.

Let's take $K$ of above and set $K_0:=K$ and $K_{i+1}$ is the set of the $g_\phi(p), p \in K_i \wedge  g_\phi$ (with $g_\phi$ a Skolem function). Let $B:= \bigcup K_i$. Then $B$ is closed under Skolem functions. And we have $|K|\leq \omega .|L|+|K|= |L|+|K|$. This comes from the fact that $|L|+|K_i|=|\emph{SF}|+|K_i|= \sum_{k \in \mathbb{N} }|(SF)_k|+|K_i|= \sum_{k \in \mathbb{N} }|(SF)_k|*|K_i|$ but we have $|K_{i+1}| \leq \sum_{k \in \mathbb{N} }|(SF)_k|*|K_i|$.
We need now to provide interpretations of relations, predicates, functions and constants so it can fit $\mathcal{A}$.

We have because $B$ is closed under $L$-terms and for an $L$-function symbol $f$, the Skolem function of the $L$-formula $f(x)=y$ takes the value $f^\mathcal{A}(p)$ at p:  

          for any n-ary relation symbol $P$:  $P^\mathcal{B}=P^\mathcal{A}\bigcap B^n$

          for an m-ary function symbol $f$ and $p \in B^m , p' \in B$ we have $f^\mathcal{A}(p)=p'$ <-> $f^\mathcal{B}(p)=p'$.

          for a constant symbol $c$, we have $c^\mathcal{A}=c^\mathcal{B}$ We have constructed a substructure $\mathcal{B}$ of $\mathcal{A}$.
%%%%%
%%%%%
\end{document}
