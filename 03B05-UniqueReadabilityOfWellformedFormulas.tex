\documentclass[12pt]{article}
\usepackage{pmmeta}
\pmcanonicalname{UniqueReadabilityOfWellformedFormulas}
\pmcreated{2013-03-22 18:52:32}
\pmmodified{2013-03-22 18:52:32}
\pmowner{CWoo}{3771}
\pmmodifier{CWoo}{3771}
\pmtitle{unique readability of well-formed formulas}
\pmrecord{9}{41723}
\pmprivacy{1}
\pmauthor{CWoo}{3771}
\pmtype{Theorem}
\pmcomment{trigger rebuild}
\pmclassification{msc}{03B05}
\pmdefines{unique readability}

\endmetadata

\usepackage{amssymb,amscd}
\usepackage{amsmath}
\usepackage{amsfonts}
\usepackage{mathrsfs}

% used for TeXing text within eps files
%\usepackage{psfrag}
% need this for including graphics (\includegraphics)
%\usepackage{graphicx}
% for neatly defining theorems and propositions
\usepackage{amsthm}
% making logically defined graphics
%%\usepackage{xypic}
\usepackage{pst-plot}

% define commands here
\newcommand*{\abs}[1]{\left\lvert #1\right\rvert}
\newtheorem{prop}{Proposition}
\newtheorem{thm}{Theorem}
\newtheorem{ex}{Example}
\newcommand{\real}{\mathbb{R}}
\newcommand{\pdiff}[2]{\frac{\partial #1}{\partial #2}}
\newcommand{\mpdiff}[3]{\frac{\partial^#1 #2}{\partial #3^#1}}
\begin{document}
Suppose $V$ is an alphabet.  Two words $w_1$ and $w_2$ over $V$ are the same if they have the same length and same symbol for every position of the word.  In other words, if $w_1 = a_1\cdots a_n$ and $w_2 = b_1\cdots b_m$, where each $a_i$ and $b_i$ are symbols in $V$, then $n=m$ and $a_i=b_i$.  In other words, every word over $V$ has a unique representation as product (concatenation) of symbols in $V$.  This is called the \emph{unique readability} of words over an alphabet.

Unique readability remains true if $V$ is infinite.  Now, suppose $V$ is a set of propositional variables, and suppose we have two well-formed formulas (wffs) $p:=\alpha p_1 \cdots p_n$ and $q:=\beta q_1 \cdots q_m$ over $V$, where $\alpha, \beta$ are logical connectives from a fixed set $F$ of connectives (for example, $F$ could be the set $\lbrace \neg, \vee, \wedge, \to, \leftrightarrow \rbrace$), and $p_i,q_j$ are wffs.  Does $p=q$ mean $\alpha = \beta$, $m=n$, and $p_i=q_i$?  This is the notion of \emph{unique readability} of well-formed formulas.  It is slightly different from the unique readability of words over an alphabet, for $p_i$ and $q_j$ are no longer symbols in an alphabet, but words themselves.

\begin{thm} Given any countable set $V$ of propositional variables and a set $F$ of logical connectives, well-formed formulas over $V$ constructed via $F$ are uniquely readable \end{thm}

\begin{proof}
Every $p\in \overline{V}$ has a representation $\alpha p_1 \cdots p_n$ for some $n$-ary connective $\alpha$ and wffs $p_i$.  The rest of the proof we show that this representation is unique, establishing unique readability.

Define a function $\phi: V\cup F \to \mathbb{Z}$ such that $\phi(v)=1$ for any $v\in V$, and $\phi(f)=1-n$ for any $n$-ary connective $f\in F$.  Defined inductively on the length of words over $V\cup F$, $\phi$ can be extended to an integer-valued function $\phi^*$ on the set of all words on $V\cup F$ so that $\phi^*(w_1w_2)=\phi^*(w_1)+\phi^*(w_2)$, for any words $w_1,w_2$ over $V\cup F$.  We have
\begin{enumerate}
\item $\phi^*$ is constant (whose value is $1$) when restricted to $\overline{V}$.

This can be proved by induction on $V_i$ (for the definition of $V_i$, see the parent entry).  By definition, this is true for any atoms.  Assume this is true for $V_i$.  Pick any $p\in V_{i+1}$.  Then $p=\alpha p_1 \cdots p_n$ for some $n$-ary $\alpha \in F$, and $p_j\in V_i$ for all $j=1,\ldots, n$  Then $\phi^*(p)= \phi^*(\alpha) + \phi^*(p_1) + \cdots + \phi^*(p_n) = 1 - n + n\times 1 = 1$.  Therefore, $\phi^*(p)=1$ for all $p\in \overline{V}$.
\item for any non-trivial suffix $s$ of a wff $p$, $\phi^*(s)>0$.  (A \emph{suffix} of a word $w$ is a word $s$ such that $w=ts$ for some word $t$; $s$ is non-trivial if $s$ is not the empty word) 

This is also proved by induction.  If $p\in V_0$, then $p$ itself is its only non-trivial final segment, so the assertion is true.  Suppose now this is true for any proposition in $V_i$.  If $p \in V_{i+1}$, then $p=\alpha p_1 \cdots p_n$, where each $p_k\in V_i$.  A non-trivial final segment $s$ of $p$ is either $p$, a final segment of $p_n$, or has the form $tp_j \cdots p_n$, where $t$ is a non-trivial final segment of $p_{j-1}$.  In the first case, $\phi^*(s)=\phi^*(p)=1$.  In the second case, $\phi^*(s)>0$ from assumption.  In the last case, $\phi^*(s)=\phi^*(t)+(n-j+1)>0$.
\end{enumerate}
Now, back to the main proof.  Suppose $p=q$.  If $p$ is an atom, so must $q$, and we are done.  Otherwise, assume $p = \alpha p_1 \cdots p_m = \beta q_1 \cdots q_n = q$.  Then $\alpha = \beta$ since the expressions are words over $V\cup F$, and $\alpha ,\beta\in F$.  Since the two connectives are the same, they have the same arity: $m=n$, and we have $\alpha p_1 \cdots p_n = \alpha q_1 \cdots q_n$.  If $n=0$, then we are done.  So assume $n>0$.  Then $p_1\cdots p_n = q_1 \cdots q_n$.  We want to show that $p_n=q_n$, and therefore $p_1 \cdots p_{n-1} = q_1 \cdots q_{n-1}$, and by induction $p_i = q_i$ for all $i< n$ as well, proving the theorem.

First, notice that $p_n$ is a non-trivial suffix of $q_1 \cdots q_n$.  So $p_n$ is either a suffix of $q_n$ or has the form $t q_j \cdots q_n$, where $j\le n$, and $t$ is a non-trivial suffix of $q_{j-1}$.  In the latter case, $1 = \phi^*(p_n) = \phi^*(t q_j \cdots q_n)= \phi^*(t) + n-j+1$.  Then $\phi^*(t) = j-n \le 0$, contradicting 2 above.  Therefore $p_n$ is a suffix of $q_n$.  By symmetry, $q_n$ is also a suffix of $p_n$, hence $p_n=q_n$.
\end{proof}

As a corollary, we see that the well-formed formulas of the classical propositional logic, written in Polish notation, are uniquely readable.  The unique readability of wffs using parentheses and infix notation requires a different proof.

\textbf{Remark}.  Unique readability will fail if the $p_i$ and $q_j$ above are not wffs, even if $V$ is finite.  For example, suppose $v\in V$ and $\#$ is binary, then $\# v \# vv$ can be read in three non-trivial ways: combining $v$ and $\#vv$, combining $v\#$ and $vv$, or combining $v\#v$ and $v$.  Notice that only the first combination do we get wffs.
%%%%%
%%%%%
\end{document}
