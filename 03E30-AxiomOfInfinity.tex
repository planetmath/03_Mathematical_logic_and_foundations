\documentclass[12pt]{article}
\usepackage{pmmeta}
\pmcanonicalname{AxiomOfInfinity}
\pmcreated{2013-03-22 13:43:52}
\pmmodified{2013-03-22 13:43:52}
\pmowner{Sabean}{2546}
\pmmodifier{Sabean}{2546}
\pmtitle{axiom of infinity}
\pmrecord{6}{34419}
\pmprivacy{1}
\pmauthor{Sabean}{2546}
\pmtype{Axiom}
\pmcomment{trigger rebuild}
\pmclassification{msc}{03E30}
\pmsynonym{infinity}{AxiomOfInfinity}

\endmetadata

% this is the default PlanetMath preamble.  as your knowledge
% of TeX increases, you will probably want to edit this, but
% it should be fine as is for beginners.

% almost certainly you want these
\usepackage{amssymb}
\usepackage{amsmath}
\usepackage{amsfonts}

% used for TeXing text within eps files
%\usepackage{psfrag}
% need this for including graphics (\includegraphics)
%\usepackage{graphicx}
% for neatly defining theorems and propositions
%\usepackage{amsthm}
% making logically defined graphics
%%%\usepackage{xypic}

% there are many more packages, add them here as you need them

% define commands h
\begin{document}
There exists an infinite set.

The Axiom of Infinity is an axiom of Zermelo-Fraenkel set theory.
At first glance, this axiom seems to be ill-defined.  How are we to know what
constitutes an infinite set when we have not yet defined the notion of a
finite set?  However, once we have a theory of ordinal numbers in hand, the axiom makes sense.

Meanwhile, we can give a definition of finiteness that does not rely upon
the concept of number.  We do this by introducing the notion of an inductive
set.  A set $S$ is said to be inductive if $\emptyset \in S$ 
and for every $x \in S$, $x \cup \{ x \} \in S$.  We may then state the 
Axiom of Infinity as follows:

There exists an inductive set.

In symbols:

\[
\exists S [\emptyset \in S \land (\forall x \in S)[x \cup \{ x \} \in S]]
\]

We shall then be able to prove that the following conditions are equivalent:
\begin{enumerate}
\item There exists an inductive set.
\item There exists an infinite set.
\item The least nonzero limit ordinal, $\omega$, is a set.
\end{enumerate}
%%%%%
%%%%%
\end{document}
