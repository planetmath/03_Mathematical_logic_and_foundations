\documentclass[12pt]{article}
\usepackage{pmmeta}
\pmcanonicalname{Intercession}
\pmcreated{2013-03-22 17:02:23}
\pmmodified{2013-03-22 17:02:23}
\pmowner{WM}{16977}
\pmmodifier{WM}{16977}
\pmtitle{intercession}
\pmrecord{10}{39327}
\pmprivacy{1}
\pmauthor{WM}{16977}
\pmtype{Definition}
\pmcomment{trigger rebuild}
\pmclassification{msc}{03E10}
\pmclassification{msc}{03E70}
\pmsynonym{intercede}{Intercession}
%\pmkeywords{cardinal number}
%\pmkeywords{equivalence}
%\pmkeywords{infinity}
%\pmkeywords{order}
%\pmkeywords{sets}

\endmetadata

% this is the default PlanetMath preamble.  as your knowledge
% of TeX increases, you will probably want to edit this, but
% it should be fine as is for beginners.

% almost certainly you want these
\usepackage{amssymb}
\usepackage{amsmath}
\usepackage{amsfonts}

% used for TeXing text within eps files
%\usepackage{psfrag}
% need this for including graphics (\includegraphics)
%\usepackage{graphicx}
% for neatly defining theorems and propositions
%\usepackage{amsthm}
% making logically defined graphics
%%%\usepackage{xypic}

% there are many more packages, add them here as you need them

% define commands here

\begin{document}
\PMlinkescapeword{words}

The intercession is an alternative measure for infinite sets of finite numbers.

Definition: Two infinite sets, \emph{A} and \emph{B}, \emph{intercede} (each other) if they can be put in an \emph{intercession}, i.e., if they can be ordered such that \emph{A} is dense in \emph{B} and \emph{B} is dense in \emph{A}. In other words, between two elements of \emph{A} there is at least one element of \emph{B} and, \emph{vice versa}, between two elements of \emph{B} there is at least one element of \emph{A}.

The intercession of sets with nonempty intersection, e.g., the intercession of a set with itself, requires distinction of identical elements. As an example an intercession of the set of positive integers and the set of even positive integers, 1, 2, 3, ... and 2', 4', 6', ..., is given by 1, 2', 2, 4', 3, 6', ....

The intercession includes Cantor's definition of equivalent (or equipotent) sets by one-to-one correspondence (or bijection): Two equivalent sets always intercede each other, i.e., they can always be put in an intercession. The intercession is an equivalence relation. All infinite sets of finite numbers (like the integers, the rationals or the reals) belong, under this relation, to the same equivalence class.

\textbf{Literature} W. M\"uckenheim: Die Mathematik des Unendlichen, Shaker, Aachen 2006.

%%%%%
%%%%%
\end{document}
