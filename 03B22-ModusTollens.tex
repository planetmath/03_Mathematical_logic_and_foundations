\documentclass[12pt]{article}
\usepackage{pmmeta}
\pmcanonicalname{ModusTollens}
\pmcreated{2013-03-22 16:56:03}
\pmmodified{2013-03-22 16:56:03}
\pmowner{rspuzio}{6075}
\pmmodifier{rspuzio}{6075}
\pmtitle{modus tollens}
\pmrecord{7}{39200}
\pmprivacy{1}
\pmauthor{rspuzio}{6075}
\pmtype{Definition}
\pmcomment{trigger rebuild}
\pmclassification{msc}{03B22}
\pmclassification{msc}{03B35}
\pmclassification{msc}{03B05}

% this is the default PlanetMath preamble.  as your knowledge
% of TeX increases, you will probably want to edit this, but
% it should be fine as is for beginners.

% almost certainly you want these
\usepackage{amssymb}
\usepackage{amsmath}
\usepackage{amsfonts}

% used for TeXing text within eps files
%\usepackage{psfrag}
% need this for including graphics (\includegraphics)
%\usepackage{graphicx}
% for neatly defining theorems and propositions
%\usepackage{amsthm}
% making logically defined graphics
%%%\usepackage{xypic}

% there are many more packages, add them here as you need them

% define commands here

\begin{document}
The law of \emph{modus tollens} is the inference rule which allows one to 
conclude $\neg P$ from $P \Rightarrow Q$ and $\neg Q$.  The name ``modus
tollens'' refers to the fact that this rule allows one to take away the 
conclusion of a conditional statement and conclude the negation of the 
condition.  As an example of this rule, we may cite the following:
\[
{{\hbox{If the postman is at the door, the doorbell will ring twice} \atop
\hbox{The bell is not ringing.}} \over
\hbox{The postman is not at the door.}}
\]

The validity of this rule may be established by means of the following
truth table:
\begin{center}
\begin{tabular}
{ccccc}
$P$ & $Q$ & $P \Rightarrow Q$ & $\neg P$ & $\neg Q$ \\
\hline
F & F & T & T & T \\
F & T & T & T & F \\
T & F & F & F & T \\
T & T & T & F & F
\end{tabular}
\end{center}

This rule can be used to justify the popular technique of proof by 
contradiction.  In this technique, one assumes a hypothesis $P$ and 
then derives a conclusion $Q$.  This is tantamount to showing that
$P \Rightarrow Q$.  Next one demonstrates $\neg Q$.  Applying modus
tollens, one then concludes $\neg P$.
%%%%%
%%%%%
\end{document}
