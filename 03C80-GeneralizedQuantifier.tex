\documentclass[12pt]{article}
\usepackage{pmmeta}
\pmcanonicalname{GeneralizedQuantifier}
\pmcreated{2013-03-22 12:59:57}
\pmmodified{2013-03-22 12:59:57}
\pmowner{Henry}{455}
\pmmodifier{Henry}{455}
\pmtitle{generalized quantifier}
\pmrecord{5}{33377}
\pmprivacy{1}
\pmauthor{Henry}{455}
\pmtype{Definition}
\pmcomment{trigger rebuild}
\pmclassification{msc}{03C80}
\pmclassification{msc}{03B15}
\pmclassification{msc}{03B10}
\pmrelated{quantifier}
\pmrelated{Quantifier}
\pmdefines{monadic}
\pmdefines{polyadic}

% this is the default PlanetMath preamble.  as your knowledge
% of TeX increases, you will probably want to edit this, but
% it should be fine as is for beginners.

% almost certainly you want these
\usepackage{amssymb}
\usepackage{amsmath}
\usepackage{amsfonts}

% used for TeXing text within eps files
%\usepackage{psfrag}
% need this for including graphics (\includegraphics)
%\usepackage{graphicx}
% for neatly defining theorems and propositions
%\usepackage{amsthm}
% making logically defined graphics
%%%\usepackage{xypic}

% there are many more packages, add them here as you need them

% define commands here
%\PMlinkescapeword{theory}
\begin{document}
\emph{Generalized quantifiers} are an abstract way of defining quantifiers.

The underlying principle is that formulas quantified by a generalized quantifier are true if the set of elements satisfying those formulas belong in some relation associated with the quantifier.

Every generalized quantifier has an arity, which is the number of formulas it takes as arguments, and a type, which for an $n$-ary quantifier is a tuple of length $n$.  The tuple represents the number of quantified variables for each argument.

The most common quantifiers are those of type $\langle 1\rangle$, including $\forall$ and $\exists$.  If $Q$ is a quantifier of type $\langle 1\rangle$, $M$ is the universe of a model, and $Q_M$ is the relation associated with $Q$ in that model, then $Qx\phi(x)\leftrightarrow \{x\in M\mid \phi(x)\}\in Q_M$.

So $\forall_M=\{M\}$, since the quantified formula is only true when all elements satisfy it.  On the other hand $\exists_M=P(M)-\{\emptyset\}$.

In general, the \emph{monadic} quantifiers are those of type $\langle 1,\ldots, 1\rangle$ and if $Q$ is an $n$-ary monadic quantifier then $Q_M\subseteq P(M)^n$.  H\"artig's quantifier, for instance, is $\langle 1,1\rangle$, and $I_M=\{\langle X,Y\rangle\mid X,Y\subseteq M\wedge |X|=|Y|\}$.

A quantifier $Q$ is \emph{polyadic} if it is of type $\langle n_1,\ldots, n_n\rangle$ where each $n_i\in\mathbb{N}$.  Then:

$$Q_M\subseteq \prod_{i} P(M^{n_i})$$

These can get quite elaborate; $Wxy\phi(x,y)$ is a $\langle 2\rangle$ quantifier where $X\in W_M\leftrightarrow X$\texttt{ is a well-ordering}.  That is, it is true if the set of pairs making $\phi$ true is a well-ordering.
%%%%%
%%%%%
\end{document}
