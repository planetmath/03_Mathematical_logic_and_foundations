\documentclass[12pt]{article}
\usepackage{pmmeta}
\pmcanonicalname{Chapter10Notes}
\pmcreated{2013-11-06 17:21:25}
\pmmodified{2013-11-06 17:21:25}
\pmowner{PMBookProject}{1000683}
\pmmodifier{PMBookProject}{1000683}
\pmtitle{Chapter 10 Notes}
\pmrecord{1}{}
\pmprivacy{1}
\pmauthor{PMBookProject}{1000683}
\pmtype{Feature}
\pmclassification{msc}{03B15}

\usepackage{xspace}
\usepackage{amssyb}
\usepackage{amsmath}
\usepackage{amsfonts}
\usepackage{amsthm}
\makeatletter
\newcommand{\choice}[1]{\ensuremath{\mathsf{AC}_{#1}}\xspace}
\newcommand{\LEM}[1]{\ensuremath{\mathsf{LEM}_{#1}}\xspace}
\newcommand{\sectionNotes}{\phantomsection\section*{Notes}\addcontentsline{toc}{section}{Notes}\markright{\textsc{\@chapapp{} \thechapter{} Notes}}}
\newcommand{\UU}{\ensuremath{\mathcal{U}}\xspace}
\let\autoref\cref
\makeatother
\begin{document}
The basic properties one expects of the category of sets date back to the early days of elementary topos theory.
The \emph{Elementary theory of the category of sets} referred to in \autoref{subsec:emacinsets} was introduced by Lawvere\index{Lawvere} in
\cite{lawvere:etcs-long}, as a category-theoretic axiomatization of set theory.
\index{Elementary Theory of the Category of Sets}%
The notion of $\Pi W$-pretopos, regarded as a predicative version of an elementary topos, was introduced in~\cite{MoerdijkPalmgren2002}; see also~\cite{palmgren:cetcs}.

The treatment of the category of sets in \autoref{sec:piw-pretopos} roughly follows that in~\cite{RijkeSpitters}.
The fact that epimorphisms are surjective (\autoref{epis-surj}) is well known in classical mathematics, but is not as trivial as it may seem to prove \emph{predicatively}.
\index{mathematics!predicative}%
The proof in~\cite{Mines/R/R:1988} uses the power set operation (which is impredicative), although it can also be seen as a predicative proof of the weaker statement that a map in a universe $\UU_i$ is surjective if it is an epimorphism in the next universe $\UU_{i+1}$.
A predicative proof for setoids was given by Wilander~\cite{Wilander2010}. 
Our proof is similar to Wilander's, but avoids setoids by using pushouts and univalence.

The implication in \autoref{thm:1surj_to_surj_to_pem} from $\choice{}$ to $\LEM{}$ is an adaptation to homotopy type
theory of a theorem from topos theory due to Diaconescu~\cite{Diaconescu}; it was posed as a problem already by Bishop~\cite[Problem~2]{Bishop1967}.

For the intuitionistic theory of ordinal numbers, see~\cite{taylor:ordinals} and also \cite{JoyalMoerdijk1995}.
Definitions of well-foundedness in type theory by an induction principle, including the inductive predicate of accessibility\index{accessibility}, were studied in~\cite{Huet80,Paulson86,Nordstrom88}, although the idea dates back to Getzen's proof of the consistency\index{consistency!of arithmetic} of arithmetic~\cite{Gentzen36}.

The idea of algebraic set theory, which informs our development in \autoref{sec:cumulative-hierarchy} of the cumulative hierarchy, is due to~\cite{JoyalMoerdijk1995}, but it derives from earlier work by~\cite{AczelCZF}.
\index{algebraic set theory}%
\index{set theory!algebraic}%


%%%%%%%%%%%%%%%%%%%%%%%%%%%%%%%%%%%%%%%%%%%%%%%%%%%%%%%%%%%%%%%%%%%%%%

\end{document}
