\documentclass[12pt]{article}
\usepackage{pmmeta}
\pmcanonicalname{Disjunction}
\pmcreated{2013-03-22 11:54:40}
\pmmodified{2013-03-22 11:54:40}
\pmowner{akrowne}{2}
\pmmodifier{akrowne}{2}
\pmtitle{disjunction}
\pmrecord{14}{30536}
\pmprivacy{1}
\pmauthor{akrowne}{2}
\pmtype{Definition}
\pmcomment{trigger rebuild}
\pmclassification{msc}{03B05}
\pmsynonym{logical or}{Disjunction}
\pmsynonym{disjunctive truth function}{Disjunction}
\pmrelated{Conjunction}
\pmrelated{PropositionalLogic}

\usepackage{amssymb}
\usepackage{amsmath}
\usepackage{amsfonts}
\usepackage{graphicx}
%%%%\usepackage{xypic}
\begin{document}
A disjunction is true if either of its parameters (called disjuncts) are true.  Disjunction does \emph{not} correspond to ``or'' in English (see exclusive or.)  Disjunction uses the symbol $\lor$ or sometimes $+$ when taken in algebraic context.  Hence, disjunction of $a$ and $b$ would be written $$ a \lor b $$ or $$ a + b $$  The truth table for  disjunction is 

\begin{center}
\begin{tabular}{ccc}
$a$ & $b$ & $a \lor b$ \\
\hline
F & F & F \\
F & T & T \\
T & F & T \\
T & T & T
\end{tabular}
\end{center}
%%%%%
%%%%%
%%%%%
%%%%%
\end{document}
