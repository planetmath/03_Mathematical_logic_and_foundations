\documentclass[12pt]{article}
\usepackage{pmmeta}
\pmcanonicalname{10SetTheory}
\pmcreated{2013-11-06 17:02:19}
\pmmodified{2013-11-06 17:02:19}
\pmowner{PMBookProject}{1000683}
\pmmodifier{PMBookProject}{1000683}
\pmtitle{10. Set theory}
\pmrecord{1}{}
\pmprivacy{1}
\pmauthor{PMBookProject}{1000683}
\pmtype{Feature}
\pmclassification{msc}{03B15}

\endmetadata

\usepackage{xspace}
\usepackage{amssyb}
\usepackage{amsmath}
\usepackage{amsfonts}
\usepackage{amsthm}
\newcommand{\choice}[1]{\ensuremath{\mathsf{AC}_{#1}}\xspace}
\newcommand{\LEM}[1]{\ensuremath{\mathsf{LEM}_{#1}}\xspace}
\newcommand{\uset}{\ensuremath{\mathcal{S}et}\xspace}
\let\autoref\cref
\begin{document}

\index{set|(}%

Our conception of sets as types with particularly simple homotopical character, cf.\
\autoref{sec:basics-sets}, is quite different from the sets of Zermelo--Fraenkel\index{set theory!Zermelo--Fraenkel} set theory, which form a
cumulative hierarchy with an intricate nested membership structure.
For many mathematical purposes, the homotopy-the\-o\-ret\-ic sets are just as good as
the Zermelo--Fraenkel ones, but there are important differences.

We begin this chapter in \autoref{sec:piw-pretopos} by showing that the category $\uset$ has (most of) the usual properties of the category of sets.
\index{mathematics!constructive}%
\index{mathematics!predicative}%
In constructive, predicative, univalent foundations, it is a ``$\Pi\mathsf{W}$-pretopos''; whereas if we assume propositional resizing
\index{propositional!resizing}%
(\autoref{subsec:prop-subsets}) it is an elementary topos,\index{topos} and if we assume \LEM{} and \choice{} then it is a model of Lawvere's \emph{Elementary Theory of the Category of Sets}\index{Lawvere}.
\index{Elementary Theory of the Category of Sets}%
This is sufficient to ensure that the sets in homotopy type theory behave like sets as used by most mathematicians outside of set theory.

In the rest of the chapter, we investigate some subjects that traditionally belong to ``set theory''.
In \autoref{sec:cardinals,sec:ordinals,sec:wellorderings} we study cardinal and ordinal numbers.
These are traditionally defined in set theory using the global membership relation, but we will see that the univalence axiom enables an equally convenient, more ``structural'' approach.

Finally, in \autoref{sec:cumulative-hierarchy} we consider the possibility of constructing \emph{inside} of homotopy type theory a cumulative hierarchy of sets, equipped with a binary membership relation akin to that of Zermelo--Fraenkel set theory.
This combines higher inductive types with ideas from the field of algebraic set theory.
\index{algebraic set theory}%
\index{set theory!algebraic}%

In this chapter we will often use the traditional logical notation described in \autoref{subsec:prop-trunc}.
In addition to the basic theory of \autoref{cha:basics},\autoref{cha:logic}, we use higher inductive types for colimits and quotients as in \autoref{sec:colimits},\autoref{sec:set-quotients}, as well as some of the theory of truncation from \autoref{cha:hlevels}, particularly the factorization system of \autoref{sec:image-factorization} in the case $n=-1$.
In \autoref{sec:ordinals} we use an inductive family (\autoref{sec:generalizations}) to describe well-foundedness, and in \autoref{sec:cumulative-hierarchy} we use a more complicated higher inductive type to present the cumulative hierarchy.


%\section{\texorpdfstring{$\set$}{Set} is a \texorpdfstring{$\Pi$}{Π}W-pretopos}

\end{document}
