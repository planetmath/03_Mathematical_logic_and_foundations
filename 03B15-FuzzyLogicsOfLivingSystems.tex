\documentclass[12pt]{article}
\usepackage{pmmeta}
\pmcanonicalname{FuzzyLogicsOfLivingSystems}
\pmcreated{2013-03-22 18:23:58}
\pmmodified{2013-03-22 18:23:58}
\pmowner{bci1}{20947}
\pmmodifier{bci1}{20947}
\pmtitle{fuzzy logics of living systems}
\pmrecord{17}{41045}
\pmprivacy{1}
\pmauthor{bci1}{20947}
\pmtype{Feature}
\pmcomment{trigger rebuild}
\pmclassification{msc}{03B15}
\pmclassification{msc}{03B10}
\pmsynonym{biological variability}{FuzzyLogicsOfLivingSystems}
\pmsynonym{logics of variable supercomplex systems}{FuzzyLogicsOfLivingSystems}
%\pmkeywords{fuzzy logics of living organisms represented as supercomplex systems}
\pmrelated{FuzzyLogic2}
\pmdefines{operational logic of super-complex systems}

\endmetadata

% this is the default PlanetMath preamble.  as your knowledge
% of TeX increases, you will probably want to edit this, but
% it should be fine as is for beginners.

% almost certainly you want these
\usepackage{amssymb}
\usepackage{amsmath}
\usepackage{amsfonts}

% used for TeXing text within eps files
%\usepackage{psfrag}
% need this for including graphics (\includegraphics)
%\usepackage{graphicx}
% for neatly defining theorems and propositions
%\usepackage{amsthm}
% making logically defined graphics
%%%\usepackage{xypic}

% there are many more packages, add them here as you need them

% define commands here
\usepackage{amsmath, amssymb, amsfonts, amsthm, amscd, latexsym}
%%\usepackage{xypic}
\usepackage[mathscr]{eucal}

\setlength{\textwidth}{6.5in}
%\setlength{\textwidth}{16cm}
\setlength{\textheight}{9.0in}
%\setlength{\textheight}{24cm}

\hoffset=-.75in     %%ps format
%\hoffset=-1.0in     %%hp format
\voffset=-.4in

\theoremstyle{plain}
\newtheorem{lemma}{Lemma}[section]
\newtheorem{proposition}{Proposition}[section]
\newtheorem{theorem}{Theorem}[section]
\newtheorem{corollary}{Corollary}[section]

\theoremstyle{definition}
\newtheorem{definition}{Definition}[section]
\newtheorem{example}{Example}[section]
%\theoremstyle{remark}
\newtheorem{remark}{Remark}[section]
\newtheorem*{notation}{Notation}
\newtheorem*{claim}{Claim}

\renewcommand{\thefootnote}{\ensuremath{\fnsymbol{footnote%%@
}}}
\numberwithin{equation}{section}

\newcommand{\Ad}{{\rm Ad}}
\newcommand{\Aut}{{\rm Aut}}
\newcommand{\Cl}{{\rm Cl}}
\newcommand{\Co}{{\rm Co}}
\newcommand{\DES}{{\rm DES}}
\newcommand{\Diff}{{\rm Diff}}
\newcommand{\Dom}{{\rm Dom}}
\newcommand{\Hol}{{\rm Hol}}
\newcommand{\Mon}{{\rm Mon}}
\newcommand{\Hom}{{\rm Hom}}
\newcommand{\Ker}{{\rm Ker}}
\newcommand{\Ind}{{\rm Ind}}
\newcommand{\IM}{{\rm Im}}
\newcommand{\Is}{{\rm Is}}
\newcommand{\ID}{{\rm id}}
\newcommand{\GL}{{\rm GL}}
\newcommand{\Iso}{{\rm Iso}}
\newcommand{\Sem}{{\rm Sem}}
\newcommand{\St}{{\rm St}}
\newcommand{\Sym}{{\rm Sym}}
\newcommand{\SU}{{\rm SU}}
\newcommand{\Tor}{{\rm Tor}}
\newcommand{\U}{{\rm U}}

\newcommand{\A}{\mathcal A}
\newcommand{\Ce}{\mathcal C}
\newcommand{\D}{\mathcal D}
\newcommand{\E}{\mathcal E}
\newcommand{\F}{\mathcal F}
\newcommand{\G}{\mathcal G}
\newcommand{\Q}{\mathcal Q}
\newcommand{\R}{\mathcal R}
\newcommand{\cS}{\mathcal S}
\newcommand{\cU}{\mathcal U}
\newcommand{\W}{\mathcal W}

\newcommand{\bA}{\mathbb{A}}
\newcommand{\bB}{\mathbb{B}}
\newcommand{\bC}{\mathbb{C}}
\newcommand{\bD}{\mathbb{D}}
\newcommand{\bE}{\mathbb{E}}
\newcommand{\bF}{\mathbb{F}}
\newcommand{\bG}{\mathbb{G}}
\newcommand{\bK}{\mathbb{K}}
\newcommand{\bM}{\mathbb{M}}
\newcommand{\bN}{\mathbb{N}}
\newcommand{\bO}{\mathbb{O}}
\newcommand{\bP}{\mathbb{P}}
\newcommand{\bR}{\mathbb{R}}
\newcommand{\bV}{\mathbb{V}}
\newcommand{\bZ}{\mathbb{Z}}

\newcommand{\bfE}{\mathbf{E}}
\newcommand{\bfX}{\mathbf{X}}
\newcommand{\bfY}{\mathbf{Y}}
\newcommand{\bfZ}{\mathbf{Z}}

\renewcommand{\O}{\Omega}
\renewcommand{\o}{\omega}
\newcommand{\vp}{\varphi}
\newcommand{\vep}{\varepsilon}

\newcommand{\diag}{{\rm diag}}
\newcommand{\grp}{{\mathbb G}}
\newcommand{\dgrp}{{\mathbb D}}
\newcommand{\desp}{{\mathbb D^{\rm{es}}}}
\newcommand{\Geod}{{\rm Geod}}
\newcommand{\geod}{{\rm geod}}
\newcommand{\hgr}{{\mathbb H}}
\newcommand{\mgr}{{\mathbb M}}
\newcommand{\ob}{{\rm Ob}}
\newcommand{\obg}{{\rm Ob(\mathbb G)}}
\newcommand{\obgp}{{\rm Ob(\mathbb G')}}
\newcommand{\obh}{{\rm Ob(\mathbb H)}}
\newcommand{\Osmooth}{{\Omega^{\infty}(X,*)}}
\newcommand{\ghomotop}{{\rho_2^{\square}}}
\newcommand{\gcalp}{{\mathbb G(\mathcal P)}}

\newcommand{\rf}{{R_{\mathcal F}}}
\newcommand{\glob}{{\rm glob}}
\newcommand{\loc}{{\rm loc}}
\newcommand{\TOP}{{\rm TOP}}

\newcommand{\wti}{\widetilde}
\newcommand{\what}{\widehat}

\renewcommand{\a}{\alpha}
\newcommand{\be}{\beta}
\newcommand{\ga}{\gamma}
\newcommand{\Ga}{\Gamma}
\newcommand{\de}{\delta}
\newcommand{\del}{\partial}
\newcommand{\ka}{\kappa}
\newcommand{\si}{\sigma}
\newcommand{\ta}{\tau}
\newcommand{\med}{\medbreak}
\newcommand{\medn}{\medbreak \noindent}
\newcommand{\bign}{\bigbreak \noindent}
\newcommand{\lra}{{\longrightarrow}}
\newcommand{\ra}{{\rightarrow}}
\newcommand{\rat}{{\rightarrowtail}}
\newcommand{\oset}[1]{\overset {#1}{\ra}}
\newcommand{\osetl}[1]{\overset {#1}{\lra}}
\newcommand{\hr}{{\hookrightarrow}}
\begin{document}
\subsection{Fuzzy logics of living organisms.}
Living organisms or biosystems can be represented as 
\PMlinkname{super-complex systems}{ComplexSystemsBiology} with dynamics
that is not reducible to that of their components, such as molecules and atoms. It is an
empirically accepted fact that living organisms exhibit a wide degree of `biological variability': 
genetic/epigenetic and also phenotypic/metabolic within the same species; their behavior 
and dynamics thus exhibit a type of `fuzziness' (refs.\cite{ICBM1,ICB77}) that unlike Zadeh's 
fuzzy sets characteristic (\cite{ZLA1,ZLA2}) is neither random nor always following a (symmetric) Gaussian distribution. 
It has been proposed that the operational logics underlying 
\PMlinkname{super-complex systems dynamics}{FundamentalDiagramsInCategoricalTheoryOfLevels} are
\PMlinkname{$LM_n$ many-valued logics}{AlgebraicCategoryOfLMnLogicAlgebras} for both genetic 
and neural networks (refs. \cite{ICB77,ICB2k4}). 


{\bf [Entry under construction]}

\begin{thebibliography}{99}

\bibitem{GG2k6}
Georgescu, G. 2006, N-valued Logics and \L ukasiewicz-Moisil
Algebras, \emph{Axiomathes}, \textbf{16} (1-2): 123-136.

\bibitem{ICBM1}
Baianu, I.C. and M. Marinescu: 1968, Organismic Supercategories:
Towards a Unitary Theory of Systems. \emph{Bulletin of Mathematical Biophysics} \textbf{30}, 148-159.

\bibitem{ICB77}
Baianu, I.C.: 1977, A Logical Model of Genetic Activities in \L ukasiewicz Algebras: The Non-linear Theory. \emph{Bulletin of Mathematical Biology}, \textbf{39}: 249-258.
    
\bibitem{ICB87a}
Baianu, I. C.: 1986--1987a, Computer Models and Automata Theory in Biology and Medicine., in M. Witten (ed.),
\emph{Mathematical Models in Medicine}, vol. 7., Ch.11 Pergamon Press, New York, 1513 -1577; URLs:
\PMlinkexternal{CERN Preprint No. EXT-2004-072}{http://doe.cern.ch//archive/electronic/other/ext/ext-2004-072.pdf} ,
and \PMlinkexternal{html Abstract}{http://en.scientificcommons.org/1857371}.

\bibitem{ICB87b}
Baianu, I. C.: 1987b, Molecular Models of Genetic and Organismic Structures, in \emph{Proceed. Relational Biology Symp.}
Argentina;
\PMlinkexternal{CERN Preprint No.EXT-2004-067}{http://doc.cern.ch//archive/electronic/other/ext/ext-2004-067.pdf} .

\bibitem{ICB2k4}
Baianu, I.C.: 2004. \L{}ukasiewicz-Topos Models of Neural Networks, Cell Genome and Interactome Nonlinear Dynamic Models (2004). Eprint: w. Cogprints at Sussex Univ. 

\bibitem{ZLA1}
Zadeh, L.A., Fuzzy Sets, {\em Information and Control}, 8 (1965) 338í-353.

\bibitem{ZLA2}
Zadeh L. A., The concept of a linguistic variable and its application to approximate reasoning I, II, III, {\em Information Sciences}, vol. 8, 9(1975), pp. 199-275, 301-357, 43-80.

\end{thebibliography}
%%%%%
%%%%%
\end{document}
