\documentclass[12pt]{article}
\usepackage{pmmeta}
\pmcanonicalname{HartigsQuantifier}
\pmcreated{2013-03-22 12:59:16}
\pmmodified{2013-03-22 12:59:16}
\pmowner{Henry}{455}
\pmmodifier{Henry}{455}
\pmtitle{H\"artig's quantifier}
\pmrecord{7}{33362}
\pmprivacy{1}
\pmauthor{Henry}{455}
\pmtype{Definition}
\pmcomment{trigger rebuild}
\pmclassification{msc}{03B15}
\pmrelated{Quantifier}
\pmdefines{Rescher quantifier}

% this is the default PlanetMath preamble.  as your knowledge
% of TeX increases, you will probably want to edit this, but
% it should be fine as is for beginners.

% almost certainly you want these
\usepackage{amssymb}
\usepackage{amsmath}
\usepackage{amsfonts}

% used for TeXing text within eps files
%\usepackage{psfrag}
% need this for including graphics (\includegraphics)
%\usepackage{graphicx}
% for neatly defining theorems and propositions
%\usepackage{amsthm}
% making logically defined graphics
%%%\usepackage{xypic}

% there are many more packages, add them here as you need them

% define commands here
%\PMlinkescapeword{theory}
\begin{document}
\emph{H\"artig's quantifier} is a quantifier which takes two variables and two formulas, written $Ixy\phi(x)\psi(y)$.  It asserts that $|\{x\mid \phi(x)\}|=|\{y\mid\psi(y)\}|$.  That is, the cardinality of the values of $x$ which make $\phi$ is the same as the cardinality of the values which make $\psi(x)$ true.  Viewed as a generalized quantifier, $I$ is a $\langle 2\rangle$ quantifier.

Closely related is the \emph{Rescher quantifier}, which also takes two variables and two formulas, is written $Jxy\phi(x)\psi(y)$, and asserts that $|\{x\mid \phi(x)\}|\leq|\{y\mid\psi(y)|$.  The Rescher quantifier is sometimes defined instead to be a similar but different quantifier, $Jx\phi(x)\leftrightarrow |\{x\mid\phi(x)\}|>|\{x\mid\neg\phi(x)\}|$.  The first definition is a $\langle 2\rangle$ quantifier while the second is a $\langle 1\rangle$ quantifier.

Another similar quantifier is Chang's quantifier $Q^C$, a $\langle 1\rangle$ quantifier defined by $Q^C_M=\{X\subseteq M\mid |X|=|M|\}$.  That is, $Q^Cx\phi(x)$ is true if the number of $x$ satisfying $\phi$ has the same cardinality as the universe; for finite models this is the same as $\forall$, but for infinite ones it is not.
%%%%%
%%%%%
\end{document}
