\documentclass[12pt]{article}
\usepackage{pmmeta}
\pmcanonicalname{2142EqualityOfSemigroups}
\pmcreated{2013-11-16 3:57:26}
\pmmodified{2013-11-16 3:57:26}
\pmowner{PMBookProject}{1000683}
\pmmodifier{rspuzio}{6075}
\pmtitle{2.14.2 Equality of semigroups}
\pmrecord{6}{87622}
\pmprivacy{1}
\pmauthor{PMBookProject}{6075}
\pmtype{Feature}
\pmclassification{msc}{03B15}

\usepackage{xspace}
\usepackage{amssyb}
\usepackage{amsmath}
\usepackage{amsfonts}
\usepackage{amsthm}
\makeatletter
\def\@dprd#1{\prod_{(#1)}\,}
\def\@dprd@noparens#1{\prod_{#1}\,}
\def\@dsm#1{\sum_{(#1)}\,}
\def\@dsm@noparens#1{\sum_{#1}\,}
\def\@eatprd\prd{\prd@parens}
\def\@eatsm\sm{\sm@parens}
\newcommand{\narrowequation}[1]{$#1$}
\newcommand{\opp}[1]{\mathord{{#1}^{-1}}}
\def\prd#1{\@ifnextchar\bgroup{\prd@parens{#1}}{\@ifnextchar\sm{\prd@parens{#1}\@eatsm}{\prd@noparens{#1}}}}
\def\prd@noparens#1{\mathchoice{\@dprd@noparens{#1}}{\@tprd{#1}}{\@tprd{#1}}{\@tprd{#1}}}
\def\prd@parens#1{\@ifnextchar\bgroup  {\mathchoice{\@dprd{#1}}{\@tprd{#1}}{\@tprd{#1}}{\@tprd{#1}}\prd@parens}  {\@ifnextchar\sm    {\mathchoice{\@dprd{#1}}{\@tprd{#1}}{\@tprd{#1}}{\@tprd{#1}}\@eatsm}    {\mathchoice{\@dprd{#1}}{\@tprd{#1}}{\@tprd{#1}}{\@tprd{#1}}}}}
\newcommand{\semigroup}[0]{\ensuremath{\mathsf{Semigroup}}}
\newcommand{\semigroupstrsym}{\ensuremath{\mathsf{SemigroupStr}}}
\def\sm#1{\@ifnextchar\bgroup{\sm@parens{#1}}{\@ifnextchar\prd{\sm@parens{#1}\@eatprd}{\sm@noparens{#1}}}}
\def\sm@noparens#1{\mathchoice{\@dsm@noparens{#1}}{\@tsm{#1}}{\@tsm{#1}}{\@tsm{#1}}}
\def\sm@parens#1{\@ifnextchar\bgroup  {\mathchoice{\@dsm{#1}}{\@tsm{#1}}{\@tsm{#1}}{\@tsm{#1}}\sm@parens}  {\@ifnextchar\prd    {\mathchoice{\@dsm{#1}}{\@tsm{#1}}{\@tsm{#1}}{\@tsm{#1}}\@eatprd}    {\mathchoice{\@dsm{#1}}{\@tsm{#1}}{\@tsm{#1}}{\@tsm{#1}}}}}
\def\@tprd#1{\mathchoice{{\textstyle\prod_{(#1)}}}{\prod_{(#1)}}{\prod_{(#1)}}{\prod_{(#1)}}}
\newcommand{\transfib}[3]{\ensuremath{\mathsf{transport}^{#1}(#2,#3)\xspace}}
\def\@tsm#1{\mathchoice{{\textstyle\sum_{(#1)}}}{\sum_{(#1)}}{\sum_{(#1)}}{\sum_{(#1)}}}
\newcommand{\ua}{\ensuremath{\mathsf{ua}}\xspace} 
\newcommand{\UU}{\ensuremath{\mathcal{U}}\xspace}
\let\autoref\cref
\let\type\UU
\makeatother

\begin{document}
Using the equations for path spaces discussed in the previous sections,
we can investigate when two semigroups are equal. Given semigroups
$(A,m,a)$ and $(B,m',a')$, by \PMlinkname{Theorem 2.7.2}{27sigmatypes#Thmprethm1}, the type of paths
\narrowequation{
  (A,m,a) =_\semigroup (B,m',a')
}
is equal to the type of pairs
\begin{align*}
p_1 &: A =_{\type} B \qquad\text{and}\\
p_2 &: \transfib{\semigroupstrsym}{p_1}{(m,a)} = {(m',a')}.
\end{align*}
By univalence, $p_1$ is $\ua(e)$ for some equivalence $e$. By
\PMlinkname{Theorem 2.7.2}{27sigmatypes#Thmprethm1}, function extensionality, and the above analysis of
transport in the type family $\semigroupstrsym$, $p_2$ is equivalent to a pair
of proofs, the first of which shows that
\begin{equation*} \label{eq:equality-semigroup-mult}
\prd{y_1,y_2:B} e(m(\opp{e}(y_1), \opp{e}(y_2))) = m'(y_1,y_2)
\end{equation*}
and the second of which shows that $a'$ is equal to the induced
associativity proof constructed from $a$ in
\PMlinkname{2.14.3}{2141liftingequivalences#S0.E1}.  But by cancellation of inverses
\PMlinkname{2.14.2}{2141liftingequivalences#S0.E2X} is equivalent to
\[
\prd{x_1,x_2:A} e(m(x_1, x_2)) = m'(e(x_2),e(x_2)).
\]
This says that $e$ commutes with the binary operation, in the sense
that it takes multiplication in $A$ (i.e.\ $m$) to multiplication in $B$
(i.e.\ $m'$).  A similar rearrangement is possible for the equation relating
$a$ and $a'$.  Thus, an equality of semigroups consists exactly of an
equivalence on the carrier types that commutes with the semigroup
structure.  

For general types, the proof of associativity is thought of as part of
the structure of a semigroup.  However, if we restrict to set-like types
(again, see \PMlinkexternal{Chapter 3}{http://planetmath.org/node/87576}), the
equation relating $a$ and $a'$ is trivially true.  Moreover, in this
case, an equivalence between sets is exactly a bijection.  Thus, we have
arrived at a standard definition of a \emph{semigroup isomorphism}:\index{isomorphism!semigroup} a
bijection on the carrier sets that preserves the multiplication
operation.  It is also possible to use the category-theoretic definition
of isomorphism, by defining a \emph{semigroup homomorphism}\index{homomorphism!semigroup} to be a map
that preserves the multiplication, and arrive at the conclusion that equality of
semigroups is the same as two mutually inverse homomorphisms; but we
will not show the details here; see \PMlinkname{\S 9.8}{98thestructureidentityprinciple}.

The conclusion is that, thanks to univalence, semigroups are equal
precisely when they are isomorphic as algebraic structures. As we will see in \PMlinkname{\S 9.8}{98thestructureidentityprinciple}, the
conclusion applies more generally: in homotopy type theory, all constructions of
mathematical structures automatically respect isomorphisms, without any
tedious proofs or abuse of notation.


\end{document}
