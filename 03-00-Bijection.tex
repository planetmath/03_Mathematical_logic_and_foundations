\documentclass[12pt]{article}
\usepackage{pmmeta}
\pmcanonicalname{Bijection}
\pmcreated{2013-03-22 11:51:35}
\pmmodified{2013-03-22 11:51:35}
\pmowner{mathcam}{2727}
\pmmodifier{mathcam}{2727}
\pmtitle{bijection}
\pmrecord{16}{30425}
\pmprivacy{1}
\pmauthor{mathcam}{2727}
\pmtype{Definition}
\pmcomment{trigger rebuild}
\pmclassification{msc}{03-00}
\pmclassification{msc}{83-00}
\pmclassification{msc}{81-00}
\pmclassification{msc}{82-00}
\pmsynonym{bijective}{Bijection}
\pmsynonym{bijective function}{Bijection}
\pmsynonym{1-1 correspondence}{Bijection}
\pmsynonym{1 to 1 correspondence}{Bijection}
\pmsynonym{one to one correspondence}{Bijection}
\pmsynonym{one-to-one correspondence}{Bijection}
%\pmkeywords{Set}
\pmrelated{Function}
\pmrelated{Permutation}
\pmrelated{InjectiveFunction}
\pmrelated{Surjective}
\pmrelated{Isomorphism2}
\pmrelated{CardinalityOfAFiniteSetIsUnique}
\pmrelated{CardinalityOfDisjointUnionOfFiniteSets}
\pmrelated{AConnectedNormalSpaceWithMoreThanOnePointIsUncountable2}
\pmrelated{AConnectedNormalSpaceWithMoreThanOnePointIsUncountable}
\pmrelated{Bo}

\endmetadata

\usepackage{amssymb}
\usepackage{amsmath}
\usepackage{amsfonts}
\usepackage{graphicx}
%%%%\usepackage{xypic}
\begin{document}
Let $X$ and $Y$ be sets. A function $f\colon X\to Y$ that is one-to-one and onto is called a \emph{bijection} or \emph{bijective function} from $X$ to $Y$.

When $X=Y$, $f$ is also called a \emph{permutation} of $X$.

An important consequence of the bijectivity of a function $f$ is the existence of an inverse function $f^{-1}$.  Specifically, a function is invertible if and only if it is bijective.  Thus if $f:X\rightarrow Y$ is a bijection, then for any $A\subset X$ and $B\subset Y$ we have 

\begin{align*}
f\circ f^{-1}(B)&=B\\
f^{-1}\circ f(A)&=A\\
\end{align*}

It easy to see the inverse of a bijection is a bijection, and that a composition of bijections is again bijective.
%%%%%
%%%%%
%%%%%
%%%%%
\end{document}
