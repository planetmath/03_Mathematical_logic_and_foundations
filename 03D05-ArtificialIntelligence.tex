\documentclass[12pt]{article}
\usepackage{pmmeta}
\pmcanonicalname{ArtificialIntelligence}
\pmcreated{2013-03-22 15:31:51}
\pmmodified{2013-03-22 15:31:51}
\pmowner{aplant}{12431}
\pmmodifier{aplant}{12431}
\pmtitle{artificial intelligence}
\pmrecord{15}{37418}
\pmprivacy{1}
\pmauthor{aplant}{12431}
\pmtype{Topic}
\pmcomment{trigger rebuild}
\pmclassification{msc}{03D05}
\pmclassification{msc}{18B20}
\pmclassification{msc}{68T40}
\pmclassification{msc}{68T27}
\pmclassification{msc}{68-00}
\pmclassification{msc}{68T01}
\pmsynonym{computer programmimg}{ArtificialIntelligence}
\pmsynonym{symbolic computation}{ArtificialIntelligence}
\pmsynonym{computer simulation}{ArtificialIntelligence}
\pmsynonym{sub-symbolic computation}{ArtificialIntelligence}
\pmsynonym{super-complex systems}{ArtificialIntelligence}
\pmsynonym{universal Turing machines}{ArtificialIntelligence}
%\pmkeywords{AI}
%\pmkeywords{parallel computation}
%\pmkeywords{symbolic computation}
%\pmkeywords{automatic theorem proving}
%\pmkeywords{computer programmimg}
%\pmkeywords{super-complex systems}
%\pmkeywords{universal Turing machines}
\pmrelated{CategoryOfAutomata}
\pmrelated{Automaton}
\pmrelated{CategoryOfMRSystems3}
\pmrelated{StateMachine}
\pmrelated{UniversalTuringMachine}
\pmrelated{StrongAIThesis}
\pmrelated{Supercomputers2}

% this is the default PlanetMath preamble.  as your knowledge
% of TeX increases, you will probably want to edit this, but
% it should be fine as is for beginners.

% almost certainly you want these
\usepackage{amssymb}
\usepackage{amsmath}
\usepackage{amsfonts}

% used for TeXing text within eps files
%\usepackage{psfrag}
% need this for including graphics (\includegraphics)
%\usepackage{graphicx}
% for neatly defining theorems and propositions
%\usepackage{amsthm}
% making logically defined graphics
%%%\usepackage{xypic}

% there are many more packages, add them here as you need them

% define commands here
\begin{document}
\emph{Artificial intelligence} aims to mimic the `operation' of the human mind using sequential machines, automata, robots, or computers. Indeed there are two different claims on how far AI can go in exhibiting human behaviors, and especially in emulating the actions of the human mind:

\begin{itemize}
\item Strong AI thesis, or 
\PMlinkexternal{`general intelligence'}{http://en.wikipedia.org/wiki/Artificial_intelligence}, 
is a very long term aim of AI research in Computer Science that might not be achievable
within the bounds of Boolean logic if it does aim to match human intelligence that often operates
beyond chrysippian, formal or even symbolic two-value logic, and indeed as it remains currently undefined (or undefinable ?) in terms of either Boolean or symbolic, two-value logic. \\
\bigbreak

\item Weak AI thesis: 

\begin{enumerate}
\item Knowledge-based AI 

\item Symbolic computation AI

\item Sub-symbolic AI

\item Computational `Intelligence' and `Neural' Networks or Nets

\end{enumerate}

\end{itemize}

{\bf Remarks:}
\begin{enumerate}
\item A \PMlinkexternal{Super-complex Computer System Architecture}{http://etd.library.pitt.edu/ETD/available/etd-03102008-120235/unrestricted/ColinIhrig-ms-3-18-08.pdf} may give one the illusion of `strong AI' in spite of its
Boolean logic limitations, as there are no stringent tests defined so far that are capable of correctly `measuring' intelligence either in humans or in computers. Until a satisfactory definition of `human intelligence' is arrived at, it will not
be possible to design acceptable means or tests to `measure' such intelligence, and therefore one could not establish
if the `strong AI' thesis is valid or not. It may indeed remain an {\em undecidable} issue on a chrysippian logic basis.

\item An universal Turing machine ({\em $AUTM$}) was shown to be able to simulate any other sequential machine, automaton, robot, or computer by employing steps that humans do not consider to require `intelligence' (i.e., 
human intelligence). 

\end{enumerate}


%%%%%
%%%%%
\end{document}
