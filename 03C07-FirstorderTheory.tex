\documentclass[12pt]{article}
\usepackage{pmmeta}
\pmcanonicalname{FirstorderTheory}
\pmcreated{2013-03-22 12:43:04}
\pmmodified{2013-03-22 12:43:04}
\pmowner{CWoo}{3771}
\pmmodifier{CWoo}{3771}
\pmtitle{first-order theory}
\pmrecord{19}{33012}
\pmprivacy{1}
\pmauthor{CWoo}{3771}
\pmtype{Definition}
\pmcomment{trigger rebuild}
\pmclassification{msc}{03C07}
\pmclassification{msc}{03B10}
\pmsynonym{first order theory}{FirstorderTheory}
\pmrelated{PropertiesOfConsistency}
\pmrelated{MaximallyConsistent}
\pmdefines{theory}
\pmdefines{complete theory}
\pmdefines{axiomatizable theory}
\pmdefines{deductively closed}
\pmdefines{finitely axiomatizable theory}

\endmetadata

% this is the default PlanetMath preamble.  as your knowledge
% of TeX increases, you will probably want to edit this, but
% it should be fine as is for beginners.

% almost certainly you want these
\usepackage{amssymb}
\usepackage{amsmath}
\usepackage{amsfonts}

% used for TeXing text within eps files
%\usepackage{psfrag}
% need this for including graphics (\includegraphics)
%\usepackage{graphicx}
% for neatly defining theorems and propositions
%\usepackage{amsthm}
% making logically defined graphics
\usepackage[arrow,curve,poly,arc,2cell,frame,web]{xypic}

% there are many more packages, add them here as you need them

% define commands here
\newcommand{\br}{[\![}
\newcommand{\rb}{]\!]}
\newcommand{\oq}{\text{``}}
\newcommand{\cq}{\text{''}}


\newcommand{\im}{\mathbf{Im}}
\newcommand{\dom}{\mathbf{Dom}}


\newcommand{\Or}{\vee}
\newcommand{\Implies}{\Rightarrow}
\newcommand{\Iff}{\Leftrightarrow}
\newcommand{\proves}{\vdash}
\renewcommand{\And}{\wedge}
\newcommand{\Sup}{\bigwedge}
\newcommand{\Inf}{\bigvee}
\newcommand{\Z}{\mathbb{Z}}
\newcommand{\F}{\mathbb{F}}
\newcommand{\Q}{\mathbb{Q}}
\newcommand{\R}{\mathbb{R}}
\newcommand{\C}{\mathbb{C}}
\newcommand{\Nat}{\mathbb{N}}
\newcommand{\M}{\mathfrak{M}}
\newcommand{\N}{\mathfrak{N}}
\newcommand{\A}{\mathfrak{A}}
\newcommand{\B}{\mathfrak{B}}
\newcommand{\K}{\mathfrak{K}}
\newcommand{\G}{\mathbb{G}}
\newcommand{\Def}{\overset{\operatorname{def}}{:=}}



\newcommand{\spec}{\text{{\bf Spec}}}
\newcommand{\stab}{\text{{\bf Stab}}}
\newcommand{\ann}{\text{{\bf Ann}}}
\newcommand{\irr}{\text{{\bf Irr}}}
\newcommand{\qt}{\text{{\bf Qt}}}
\newcommand{\st}{\mathcal{Qt}}
\newcommand{\ro}{\mathbf{r.o.}}


\newcommand{\Endo}{\text{{\bf End}}}
\newcommand{\mat}{\text{{\bf Mat}}}
\newcommand{\der}{\text{{\bf Der}}}
\newcommand{\rad}{\text{{\bf Rad}}}
\newcommand{\trd}{\text{{\bf tr.d.}}}
\newcommand{\cl}{\text{{\bf acl}}}
\newcommand{\Int}{\text{{\bf int}}}
\newcommand{\V}{\mathbb{V}}
\newcommand{\D}{\mathbf{D}}

\newcommand{\del}{\partial}
\renewcommand{\O}{\mathcal{O}}
\newcommand{\aut}{\mathbf{Aut}}
\newcommand{\height}{\text{\bf Height}}
\newcommand{\coheight}{\text{\bf Co-height}}

\newcommand{\lcm}{\operatorname{lcm}}

\newcommand{\Gal}{\operatorname{Gal}}
\newcommand{\x}{\mathbf{x}}
\newcommand{\y}{\mathbf{y}}
\newcommand{\inner}[2]{\langle #1|#2\rangle}
\renewcommand{\r}{{r}}
\renewcommand{\t}{{t}}

\newcommand{\restr}{\upharpoonright}
\newcommand{\Matrix}[4]{\left(\begin{array}{cc} #1 & #2 \\ #3 & #4 
\end{array}\right)}

\begin{document}
In what follows, references to sentences and sets of sentences are
all relative to some fixed first-order language $L$. \\

\textbf{Definition.} A \textbf{theory} $T$ is a \emph{deductively
closed} set of sentences in $L$; that is, a set $T$ such that for each
sentence $\varphi$, $T \vdash \varphi$ only if $\varphi \in
T$.\\

\textbf{Remark}.  Some authors do not require that a theory be deductively closed.  Therefore, a theory is simply a set of sentences.  This is not a cause for alarm, since every theory $T$ under this definition can be ``extended'' to a deductively closed theory $T^{\vdash}:=\lbrace \varphi \in L\mid T\vdash \varphi\rbrace$.  Furthermore, $T^{\vdash}$ is unique (it is the smallest deductively closed theory including $T$), and any structure $M$ is a model of $T$ iff it is a model of $T^{\vdash}$.

\textbf{Definition.} A theory $T$ is \emph{consistent} if and only
if for some sentence $\varphi$, $T \not \vdash \varphi$.
Otherwise, $T$ is \emph{inconsistent}. A sentence
$\varphi$ is \emph{consistent with $T$} if and only if the
theory $T \cup \lbrace \varphi \rbrace$ is consistent.\\

\textbf{Definition.} A theory $T$ is \emph{complete} if and only
if $T$ is consistent and for each sentence $\varphi$, either $\varphi \in T$
or $\neg \varphi \in T$.\\

\textbf{Lemma.} A consistent theory $T$ is complete if and only if $T$ is
maximally consistent. That is, $T$ is complete if and only if for
each sentence $\varphi$, $\varphi \not \in T$ only if
$T \cup \lbrace \varphi \rbrace$ is inconsistent.  See \PMlinkname{this entry}{MaximallyConsistent} for a proof. \\

\textbf{Theorem. (Tarski)} Every consistent theory $T$ is included
in a complete theory.

\textbf{Proof :} Use Zorn's lemma on the set of consistent
theories that include $T$.\\

\textbf{Remark}. A theory $T$ is \emph{axiomatizable} if and only
if $T$ includes a \PMlinkname{decidable}{DecidableSet} subset $\Delta$ such that $\Delta
\vdash T$ (every sentence of $T$ is a logical consequence of
$\Delta$), and \emph{finitely axiomatizable} if $\Delta$ can be made finite. Every complete axiomatizable theory $T$ is decidable;
that is, there is an algorithm that given a sentence $\varphi$ as
input yields $0$ if $\varphi \in T$, and $1$ otherwise.

%%%%%
%%%%%
\end{document}
