\documentclass[12pt]{article}
\usepackage{pmmeta}
\pmcanonicalname{MathematicalPlatonism}
\pmcreated{2013-03-22 18:37:04}
\pmmodified{2013-03-22 18:37:04}
\pmowner{gribskoff}{21395}
\pmmodifier{gribskoff}{21395}
\pmtitle{mathematical platonism}
\pmrecord{12}{41352}
\pmprivacy{1}
\pmauthor{gribskoff}{21395}
\pmtype{Topic}
\pmcomment{trigger rebuild}
\pmclassification{msc}{03A05}
\pmclassification{msc}{03-01}
\pmsynonym{mathematical realism}{MathematicalPlatonism}
%\pmkeywords{realism}
%\pmkeywords{universals}
%\pmkeywords{Bernays measure}
%\pmkeywords{complex numbers}
%\pmkeywords{creationism}
\pmrelated{FOUNDATIONSOFMATHEMATICSOVERVIEW}
\pmrelated{Logicism}
\pmrelated{IntuitionisticLogic}
\pmrelated{Predicativism}
\pmrelated{BeyondFormalism}
\pmrelated{PlatosMathematics}
\pmdefines{Bernays measure}

\usepackage{amssymb}
\usepackage{amsmath}
\usepackage{amsfonts}
\usepackage{amsmath}
\usepackage{graphicx}
%%\usepackage{xypic}
\usepackage{psfrag}
\usepackage{amsthm}
\usepackage[T1]{fontenc}
\usepackage{yfonts}

% define commands here
\begin{document}
\title{Mathematical platonism}\Large 
\maketitle

\tableofcontents

\section{Platonism and realism}\normalsize

In spite of the fact that the term "platonism" is part of the vocabulary used in some traditional philosophical domains, in the philosophy of mathematics discourse, however, the term is officially a creation of Paul Bernays, even if platonist positions in mathematical philosophy were taken \emph{before} Bernays coined the term.

It denotes the doctrine according to which the variables of true mathematical propositions denote objects with an autonomous identity, in the sense that their existence is independent of the cognitive subject.

The position in the philosophy of mathematics is logically equivalent to the position in epistemology traditionally  known as realism.

But mathematical platonism also has what might be called a metaphysical component, since it asserts a special kind of existence, in the space-time continuum, that the objects of mathematical knowledge are supposed to have.

Turning now to the source of the evidence of mathematical knowledge, platonism is a doctrine opposed to constructivism, since in constructivism mathematical objects are considered the result of a creative mental act carried out by the cognitive subject.

There is an interesting remark by Kreisel, according to which platonism is the position unconsciously adopted by a silent majority of mathematicians in their daily work. But the same mathematicians, when they are questioned about their daily practice, soon turn to constructivism, as being the  mathematical philosophy of their conviction.

We are then confronted with a naive contradiction between theory and experience (in the sense of \emph{Erlebnis}), which could be disregarded in some other philosophical discipline, but which is not acceptable in a domain which aims at the definition of a state of \emph{reflected consistency} between theory and experience.

\section{Historical overview}\normalsize

The historical background of mathematical platonism goes, as the name suggests, back to Plato \emph{Republic} (596A and 522C) and it became later part and parcel of the Oxford mediaeval philosophy of logic, where it was known as the problem of (the existence of) universals. The term then used for platonism was again realism and it was contrasted with conceptualism and nominalism, its rival positions in the philosophy of logic.

A universal is what we would call today a monadic predicate like "Athena is wise". Plato is supposed to have advanced the theory that it is our grasp of the \emph{concept of wisdom} that secures our understanding of the meaning of the instantiations "Athena is wise", "Socrates is wise", etc.
 
Whereas realists maintained that such concepts have an objective existence of their own, like the objects of the material world, conceptualists, like the anti-realists of today, argued that universals had no autonomous existence but were a creation of the cognitive subject, the purpose of which was to bring the mass of the instantiations under some order. The Oxford nominalists, v.g. Occam, like the nominalists of today, saw in universals simply an expedient lexicographical convention, with no extra-linguistic reality.

The first modern formulation of mathematical platonism is due to Frege, in particular in \emph{Grundlagen der Arithmetik}, where the objectivity of concepts is dissociated from any cognitive performance on the subject's part. We shall return to Frege's position later.

\section{Bernays measure}\normalsize

To carry out an analysis of the mathematical content of the platonist position one uses the so-called \emph{Bernays measure} of the degree of Platonism involved in a mathematical system.

The initial definition is that the degree of platonism of a mathematical system is the kind of totalities admitted in the system.

These totalities are then the mathematical objects of the system, reference to which secures the meaning for the propositions of the system. We will distinguish three degrees to be ordered in the following way.

To the lowest degree belongs a theory that accepts the natural numbers as a completed totality and, by our discussion above, considers the use of the \emph{tertium non datur} as well defined in propositions involving quantifiers over all natural numbers. Well known propositions of analysis use this resource, v.g., in the proof that either the limit of every sequence of rational numbers tends to 0 or it doesn't, needed to prove that

\begin{center}
$(\forall x) (x \in \mathbb{R}) [(x = 0) \vee \neg (x = 0)].$
\end{center}

This leads to an intermediate and higher degree to which belongs classical analysis, where one admits as meaningful  totalities like the set of all the points of the continuum or the totality of all subsets of the natural numbers.

The classical theory of real numbers uses concepts like \emph{arbitrary} sequences of natural numbers or \emph{arbitrary} sets of natural numbers, where these notions are used \emph{via an analogy} with the finite case.

For example, if the set of numbers $S = {1, \ldots, n}$ is given, there are $n^n$ functions which map to each element of $S$ an element of $S$.

But if $S$ is infinite we allow the conception of a function fixed by an infinite number of independent determinations, which maps an integer to an integer and consider the totality of such functions as being well defined.

The same applies to sets of integers $M$, conceived as the result of an infinite number of decisions as to the membership of an element in $M$. We then form the totality of such sets $M$.
 
The generalization of this procedure finds its expression in the classical axiom of choice. If $S_1, S_2, \ldots$ is a sequence of non-empty sets of real numbers, one postulates the existence of the sequence $x_1, x_2, \ldots$ such that for each $i$, $x_i \in S_i$, although a method to construct $x_1, x_2, \ldots$ effectively is not provided.

In impredicative concept formation one uses the existence of sets of integers as a given totality, for example in the definition of a real number as the least upper bound of a bounded set of real numbers.

Finally there is a third and still higher degree of platonism which goes beyond the classical definition of the real numbers. This new domain is to be found in the concept formation and methods of Cantor's set theory.

The best known alternative to platonism of the second degree is Hermann Weyl's reconstruction of classical analysis, using only the totality of the natural numbers as given, that is platonism of the lowest degree.
 
Also a moderate use of platonism can be found in predicative analysis. (See the PM entry \PMlinkname{"Predicativism"}{Predicativism}, for a discussion of impredicative definitions and their elimination.)

\section{The postulates of platonism}\normalsize

Before we start a discussion of individual platonist positions, it may prove useful to spell out the structure of any platonist theory. Any platonist position has to satisfy the following propositions:

\begin{enumerate}
\item There is a mathematical  reality.

\item The existence of the objects of mathematical knowledge does not depend on actions or reactions by the cognitive subject.

\item In particular, the objective existence of these objects does not result from the success of the subject's  cognitive performance.

\item The existence of the objects of mathematical knowledge does not depend on the conceptual scheme which the cognitive subject happens to be inserted in.

\item In particular, does not depend on the language used by the cognitive subject.

\item The meaning of a mathematical proposition is given by the truth-conditions which correspond to it, since it describes a fact of the mathematical reality.

\item The truth of a mathematical proposition is independent of its being verified, either effectively or only in principle.
\end{enumerate}

Under these propositions we consider that totalities of mathematical objects are \emph{well defined} when statements which use quantification over such totalities are given a truth-value.

In particular this is equivalent to consider the use of the \emph{tertium non datur} over such totalities as well defined.

\section{Frege on complex numbers}\normalsize

The best known strategies to argue in favour of the platonist theory revolve around establishing Proposition 2 above.

It usually takes the form of a refutation of the position called creationism or postulationism, according to which the objects of mathematical discourse are creations of the individual mathematician, or simply postulated by him to exist.

Its first modern expression is to be found in Frege, whose \emph{Grundlagen der Arithmetik} combine a surprising mixture of logicism and platonism. 

Frege uses two main lines of argument: the first line, from $\S$ 47 onwards, in a very persuasive defense of the objective existence of concepts and, the second line, from $\S$ 96 onwards, in his refutation of creationism or postulationism.

His test case was the problem of providing a denotation for imaginary terms like $\text{b}i$ and complex expressions like $\text{a + b}i$.

Not only the familiar text books of his day avoided the question altogether, but also the leaders of the then prevailing mathematical theory tried to solve the question in tune with the formalist outlook, in following manner: one introduces a system of notation and a system of rules to operate with the kind of entities called imaginary numbers, like $i^2 = -1$, without having to decide what $i$ actually denotes; and one develops out of this system the theory of complex numbers and, as  long as this theory proves to be \emph{formally} consistent, one is entitled to assert the existence of the postulated imaginary numbers. Existence is only freedom from formal contradiction.

Against this kind of formalism Frege countered with two main arguments.

The first was to argue that to create a system of notation and a system of rules for a set of entities like the imaginary numbers does not secure their existence. It is required that the terms and the expressions of the system \emph{be satisfied} in a model, where they receive their truth-values. If nothing satisfies the expressions of the system, nothing secures their truth. Formalists delude themselves into believing that postulating imaginary numbers to exist as a notation without content spares them the effort of providing a model where propositions about such numbers can be seen to be consistent.

Frege's second argument was that mathematicians like geographers can not create anything out of nothing.

Mathematicians can at most discover what already existed, prior to its being discovered, and can give it a name and describe its properties.

We have then the following disjunction:

\begin{quote}
either the imaginary numbers really existed prior to their creation by the mathematician, in which case his creating them in a system of content-free notation is redundant, or imaginary numbers did not exist prior to their being created by the mathematician, in which case his postulating them in a system of content-free notation does not make them any more real.  
\end{quote}

As to the theme of providing a denotation for imaginary terms like $\text{b}i$, Frege seems to have found a solution in the geometrical representation of complex numbers, as it is discussed in $\S$ 101. This obviously implies the insight that $i$ is not the name of a number. 

\section{G\"{o}del against creationism}\normalsize

The fascinating mixture between logicism and platonism survived the work of Frege, at least in another well known logicist, Bertrand Russell. His platonism is associated with some of his earlier work, still under the influence of Frege, but it did not become a defining feature of his overall outlook. His best platonist moment is to be found at the time of the \emph{Introduction to Mathematical Philosophy}.

There he changes Frege's metaphor of the homology between mathematics and geography to that between mathematics and zoology.

The legacy of Frege was continued in G\"{o}del's philosophical work and we report now on his analysis of the already mentioned Frege contention according to which there is no creation out of nothing.

From his analysis of Hilbert's tenth problem and the discovery of a Diophantine equivalent for the undecidable proposition, G\"{o}del ended up by proposing that there are Diophantine problems whose solution is inaccessible to the human mind. This proves Propositions 2 and 3 above and thereby the falsity of creationism.

G\"{o}del's reformulation of Frege's idea is that the created object can not have predicates other than those given to it by the creator. This provides the evidence for the epistemic assertion:

\begin{center}
$(\heartsuit)$ \qquad Necessarily the creator knows all predicates of his creature.
\end{center}

But $(\heartsuit)$ implies that the existence of absolutely undecidable propositions shows that the objects denoted by them do not result from our creation, since it is not possible that there exists a predicate the truth of which is not known to the creator.

We consider now two lines of objections raised against the Frege-G\"{o}del thesis, the first in tune with the formalist philosophy of mathematics and the second coming from intuitionism.

\begin{enumerate}
\item Against the epistemic assertion $(\heartsuit)$ the first objection is then to produce a predicate the truth of which is not necessarily known by the creator.

This will be better motivated by using as a paradigm the case in which the creator of a mechanical device can not anticipate at least one of the predicates of his invention.

So let us assume that a mathematical creation, for example a formal system, is like the creation of a mechanical device. In that case: 

\begin{itemize}
\item Some initial material would be necessary (and therefore would not be a creation out of nothing);

\item On the other hand, to the initial material needed to construct the mechanical device, would correspond, in the mathematical case, something really existent, prior to the mathematical construction.
\end{itemize}

But let us assume, for the sake of the argument, that the initial material, v.g., axioms about the integers, is also an invention. In that case the theorems are likewise an invention, since the inference rules preserve the properties of the axioms. However the validity of the theorems proved, could not itself be an invention, because in that case we would have a \emph{petitio principii}.

But this rather proves again Proposition 2 and therefore defeats the purpose of the comparison.

\item Intuitionism rejects Propositions 1 to 7 and, in particular, a meaningful undecidable proposition is a contradiction in terms, since what first constitutes the meaning of a proposition is the proof (or the construction) that compels us to assert it.

Since the undecidable proposition does not have a proof, one is not entitled to say that the terms occurring in it  denote anything, least of all the objects of mathematical reality.

Of course the intuitionistic objection is only tenable if one equates meaning with proof.

But there is no reason to believe that this equation is convincing. On the contrary, the existence of undecidable propositions is evidence against it, since the undecidable proposition is both true and unprovable. Proof is not a necessary condition of meaning. 
\end{enumerate}

\begin{thebibliography} {90}
\bibitem{BP} Bernays, P., "Sur le platonism dans les mathematiques", \emph{L'Enseignement Mathematique}, vol. 34; pp. 52-69; 1935. [See Charles Parsons translation here: \PMlinkexternal{The Bernays Project}{http://www.phil.cmu.edu/projects/bernays/}.]
\bibitem{MD} Dummett, M., \emph{Frege: Philosophy of mathematics}, Duckworth, 1991.
\bibitem{GF} Frege, G., \emph{Grundlagen der Arithmetik}, Breslau, 1884.
\bibitem{KG} G\"{o}del, K., \emph{Collected works}, ed. S. Feferman, Oxford, 1987-2003.
\bibitem{HW1} Wang, H., \emph{From mathematics to philosophy}, London, 1974.
\bibitem{HW2} Weyl, H., \emph{Philosophy of mathematics and natural science}, Princeton University Press, 1949.
\end{thebibliography}

\end{document}
%%%%%
%%%%%
\end{document}
