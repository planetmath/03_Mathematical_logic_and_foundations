\documentclass[12pt]{article}
\usepackage{pmmeta}
\pmcanonicalname{PresburgerArithmetic}
\pmcreated{2013-03-22 12:51:11}
\pmmodified{2013-03-22 12:51:11}
\pmowner{Henry}{455}
\pmmodifier{Henry}{455}
\pmtitle{Presburger arithmetic}
\pmrecord{8}{33185}
\pmprivacy{1}
\pmauthor{Henry}{455}
\pmtype{Definition}
\pmcomment{trigger rebuild}
\pmclassification{msc}{03B10}
\pmrelated{PeanoArithmetic}

% this is the default PlanetMath preamble.  as your knowledge
% of TeX increases, you will probably want to edit this, but
% it should be fine as is for beginners.

% almost certainly you want these
\usepackage{amssymb}
\usepackage{amsmath}
\usepackage{amsfonts}

% used for TeXing text within eps files
%\usepackage{psfrag}
% need this for including graphics (\includegraphics)
%\usepackage{graphicx}
% for neatly defining theorems and propositions
%\usepackage{amsthm}
% making logically defined graphics
%%%\usepackage{xypic}

% there are many more packages, add them here as you need them

% define commands here
\begin{document}
\emph{Presburger arithmetic} is a weakened form of arithmetic which includes the structure $\mathbb{N}$, the constant $0$, the unary function $S$, the binary function $+$, and the binary relation $<$.  Essentially, it is Peano arithmetic without multiplication.

The axioms are:
\begin{enumerate}
\item $0\not= Sx$
\item $Sx=Sy\rightarrow x=y$
\item $x+0=x$
\item $x+Sy=S(x+y)$
\item For each first order formula $P(x)$, $P(0)\wedge\forall x[P(x)\rightarrow P(x+1)]\rightarrow\forall x P(x)$
\end{enumerate}

Presburger arithmetic is decidable, but is consequently very limited in what it can express.
%%%%%
%%%%%
\end{document}
