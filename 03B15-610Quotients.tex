\documentclass[12pt]{article}
\usepackage{pmmeta}
\pmcanonicalname{610Quotients}
\pmcreated{2013-11-18 15:49:42}
\pmmodified{2013-11-18 15:49:42}
\pmowner{PMBookProject}{1000683}
\pmmodifier{rspuzio}{6075}
\pmtitle{6.10 Quotients}
\pmrecord{3}{87692}
\pmprivacy{1}
\pmauthor{PMBookProject}{6075}
\pmtype{Feature}
\pmclassification{msc}{03B15}

\usepackage{xspace}
\usepackage{amssyb}
\usepackage{amsmath}
\usepackage{amsfonts}
\usepackage{amsthm}
\usepackage{stmaryrd}
\makeatletter
\newcommand{\base}{\ensuremath{\mathsf{base}}\xspace}
\newcommand{\blank}{\mathord{\hspace{1pt}\text{--}\hspace{1pt}}}
\newcommand{\ct}{  \mathchoice{\mathbin{\raisebox{0.5ex}{$\displaystyle\centerdot$}}}             {\mathbin{\raisebox{0.5ex}{$\centerdot$}}}             {\mathbin{\raisebox{0.25ex}{$\scriptstyle\,\centerdot\,$}}}             {\mathbin{\raisebox{0.1ex}{$\scriptscriptstyle\,\centerdot\,$}}}}
\newcommand{\defeq}{\vcentcolon\equiv}  
\newcommand{\define}[1]{\textbf{#1}}
\def\@dprd#1{\prod_{(#1)}\,}
\def\@dprd@noparens#1{\prod_{#1}\,}
\def\@dsm#1{\sum_{(#1)}\,}
\def\@dsm@noparens#1{\sum_{#1}\,}
\def\@eatprd\prd{\prd@parens}
\def\@eatsm\sm{\sm@parens}
\newcommand{\eqr}{\sim}         
\newcommand{\eqv}[2]{\ensuremath{#1 \simeq #2}\xspace}
\newcommand{\eqvspaced}[2]{\ensuremath{#1 \;\simeq\; #2}\xspace}
\newcommand{\eqvsym}{\simeq}    
\newcommand{\id}[3][]{\ensuremath{#2 =_{#1} #3}\xspace}
\newcommand{\indexdef}[1]{\index{#1|defstyle}}   
\newcommand{\indexsee}[2]{\index{#1|see{#2}}}    
\newcommand{\jdeq}{\equiv}      
\newcommand{\N}{\ensuremath{\mathbb{N}}\xspace}
\newcommand{\nameless}{\mathord{\hspace{1pt}\underline{\hspace{1ex}}\hspace{1pt}}}
\newcommand{\narrowequation}[1]{$#1$}
\newcommand{\opp}[1]{\mathord{{#1}^{-1}}}
\newcommand{\Parens}[1]{\Bigl(#1\Bigr)}
\def\prd#1{\@ifnextchar\bgroup{\prd@parens{#1}}{\@ifnextchar\sm{\prd@parens{#1}\@eatsm}{\prd@noparens{#1}}}}
\def\prd@noparens#1{\mathchoice{\@dprd@noparens{#1}}{\@tprd{#1}}{\@tprd{#1}}{\@tprd{#1}}}
\def\prd@parens#1{\@ifnextchar\bgroup  {\mathchoice{\@dprd{#1}}{\@tprd{#1}}{\@tprd{#1}}{\@tprd{#1}}\prd@parens}  {\@ifnextchar\sm    {\mathchoice{\@dprd{#1}}{\@tprd{#1}}{\@tprd{#1}}{\@tprd{#1}}\@eatsm}    {\mathchoice{\@dprd{#1}}{\@tprd{#1}}{\@tprd{#1}}{\@tprd{#1}}}}}
\newcommand{\proj}[1]{\ensuremath{\mathsf{pr}_{#1}}\xspace}
\newcommand{\prop}{\ensuremath{\mathsf{Prop}}\xspace}
\newcommand{\refl}[1]{\ensuremath{\mathsf{refl}_{#1}}\xspace}
\def\sm#1{\@ifnextchar\bgroup{\sm@parens{#1}}{\@ifnextchar\prd{\sm@parens{#1}\@eatprd}{\sm@noparens{#1}}}}
\def\sm@noparens#1{\mathchoice{\@dsm@noparens{#1}}{\@tsm{#1}}{\@tsm{#1}}{\@tsm{#1}}}
\def\sm@parens#1{\@ifnextchar\bgroup  {\mathchoice{\@dsm{#1}}{\@tsm{#1}}{\@tsm{#1}}{\@tsm{#1}}\sm@parens}  {\@ifnextchar\prd    {\mathchoice{\@dsm{#1}}{\@tsm{#1}}{\@tsm{#1}}{\@tsm{#1}}\@eatprd}    {\mathchoice{\@dsm{#1}}{\@tsm{#1}}{\@tsm{#1}}{\@tsm{#1}}}}}
\newcommand{\suc}{\mathsf{succ}}
\newcommand{\symlabel}[1]{\refstepcounter{symindex}\label{#1}}
\def\@tprd#1{\mathchoice{{\textstyle\prod_{(#1)}}}{\prod_{(#1)}}{\prod_{(#1)}}{\prod_{(#1)}}}
\def\tsm#1{\@tsm{#1}\@ifnextchar\bgroup{\tsm}{}}
\def\@tsm#1{\mathchoice{{\textstyle\sum_{(#1)}}}{\sum_{(#1)}}{\sum_{(#1)}}{\sum_{(#1)}}}
\newcommand{\UU}{\ensuremath{\mathcal{U}}\xspace}
\newcommand{\vcentcolon}{:\!\!}
\newcommand{\Z}{\ensuremath{\mathbb{Z}}\xspace}
\newcounter{mathcount}
\setcounter{mathcount}{1}
\newtheorem{precor}{Corollary}
\newenvironment{cor}{\begin{precor}}{\end{precor}\addtocounter{mathcount}{1}}
\renewcommand{\theprecor}{6.10.\arabic{mathcount}}
\newtheorem{predefn}{Definition}
\newenvironment{defn}{\begin{predefn}}{\end{predefn}\addtocounter{mathcount}{1}}
\renewcommand{\thepredefn}{6.10.\arabic{mathcount}}
\newenvironment{myeqn}{\begin{equation}}{\end{equation}\addtocounter{mathcount}{1}}
\renewcommand{\theequation}{6.10.\arabic{mathcount}}
\newtheorem{prelem}{Lemma}
\newenvironment{lem}{\begin{prelem}}{\end{prelem}\addtocounter{mathcount}{1}}
\renewcommand{\theprelem}{6.10.\arabic{mathcount}}
\newtheorem{prermk}{Remark}
\newenvironment{rmk}{\begin{prermk}}{\end{prermk}\addtocounter{mathcount}{1}}
\renewcommand{\theprermk}{6.10.\arabic{mathcount}}
\newtheorem{prethm}{Theorem}
\newenvironment{thm}{\begin{prethm}}{\end{prethm}\addtocounter{mathcount}{1}}
\renewcommand{\theprethm}{6.10.\arabic{mathcount}}
\let\autoref\cref
\let\setof\Set    
\let\type\UU
\makeatother

\begin{document}

A particularly important sort of colimit of sets is the \emph{quotient} by a relation.
That is, let $A$ be a set and $R:A\times A \to \prop$ a family of mere propositions (a \define{mere relation}).
\indexdef{relation!mere}%
\indexdef{mere relation}%
Its quotient should be the set-coequalizer of the two projections
\[ \tsm{a,b:A} R(a,b) \rightrightarrows A. \]
We can also describe this directly, as the higher inductive type $A/R$ generated by
\index{set-quotient|(defstyle}%
\indexsee{quotient of sets}{set-quotient}%
\indexsee{type!quotient}{set-quotient}%
\begin{itemize}
\item A function $q:A\to A/R$;
\item For each $a,b:A$ such that $R(a,b)$, an equality $q(a)=q(b)$; and
\item The $0$-truncation constructor: for all $x,y:A/R$ and $r,s:x=y$, we have $r=s$.
\end{itemize}
We may sometimes refer to $A/R$ as the \define{set-quotient} of $A$ by $R$, to emphasize that it produces a set by definition.
(There are more general notions of ``quotient'' in homotopy theory, but they are mostly beyond the scope of this book.
However, in \PMlinkname{\S 9.9}{99therezkcompletion} we will consider the ``quotient'' of a type by a 1-groupoid, which is the next level up from set-quotients.)

\begin{rmk}
  It is not actually necessary for the definition of set-quotients, and most of their properties, that $A$ be a set.
  However, this is generally the case of most interest.
\end{rmk}

\begin{lem}\label{thm:quotient-surjective}
  The function $q:A\to A/R$ is surjective.
\end{lem}
\begin{proof}
  We must show that for any $x:A/R$ there merely exists an $a:A$ with $q(a)=x$.
  We use the induction principle of $A/R$.
  The first case is trivial: if $x$ is $q(a)$, then of course there merely exists an $a$ such that $q(a)=q(a)$.
  And since the goal is a mere proposition, it automatically respects all path constructors, so we are done.
\end{proof}

\begin{lem}\label{thm:quotient-ump}
  For any set $B$, precomposing with $q$ yields an equivalence
  \[ \eqvspaced{(A/R \to B)}{\Parens{\sm{f:A\to B} \prd{a,b:A} R(a,b) \to (f(a)=f(b))}}.\]
\end{lem}
\begin{proof}
  The quasi-inverse of $\blank\circ q$, going from right to left, is just the recursion principle for $A/R$.
  That is, given $f:A\to B$ such that
  \narrowequation{\prd{a,b:A} R(a,b) \to (f(a)=f(b)),} we define $\bar f:A/R\to B$ by $\bar f(q(a))\defeq f(a)$.
  This defining equation says precisely that $(f\mapsto \bar f)$ is a right inverse to $(\blank\circ q)$.

  For it to also be a left inverse, we must show that for any $g:A/R\to B$ and $x:A/R$ we have $g(x) = \overline{g\circ q}$.
  However, by \PMlinkname{Lemma 6.10.2}{610quotients#Thmprelem1} there merely exists $a$ such that $q(a)=x$.
  Since our desired equality is a mere proposition, we may assume there purely exists such an $a$, in which case $g(x) = g(q(a)) = \overline{g\circ q}(q(a)) = \overline{g\circ q}(x)$.  
\end{proof}

Of course, classically the usual case to consider is when $R$ is an \define{equivalence relation}, i.e.\ we have
\indexdef{relation!equivalence}%
\indexsee{equivalence!relation}{relation, equivalence}%
%
\begin{itemize}
\item \define{reflexivity}: $\prd{a:A} R(a,a)$,
  \indexdef{reflexivity!of a relation}%
  \indexdef{relation!reflexive}%
\item \define{symmetry}: $\prd{a,b:A} R(a,b) \to R(b,a)$, and
  \indexdef{symmetry!of a relation}%
  \indexdef{relation!symmetric}%
\item \define{transitivity}: $\prd{a,b,c:C} R(a,b) \times R(b,c) \to R(a,c)$.
  \indexdef{transitivity!of a relation}%
  \indexdef{relation!transitive}%
\end{itemize}
%
In this case, the set-quotient $A/R$ has additional good properties, as we will see in \PMlinkname{\S 10.1}{101thecategoryofsets}: for instance, we have $R(a,b) \eqvsym (\id[A/R]{q(a)}{q(b)})$.
\symlabel{equivalencerelation}
We often write an equivalence relation $R(a,b)$ infix as $a\eqr b$.

The quotient by an equivalence relation can also be constructed in other ways.
The set theoretic approach is to consider the set of equivalence classes, as a subset of the power set\index{power set} of $A$.
We can mimic this ``impredicative'' construction in type theory as well.
\index{impredicative!quotient}

\begin{defn}
  A predicate $P:A\to\prop$ is an \define{equivalence class}
  \indexdef{equivalence!class}%
  of a relation $R : A \times A \to \prop$ if there merely exists an $a:A$ such that for all $b:A$ we have $\eqv{R(a,b)}{P(b)}$.
\end{defn}

As $R$ and $P$ are mere propositions, the equivalence $\eqv{R(a,b)}{P(b)}$ is the same thing as implications $R(a,b) \to P(b)$ and $P(b) \to R(a,b)$.
And of course, for any $a:A$ we have the canonical equivalence class $P_a(b) \defeq R(a,b)$.

\begin{defn}\label{def:VVquotient}
  We define
  \begin{equation*}
    A\sslash R \defeq \setof{ P:A\to\prop | P \text{ is an equivalence class of } R}.
  \end{equation*}
  The function $q':A\to A\sslash R$ is defined by $q'(a) \defeq P_a$.
\end{defn}

\begin{thm}
  For any equivalence relation $R$ on $A$, the two set-quotients $A/R$ and $A\sslash R$ are equivalent.
\end{thm}
\begin{proof}
  First, note that if $R(a,b)$, then since $R$ is an equivalence relation we have $R(a,c) \Leftrightarrow R(b,c)$ for any $c:A$.
  Thus, $R(a,c) = R(b,c)$ by univalence, hence $P_a=P_b$ by function extensionality, i.e.\ $q'(a)=q'(b)$.
  Therefore, by \PMlinkname{Lemma 6.10.3}{610quotients#Thmprelem2} we have an induced map $f:A/R \to A\sslash R$ such that $f\circ q = q'$.

  We show that $f$ is injective and surjective, hence an equivalence.
  Surjectivity follows immediately from the fact that $q'$ is surjective, which in turn is true essentially by definition of $A\sslash R$.
  For injectivity, if $f(x)=f(y)$, then to show the mere proposition $x=y$, by surjectivity of $q$ we may assume $x=q(a)$ and $y=q(b)$ for some $a,b:A$.
  Then $R(a,c) = f(q(a))(c) = f(q(b))(c) = R(b,c)$ for any $c:A$, and in particular $R(a,b) = R(b,b)$.
  But $R(b,b)$ is inhabited, since $R$ is an equivalence relation, hence so is $R(a,b)$.
  Thus $q(a)=q(b)$ and so $x=y$.
\end{proof}

In \PMlinkname{\S 10.1.3}{1013quotients} we will give an alternative proof of this theorem.
Note that unlike $A/R$, the construction $A\sslash R$ raises universe level: if $A:\UU_i$ and $R:A\to A\to \prop_{\UU_i}$, then in the definition of $A\sslash R$ we must also use $\prop_{\UU_i}$ to include all the equivalence classes, so that $A\sslash R : \UU_{i+1}$.
Of course, we can avoid this if we assume the propositional resizing axiom from \PMlinkname{\S 3.5}{35subsetsandpropositionalresizing}.

\begin{rmk}\label{defn-Z}
The previous two constructions provide quotients in generality, but in particular cases there may be easier constructions.
For instance, we may define the integers \Z as a set-quotient
\indexdef{integers}%
\indexdef{number!integers}%
%
\[ \Z \defeq (\N \times \N)/{\eqr} \]
%
where $\eqr$ is the equivalence relation defined by
%
\[ (a,b) \eqr (c,d) \defeq (a + d = b + c). \]
%
In other words, a pair $(a,b)$ represents the integer $a - b$.
In this case, however, there are \emph{canonical representatives} of the equivalence classes: those of the form $(n,0)$ or $(0,n)$.
\end{rmk}

The following lemma says that when this sort of thing happens, we don't need either general construction of quotients.
(A function $r:A\to A$ is called \define{idempotent}
\indexdef{function!idempotent}%
\indexdef{idempotent!function}%
if $r\circ r = r$.)

\begin{lem}\label{lem:quotient-when-canonical-representatives}
  Suppose $\eqr$ is an equivalence relation on a set $A$, and there exists an idempotent $r
  : A \to A$ such that, for all $x, y \in A$, $\eqv{(r(x) = r(y))}{(x \eqr y)}$. Then the
  type
  %
  \begin{equation*}
    (A/{\eqr}) \defeq \sm{x : A} r(x) = x
  \end{equation*}
  %
  is the set-quotient of $A$ by~$\eqr$.
  In other words, there is a map $q : A \to (A/{\eqr})$ such that for every set $B$, the type $(A/{\eqr}) \to B$ is equivalent to
  %
  \begin{myeqn}
    \label{eq:quotient-when-canonical}
    \sm{g : A \to B} \prd{x, y : A} (x \eqr y) \to (g(x) = g(y))
  \end{myeqn}
  with the map being induced by precomposition with $q$.
\end{lem}

\begin{proof}
  Let $i : \prd{x : A} r(r(x)) = r(x)$ witness idempotence of~$r$.
  The map $q : A \to A/{\eqr}$ is defined by $q(x) \defeq (r(x), i(x))$. An equivalence $e$
  from $A/{\eqr} \to B$ to~\eqref{eq:quotient-when-canonical} is defined by
  %
  \[ e(f) \defeq (f \circ q, \nameless), \]
  %
  where the underscore $\nameless$ denotes the following proof: if $x, y : A$ and $x \eqr y$ then by assumption
  $r(x) = r(y)$, hence $(r(x), i(x)) = (r(y), i(y))$ as $A$ is a set, therefore $f(q(x)) =
  f(q(y))$. To see that $e$ is an equivalence, consider the map $e'$ in the opposite
  direction,
  %
  \[ e'(g, p) (x, q) \jdeq g(x). \]
  %
  Given any $f : A/{\eqr} \to B$,
  %
  \[ e'(e(f))(x, p) \jdeq f(q(x)) \jdeq f(r(x), i(x)) = f(x, p) \]
  %
  where the last equality holds because $p : r(x) = x$ and so $(x,p) = (r(x), i(x))$
  because $A$ is a set. Similarly we compute
  %
  \[ e(e'(g, p)) \jdeq e(g \circ \proj{1}) \jdeq (f \circ \proj{1} \circ q, {\nameless}). \]
  %
  Because $B$ is a set we need not worry about the $\nameless$ part, while for the first
  component we have
  %
  \[ f(\proj{1}(q(x))) \defeq f(r(x)) = f(x), \]
  %
  where the last equation holds because $r(x) \eqr x$ and $f$ respects $\eqr$ by
  assumption.
\end{proof}

\begin{cor}\label{thm:retraction-quotient}
  Suppose $p:A\to B$ is a retraction between sets.
  Then $B$ is the quotient of $A$ by the equivalence relation $\eqr$ defined by
  \[ (a_1 \eqr a_2) \defeq (p(a_1) = p(a_2)). \]
\end{cor}
\begin{proof}
  Suppose $s:B\to A$ is a section of $p$.
  Then $s\circ p : A\to A$ is an idempotent which satisfies the condition of \PMlinkname{Lemma 6.10.8}{610quotients#Thmprelem3} for this $\eqr$, and $s$ induces an isomorphism from $B$ to its set of fixed points.
\end{proof}

\begin{rmk}\label{Z-quotient-by-canonical-representatives}
\PMlinkname{Lemma 6.10.8}{610quotients#Thmprelem3} applies to $\Z$ with the idempotent $r : \N \times \N \to \N \times \N$
defined by
%
\begin{equation*}
  r(a, b) =
  \begin{cases}
    (a - b, 0) & \text{if $a \geq b$,} \\
    (0, b - a) & \text{otherwise.}
  \end{cases}  
\end{equation*}
%
(This is a valid definition even constructively, since the relation $\geq$ on $\N$ is decidable.)
Thus a non-negative integer is canonically represented as $(k, 0)$ and a non-positive one by $(0, m)$, for $k,m:\N$.
This division into cases implies the following induction principle for integers, which will be useful in \PMlinkexternal{Chapter 8}{http://planetmath.org/node/87582}.
\index{natural numbers}%
(As usual, we identify natural numbers with the corresponding non-negative integers.)
\end{rmk}

\begin{lem}\label{thm:sign-induction}
  \index{integers!induction principle for}%
  \index{induction principle!for integers}%
  Suppose $P:\Z\to\type$ is a type family and that we have
  \begin{itemize}
  \item $d_0: P(0)$,
  \item $d_+: \prd{n:\N} P(n) \to P(\suc(n))$, and
  \item $d_- : \prd{n:\N} P(-n) \to P(-\suc(n))$.
  \end{itemize}
  Then we have $f:\prd{z:\Z} P(z)$ such that $f(0)\jdeq d_0$ and $f(\suc(n))\jdeq d_+(f(n))$, and $f(-\suc(n))\jdeq d_-(f(-n))$ for all $n:\N$.
\end{lem}
\begin{proof}
  We identify $\Z$ with $\sm{x:\N\times\N}(r(x)=x)$, where $r$ is the above idempotent.
  Now define $Q\defeq P\circ r:\N\times \N \to \type$.
  We can construct $g:\prd{x:\N\times \N} Q(x)$ by double induction on $n$:
  \begin{align*}
    g(0,0) &\defeq d_0,\\
    g(\suc(n),0) &\defeq d_+(g(n,0)),\\
    g(0,\suc(m)) &\defeq d_-(g(0,m)),\\
    g(\suc(n),\suc(m)) &\defeq g(n,m).
  \end{align*}
  Let $f$ be the restriction of $g$ to $\Z$.
\end{proof}

For example, we can define the $n$-fold concatenation of a loop for any integer $n$.

\begin{cor}\label{thm:looptothe}
  \indexdef{path!concatenation!n-fold@$n$-fold}%
  Let $A$ be a type with $a:A$ and $p:a=a$.
  There is a function $\prd{n:\Z} (a=a)$, denoted $n\mapsto p^n$, defined by
  \begin{align*}
    p^0 &\defeq \refl{\base}\\
    p^{n+1} &\defeq p^n \ct p
    & &\text{for $n\ge 0$}\\
    p^{n-1} &\defeq p^n \ct \opp p
    & &\text{for $n\le 0$.}
  \end{align*}
\end{cor}

We will discuss the integers further in \PMlinkname{\S 6.11}{611algebra},\PMlinkname{\S 11.1}{111thefieldofrationalnumbers}.

\index{set-quotient|)}%


\end{document}
