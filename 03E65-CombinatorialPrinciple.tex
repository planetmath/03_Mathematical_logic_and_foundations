\documentclass[12pt]{article}
\usepackage{pmmeta}
\pmcanonicalname{CombinatorialPrinciple}
\pmcreated{2013-03-22 14:17:41}
\pmmodified{2013-03-22 14:17:41}
\pmowner{mps}{409}
\pmmodifier{mps}{409}
\pmtitle{combinatorial principle}
\pmrecord{6}{35750}
\pmprivacy{1}
\pmauthor{mps}{409}
\pmtype{Definition}
\pmcomment{trigger rebuild}
\pmclassification{msc}{03E65}
%\pmkeywords{independence}
%\pmkeywords{combinatorial}
%\pmkeywords{set theory}
\pmrelated{Diamond}
\pmrelated{Clubsuit}

% this is the default PlanetMath preamble.  as your knowledge
% of TeX increases, you will probably want to edit this, but
% it should be fine as is for beginners.

% almost certainly you want these
\usepackage{amssymb}
\usepackage{amsmath}
\usepackage{amsfonts}

% used for TeXing text within eps files
%\usepackage{psfrag}
% need this for including graphics (\includegraphics)
%\usepackage{graphicx}
% for neatly defining theorems and propositions
%\usepackage{amsthm}
% making logically defined graphics
%%%\usepackage{xypic}

% there are many more packages, add them here as you need them

% define commands here
\begin{document}
\PMlinkescapeword{utility}
\PMlinkescapeword{implications}
\PMlinkescapeword{independent}
A \emph{combinatorial principle} is any statement $\Phi$ of set theory proved to be independent of Zermelo-Fraenkel (ZF) set theory, usually one with interesting consequences.

If $\Phi$ is a combinatorial principle, then whenever we have implications 
of the form
\[P\implies \Phi\implies Q,\]
we automatically know that $P$ is unprovable in ZF and $Q$ is relatively consistent with ZF.

Some examples of combinatorial principles are the \PMlinkname{axiom of choice}{AxiomOfChoice}, the continuum hypothesis, $\Diamond$, $\clubsuit$, and Martin's axiom.  

\begin{thebibliography}{9}
\bibitem{WJ}
Just, W., \PMlinkexternal{http://www.math.ohiou.edu/\textasciitilde just/resint.html\#principles}{http://www.math.ohiou.edu/~just/resint.html#principles}.
\end{thebibliography}
%%%%%
%%%%%
\end{document}
