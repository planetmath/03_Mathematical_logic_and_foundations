\documentclass[12pt]{article}
\usepackage{pmmeta}
\pmcanonicalname{PermutationModel}
\pmcreated{2013-03-22 14:46:48}
\pmmodified{2013-03-22 14:46:48}
\pmowner{ratboy}{4018}
\pmmodifier{ratboy}{4018}
\pmtitle{permutation model}
\pmrecord{13}{36428}
\pmprivacy{1}
\pmauthor{ratboy}{4018}
\pmtype{Definition}
\pmcomment{trigger rebuild}
\pmclassification{msc}{03E25}
\pmdefines{G\"odel Operations}

\endmetadata

% this is the default PlanetMath preamble.  as your knowledge
% of TeX increases, you will probably want to edit this, but
% it should be fine as is for beginners.

% almost certainly you want these
\usepackage{amssymb}
\usepackage{amsmath}
\usepackage{amsfonts}

% used for TeXing text within eps files
%\usepackage{psfrag}
% need this for including graphics (\includegraphics)
%\usepackage{graphicx}
% for neatly defining theorems and propositions
\usepackage{amsthm}
% making logically defined graphics
%%\usepackage{xypic}

\usepackage{eucal}

% there are many more packages, add them here as you need them

% define commands here
\newcommand{\w}{\omega}
\newcommand{\proves}{\vdash}
\newcommand{\nproves}{\nvdash}
\renewcommand{\subset}{\subseteq}

\DeclareMathOperator{\fix}{fix}
\DeclareMathOperator{\dom}{dom}
\DeclareMathOperator{\rank}{rank}


\newcommand{\R}{\mathbb{R}}

\newtheorem*{thm}{Theorem}
\newtheorem*{lem}{Lemma}
\begin{document}
A permutation model is a model of the axioms of set theory in which there is a non trivial automorphism of the set theoretic universe.  Such models are used to show the consistency of the negation of the Axiom of Choice (AC).  

A typical construction of a permutation model is done here.  By $ZF^-$ we denote the axioms of $ZF$ minus the axiom of foundation.  In particular we allow sets $a$ such that $a = \{a\}$ which we will call atoms.  Let $A$ be an infinite set of atoms.

Define $V_\alpha(A)$ by induction on $\alpha$ as follows:
\begin{align*}
V_0(A) &= A \\
V_{\alpha+1}(A) &= \mathcal{P}(V_\alpha) \\
V_\alpha(A) &= \bigcup_{\gamma < \alpha}V_\gamma(A)  \text{ for $\alpha$ limit}
\end{align*}

Finally define $V = \bigcup_{\alpha \in \text{ON}} V_\alpha(A)$.  Then we have
\[
A = V_0(A) \subset V_1(A) \subset \cdots \subset V_\alpha(A) \cdots \subset V
\]
For any $x \in V$ we can assign a rank,
\[
\rank(x) = \text{ least } \alpha [ x \in V_{\alpha+1}(A)]
\]
Let $G$ be the group of permutations of $A$. For $\pi \in G$ we extend $\pi$ to
a permutation of $V$ by induction on $\in$ by defining
\[
\pi(x) = \{ \pi(y) : y \in x \}
\]
and letting $\pi(\emptyset) = \emptyset$. Then $G$ permutes $V$ and fixes the well founded sets $WF \subset V$.
                                                                                
\begin{lem}
For all $x,y \in V$ and any $\pi \in G$.
\[
x \in y \iff \pi(x) \in \pi(y)
\]
\end{lem}
That is, $\pi$ is an $\in$-automorphism of $V$.  From this we can prove that $\pi(\{X,Y\}) = \{\pi(X), \pi(Y)\}$ and so
\begin{align*}
\pi((X,Y)) &= (\pi(X),\pi(Y))\\
\pi((X,Y,Z)) &= (\pi(X),\pi(Y),\pi(Z))
\end{align*}

Also by induction on $\alpha$ it is easy to show that
\[
\rank(x) = \rank(\pi(x))
\]
for all $x \in V$.

Let $a_1,\cdots,a_n \in A$ and define
\[
[ a_1, \cdots, a_n ] = \{\pi \in G : \pi(a_i) = a_i,\text{ for } i = 1, \cdots, n \}
\]
Call a set $X \in V$ symmetric if there exists $a_1,\cdots,a_n \in A$ such that $\pi(X) = X$ for all $\pi \in [a_1, \cdots, a_n]$.  Define the class $HS \subset V$ of hereditarily symmetric sets
\[
HS = \{x \in V : x \text{ is symmetric and } x \subset HS \}
\]
                                                                                
Call a class $N$ transitive if
\[
\forall x \in N [ x \subset N]
\]
and call $N$ almost universal if (for sets S)
\[
\forall S \subset N [ \exists Y \in N (S \subset Y) ]
\]
                                                                                
$HS$ is transitive and almost universal.
                                                                                
To show that a class $N \models ZF^-$ is straightforward for most axioms of $ZF^-$ except for the axiom of Comprehension.  To show $N$ is a model of Comprehension it suffices to show that $N$ is closed under \textbf{G\"odel Operations}:
                                                                                
\begin{align*}
G_1(X,Y) &= \{X,Y\} \\
G_2(X,Y) &= X \setminus Y \\
G_3(X,Y) &= X \times Y \\
G_4(X) &= \text{dom}(X)\\
G_5(X) &= \ \in \!\cap X^2 \\
G_6(X) &= \{(a,b,c) : (b,c,a) \in X \}\\
G_7(X) &= \{(a,b,c) : (c,b,a) \in X \}\\
G_8(X) &= \{(a,b,c) : (a,c,b) \in X \}
\end{align*}
                                                                                
\begin{thm} ($ZF$)  If $N$ is transitive, almost universal and closed under G\"odel Operations, then $N \models ZF$.
\end{thm}
                                                                                
$HS$ is closed under G\"odel operations and so $HS \models ZF^-$.  The class $HS$ is a permutation model.  The set of atoms $A \in HS$ and furthermore:
                                                                                
\begin{lem} Let $f : \w \rightarrow A$ be a one to one function.  Then $f \notin HS$ and so $A$ cannot be well ordered in $HS$.
\end{lem}
                                                                                
Which proves the   theorem:                                                                                
                                                                                
\begin{thm}
$HS \models ZF^- + \neg AC$.
\end{thm}
                                                                                
which completes the proof that $\text{Con}(ZF^-) \implies \text{Con}(ZF^- + \neg AC)$.  In particular we have that $ZF^- \nproves AC$.
%%%%%
%%%%%
\end{document}
