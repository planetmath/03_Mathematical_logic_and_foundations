\documentclass[12pt]{article}
\usepackage{pmmeta}
\pmcanonicalname{CompositionOfForcingNotions}
\pmcreated{2013-03-22 12:54:20}
\pmmodified{2013-03-22 12:54:20}
\pmowner{Henry}{455}
\pmmodifier{Henry}{455}
\pmtitle{composition of forcing notions}
\pmrecord{4}{33256}
\pmprivacy{1}
\pmauthor{Henry}{455}
\pmtype{Definition}
\pmcomment{trigger rebuild}
\pmclassification{msc}{03E35}
\pmclassification{msc}{03E40}
\pmrelated{Forcing}

% this is the default PlanetMath preamble.  as your knowledge
% of TeX increases, you will probably want to edit this, but
% it should be fine as is for beginners.

% almost certainly you want these
\usepackage{amssymb}
\usepackage{amsmath}
\usepackage{amsfonts}

% used for TeXing text within eps files
%\usepackage{psfrag}
% need this for including graphics (\includegraphics)
%\usepackage{graphicx}
% for neatly defining theorems and propositions
%\usepackage{amsthm}
% making logically defined graphics
%%%\usepackage{xypic}

% there are many more packages, add them here as you need them

% define commands here
%\PMlinkescapeword{theory}
\begin{document}
Suppose $P$ is a forcing notion in $\mathfrak{M}$ and $\hat{Q}$ is some $P$-name such that $\Vdash_P \hat{Q}$\texttt{ is a forcing notion}.

Then take a set of $P$-names $Q$ such that given a $P$ name $\tilde{Q}$ of $Q$, $\Vdash_P \tilde{Q}=\hat{Q}$ (that is, no matter which generic subset $G$ of $P$ we force with, the names in $Q$ correspond precisely to the elements of $\hat{Q}[G]$).  We can define

$$P*Q=\{\langle p,\hat{q}\rangle \mid p\in P, \hat{q}\in Q\}$$

We can define a partial order on $P*Q$ such that $\langle p_1,\hat{q}_1\rangle\leq \langle p_2,\hat{q}_2\rangle$ iff $p_1\leq_P p_2$ and $p_1\Vdash \hat{q}_1\leq_{\hat{Q}} \hat{q}_2$.  (A note on interpretation: $q_1$ and $q_2$ are $P$ names; this requires only that $\hat{q}_1\leq \hat{q}_2$ in generic subsets contain $p_1$, so in other generic subsets that fact could fail.)

Then $P*\hat{Q}$ is itself a forcing notion, and it can be shown that forcing by $P*\hat{Q}$ is equivalent to forcing first by $P$ and then by $\hat{Q}[G]$.
%%%%%
%%%%%
\end{document}
