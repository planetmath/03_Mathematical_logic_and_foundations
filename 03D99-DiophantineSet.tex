\documentclass[12pt]{article}
\usepackage{pmmeta}
\pmcanonicalname{DiophantineSet}
\pmcreated{2013-03-22 18:02:50}
\pmmodified{2013-03-22 18:02:50}
\pmowner{CWoo}{3771}
\pmmodifier{CWoo}{3771}
\pmtitle{Diophantine set}
\pmrecord{13}{40571}
\pmprivacy{1}
\pmauthor{CWoo}{3771}
\pmtype{Definition}
\pmcomment{trigger rebuild}
\pmclassification{msc}{03D99}
\pmclassification{msc}{03D80}
\pmdefines{Diophantine function}

\endmetadata

\usepackage{amssymb,amscd}
\usepackage{amsmath}
\usepackage{amsfonts}
\usepackage{mathrsfs}

% used for TeXing text within eps files
%\usepackage{psfrag}
% need this for including graphics (\includegraphics)
%\usepackage{graphicx}
% for neatly defining theorems and propositions
\usepackage{amsthm}
% making logically defined graphics
%%\usepackage{xypic}
\usepackage{pst-plot}

% define commands here
\newcommand*{\abs}[1]{\left\lvert #1\right\rvert}
\newtheorem{prop}{Proposition}
\newtheorem{thm}{Theorem}
\newtheorem{ex}{Example}
\newcommand{\real}{\mathbb{R}}
\newcommand{\pdiff}[2]{\frac{\partial #1}{\partial #2}}
\newcommand{\mpdiff}[3]{\frac{\partial^#1 #2}{\partial #3^#1}}
\begin{document}
\PMlinkescapeword{diophantine}

A set $S$ is said to be \emph{Diophantine} if 
\begin{itemize}
\item
it is a subset of $\mathbb{N}^n$, the set of all $n$-tuples of positive integers, and 
\item
there is a polynomial $p$ over $\mathbb{Z}$ in $n+k$ variables, $k\ge 0$, such that $x\in S$ iff there is some $y\in \mathbb{N}^k$, such that $p(x,y)=0$.
\end{itemize}

So $S$ can be thought of as a set such that, there is a Diophantine equation $p=0$ and a non-negative integer $k$, so that when each element in $S$ is ``combined'' with some $k$-tuple, makes up a solution to a Diophantine equation $p=0$.  In other words, if $f_n^{n+k}:\mathbb{N}^{n+k}\to \mathbb{N}^n$ is a projection function given by $f_n^{n+k}(x,y)=x$ where $x\in \mathbb{N}^n$ and $y\in \mathbb{N}^k$, then $S$ is a Diophantine set iff $S=f_n^{n+k}(Z)$, where $Z$ is the zero set of some Diophantine equation $p=0$.  Equivalently, a set $S\in \mathbb{N}^n$ is Diophantine if there is a $p\in \mathbb{Z}[X_1,\ldots , X_{n+k}]$, such that $$S=\lbrace x\in \mathbb{N}^n\mid \exists y\in \mathbb{N}^k \; p(x,y)=0\rbrace.$$

For example, $\mathbb{N}$ itself is Diophantine, for the polynomial $p(x,y)=x-y$ works.  Another trivial example: the set of all positive integers divisible by $3$ is Diophantine, for the polynomial $p(x,y)=x-3y$ works.  

For a less trivial example, let us show that the set of all triples $(a,b,c)$ such that $a\le b\le c$ is Diophantine.  For the inequality $a\le b$, let $p(x_1,x_2,y)= x_2-x_1-(y-1)$.  Then the sentence $\exists y \; p(x_1,x_2,y)=0$ is equivalent to $x_1\le x_2$.  Similarly, for the inequality $b\le c$, we have the same polynomial $p$.  Putting the two inequality together amounts to setting $q(x_1,x_2,x_3,y_1,y_2)=p(x_1,x_2,y_1)^2+p(x_2,x_3,y_2)^2$.  Thus, the sentence $\exists  (y_1,y_2)\; q(x,y)=0$, where $x=(x_1,x_2,x_3)$ and $y=(y_1,y_2)$ is the same as the inequality $x_1\le x_2\le x_3$.

Some other Diophantine sets are:
\begin{itemize}
\item the set $A=B\cup C$, where $B$ and $C$ are Diophantine
\item the set $A=B\cap C$, where $B$ and $C$ are Diophantine
\item the set $\lbrace a \mid a \equiv b \pmod n\rbrace$
\item the set of composite numbers
\item the set of prime numbers 
\item the set of powers of a positive integer $\lbrace m^n\mid n=0,1,\ldots \rbrace$
\end{itemize}

\textbf{Remark}.  Associated with the concept of a Diophantine set is that of a \emph{Diophantine function}: a function $f$ is said to be \emph{Diophantine} if its graph $\lbrace (x,f(x)) \mid x\in \operatorname{dom}(f)\rbrace$ is a Diophantine set.  Some well-know Diophantine functions are the exponential functions $f(x)=n^x$ and the factorial function $f(x)=x!$, where $n,x$ are positive integers.

It turns out that a function is Diophantine iff it is recursive.  From this, it is possible to prove that Hilbert's 10th problem is unsolvable.  

The idea behind using Diophantine sets to prove the unsolvability of Hilbert's 10th problem comes from Yuri Matiyasevi\^c, and hence the theorem is known as Matiyasevi\^c's theorem.

\begin{thebibliography}{8}
\bibitem{md} M Davis, {\em Computability and Unsolvability}. Dover Publications, Inc. New York, 1982
\end{thebibliography}
%%%%%
%%%%%
\end{document}
