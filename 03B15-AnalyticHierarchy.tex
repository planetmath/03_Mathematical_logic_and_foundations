\documentclass[12pt]{article}
\usepackage{pmmeta}
\pmcanonicalname{AnalyticHierarchy}
\pmcreated{2013-03-22 12:56:48}
\pmmodified{2013-03-22 12:56:48}
\pmowner{Henry}{455}
\pmmodifier{Henry}{455}
\pmtitle{analytic hierarchy}
\pmrecord{4}{33305}
\pmprivacy{1}
\pmauthor{Henry}{455}
\pmtype{Definition}
\pmcomment{trigger rebuild}
\pmclassification{msc}{03B15}
\pmsynonym{analytical hierarchy}{AnalyticHierarchy}
\pmrelated{ArithmeticalHierarchy}

% this is the default PlanetMath preamble.  as your knowledge
% of TeX increases, you will probably want to edit this, but
% it should be fine as is for beginners.

% almost certainly you want these
\usepackage{amssymb}
\usepackage{amsmath}
\usepackage{amsfonts}

% used for TeXing text within eps files
%\usepackage{psfrag}
% need this for including graphics (\includegraphics)
%\usepackage{graphicx}
% for neatly defining theorems and propositions
%\usepackage{amsthm}
% making logically defined graphics
%%%\usepackage{xypic}

% there are many more packages, add them here as you need them

% define commands here
%\PMlinkescapeword{theory}
\begin{document}
The \emph{analytic hierarchy} is a hierarchy of either (depending on context) formulas or relations similar to the arithmetical hierarchy.  It is essentially the second order equivalent.  Like the arithmetical hierarchy, the relations in each level are exactly the relations defined by the formulas of that level.

The first level can be called $\Delta^1_0$, $\Delta^1_1$, $\Sigma^1_0$, or $\Pi^1_0$, and consists of the arithmetical formulas or relations.

A formula $\phi$ is $\Sigma^1_n$ if there is some arithmetical formula $\psi$ such that:

$$\phi(\vec k)=\exists X_1\forall X_2\cdots Q X_n\psi(\vec k,\vec X_n)$$
$$\text{ where }Q\text{ is either }\forall\text{ or }\exists\text{, whichever maintains the pattern of alternating quantifiers, and each } X_i \text{ is a set variable (that is, second order)}
$$

Similarly, a formula $\phi$ is $\Pi^1_n$ if there is some arithmetical formula $\psi$ such that:

$$\phi(\vec k)=\forall X_1\exists X_2\cdots Q X_n\psi(\vec k,\vec X_n)$$
$$\text{ where }Q\text{ is either }\forall\text{ or }\exists\text{, whichever maintains the pattern of alternating quantifiers, and each } X_i \text{ is a set variable (that is, second order)}
$$
%%%%%
%%%%%
\end{document}
