\documentclass[12pt]{article}
\usepackage{pmmeta}
\pmcanonicalname{PointedSet}
\pmcreated{2013-03-22 15:55:42}
\pmmodified{2013-03-22 15:55:42}
\pmowner{CWoo}{3771}
\pmmodifier{CWoo}{3771}
\pmtitle{pointed set}
\pmrecord{10}{37936}
\pmprivacy{1}
\pmauthor{CWoo}{3771}
\pmtype{Definition}
\pmcomment{trigger rebuild}
\pmclassification{msc}{03E20}
\pmsynonym{base point}{PointedSet}
\pmsynonym{base-point}{PointedSet}
\pmdefines{basepoint}
\pmdefines{pointed subset}

\endmetadata

\usepackage{amssymb,amscd}
\usepackage{amsmath}
\usepackage{amsfonts}

% used for TeXing text within eps files
%\usepackage{psfrag}
% need this for including graphics (\includegraphics)
%\usepackage{graphicx}
% for neatly defining theorems and propositions
%\usepackage{amsthm}
% making logically defined graphics
%%\usepackage{xypic}

% define commands here

\begin{document}
\subsection{Definition}
A \emph{pointed set} is an ordered pair $(A,a)$ such that $A$ is a set and $a\in A$.  The element $a$ is called the \emph{basepoint} of $(A,a)$.  At first glance, it seems appropriate enough to call any non-empty set a pointed set.  However, the basepoint plays an important role in that if we select a different element $a^{\prime}\in A$, the ordered pair $(A,a^{\prime})$ forms a different pointed set from $(A,a)$.  In fact, given any non-empty set $A$ with $n$ elements, $n$ pointed sets can be formed from $A$.

A function $f$ between two pointed sets $(A,a)$ and $(B,b)$ is just a function from $A$ to $B$ such that $f(a)=b$.  Whereas there are $|B|^{\mid A\mid}$ functions from $A$ to $B$, only $|B|^{\mid A\mid-1}$ of them are from $(A,a)$ to $(B,b)$.

Pointed sets are mainly used as illustrative examples in the study of universal algebra as algebras with a single constant operator.  This operator takes every element in the algebra to a unique constant, which is clearly the basepoint in our definition above.  Any \PMlinkname{homomorphism}{HomomorphismBetweenAlgebraicSystems} between two algebras preserves basepoints (taking the basepoint of the domain algebra to the basepoint of the codomain algebra).

From the above discussion, we see that a pointed set can alternatively described as any constant function $p$ where the its domain is the underlying set, and its range consists of a single element $p_0\in \operatorname{dom}(p)$.  A function $f$ from one pointed set $p$ to another pointed set $q$ can be seen as a function from the domain of $p$ to the domain of $q$ such that the following diagram commutes:

$$\xymatrix{
{\operatorname{dom}(p)}\ar[r]^{f}\ar[d]_{p}&{\operatorname{dom}(q)}\ar[d]^{q}\\
{\lbrace p_0\rbrace}\ar[r]_{c}&{\lbrace q_0\rbrace}}
$$


\subsection{Creation of Pointed Sets from Existing Ones}
\textbf{Pointed Subsets}.  Given a pointed set $(A,a)$, a pointed subset of $(A,a)$ is an ordered pair $(A^{\prime},a)$, where $A^{\prime}$ is a subset of $A$.  A pointed subset is clearly a pointed set.

\textbf{Products of Pointed Sets}.  Given two pointed sets $(A,a)$ and $(B,b)$, their product is defined to be the ordered pair $(A\times B,(a,b))$.  More generally, given a family of pointed sets $(A_i,a_i)$ indexed by $I$, we can form their Cartesian product to be the ordered pair $(\prod A_i, (a_i))$.  Both the finite and the arbitrary cases produce pointed sets.

\textbf{Quotients}.  Given a pointed set $(A,a)$ and an equivalence relation $R$ defined on $A$.  For each $x\in A$, define $\overline{x}:=\lbrace y\in A \mid y R x\rbrace$.  Then $A/R:=\lbrace \overline{x}\mid x\in A\rbrace$ is a subset of the power set $2^A$ of $A$, called the quotient of $A$ by $R$.  Then $(A/R,\overline{a})$ is a pointed set.
%%%%%
%%%%%
\end{document}
