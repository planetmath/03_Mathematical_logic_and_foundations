\documentclass[12pt]{article}
\usepackage{pmmeta}
\pmcanonicalname{ExampleOfGodelNumbering}
\pmcreated{2013-03-22 12:58:28}
\pmmodified{2013-03-22 12:58:28}
\pmowner{Henry}{455}
\pmmodifier{Henry}{455}
\pmtitle{example of G\"odel numbering}
\pmrecord{7}{33345}
\pmprivacy{1}
\pmauthor{Henry}{455}
\pmtype{Example}
\pmcomment{trigger rebuild}
\pmclassification{msc}{03B10}

\endmetadata

% this is the default PlanetMath preamble.  as your knowledge
% of TeX increases, you will probably want to edit this, but
% it should be fine as is for beginners.

% almost certainly you want these
\usepackage{amssymb}
\usepackage{amsmath}
\usepackage{amsfonts}

% used for TeXing text within eps files
%\usepackage{psfrag}
% need this for including graphics (\includegraphics)
%\usepackage{graphicx}
% for neatly defining theorems and propositions
%\usepackage{amsthm}
% making logically defined graphics
%%%\usepackage{xypic}

% there are many more packages, add them here as you need them

% define commands here
%\PMlinkescapeword{theory}
\begin{document}
We can define by recursion a function $e$ from formulas of arithmetic to numbers, and the corresponding G\"odel numbering as the inverse.

The symbols of the language of arithmetic are $=$, $\forall$, $\neg$, $\rightarrow$, $0$, $S$, $<$, $+$, $\cdot$, the variables $v_i$ for any integer $i$, and $($ and $)$.  $($ and $)$ are only used to define the order of operations, and should be inferred where appropriate in the definition below.

We can define a function $e$ by recursion as follows:

\begin{itemize}
\item $e(v_i)=\langle 0,i\rangle$

\item $e(\phi=\psi)=\langle 1,e(\phi),e(\psi)\rangle$

\item $e(\forall v_i \phi)=\langle 2,e(v_i),e(\phi)\rangle$

\item $e(\neg\phi)=\langle 3,e(\phi)\rangle$

\item $e(\phi\rightarrow\psi)=\langle 4,e(\phi),e(\psi)\rangle$

\item $e(0)=\langle 5\rangle$

\item $e(S\phi)=\langle 6,e(\phi)\rangle$

\item $e(\phi<\psi)=\langle 7,e(\phi),e(\psi)\rangle$

\item $e(\phi+\psi)=\langle 8,e(\phi),e(\psi)\rangle$

\item $e(\phi\cdot\psi)=\langle 9,e(\phi),e(\psi)\rangle$
\end{itemize}

Clearly $e^{-1}$ is a G\"odel numbering, with $\ulcorner\phi\urcorner=e(\phi)$.
%%%%%
%%%%%
\end{document}
