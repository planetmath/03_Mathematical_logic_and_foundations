\documentclass[12pt]{article}
\usepackage{pmmeta}
\pmcanonicalname{A22StructuralRules}
\pmcreated{2013-11-09 5:20:59}
\pmmodified{2013-11-09 5:20:59}
\pmowner{PMBookProject}{1000683}
\pmmodifier{PMBookProject}{1000683}
\pmtitle{A.2.2 Structural rules}
\pmrecord{1}{}
\pmprivacy{1}
\pmauthor{PMBookProject}{1000683}
\pmtype{Feature}
\pmclassification{msc}{03B15}

\usepackage{xspace}
\usepackage{amssyb}
\usepackage{amsmath}
\usepackage{amsfonts}
\usepackage{amsthm}
\makeatletter
\newcommand{\ctx}{\ensuremath{\mathsf{ctx}}}
\newcommand{\define}[1]{\textbf{#1}}
\newcommand{\indexdef}[1]{\index{#1|defstyle}}   
\newcommand{\indexsee}[2]{\index{#1|see{#2}}}    
\newcommand{\intro}{\textsc{intro}}
\newcommand{\jdeq}{\equiv}      
\newcommand{\jdeqtp}[4]{#1 \vdash #2 \jdeq #3 : #4}
\def\lamu#1{{\lambda}\@lamuarg#1:\@endlamuarg\@ifnextchar\bgroup{.\,\lamu}{.\,}}
\def\@lamuarg#1:#2\@endlamuarg{#1}
\newcommand{\oftp}[3]{#1 \vdash #2 : #3}
\newcommand{\Subst}{\mathsf{Subst}}
\newcommand{\tmtp}[2]{#1 \mathord{:} #2}
\def\tprd#1{\@tprd{#1}\@ifnextchar\bgroup{\tprd}{}}
\def\@tprd#1{\mathchoice{{\textstyle\prod_{(#1)}}}{\prod_{(#1)}}{\prod_{(#1)}}{\prod_{(#1)}}}
\newcommand{\UU}{\ensuremath{\mathcal{U}}\xspace}
\newcommand{\Vble}{\mathsf{Vble}}
\newcommand{\Weak}{\mathsf{Wkg}}
\newcommand{\wfctx}[1]{#1\ \ctx}
\makeatother

\begin{document}
\index{structural!rules|(}%
\index{rule!structural|(}%

The fact that the context holds assumptions is expressed by the rule which says that we may derive those typing judgments which are listed in the context:
%
\begin{mathpar}
  \inferrule*[right=$\Vble$]
  {\wfctx {(\tmtp{x_1}{A_1}, \ldots, \tmtp{x_n}{A_n})} }
  {\oftp{\tmtp{x_1}{A_1}, \ldots, \tmtp{x_n}{A_n}}{x_i}{A_i}}
\end{mathpar}
%
As with $\ctx$-\textsc{ext}, the hypothesis and conclusion of the rule $\Vble$ are judgments of different forms, only now they are reversed: we start with a well-formed context and derive a typing judgment.

The following important principles, called \define{substitution}
\indexdef{rule!of substitution}%
and
\define{weakening},
\indexdef{rule!of weakening}%
need not be explicitly assumed. Rather, it is possible to
show, by induction on the structure of all possible derivations, that whenever
the hypotheses of these rules are derivable, their conclusion is also
derivable.\footnote{Such rules are called \define{admissible}\indexdef{rule!admissible}\indexsee{admissible!rule}{rule, admissible}.}
For the typing judgments these principles are manifested as
%
\begin{mathpar}
  \inferrule*[right=$\Subst_1$]
  {\oftp\Gamma{a}{A} \\ \oftp{\Gamma,\tmtp xA,\Delta}{b}{B}}
  {\oftp{\Gamma,\Delta[a/x]}{b[a/x]}{B[a/x]}}
\and
  \inferrule*[right=$\Weak_1$]
  {\oftp\Gamma{A}{\UU_i} \\ \oftp{\Gamma,\Delta}{b}{B}}
  {\oftp{\Gamma,\tmtp xA,\Delta}{b}{B}}
\end{mathpar}
and for judgmental equalities they become
\begin{mathpar}
  \inferrule*[right=$\Subst_2$]
  {\oftp\Gamma{a}{A} \\ \jdeqtp{\Gamma,\tmtp xA,\Delta}{b}{c}{B}}
  {\jdeqtp{\Gamma,\Delta[a/x]}{b[a/x]}{c[a/x]}{B[a/x]}}
\and
  \inferrule*[right=$\Weak_2$]
  {\oftp\Gamma{A}{\UU_i} \\ \jdeqtp{\Gamma,\Delta}{b}{c}{B}}
  {\jdeqtp{\Gamma,\tmtp xA,\Delta}{b}{c}{B}}
\end{mathpar}
%
In addition to the judgmental equality rules given for each type former, we also
assume that judgmental equality is an equivalence relation respected by typing.
\begin{mathparpagebreakable}
  \inferrule*{\oftp\Gamma{a}{A}}{\jdeqtp\Gamma{a}{a}{A}}
\and
  \inferrule*{\jdeqtp\Gamma{a}{b}{A}}{\jdeqtp\Gamma{b}{a}{A}}
\and
  \inferrule*{\jdeqtp\Gamma{a}{b}{A} \\ \jdeqtp\Gamma{b}{c}{A}}{\jdeqtp\Gamma{a}{c}{A}}
\and
  \inferrule*{\oftp\Gamma{a}{A} \\ \jdeqtp\Gamma{A}{B}{\UU_i}}{\oftp\Gamma{a}{B}}
\and
  \inferrule*{\jdeqtp\Gamma{a}{b}{A} \\ \jdeqtp\Gamma{A}{B}{\UU_i}}{\jdeqtp\Gamma{a}{b}{B}}
\end{mathparpagebreakable}
%
Additionally, for all the type formers below, we assume rules stating that each constructor preserves definitional equality in each of its arguments; for instance, along with the $\Pi$-\intro\ rule, we assume the rule
\[
  \inferrule*[right=$\Pi$-\intro-eq]
  {\jdeqtp\Gamma{A}{A'}{\UU_i} \\
   \jdeqtp{\Gamma,\tmtp xA}{B}{B'}{\UU_i} \\
   \jdeqtp{\Gamma,\tmtp xA}{b}{b'}{B}}
  {\jdeqtp\Gamma{\lamu{x:A} b}{\lamu{x:A'} b'}{\tprd{x:A} B}}
\]
However, we omit these rules for brevity.

\index{rule!structural|)}%
\index{structural!rules|)}%

\end{document}
