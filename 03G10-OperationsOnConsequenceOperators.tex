\documentclass[12pt]{article}
\usepackage{pmmeta}
\pmcanonicalname{OperationsOnConsequenceOperators}
\pmcreated{2013-03-22 16:29:38}
\pmmodified{2013-03-22 16:29:38}
\pmowner{rspuzio}{6075}
\pmmodifier{rspuzio}{6075}
\pmtitle{operations on consequence operators}
\pmrecord{6}{38669}
\pmprivacy{1}
\pmauthor{rspuzio}{6075}
\pmtype{Definition}
\pmcomment{trigger rebuild}
\pmclassification{msc}{03G10}
\pmclassification{msc}{03B22}
\pmclassification{msc}{03G25}

\endmetadata

% this is the default PlanetMath preamble.  as your knowledge
% of TeX increases, you will probably want to edit this, but
% it should be fine as is for beginners.

% almost certainly you want these
\usepackage{amssymb}
\usepackage{amsmath}
\usepackage{amsfonts}

% used for TeXing text within eps files
%\usepackage{psfrag}
% need this for including graphics (\includegraphics)
%\usepackage{graphicx}
% for neatly defining theorems and propositions
%\usepackage{amsthm}
% making logically defined graphics
%%%\usepackage{xypic}

\newtheorem{definition}{Definition}
\begin{document}
Let $L$ be a set and let $\mathcal{C}$ be the set of all consequence operators
on $S$.  Then we may define a binary relation $\le \,\subset \mathcal{C} \times
\mathcal{C}$ and binary operations $\wedge, \vee, \veebar \colon \mathcal{C}\times \mathcal{C} \to \mathcal{C}$ as follows:

\begin{definition}
For $C_1, C_2 \in \mathcal{C}$, we have $C_1 \le C_2$ when, for all 
$X \subseteq L$, we have $C_1 (X) \subseteq C_2 (X)$
\end{definition}

\begin{definition}
For $C_1, C_2 \in \mathcal{C}$, we have $(C_1 \wedge C_2) (X) = 
C_1 (X) \cap C_2 (X)$ for all $X \subseteq L$.
\end{definition}

\begin{definition}
For $C_1, C_2 \in \mathcal{C}$, we have $(C_1 \vee C_2) (X) = 
C_1 (X) \cup C_2 (X)$ for all $X \subseteq L$.
\end{definition}

\begin{definition}
For $C_1, C_2 \in \mathcal{C}$, we have $(C_1 \veebar C_2) (X) =
\cap \{ Y \mid X \subseteq Y \subseteq L \land C_1 (Y) = C_2 (Y)
= Y \}$ for all $X \subseteq L$.
\end{definition}
%%%%%
%%%%%
\end{document}
