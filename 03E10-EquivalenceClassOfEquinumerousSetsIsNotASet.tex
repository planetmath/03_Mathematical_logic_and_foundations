\documentclass[12pt]{article}
\usepackage{pmmeta}
\pmcanonicalname{EquivalenceClassOfEquinumerousSetsIsNotASet}
\pmcreated{2013-03-22 18:50:34}
\pmmodified{2013-03-22 18:50:34}
\pmowner{CWoo}{3771}
\pmmodifier{CWoo}{3771}
\pmtitle{equivalence class of equinumerous sets is not a set}
\pmrecord{4}{41648}
\pmprivacy{1}
\pmauthor{CWoo}{3771}
\pmtype{Result}
\pmcomment{trigger rebuild}
\pmclassification{msc}{03E10}

\endmetadata

\usepackage{amssymb,amscd}
\usepackage{amsmath}
\usepackage{amsfonts}
\usepackage{mathrsfs}

% used for TeXing text within eps files
%\usepackage{psfrag}
% need this for including graphics (\includegraphics)
%\usepackage{graphicx}
% for neatly defining theorems and propositions
\usepackage{amsthm}
% making logically defined graphics
%%\usepackage{xypic}
\usepackage{pst-plot}

% define commands here
\newcommand*{\abs}[1]{\left\lvert #1\right\rvert}
\newtheorem{prop}{Proposition}
\newtheorem{thm}{Theorem}
\newtheorem{ex}{Example}
\newcommand{\real}{\mathbb{R}}
\newcommand{\pdiff}[2]{\frac{\partial #1}{\partial #2}}
\newcommand{\mpdiff}[3]{\frac{\partial^#1 #2}{\partial #3^#1}}
\begin{document}
Recall that two sets are equinumerous iff there is a bijection between them.

\begin{prop} Let $A$ be a non-empty set, and $E(A)$ the class of all sets equinumerous to $A$.  Then $E(A)$ is a proper class. \end{prop}
\begin{proof}  $E(A)\ne \varnothing$ since $A$ is in $E(A)$.  Since $A\ne \varnothing$, pick an element $a\in A$, and let $B=A-\lbrace a\rbrace$.  Then $C:=\lbrace y\mid y\mbox{ is a set, and }y\notin B\rbrace$ is a proper class, for otherwise $C\cup B$ would be the ``set'' of all sets, which is impossible.  For each $y$ in $C$, the set $F(y):=B\cup \lbrace y\rbrace$ is in one-to-one correspondence with $A$, with the bijection $f:F(y)\to A$ given by $f(x)=x$ if $x\in B$, and $f(y)=a$.  Therefore $E(A)$ contains $F(y)$ for every $y$ in the proper class $C$.  Furthermore, since $F(y_1)\ne F(y_2)$ whenever $y_1\ne y_2$, we have that $E(A)$ is a proper class as a result.
\end{proof}

\textbf{Remark}.  In the proof above, one can think of $F$ as a class function from $C$ to $E(A)$, taking every $y\in C$ into $F(y)$.  This function is one-to-one, so $C$ embeds in $E(A)$, and hence $E(A)$ is a proper class.
%%%%%
%%%%%
\end{document}
