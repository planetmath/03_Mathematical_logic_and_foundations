\documentclass[12pt]{article}
\usepackage{pmmeta}
\pmcanonicalname{ElementarilyEquivalent}
\pmcreated{2013-03-22 13:00:26}
\pmmodified{2013-03-22 13:00:26}
\pmowner{CWoo}{3771}
\pmmodifier{CWoo}{3771}
\pmtitle{elementarily equivalent}
\pmrecord{8}{33388}
\pmprivacy{1}
\pmauthor{CWoo}{3771}
\pmtype{Definition}
\pmcomment{trigger rebuild}
\pmclassification{msc}{03C99}
%\pmkeywords{sentence}
\pmdefines{theory}

\endmetadata

% this is the default PlanetMath preamble.  as your knowledge
% of TeX increases, you will probably want to edit this, but
% it should be fine as is for beginners.

% almost certainly you want these
\usepackage{amssymb}
\usepackage{amsmath}
\usepackage{amsfonts}

% used for TeXing text within eps files
%\usepackage{psfrag}
% need this for including graphics (\includegraphics)
%\usepackage{graphicx}
% for neatly defining theorems and propositions
%\usepackage{amsthm}
% making logically defined graphics
%%%\usepackage{xypic}

% there are many more packages, add them here as you need them

% define commands here
%\PMlinkescapeword{theory}
\newtheorem{definition}{Definition}
\renewcommand{\thedefinition}{}
\newtheorem{conv}{Conventions}
\renewcommand{\theconv}{}

\newcommand{\theory}{\operatorname{Th}}


\begin{document}
\begin{conv} All structures share a common signature; the first-order
language $\mathcal{L}$ is the language determined by that signature. 
\end{conv}
\begin{definition}The \emph{theory} of a structure $\mathcal{M}\text{, }\theory(\mathcal{M})\text{,}$ 
is the set of all sentences
of $\mathcal{L}$ that are true in $\mathcal{M}.$
\end{definition}
\begin{definition}Structures $\mathcal{M}$ and $\mathcal{N}$ are \emph{elementarily equivalent},
(in symbols: $\mathcal{M} \equiv \mathcal{N})$ if and only if
$\theory(\mathcal{M}) = \theory(\mathcal{N})$.
\end{definition}

%%%%%
%%%%%
\end{document}
