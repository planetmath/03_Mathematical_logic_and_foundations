\documentclass[12pt]{article}
\usepackage{pmmeta}
\pmcanonicalname{DeMorgansLawsForSetsproof}
\pmcreated{2013-03-22 13:32:16}
\pmmodified{2013-03-22 13:32:16}
\pmowner{mathcam}{2727}
\pmmodifier{mathcam}{2727}
\pmtitle{de Morgan's laws for sets (proof)}
\pmrecord{7}{34134}
\pmprivacy{1}
\pmauthor{mathcam}{2727}
\pmtype{Proof}
\pmcomment{trigger rebuild}
\pmclassification{msc}{03E30}

\endmetadata

% this is the default PlanetMath preamble.  as your knowledge
% of TeX increases, you will probably want to edit this, but
% it should be fine as is for beginners.

% almost certainly you want these
\usepackage{amssymb}
\usepackage{amsmath}
\usepackage{amsfonts}

% used for TeXing text within eps files
%\usepackage{psfrag}
% need this for including graphics (\includegraphics)
%\usepackage{graphicx}
% for neatly defining theorems and propositions
%\usepackage{amsthm}
% making logically defined graphics
%%%\usepackage{xypic}

% there are many more packages, add them here as you need them

% define commands here
\begin{document}
Let $X$ be a set with subsets $A_i \subset X$ for $i\in I$, where
$I$ is an arbitrary index-set. In other words, $I$ can be finite, 
countable, or uncountable. We first show that 
\begin{eqnarray*}
\displaystyle \big( \cup_{i\in I} A_i \big)' &=& \cap_{i\in I} A_i',
\end{eqnarray*}
where $A'$ denotes the complement of $A$. 

Let us define $S=\big( \cup_{i\in I} A_i \big)'$ 
and $T=\cap_{i\in I} A_i'$. To establish the equality $S=T$, we shall 
use a standard argument for proving equalities in set theory. Namely, 
we show that $S\subset T$ and $T\subset S$. 
For the first claim, suppose $x$ is an 
element in $S$. 
Then $x\notin \cup_{i\in I} A_i$, so $x\notin A_i$ for any $i\in I$. 
Hence $x\in A_i'$ for all $i\in I$, and $x\in \cap_{i\in I} A_i'=T$. 
Conversely, suppose $x$ is an 
element in $T=\cap_{i\in I} A_i'$. Then  $x\in A_i'$ for all $i\in I$. 
Hence $x\notin A_i$ for any $i\in I$, so $x\notin \cup_{i\in I} A_i$,
and $x\in S$. 

The second claim,
\begin{eqnarray*}
\big( \cap_{i\in I} A_i \big)' &=& \cup_{i\in I} A_i',
\end{eqnarray*}
follows by applying the first claim to the sets $A_i'$.
%%%%%
%%%%%
\end{document}
