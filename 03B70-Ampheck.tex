\documentclass[12pt]{article}
\usepackage{pmmeta}
\pmcanonicalname{Ampheck}
\pmcreated{2013-03-22 17:47:21}
\pmmodified{2013-03-22 17:47:21}
\pmowner{Jon Awbrey}{15246}
\pmmodifier{Jon Awbrey}{15246}
\pmtitle{ampheck}
\pmrecord{18}{40250}
\pmprivacy{1}
\pmauthor{Jon Awbrey}{15246}
\pmtype{Definition}
\pmcomment{trigger rebuild}
\pmclassification{msc}{03B70}
\pmclassification{msc}{03B35}
\pmclassification{msc}{03B22}
\pmclassification{msc}{03B05}
\pmclassification{msc}{03-03}
\pmclassification{msc}{01A55}
\pmrelated{LogicalConnective}
\pmrelated{LogicalGraph}
\pmrelated{LogicalGraphFormalDevelopment}
\pmrelated{SoleSufficientOperator}
\pmdefines{NAND}
\pmdefines{NNOR}
\pmdefines{Peirce arrow}
\pmdefines{Sheffer stroke}

\endmetadata

% this is the default PlanetMath preamble.  as your knowledge
% of TeX increases, you will probably want to edit this, but
% it should be fine as is for beginners.

% almost certainly you want these
\usepackage{amssymb}
\usepackage{amsmath}
\usepackage{amsfonts}

% used for TeXing text within eps files
%\usepackage{psfrag}
% need this for including graphics (\includegraphics)
%\usepackage{graphicx}
% for neatly defining theorems and propositions
%\usepackage{amsthm}
% making logically defined graphics
%%%\usepackage{xypic}

% there are many more packages, add them here as you need them

% define commands here

\begin{document}
\PMlinkescapeword{calculus}
\PMlinkescapeword{Calculus}
\PMlinkescapeword{clear}
\PMlinkescapeword{Clear}
\PMlinkescapeword{ma}
\PMlinkescapeword{MA}
\PMlinkescapeword{volume}
\PMlinkescapeword{Volume}
\PMlinkescapeword{wedge}
\PMlinkescapeword{Wedge}

\textbf{Ampheck}, from the Greek $\alpha\mu\phi\eta\kappa\eta\varsigma$, \textit{double-edged}, is a term coined by Charles Sanders Peirce for either one of the pair of logically dual operators, variously referred to as \textit{Peirce arrows}, \textit{Sheffer strokes}, or logical NAND and logical NNOR.  Either of these logical operators is a \textit{\PMlinkname{sole sufficient operator}{SoleSufficientOperator}} for defining all of the other operators in the subject matter variously described as boolean functions, monadic predicate calculus, propositional calculus, sentential calculus, or zeroth order logic.  See \textit{\PMlinkname{logical connective}{LogicalConnective}} for further discussion.

\begin{quote}
For example, $x \curlywedge y$ signifies that $x$ is $\mathbf{f}$ and $y$ is $\mathbf{f}$.  Then $(x \curlywedge y) \curlywedge z$, or $\underline{x \curlywedge y} \curlywedge z$, will signify that $z$ is $\mathbf{f}$, but that the statement that $x$ and $y$ are both $\mathbf{f}$ is itself $\mathbf{f}$, that is, is \textit{false}.  Hence, the value of $x \curlywedge x$ is the same as that of $\overline{x}$;  and the value of $\underline{x \curlywedge x} \curlywedge x$ is $\mathbf{f}$, because it is necessarily false;  while the value of $\underline{x \curlywedge y} \curlywedge \underline{x \curlywedge y}$ is only $\mathbf{f}$ in case $x \curlywedge y$ is $\mathbf{v}$;  and $(\underline{x \curlywedge x} \curlywedge x) \curlywedge (x \curlywedge \underline{x \curlywedge x})$ is necessarily true, so that its value is $\mathbf{v}$.
\\
With these two signs, the vinculum (with its equivalents, parentheses, brackets, braces, etc.) and the sign $\curlywedge$, which I will call the \textit{ampheck} (from $\alpha\mu\phi\eta\kappa\eta\varsigma$, cutting both ways), all assertions as to the values of quantities can be expressed. (C.S. Peirce, CP 4.264).
\end{quote}

In the above passage, Peirce introduces the term \textit{ampheck} for the 2-place logical connective or the binary logical operator that is currently called the \textit{joint denial} in logic, the NNOR operator in computer science, or indicated by means of  phrases like ``neither-nor'' or ``both not'' in ordinary language.  In his handwritten manuscripts Peirce used a cursive symbol for the amphecks that he derived from his \textit{dot-cross} notation for truth tables, one that the typographer most likely set by inverting the zodiac symbol for Aries, and that is set in the text above by using the so-called \textit{curly wedge} symbol.

In the same paper, Peirce introduces a symbol for the logically dual operator.  This was rendered by the editors of his \textit{Collected Papers} as an inverted Aries symbol with a bar or a serif at the top, in this way denoting the connective or logical operator that is currently called the \textit{alternative denial} in logic, the NAND operator in computer science, or invoked by means of phrases like ``not-and'' or ``not both'' in ordinary language.  It is not clear whether it was Peirce himself or later writers who initiated the practice, but on account of their dual relationship it became common to refer to these two operators in the plural, as the \textit{amphecks}.

\section{Bibliography}

\begin{itemize}
\item
Clark, Glenn (1997), ``New Light on Peirce's Iconic Notation for the Sixteen Binary Connectives'', pp. 304--333 in Houser, Roberts, Van Evra (eds.), \textit{Studies in the Logic of Charles Sanders Peirce}, Indiana University Press, Bloomington, IN.
\item
Houser, Nathan; Roberts, Don D.; and Van Evra, James (eds., 1997), \textit{Studies in the Logic of Charles Sanders Peirce}, Indiana University Press, Bloomington, IN.
\item
McCulloch, Warren Sturgis (1961), ``What Is a Number, that a Man May Know It, and a Man, that He May Know a Number?'' (Ninth Alfred Korzybski Memorial Lecture), \textit{General Semantics Bulletin}, Nos. 26 \& 27, 7--18, Institute of General Semantics, Lakeville, CT.  Reprinted, pp. 1--18 in \textit{Embodiments of Mind}.
\item
McCulloch, Warren Sturgis (1965), \textit{Embodiments of Mind}, MIT Press, Cambridge, MA.
\item
Peirce, Charles Sanders, \textit{Collected Papers of Charles Sanders Peirce}, vols. 1--6, Charles Hartshorne and Paul Weiss (eds.), vols. 7--8, Arthur W. Burks (ed.), Harvard University Press, Cambridge, MA, 1931--1935, 1958.  (Cited as CP volume.paragraph).
\item
Peirce, Charles Sanders (1902), ``The Simplest Mathematics''.  First published as CP 4.227--323 in \textit{Collected Papers}.
\item
Zellweger, Shea (1997), ``Untapped Potential in Peirce's Iconic Notation for the Sixteen Binary Connectives'', pp. 334--386 in Houser, Roberts, Van Evra (eds.), \textit{Studies in the Logic of Charles Sanders Peirce}, Indiana University Press, Bloomington, IN, 
1997.
\end{itemize}

%%%%%
%%%%%
\end{document}
