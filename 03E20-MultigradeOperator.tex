\documentclass[12pt]{article}
\usepackage{pmmeta}
\pmcanonicalname{MultigradeOperator}
\pmcreated{2013-03-22 17:48:11}
\pmmodified{2013-03-22 17:48:11}
\pmowner{Jon Awbrey}{15246}
\pmmodifier{Jon Awbrey}{15246}
\pmtitle{multigrade operator}
\pmrecord{4}{40265}
\pmprivacy{1}
\pmauthor{Jon Awbrey}{15246}
\pmtype{Definition}
\pmcomment{trigger rebuild}
\pmclassification{msc}{03E20}
\pmclassification{msc}{03C05}
\pmclassification{msc}{08A40}
\pmclassification{msc}{08A70}
\pmrelated{ParametricOperator}

\endmetadata

% this is the default PlanetMath preamble.  as your knowledge
% of TeX increases, you will probably want to edit this, but
% it should be fine as is for beginners.

% almost certainly you want these
\usepackage{amssymb}
\usepackage{amsmath}
\usepackage{amsfonts}

% used for TeXing text within eps files
%\usepackage{psfrag}
% need this for including graphics (\includegraphics)
%\usepackage{graphicx}
% for neatly defining theorems and propositions
%\usepackage{amsthm}
% making logically defined graphics
%%%\usepackage{xypic}

% there are many more packages, add them here as you need them

% define commands here

\begin{document}
A \textbf{multigrade operator} $\Omega$ is a parametric operator with parameter $k$ in the set $\mathbb{N}$ of non-negative integers.

The application of a multigrade operator $\Omega$ to a finite sequence of operands $(x_1, \ldots, x_k)$ is typically denoted with the parameter $k$ left tacit, as the appropriate application is implicit in the number of operands listed.  Thus $\Omega(x_1, \ldots, x_k)$ may be taken for $\Omega_k(x_1, \ldots, x_k).$

%%%%%
%%%%%
\end{document}
