\documentclass[12pt]{article}
\usepackage{pmmeta}
\pmcanonicalname{FirstIsomorphismTheorem1}
\pmcreated{2013-03-22 13:50:42}
\pmmodified{2013-03-22 13:50:42}
\pmowner{almann}{2526}
\pmmodifier{almann}{2526}
\pmtitle{first isomorphism theorem}
\pmrecord{10}{34582}
\pmprivacy{1}
\pmauthor{almann}{2526}
\pmtype{Theorem}
\pmcomment{trigger rebuild}
\pmclassification{msc}{03C07}

\endmetadata

% this is the default PlanetMath preamble.  as your knowledge
% of TeX increases, you will probably want to edit this, but
% it should be fine as is for beginners.

% almost certainly you want these
\usepackage{amssymb}
\usepackage{amsmath}
\usepackage{amsfonts}

% used for TeXing text within eps files
%\usepackage{psfrag}
% need this for including graphics (\includegraphics)
%\usepackage{graphicx}
% for neatly defining theorems and propositions
\usepackage{amsthm}
% making logically defined graphics
%%%\usepackage{xypic}

% there are many more packages, add them here as you need them

% define commands here
\let	\proves	=	\vdash
\let	\implies	=	\rightarrow
\let	\Implies	=	\Rightarrow
\let	\iff		=	\leftrightarrow
\let	\Iff		=	\Leftrightarrow
\let	\n		=	\overline
\let 	\cl 		=	\overline
\let	\iso		=	\cong
\let	\embeds	=	\hookrightarrow
\let	\normal	=	\lhd
\let	\normaleq		=	\unlhd
\let	\ideal	=	\lhd
\let	\idealeq	=	\unlhd
\let	\divides	=	\mid
\let	\submodel	=	\subset
\let	\submodeleq	=	\subseteq

\newcommand{\eqclass}[1]{[\![#1]\!]}
\newcommand{\set}[1]{\{#1\}}
\newcommand{\setof}[2]{\{\,#1 \mid #2\,\}}
\newcommand{\indexof}[2]{\abs{#1 : #2}}
\newcommand{\tuple}[1]{\langle#1\rangle}
\newcommand{\gen}[1]{\langle\!\langle #1 \rangle\!\rangle}
\newcommand{\genrel}[2]{\langle\!\langle\, #1 \mid #2 \,\rangle\!\rangle}
\newcommand{\ceil}[1]{\lceil#1\rceil}
\newcommand{\floor}[1]{\lfloor#1\rfloor}
\newcommand{\restrict}[1]{\upharpoonright{#1}}
\newcommand{\qapprox}{/\!\!\approx}
\newcommand{\qsim}{/\!\!\sim}

% Math operators
\DeclareMathOperator{\dom}{dom}
\DeclareMathOperator{\range}{range}
\DeclareMathOperator{\im}{im}
\DeclareMathOperator{\sgn}{sgn}
\DeclareMathOperator{\Hom}{Hom}
\DeclareMathOperator{\End}{End}
\DeclareMathOperator{\Aut}{Aut}
\DeclareMathOperator{\Inn}{Inn}
\DeclareMathOperator{\Sub}{Sub}
\DeclareMathOperator{\Con}{Con}
\DeclareMathOperator{\Cong}{Cong}
\DeclareMathOperator{\Syl}{Syl}


%Axiom systems
\newcommand{\PA}{\mathrm{PA}}
\newcommand{\ZFC}{\mathrm{ZFC}}

% Math script
\newcommand{\powerset}{\mathscr P}

% Fraktur
\newcommand{\A}{\mathfrak{A}}
\newcommand{\B}{\mathfrak{B}}
\newcommand{\C}{\mathfrak{C}}
\newcommand{\D}{\mathfrak{D}}
\newcommand{\X}{\mathfrak{X}}

% Mathbb
\newcommand{\CC}{\mathbb{C}}
\newcommand{\FF}{\mathbb{F}}
\newcommand{\NN}{\mathbb{N}}
\newcommand{\PP}{\mathbb{P}}
\newcommand{\QQ}{\mathbb{Q}}
\newcommand{\RR}{\mathbb{R}}
\newcommand{\ZZ}{\mathbb{Z}}

% Typography
\newcommand{\AM}{\textsc{a.m.}}
\newcommand{\PM}{\textsc{p.m.}}
\newcommand{\latin}[1]{\textit{#1}} 
\newcommand{\book}[1]{\textit{#1}} 
\begin{document}
Let $\Sigma$ be a fixed signature, and $\A$ and $\B$ structures for $\Sigma$. If $f \colon \A \to \B$ is a homomorphism, then there is a unique bimorphism $\phi\colon\A/\!\ker(f) \to \im(f)$ such that for all \(a \in \A\), \(\phi(\eqclass{a}) = f(a)\). Furthermore, if $f$ has the additional property that for each \(n \in \NN\) and each $n$-ary relation symbol $R$  of $\Sigma$, 
\[
R^\B(f(a_1), \ldots, f(a_n)) \Implies \exists a_i'[ f(a_i) = f(a_i') \land R^\A(a_1', \ldots, a_n')],
\]
then $\phi$ is an isomorphism.

\begin{proof}
Since the homomorphic image of a $\Sigma$-structure is also a $\Sigma$-structure, we may assume that \(\im(f) = \B\).

Let \(\sim\ = \ker(f)\). Define a bimorphism $\phi \colon \A\qsim\to \B : \eqclass{a} \mapsto f(a)$. To verify that $\phi$ is well defined, let \(a \sim a'\). Then \(\phi(\eqclass{a}) = f(a) = f(a') = \phi(\eqclass{a'})\). To show that $\phi$ is injective, suppose \(\phi(\eqclass{a}) = \phi(\eqclass{a'})\). Then \(f(a) = f(a')\), so \(a \sim a'\). Hence \(\eqclass{a} = \eqclass{a'}\). To show that $\phi$ is a homomorphism, observe that for any constant symbol $c$ of $\Sigma$ we have \(\phi(\eqclass{c^\A}) = f(c^\A) = c^\B\). For each \(n \in \NN\) and each $n$-ary function symbol $F$ of $\Sigma$,
\begin{align*}
\phi(F^{\A/\!\sim}(\eqclass{a_1}, \ldots, \eqclass{a_n})) &= \phi(\eqclass{F^\A(a_1, \ldots, a_n)}) \\
&= f(F^\A(a_1, \ldots, a_n)) \\
&= F^\B(f(a_1), \ldots, f(a_n)) \\
&= F^\B(\phi(\eqclass{a_1}, \ldots, \phi(\eqclass{a_n})).
\end{align*}
For each \(n \in \NN\) and each $n$-ary relation symbol $R$ of $\Sigma$,
\begin{align*}
R^{\A/\!\sim}(\eqclass{a_1}, \ldots, \eqclass{a_n}) &\Implies R^\A(a_1, \ldots, a_n) \\
&\Implies R^\B(f(a_1), \ldots, f(a_n)) \\
&\Implies R^\B(\phi(\eqclass{a_1}, \ldots, \phi(\eqclass{a_n})).
\end{align*}
Thus $\phi$ is a bimorphism.

Now suppose $f$ has the additional property mentioned in the statement of the theorem. Then 
\begin{align*}
R^\B(\phi(\eqclass{a_1}), \ldots, \phi(\eqclass{a_n})) 
&\Implies R^\B(f(a_1), \ldots, f(a_n)) \\
&\Implies  \exists a_i'[ a_i \sim a_i' \land R^\A(a_1', \ldots, a_n')] \\
&\Implies R^{\A/\!\sim}(\eqclass{a_1}, \ldots, \eqclass{a_n}) .
\end{align*}
Thus $\phi$ is an isomorphism.
\end{proof}
%%%%%
%%%%%
\end{document}
