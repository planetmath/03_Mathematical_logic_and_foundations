\documentclass[12pt]{article}
\usepackage{pmmeta}
\pmcanonicalname{EmptySet}
\pmcreated{2013-03-22 11:49:55}
\pmmodified{2013-03-22 11:49:55}
\pmowner{djao}{24}
\pmmodifier{djao}{24}
\pmtitle{empty set}
\pmrecord{8}{30382}
\pmprivacy{1}
\pmauthor{djao}{24}
\pmtype{Definition}
\pmcomment{trigger rebuild}
\pmclassification{msc}{03-00}
\pmclassification{msc}{65H05}
\pmclassification{msc}{65H10}
\pmsynonym{null set}{EmptySet}

\usepackage{amssymb}
\usepackage{amsmath}
\usepackage{amsfonts}
\usepackage{graphicx}
%%%%\usepackage{xypic}
\begin{document}
An {\em empty set} is a set $\emptyset$ that contains no elements. The Zermelo-Fraenkel Axioms of set theory imply that there exists an empty set. One constructs an empty set by starting with any set $X$ and then applying the axiom of separation to form the empty set $\emptyset := \{ x \in X \mid x \neq x\}$.

An empty set is a subset of every other set, and any two empty sets are equal. Alternative notations for the empty set include $\{\}$ and $\varnothing$.
%%%%%
%%%%%
%%%%%
%%%%%
\end{document}
