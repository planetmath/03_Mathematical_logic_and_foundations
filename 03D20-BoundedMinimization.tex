\documentclass[12pt]{article}
\usepackage{pmmeta}
\pmcanonicalname{BoundedMinimization}
\pmcreated{2013-03-22 19:05:18}
\pmmodified{2013-03-22 19:05:18}
\pmowner{CWoo}{3771}
\pmmodifier{CWoo}{3771}
\pmtitle{bounded minimization}
\pmrecord{11}{41978}
\pmprivacy{1}
\pmauthor{CWoo}{3771}
\pmtype{Definition}
\pmcomment{trigger rebuild}
\pmclassification{msc}{03D20}
\pmsynonym{bounded sum}{BoundedMinimization}
\pmrelated{UnboundedMinimization}
\pmrelated{RecursiveFunction}
\pmrelated{BoundedMaximization}
\pmdefines{bounded summation}
\pmdefines{bounded product}

\endmetadata

\usepackage{amssymb,amscd}
\usepackage{amsmath}
\usepackage{amsfonts}
\usepackage{mathrsfs}

% used for TeXing text within eps files
%\usepackage{psfrag}
% need this for including graphics (\includegraphics)
%\usepackage{graphicx}
% for neatly defining theorems and propositions
\usepackage{amsthm}
% making logically defined graphics
%%\usepackage{xypic}
\usepackage{pst-plot}

% define commands here
\newcommand*{\abs}[1]{\left\lvert #1\right\rvert}
\newtheorem{prop}{Proposition}
\newtheorem{thm}{Theorem}
\newtheorem{ex}{Example}
\newcommand{\real}{\mathbb{R}}
\newcommand{\pdiff}[2]{\frac{\partial #1}{\partial #2}}
\newcommand{\mpdiff}[3]{\frac{\partial^#1 #2}{\partial #3^#1}}
\begin{document}
One useful way of generating more primitive recursive functions from existing ones is through what is known as \emph{bounded summation} and \emph{bounded product}.  Given a primitive recursive function $f:\mathbb{N}^{m+1} \to \mathbb{N}$, define two functions $f_s,f_p:\mathbb{N}^{m+1} \to \mathbb{N}$ as follows: for $\boldsymbol{x}\in \mathbb{N}^m$ and $y\in \mathbb{N}$:
$$f_s(\boldsymbol{x},y):=\sum_{i=0}^y f(\boldsymbol{x},i)$$
$$f_p(\boldsymbol{x},y):=\prod_{i=0}^y f(\boldsymbol{x},i)$$

These are easily seen to be primitive recursive, because they are defined by primitive recursion.  For example,
$$f_s(\boldsymbol{x},0)=f(\boldsymbol{x},0),\quad \mbox{and}\quad f_s(\boldsymbol{x},n+1)= g(\boldsymbol{x},n,f_s(\boldsymbol{x},n)),$$
where $g(\boldsymbol{x},n,y)=\operatorname{add}(f(\boldsymbol{x},n),y)$, which is primitive recursive by functional composition.

\textbf{Definition}. We call $f_s$ and $f_p$ functions obtained from $f$ by \emph{bounded sum} and \emph{bounded product} respectively.

Using bounded summation and bounded product, another useful class of primitive recursive functions can be generated:

\textbf{Definition}.  Let $f:\mathbb{N}^{m+1}\to \mathbb{N}$ be a function.  For each $y\in \mathbb{N}$, set $$A_f(\boldsymbol{x},y):=\lbrace z\in \mathbb{N}\mid z\le y \mbox{ and }f(\boldsymbol{x},z)=0\rbrace.$$  Define
\begin{displaymath}
f_{bmin}(\boldsymbol{x},y):= \left\{
\begin{array}{ll}
\min A_f(\boldsymbol{x},y) & \textrm{if } A_f(\boldsymbol{x},y) \ne \varnothing, \\
s(y) & \textrm{otherwise.}
\end{array}
\right.
\end{displaymath}
$f_{bmin}$ is called the function obtained from $f$ by \emph{bounded minimization}, and is usually denoted $$\mu z\le y (f(\boldsymbol{x},z)=0).$$

\begin{prop}  If $f:\mathbb{N}^{m+1}\to \mathbb{N}$ is primitive recursive, so is $f_{bmin}$. \end{prop}
\begin{proof}
Define $g:=\operatorname{sgn}\circ f$.  Then 
\begin{displaymath}
g(\boldsymbol{x},y):= \left\{
\begin{array}{ll}
0 & \textrm{if } f(\boldsymbol{x},y)=0, \\
1 & \textrm{otherwise.}
\end{array}
\right.
\end{displaymath}
As $f$ is primitive recursive, so is $g$, since the sign function $\operatorname{sgn}$ is primitive recursive (see \PMlinkname{this entry}{ExamplesOfPrimitiveRecursiveFunctions}).

Next, the function $g_p$ obtained from $g$ by bounded product has the following properties:
\begin{itemize}
\item if $g_p(\boldsymbol{x},y)=1$, then $g_p(\boldsymbol{x},z)=1$ for all $z<y$,
\item if $g_p(\boldsymbol{x},y)=0$, then $g_p(\boldsymbol{x},z)=0$ for all $z\ge y$.
\end{itemize}

Finally, the function $(g_p)_s$ obtained from $g_p$ by bounded sum has the property that, when applied to $(\boldsymbol{x},y)$, counts the number of $z\le y$ such that $g_p(\boldsymbol{x},z)=1$.  Based on the property of $g_p$, this count is then exactly the least $z\le y$ such that $g_p(\boldsymbol{x},z)=1$.  This means that $(g_p)_s=f_{bmin}$ for all $(\boldsymbol{x},y)\in \mathbb{N}^{m+1}$.  Since $g_p$ is primitive recursive, so is $(g_p)_s$, or that $f_{bmin}$ is primitive recursive.
\end{proof}

In fact, if $f$ is a (total) recursive function, so is $f_{bmin}$, because all of the derived functions in the proof above preserve primitive recursiveness as well as totalness.

\textbf{Remarks}.  
\begin{itemize}
\item
In the definition of bounded minimization, if we take the $y$ out, then we arrive at the notion of \emph{unbounded minimization}, or just \emph{minimization}.  The proposition above shows that the set $\mathcal{PR}$ of primitive recursive functions is closed under bounded minimization.  However, $\mathcal{PR}$ is not closed under minimization.  The closure of $\mathcal{PR}$ under minimization is the set $\mathcal{R}$ of recursive functions (total or not).
\item
It is not hard to show that $\mathcal{ER}$, the set of all elementary recursive functions, is closed under bounded minimization.
\end{itemize}
%%%%%
%%%%%
\end{document}
