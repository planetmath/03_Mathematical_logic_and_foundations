\documentclass[12pt]{article}
\usepackage{pmmeta}
\pmcanonicalname{ExistentialTheorem}
\pmcreated{2013-03-22 17:08:57}
\pmmodified{2013-03-22 17:08:57}
\pmowner{Wkbj79}{1863}
\pmmodifier{Wkbj79}{1863}
\pmtitle{existential theorem}
\pmrecord{5}{39458}
\pmprivacy{1}
\pmauthor{Wkbj79}{1863}
\pmtype{Definition}
\pmcomment{trigger rebuild}
\pmclassification{msc}{03F07}
\pmclassification{msc}{00A35}
\pmsynonym{existence theorem}{ExistentialTheorem}
\pmrelated{TechniquesInMathematicalProofs}

\endmetadata

\usepackage{amssymb}
\usepackage{amsmath}
\usepackage{amsfonts}
\usepackage{pstricks}
\usepackage{psfrag}
\usepackage{graphicx}
\usepackage{amsthm}
%%\usepackage{xypic}

\begin{document}
An \emph{existential theorem} is a theorem which \PMlinkescapetext{states that a certain mathematical object or property} exists.

In general, there are two ways to prove an existential theorem.  The most convincing method is a constructive proof, and another common method is an existential proof.  The reason that a constructive proof is most convincing is that, after reading such a proof, readers can actually get their hands on the mathematical \PMlinkescapetext{object or property} in question.  In some cases, however, constructing the mathematical \PMlinkescapetext{object or property} is difficult, if not impossible.  In this case, an existential proof may be the only \PMlinkescapetext{feasible} method for proving an existential theorem.  An example of this is the primitive element theorem.
%%%%%
%%%%%
\end{document}
