\documentclass[12pt]{article}
\usepackage{pmmeta}
\pmcanonicalname{1012Images}
\pmcreated{2013-11-06 17:06:09}
\pmmodified{2013-11-06 17:06:09}
\pmowner{PMBookProject}{1000683}
\pmmodifier{PMBookProject}{1000683}
\pmtitle{10.1.2 Images}
\pmrecord{1}{}
\pmprivacy{1}
\pmauthor{PMBookProject}{1000683}
\pmtype{Feature}
\pmclassification{msc}{03B15}

\usepackage{xspace}
\usepackage{amssyb}
\usepackage{amsmath}
\usepackage{amsfonts}
\usepackage{amsthm}
\makeatletter
\newcommand{\blank}{\mathord{\hspace{1pt}\text{--}\hspace{1pt}}}
\newcommand{\bproj}[1]{\tproj{}{#1}}
\newcommand{\brck}[1]{\trunc{}{#1}}
\newcommand{\ct}{  \mathchoice{\mathbin{\raisebox{0.5ex}{$\displaystyle\centerdot$}}}             {\mathbin{\raisebox{0.5ex}{$\centerdot$}}}             {\mathbin{\raisebox{0.25ex}{$\scriptstyle\,\centerdot\,$}}}             {\mathbin{\raisebox{0.1ex}{$\scriptscriptstyle\,\centerdot\,$}}}}
\newcommand{\defeq}{\vcentcolon\equiv}  
\newcommand{\define}[1]{\textbf{#1}}
\def\@dprd#1{\prod_{(#1)}\,}
\def\@dprd@noparens#1{\prod_{#1}\,}
\def\@dsm#1{\sum_{(#1)}\,}
\def\@dsm@noparens#1{\sum_{#1}\,}
\def\@eatprd\prd{\prd@parens}
\def\@eatsm\sm{\sm@parens}
\newcommand{\eqvsym}{\simeq}    
\def\exis#1{\exists (#1)\@ifnextchar\bgroup{.\,\exis}{.\,}}
\def\fall#1{\forall (#1)\@ifnextchar\bgroup{.\,\fall}{.\,}}
\newcommand{\happly}{\mathsf{happly}}
\newcommand{\hfib}[2]{{\mathsf{fib}}_{#1}(#2)}
\newcommand{\htpy}{\sim}
\newcommand{\im}{\ensuremath{\mathsf{im}}} 
\newcommand{\indexdef}[1]{\index{#1|defstyle}}   
\newcommand{\indexsee}[2]{\index{#1|see{#2}}}    
\newcommand{\jdeq}{\equiv}      
\def\lam#1{{\lambda}\@lamarg#1:\@endlamarg\@ifnextchar\bgroup{.\,\lam}{.\,}}
\def\@lamarg#1:#2\@endlamarg{\if\relax\detokenize{#2}\relax #1\else\@lamvar{\@lameatcolon#2},#1\@endlamvar\fi}
\def\@lameatcolon#1:{#1}
\def\@lamvar#1,#2\@endlamvar{(#2\,{:}\,#1)}
\newcommand{\mapfunc}[1]{\ensuremath{\mathsf{ap}_{#1}}\xspace} 
\newcommand{\narrowequation}[1]{$#1$}
\newcommand{\opp}[1]{\mathord{{#1}^{-1}}}
\newcommand{\pairr}[1]{{\mathopen{}(#1)\mathclose{}}}
\newcommand{\Pairr}[1]{{\mathopen{}\left(#1\right)\mathclose{}}}
\newcommand{\Parens}[1]{\Bigl(#1\Bigr)}
\newcommand{\power}[1]{\mathcal{P}(#1)} 
\def\prd#1{\@ifnextchar\bgroup{\prd@parens{#1}}{\@ifnextchar\sm{\prd@parens{#1}\@eatsm}{\prd@noparens{#1}}}}
\def\prd@noparens#1{\mathchoice{\@dprd@noparens{#1}}{\@tprd{#1}}{\@tprd{#1}}{\@tprd{#1}}}
\def\prd@parens#1{\@ifnextchar\bgroup  {\mathchoice{\@dprd{#1}}{\@tprd{#1}}{\@tprd{#1}}{\@tprd{#1}}\prd@parens}  {\@ifnextchar\sm    {\mathchoice{\@dprd{#1}}{\@tprd{#1}}{\@tprd{#1}}{\@tprd{#1}}\@eatsm}    {\mathchoice{\@dprd{#1}}{\@tprd{#1}}{\@tprd{#1}}{\@tprd{#1}}}}}
\newcommand{\proj}[1]{\ensuremath{\mathsf{pr}_{#1}}\xspace}
\newcommand{\projpath}[1]{\ensuremath{\apfunc{\proj{#1}}}\xspace}
\newcommand{\prop}{\ensuremath{\mathsf{Prop}}\xspace}
\newcommand{\refl}[1]{\ensuremath{\mathsf{refl}_{#1}}\xspace}
\def\sm#1{\@ifnextchar\bgroup{\sm@parens{#1}}{\@ifnextchar\prd{\sm@parens{#1}\@eatprd}{\sm@noparens{#1}}}}
\def\sm@noparens#1{\mathchoice{\@dsm@noparens{#1}}{\@tsm{#1}}{\@tsm{#1}}{\@tsm{#1}}}
\def\sm@parens#1{\@ifnextchar\bgroup  {\mathchoice{\@dsm{#1}}{\@tsm{#1}}{\@tsm{#1}}{\@tsm{#1}}\sm@parens}  {\@ifnextchar\prd    {\mathchoice{\@dsm{#1}}{\@tsm{#1}}{\@tsm{#1}}{\@tsm{#1}}\@eatprd}    {\mathchoice{\@dsm{#1}}{\@tsm{#1}}{\@tsm{#1}}{\@tsm{#1}}}}}
\newcommand{\symlabel}[1]{\refstepcounter{symindex}\label{#1}}
\def\@tprd#1{\mathchoice{{\textstyle\prod_{(#1)}}}{\prod_{(#1)}}{\prod_{(#1)}}{\prod_{(#1)}}}
\newcommand{\tproj}[3][]{\mathopen{}\left|#3\right|_{#2}^{#1}\mathclose{}}
\newcommand{\trans}[2]{\ensuremath{{#1}_{*}\mathopen{}\left({#2}\right)\mathclose{}}\xspace}
\newcommand{\trunc}[2]{\mathopen{}\left\Vert #2\right\Vert_{#1}\mathclose{}}
\def\@tsm#1{\mathchoice{{\textstyle\sum_{(#1)}}}{\sum_{(#1)}}{\sum_{(#1)}}{\sum_{(#1)}}}
\newcommand{\unit}{\ensuremath{\mathbf{1}}\xspace}
\newcommand{\uset}{\ensuremath{\mathcal{S}et}\xspace}
\newcommand{\UU}{\ensuremath{\mathcal{U}}\xspace}
\newcommand{\vcentcolon}{:\!\!}
\newcounter{mathcount}
\setcounter{mathcount}{1}
\newtheorem{prelem}{Lemma}
\newenvironment{lem}{\begin{prelem}}{\end{prelem}\addtocounter{mathcount}{1}}
\renewcommand{\theprelem}{10.1.\arabic{mathcount}}
\newtheorem{prethm}{Theorem}
\newenvironment{thm}{\begin{prethm}}{\end{prethm}\addtocounter{mathcount}{1}}
\renewcommand{\theprethm}{10.1.\arabic{mathcount}}
\let\apfunc\mapfunc
\let\autoref\cref
\let\hfiber\hfib
\let\setof\Set    
\let\type\UU
\makeatother

\begin{document}

%We will show that $\uset$ is a $\Pi$W-pretopos. 
Next, we show that $\uset$ is a \define{regular category}, i.e.:
\indexdef{category!regular}%
\indexdef{regular!category}%
%
\begin{enumerate}
\item $\uset$ is finitely complete.\label{item:reg1}
\item The kernel pair $\proj1,\proj2: (\sm{x,y:A} f(x)= f(y)) \to A$ of any
  function $f : A \to B$ has a coequalizer.\label{item:reg2}
  \indexdef{kernel!pair}
\item Pullbacks of regular epimorphisms are again regular epimorphisms.\label{item:reg3}
\end{enumerate}
%
Recall that a \define{regular epimorphism}
\indexdef{epimorphism!regular}%
\indexdef{regular!epimorphism}%
is a morphism that is the coequalizer of \emph{some} pair of maps.
Thus in~\ref{item:reg3} the pullback of a coequalizer is required to again be a coequalizer, but not necessarily of the pulled-back pair.

\index{set-coequalizer}%
\index{image}
The obvious candidate for the coequalizer of the kernel pair of $f:A\to B$ is the \emph{image} of $f$, as defined in \autoref{sec:image-factorization}.
Recall that we defined $\im(f)\defeq \sm{b:B} \brck{\hfib f b}$, with functions 
$\tilde{f}:A\to\im(f)$ and $i_f:\im(f)\to B$ defined by
\begin{align*}
  \tilde{f} & \defeq \lam{a} \Pairr{f(a),\,\bproj{\pairr{a,\refl{f(a)}}}}\\
i_f & \defeq \proj1
\end{align*}
fitting into a diagram:
\begin{equation*}
  \xymatrix{
    **[l]{\sm{x,y:A} f(x)= f(y)}
    \ar@<0.25em>[r]^{\proj1}
    \ar@<-0.25em>[r]_{\proj2}
    &
    {A}
    \ar[r]^(0.4){\tilde{f}}
    \ar[rd]_{f}
    &
    {\im(f)}
    \ar@{..>}[d]^{i_f}
    \\ & &
    B
  }
\end{equation*}

Recall that a function $f:A\to B$ is called \emph{surjective} if
\index{function!surjective}%
\narrowequation{\fall{b:B}\brck{\hfib f b},}
or equivalently $\fall{b:B} \exis{a:A} f(a)=b$.
We have also said that a function $f:A\to B$ between sets is called \emph{injective} if
\index{function!injective}%
$\fall{a,a':A} (f(a) = f(a')) \Rightarrow (a=a')$, or equivalently if each of its fibers is a mere proposition.
Since these are the $(-1)$-connected and $(-1)$-truncated maps in the sense of \autoref{cha:hlevels}, the general theory there implies that $\tilde f$ above is surjective and $i_f$ is injective, and that this factorization is stable under pullback.

We now identify surjectivity and injectivity with the appropriate cat\-e\-go\-ry-theoretic notions.
First we observe that categorical monomorphisms and epimorphisms have a slightly stronger equivalent formulation.

\begin{lem}\label{thm:mono}
  For a morphism $f:\hom_A(a,b)$ in a category $A$, the following are equivalent.
  \begin{enumerate}
  \item $f$ is a \define{monomorphism}:
    \indexdef{monomorphism}%
    for all $x:A$ and ${g,h:\hom_A(x,a)}$, if $f\circ g = f\circ h$ then $g=h$.\label{item:mono1}
  \item (If $A$ has pullbacks) the diagonal map $a\to a\times_b a$ is an isomorphism.\label{item:mono4}
  \item For all $x:A$ and $k:\hom_A(x,b)$, the type $\sm{h:\hom_A(x,a)} (k = f\circ h)$ is a mere proposition.\label{item:mono2}
  \item For all $x:A$ and ${g:\hom_A(x,a)}$, the type $\sm{h:\hom_A(x,a)} (f\circ g = f\circ h)$ is contractible.\label{item:mono3}
  \end{enumerate}
\end{lem}
\begin{proof}
  The equivalence of conditions~\ref{item:mono1} and~\ref{item:mono4} is standard category theory.
  Now consider the function $(f\circ \blank ):\hom_A(x,a) \to \hom_A(x,b)$ between sets.
  Condition~\ref{item:mono1} says that it is injective, while~\ref{item:mono2} says that its fibers are mere propositions; hence they are equivalent.
  And~\ref{item:mono2} implies~\ref{item:mono3} by taking $k\defeq f\circ g$ and recalling that an inhabited mere proposition is contractible.
  Finally,~\ref{item:mono3} implies~\ref{item:mono1} since if $p:f\circ g= f\circ h$, then $(g,\refl{})$ and $(h,p)$ both inhabit the type in~\ref{item:mono3}, hence are equal and so $g=h$.
\end{proof}

\begin{lem}
  A function $f:A\to B$ between sets is injective if and only if it is a monomorphism\index{monomorphism} in \uset.
\end{lem}
\begin{proof}
  Left to the reader.
\end{proof}

Of course, an \define{epimorphism}
\indexdef{epimorphism}
\indexsee{epi}{epimorphism}
is a monomorphism in the opposite category.
We now show that in \uset, the epimorphisms are precisely the surjections, and also precisely the coequalizers (regular epimorphisms).

The coequalizer of a pair of maps $f,g:A\to B$ in $\uset$ is defined as the 0-truncation of a general (homotopy) coequalizer.
For clarity, we may call this the \define{set-coequalizer}.
\indexdef{set-coequalizer}%
\indexsee{coequalizer!of sets}{set-coequalizer}%
It is convenient to express its universal property as follows.

\begin{lem}
\index{universal!property!of set-coequalizer}%
Let $f,g:A\to B$ be functions between sets $A$ and $B$. The 
{set-co}equalizer $c_{f,g}:B\to Q$ has the property that, for any set $C$ and any $h:B\to C$ with $h\circ f = h\circ g$, the type
\begin{equation*}
\sm{k:Q\to C} (k\circ c_{f,g} = h)
\end{equation*}
is contractible.
\end{lem}

\begin{lem}\label{epis-surj}
For any function $f:A\to B$ between sets, the following are equivalent:
\begin{enumerate}
\item $f$ is an epimorphism.
\item Consider the pushout diagram
\begin{equation*}
  \xymatrix{
    {A}
    \ar[r]^{f}
    \ar[d]
    &
    {B}
    \ar[d]^{\iota}
    \\
    {\unit}
    \ar[r]_{t}
    &
    {C_f}
  }
\end{equation*}
in $\uset$ defining the mapping cone\index{cone!of a function}. Then the type $C_f$ is contractible.
\item $f$ is surjective.
\end{enumerate}
\end{lem}

\begin{proof}
Let $f:A\to B$ be a function between sets, and suppose it to be an epimorphism; we show $C_f$ is contractible.
The constructor $\unit\to C_f$ of $C_f$ gives us an element $t:C_f$.
We have to show that
\begin{equation*}
\prd{x:C_f} x= t.
\end{equation*}
Note that $x= t$ is a mere proposition, hence we can use induction on $C_f$.
Of course when $x$ is $t$ we have $\refl{t}:t=t$, so it suffices to find
\begin{align*}
I_0 & : \prd{b:B} \iota(b)= t\\
I_1 & : \prd{a:A} \opp{\alpha_1(a)} \ct I_0(f(a))=\refl{t}.
\end{align*}
where $\iota:B\to C_f$ and $\alpha_1:\prd{a:A} \iota(f(a))= t$ are the other constructors
of $C_f$. Note that $\alpha_1$ is a homotopy from $\iota\circ f$ to
$\mathsf{const}_t\circ f$, so we find the elements
\begin{equation*}
\pairr{\iota,\refl{\iota\circ f}},\pairr{\mathsf{const}_t,\alpha_1}:
\sm{h:B\to C_f} \iota\circ f \htpy h\circ f.
\end{equation*}
By the dual of \autoref{thm:mono}\ref{item:mono3} (and function extensionality), there is a path
\begin{equation*}
\gamma:\pairr{\iota,\refl{\iota\circ f}}=\pairr{\mathsf{const}_t,\alpha_1}.
\end{equation*}
Hence, we may define $I_0(b)\defeq \happly(\projpath1(\gamma),b):\iota(b)=t$.
We also have
\[\projpath2(\gamma) : \trans{\projpath1(\gamma)}{\refl{\iota\circ f}} = \alpha_1. \]
This transport involves precomposition with $f$, which commutes with $\happly$.
Thus, from transport in path types we obtain $I_0(f(a)) = \alpha_1(a)$ for any $a:A$, which gives us $I_1$.

Now suppose $C_f$ is contractible; we show $f$ is surjective.
We first construct a type family $P:C_f\to\prop$ by recursion on $C_f$, which is valid since \prop is a set.
On the point constructors, we define
\begin{align*}
P(t) & \defeq \unit\\
P(\iota(b)) & \defeq \brck{\hfiber{f}b}.
\end{align*}
To complete the construction of $P$, it remains to give a path
\narrowequation{\brck{\hfiber{f}{f(a)}} =_\prop \unit}
for all $a:A$.
However, $\brck{\hfiber{f}{f(a)}}$ is inhabited by $(f(a),\refl{f(a)})$.
Since it is a mere proposition, this means it is contractible --- and thus equivalent, hence equal, to \unit.
This completes the definition of $P$.
Now, since $C_f$ is assumed to be contractible, it follows that $P(x)$ is equivalent to $P(t)$ for any $x:C_f$.
In particular, $P(\iota(b))\jdeq \brck{\hfiber{f}b}$ is equivalent to $P(t)\jdeq \unit$ for each $b:B$, and hence contractible.
Thus, $f$ is surjective.

Finally, suppose $f:A\to B$ to be surjective, and consider a set $C$ and two functions
$g,h:B\to C$ with the property that $g\circ f = h\circ f$. Since $f$ 
is assumed to be surjective, for all $b:B$ the type $\brck{\hfib f b}$ is contractible.
Thus we have the following equivalences:
\begin{align*}
\prd{b:B} (g(b)= h(b))
& \eqvsym \prd{b:B} \Parens{\brck{\hfib f b} \to (g(b)= h(b))}\\
& \eqvsym \prd{b:B} \Parens{\hfib f b \to (g(b)= h(b))}\\
& \eqvsym \prd{b:B}{a:A}{p:f(a)= b} g(b)= h(b)\\
& \eqvsym \prd{a:A} g(f(a))= h(f(a))
\end{align*}
using on the second line the fact that $g(b)=h(b)$ is a mere proposition, since $C$ is a set.
But by assumption, there is an element of the latter type.
\end{proof}

% \begin{rem}
% The above theorem is not true when we replace $\set$ by $\type$
% (replacing it also in the definition of $\mathsf{epi}$ and $\mathsf{epi}'$). 
% However, we do
% get the implications $\textit{ii.}\Rightarrow\textit{iii.}\Rightarrow
% \textit{iv.}$
% \end{rem}

\begin{thm}\label{thm:set_regular}\label{lem:images_are_coequalizers}
The category $\uset$ is regular. Moreover, surjective functions between sets are regular epimorphisms.
\end{thm}

\begin{proof}
It is a standard lemma in category theory that a category is regular as soon as it admits finite limits and a pullback-stable orthogonal 
factorization system\index{orthogonal factorization system} $(\mathcal{E},\mathcal{M})$ with $\mathcal{M}$ the monomorphisms, in which case $\mathcal{E}$ consists automatically of 
the regular epimorphisms.
(See e.g.~\cite[A1.3.4]{elephant}.)
The existence of the factorization system was proved in \autoref{thm:orth-fact}.
\end{proof}

\begin{lem}\label{lem:pb_of_coeq_is_coeq}
Pullbacks of regular epis in \uset are regular epis.
\end{lem}
\begin{proof}
  We showed in \autoref{thm:stable-images} that pullbacks of $n$-connected functions are $n$-connected.
  By \autoref{lem:images_are_coequalizers}, it suffices to apply this when $n=-1$.
\end{proof}

\indexdef{image!of a subset}
One of the consequences of \uset being a regular category is that we have an ``image'' operation on subsets.
That is, given $f:A\to B$, any subset $P:\power A$ (i.e.\ a predicate $P:A\to \prop$) has an \define{image} which is a subset of $B$.
This can be defined directly as $\setof{ y:B | \exis{x:A} f(x)=y \land P(x)}$, or indirectly as the image (in the previous sense) of the composite function
\[ \setof{ x:A | P(x) } \to A \xrightarrow{f} B.\]
\symlabel{subset-image}
We will also sometimes use the common notation $\setof{f(x) | P(x)}$ for the image of $P$.



\end{document}
