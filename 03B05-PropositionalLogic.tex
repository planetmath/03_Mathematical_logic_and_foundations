\documentclass[12pt]{article}
\usepackage{pmmeta}
\pmcanonicalname{PropositionalLogic}
\pmcreated{2013-03-22 13:04:01}
\pmmodified{2013-03-22 13:04:01}
\pmowner{Henry}{455}
\pmmodifier{Henry}{455}
\pmtitle{propositional logic}
\pmrecord{6}{33475}
\pmprivacy{1}
\pmauthor{Henry}{455}
\pmtype{Definition}
\pmcomment{trigger rebuild}
\pmclassification{msc}{03B05}
\pmrelated{Implication}
\pmrelated{Biconditional}
\pmrelated{Conjunction}
\pmrelated{Disjunction}
\pmrelated{PropositionalCalculus}
\pmrelated{ExclusiveOr}
\pmrelated{InterpretationOfPropositions}
\pmdefines{proposition}

\endmetadata

% this is the default PlanetMath preamble.  as your knowledge
% of TeX increases, you will probably want to edit this, but
% it should be fine as is for beginners.

% almost certainly you want these
\usepackage{amssymb}
\usepackage{amsmath}
\usepackage{amsfonts}

% used for TeXing text within eps files
%\usepackage{psfrag}
% need this for including graphics (\includegraphics)
%\usepackage{graphicx}
% for neatly defining theorems and propositions
%\usepackage{amsthm}
% making logically defined graphics
%%%\usepackage{xypic}

% there are many more packages, add them here as you need them

% define commands here
%\PMlinkescapeword{theory}
\begin{document}
A \emph{propositional logic} is a logic in which the only objects are \emph{propositions}, that is, objects which themselves have truth values.  Variables represent propositions, and there are no relations, functions, or quantifiers except for the constants $T$ and $\bot$ (representing true and false respectively).  The connectives are typically $\neg$, $\wedge$, $\vee$, and $\rightarrow$ (representing negation, conjunction, disjunction, and implication), however this set is redundant, and other choices can be used ($T$ and $\bot$ can also be considered $0$-ary connectives).

A model for propositional logic is just a truth function $\nu$ on a set of variables.  Such a truth function can be easily extended to a truth function $\overline{\nu}$ on all formulas which contain only the variables $\nu$ is defined on by adding recursive clauses for the usual definitions of connectives.  For instance $\overline{\nu}(\alpha\wedge\beta)=T$ iff $\overline{\nu}(\alpha)=\overline{\nu}(\beta)=T$.

Then we say $\nu\models\phi$ if $\overline{\nu}(\phi)=T$, and we say $\models\phi$ if for every $\nu$ such that $\overline{\nu}(\phi)$ is defined, $\nu\models\phi$ (and say that $\phi$ is a tautology).

Propositional logic is decidable: there is an easy way to determine whether a sentence is a tautology.  It can be done using truth tables, since a truth table for a particular formula can be easily produced, and the formula is a tautology if every assignment of truth values makes it true.  It is not known whether this method is efficient: the equivalent problem of whether a formula is satisfiable (that is, whether its negation is a tautology) is a canonical example of an $\mathcal{NP}$-complete problem.
%%%%%
%%%%%
\end{document}
