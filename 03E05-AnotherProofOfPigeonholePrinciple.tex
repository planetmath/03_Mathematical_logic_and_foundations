\documentclass[12pt]{article}
\usepackage{pmmeta}
\pmcanonicalname{AnotherProofOfPigeonholePrinciple}
\pmcreated{2013-03-22 16:02:12}
\pmmodified{2013-03-22 16:02:12}
\pmowner{ratboy}{4018}
\pmmodifier{ratboy}{4018}
\pmtitle{another proof of pigeonhole principle}
\pmrecord{5}{38086}
\pmprivacy{1}
\pmauthor{ratboy}{4018}
\pmtype{Proof}
\pmcomment{trigger rebuild}
\pmclassification{msc}{03E05}

% this is the default PlanetMath preamble.  as your knowledge
% of TeX increases, you will probably want to edit this, but
% it should be fine as is for beginners.

% almost certainly you want these
\usepackage{amssymb}
\usepackage{amsmath}
\usepackage{amsfonts}

% used for TeXing text within eps files
%\usepackage{psfrag}
% need this for including graphics (\includegraphics)
%\usepackage{graphicx}
% for neatly defining theorems and propositions
%\usepackage{amsthm}
% making logically defined graphics
%%%\usepackage{xypic}

% there are many more packages, add them here as you need them

% define commands here

\begin{document}
By induction on $n$. It is harmless to let $n$ = $m + 1$, since
$0$ lacks proper subsets. Suppose that $f : n \to n$ is injective.

To begin, note that $m \in f[n]$. Otherwise, $f[m] \subseteq m$,
so that by the induction hypothesis, $f[m] = m$. Then $f[n] =
f[m]$, since $f[n] \subseteq m$. Therefore, for some $k < m$,
$f(k) = f(m)$.

Let $g : f[n] \to f[n]$ transpose $m$ and $f(m)$. Then $h \vert_m
: m \to m$ is injective, where $h = g \circ f$. By the induction
hypothesis, $h \vert_m[m] = m$. Therefore:
\begin{align*}
f[n] &= g \circ h[n]
\\
&= h[n]
\\
&= m \cup \{m\}
\\
&= m + 1
\\
&= n \text{.}
\end{align*}

%%%%%
%%%%%
\end{document}
