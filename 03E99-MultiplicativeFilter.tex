\documentclass[12pt]{article}
\usepackage{pmmeta}
\pmcanonicalname{MultiplicativeFilter}
\pmcreated{2013-03-22 16:48:22}
\pmmodified{2013-03-22 16:48:22}
\pmowner{jocaps}{12118}
\pmmodifier{jocaps}{12118}
\pmtitle{multiplicative filter}
\pmrecord{6}{39040}
\pmprivacy{1}
\pmauthor{jocaps}{12118}
\pmtype{Example}
\pmcomment{trigger rebuild}
\pmclassification{msc}{03E99}
\pmclassification{msc}{54A99}
\pmdefines{Gabriel Filter}
\pmdefines{Multiplicative Filter}

\endmetadata

% this is the default PlanetMath preamble.  as your knowledge
% of TeX increases, you will probably want to edit this, but
% it should be fine as is for beginners.

% almost certainly you want these
\usepackage{amssymb}
\usepackage{amsmath}
\usepackage{amsfonts}

% used for TeXing text within eps files
%\usepackage{psfrag}
% need this for including graphics (\includegraphics)
%\usepackage{graphicx}
% for neatly defining theorems and propositions
%\usepackage{amsthm}
% making logically defined graphics
%%%\usepackage{xypic}

% there are many more packages, add them here as you need them

% define commands here
\newcommand\sep{\,:\,}
\begin{document}
For any ring $A$, any set $S\subset A$ and any element $x\in A$,
we use the notation 
$$(S:x):=\{ a\in A\ ax\in S\}$$

Let $A$ be a commutative ring with unity, and let $\mathcal{I}(A)$
be the set of all ideals of $A$.
\begin{itemize}
\item A \emph{Multiplicative Filter} of $A$ is a filter $\mathcal{F}$
on $\mathcal{I}(A)$ such that $I,J\in\mathcal{F}\Rightarrow IJ\in\mathcal{F}$. 
\item A \emph{Gabriel Filter} of $A$ is a filter $\mathcal{F}$ on $\mathcal{I}(A)$
such that

$$ [I\in\mathcal{F},J\in\mathcal{I}(A)\textrm{ and }\forall x\in I,(J:x)\in\mathcal{F}]\Rightarrow J\in\mathcal{F} $$
\end{itemize}

Note that Gabriel Filters are also Multiplicative Filters.

%%%%%
%%%%%
\end{document}
