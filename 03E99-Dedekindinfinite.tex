\documentclass[12pt]{article}
\usepackage{pmmeta}
\pmcanonicalname{Dedekindinfinite}
\pmcreated{2013-03-22 12:05:25}
\pmmodified{2013-03-22 12:05:25}
\pmowner{yark}{2760}
\pmmodifier{yark}{2760}
\pmtitle{Dedekind-infinite}
\pmrecord{11}{31182}
\pmprivacy{1}
\pmauthor{yark}{2760}
\pmtype{Definition}
\pmcomment{trigger rebuild}
\pmclassification{msc}{03E99}
\pmsynonym{Dedekind infinite}{Dedekindinfinite}
\pmrelated{Cardinality}
\pmdefines{Dedekind-finite}
\pmdefines{Dedekind finite}

\endmetadata

\usepackage{amssymb}
\usepackage{amsmath}
\usepackage{amsfonts}
\begin{document}
\PMlinkescapeword{clearly}
\PMlinkescapeword{even}

A set $A$ is said to be \emph{Dedekind-infinite}
if there is an injective function $f\colon\omega\to A$,
where $\omega$ denotes the set of natural numbers.
A set that is not Dedekind-infinite is said to be \emph{Dedekind-finite}.

A Dedekind-infinite set is clearly infinite,
and in ZFC it can be shown that
a set is Dedekind-infinite if and only if it is infinite.

It is consistent with ZF that
there is an infinite set that is not Dedekind-infinite.
However, the existence of such a set requires the failure
not just of the full Axiom of Choice, but even of the Axiom of Countable Choice.
%%%%%
%%%%%
%%%%%
\end{document}
