\documentclass[12pt]{article}
\usepackage{pmmeta}
\pmcanonicalname{PolyadicAlgebraWithEquality}
\pmcreated{2013-03-22 17:51:37}
\pmmodified{2013-03-22 17:51:37}
\pmowner{CWoo}{3771}
\pmmodifier{CWoo}{3771}
\pmtitle{polyadic algebra with equality}
\pmrecord{10}{40337}
\pmprivacy{1}
\pmauthor{CWoo}{3771}
\pmtype{Definition}
\pmcomment{trigger rebuild}
\pmclassification{msc}{03G15}
\pmsynonym{equality algebra}{PolyadicAlgebraWithEquality}
\pmrelated{CylindricAlgebra}
\pmdefines{equality predicate}
\pmdefines{substitutive}
\pmdefines{reflexive}
\pmdefines{symmetric}
\pmdefines{transitive}

\endmetadata

\usepackage{amssymb,amscd}
\usepackage{amsmath}
\usepackage{amsfonts}
\usepackage{mathrsfs}

% used for TeXing text within eps files
%\usepackage{psfrag}
% need this for including graphics (\includegraphics)
%\usepackage{graphicx}
% for neatly defining theorems and propositions
\usepackage{amsthm}
% making logically defined graphics
%%\usepackage{xypic}
\usepackage{pst-plot}

% define commands here
\newcommand*{\abs}[1]{\left\lvert #1\right\rvert}
\newtheorem{prop}{Proposition}
\newtheorem{thm}{Theorem}
\newtheorem{ex}{Example}
\newcommand{\real}{\mathbb{R}}
\newcommand{\pdiff}[2]{\frac{\partial #1}{\partial #2}}
\newcommand{\mpdiff}[3]{\frac{\partial^#1 #2}{\partial #3^#1}}
\begin{document}
Let $A=(B,V,\exists,S)$ be a polyadic algebra.  An \emph{equality predicate} on $A$ is a function $E:V\times V\to B$ such that
\begin{enumerate}
\item $S(f)\circ E(x,y) = E(f(x),f(y))$ for any $f:V\to V$ and any $x,y\in V$
\item $E(x,x)=1$ for every $x\in V$, and
\item $E(x,y)\wedge a\le S(x/y)a$, where $a\in B$, and $(x/y)$ denotes the function $V\to V$ that maps $x$ to $y$, and constant everywhere else.
\end{enumerate}
Heuristically, we can interpret the conditions above as follows: 
\begin{enumerate}
\item if $x=y$ and if we replace $x$ by, say $x_1$, and $y$ by $y_1$, then $x_1=y_1$.  
\item $x=x$ for every variable $x$
\item if we have a propositional function $a$ that is true, and $x=y$, then the proposition obtained from $a$ by replacing all occurrences of $x$ by $y$ is also true.
\end{enumerate}
The second condition is also known as the \emph{reflexive property} of the equality predicate $E$, and the third is known as the \emph{substitutive property} of $E$

A \emph{polyadic algebra with equality} is a pair $(A,E)$ where $A$ is a polyadic algebra and $E$ is an equality predicate on $A$.  Paul Halmos introduced this concept and called this simply an \emph{equality algebra}.

Below are some basic properties of the equality predicate $E$ in an equality algebra $(A,E)$:
\begin{itemize}
\item (symmetric property) $E(x,y)\le E(y,x)$
\item (transitive property) $E(x,y)\wedge E(y,z)\le E(x,z)$
\item $E(x,y) \wedge a = E(x,y) \wedge S(x,y)a $, where $(x,y)$ in the $S$ is the transposition on $V$ that swaps $x$ and $y$ and leaves everything else fixed.
\item if a variable $x\in V$ is not in the support of $a\in A$, then $a=\exists(x) (E(x,y)\wedge S(y/x)a)$.
\item $\exists(x)(E(x,y)\wedge a)\wedge \exists(x)(E(x,y)\wedge a')=0$ for all $a\in A$ and all $x,y\in V$ whenever $x\ne y$.
\item $\exists(x)(E(x,y)\wedge E(x,z))=E(y,z)$ for all $x,y,z\in V$ where $x\notin \lbrace y,z\rbrace$.
\end{itemize}

\textbf{Remarks}
\begin{itemize}
\item The degree and local finiteness of a polyadic algebra $(A,E)$ are defined as the degree and the local finiteness and degree of its underlying polyadic algebra $A$.
\item It can be shown that every locally finite polyadic algebra of infinite degree can be embedded (as a polyadic subalgebra) in a locally finite polyadic algebra with equality of infinite degree.
\item Like cylindric algebras, polyadic algebras with equality is an attempt at ``converting'' a first order logic (with equality) into algebraic form, so that the logic can be studied using algebraic means.
\end{itemize}

\begin{thebibliography}{8}
\bibitem{ph} P. Halmos, \emph{Algebraic Logic}, Chelsea Publishing Co. New York (1962).
\bibitem{bp} B. Plotkin, \emph{Universal Algebra, Algebraic Logic, and Databases}, Kluwer Academic Publishers (1994).
\end{thebibliography}
%%%%%
%%%%%
\end{document}
