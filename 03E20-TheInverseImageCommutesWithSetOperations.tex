\documentclass[12pt]{article}
\usepackage{pmmeta}
\pmcanonicalname{TheInverseImageCommutesWithSetOperations}
\pmcreated{2013-03-22 13:35:24}
\pmmodified{2013-03-22 13:35:24}
\pmowner{matte}{1858}
\pmmodifier{matte}{1858}
\pmtitle{the inverse image commutes with set operations}
\pmrecord{11}{34213}
\pmprivacy{1}
\pmauthor{matte}{1858}
\pmtype{Proof}
\pmcomment{trigger rebuild}
\pmclassification{msc}{03E20}
\pmrelated{SetDifference}

% this is the default PlanetMath preamble.  as your knowledge
% of TeX increases, you will probably want to edit this, but
% it should be fine as is for beginners.

% almost certainly you want these
\usepackage{amssymb}
\usepackage{amsmath}
\usepackage{amsfonts}

% used for TeXing text within eps files
%\usepackage{psfrag}
% need this for including graphics (\includegraphics)
%\usepackage{graphicx}
% for neatly defining theorems and propositions
%\usepackage{amsthm}
% making logically defined graphics
%%%\usepackage{xypic}

% there are many more packages, add them here as you need them

% define commands here
\begin{document}
{\bf Theorem.} 
Let $f$ be a mapping from $X$ to $Y$. If $\{B_i\}_{i\in I}$ is
a (possibly uncountable) collection of subsets in  $Y$, then 
the following relations hold for the inverse image:
\begin{enumerate}
\item[(1)] $ \displaystyle f^{-1}\big(\bigcup_{i\in I} B_i\big) = \bigcup_{i\in I} f^{-1}\big(B_i\big) $
\item[(2)] $ \displaystyle f^{-1}\big(\bigcap_{i\in I} B_i\big) = \bigcap_{i\in I} f^{-1}\big(B_i\big) $
\end{enumerate}
If $A$ and $B$ are subsets in $Y$, then we also have:
\begin{enumerate}
\item[(3)] For the set complement, 
$$\big(f^{-1}(A)\big)^\complement=f^{-1}(A^\complement).$$
\item[(4)] For the set difference, 
$$f^{-1}(A\setminus B) = f^{-1}(A)\setminus f^{-1}(B).$$
\item[(5)] For the symmetric difference, 
$$f^{-1}(A \bigtriangleup B)=f^{-1}(A) \bigtriangleup f^{-1}(B).$$
\end{enumerate}
 
\emph{Proof.} 
For part (1), we have
\begin{eqnarray*}
f^{-1}\big(\bigcup_{i\in I} B_i\big) &=& \Big\{ x\in X\mid f(x) \in \bigcup_{i\in I} B_i\Big\} \\
&=& \left\{ x\in X \mid  f(x) \in B_i\ \mbox{for some}\ i\in I\right\} \\
&=& \bigcup_{i\in I}\left\{ x\in X \mid  f(x) \in B_i \right\} \\
&=& \bigcup_{i\in I} f^{-1}\big(B_i\big).
\end{eqnarray*}
Similarly, for part (2), we have
\begin{eqnarray*}
f^{-1}\big(\bigcap_{i\in I} B_i\big) &=& \big\{ x\in X \mid f(x) \in \bigcap_{i\in I} B_i\big\} \\
&=& \left\{ x\in X \mid f(x) \in B_i\ \mbox{for all}\ i\in I\right\} \\
&=& \bigcap_{i\in I}\left\{ x\in X \mid  f(x) \in B_i \right\} \\
&=& \bigcap_{i\in I} f^{-1}\big(B_i\big).
\end{eqnarray*}
For the set complement, suppose $x\notin f^{-1}(A)$. This is equivalent to
$f(x)\notin A$, or $f(x)\in A^\complement$, which is equivalent to 
$x\in f^{-1}(A^\complement)$. Since the set difference  $A\setminus B$ can be
written as $A\cap B^c$, part (4) follows from parts (2) and 
(3). Similarly, since $A\bigtriangleup B=(A\setminus B) \cup (B\setminus A)$,
part (5) follows from parts (1) and 
(4). $\Box$
%%%%%
%%%%%
\end{document}
