\documentclass[12pt]{article}
\usepackage{pmmeta}
\pmcanonicalname{SetTheory}
\pmcreated{2013-03-22 13:20:53}
\pmmodified{2013-03-22 13:20:53}
\pmowner{mathwizard}{128}
\pmmodifier{mathwizard}{128}
\pmtitle{set theory}
\pmrecord{12}{33866}
\pmprivacy{1}
\pmauthor{mathwizard}{128}
\pmtype{Topic}
\pmcomment{trigger rebuild}
\pmclassification{msc}{03E30}
\pmsynonym{theory of sets}{SetTheory}
\pmrelated{Set}
\pmrelated{ZermeloFraenkelAxioms}
\pmrelated{Superset}
\pmrelated{AbstractRelationalBiology}
\pmrelated{Definition}

\endmetadata

% this is the default PlanetMath preamble.  as your knowledge
% of TeX increases, you will probably want to edit this, but
% it should be fine as is for beginners.

% almost certainly you want these
\usepackage{amssymb}
\usepackage{amsmath}
\usepackage{amsfonts}

% used for TeXing text within eps files
%\usepackage{psfrag}
% need this for including graphics (\includegraphics)
%\usepackage{graphicx}
% for neatly defining theorems and propositions
%\usepackage{amsthm}
% making logically defined graphics
%%%\usepackage{xypic}

% there are many more packages, add them here as you need them

% define commands here
\begin{document}
%03E30 03E15
\PMlinkescapeword{complex}
\PMlinkescapeword{simple}
\PMlinkescapeword{level}
\PMlinkescapeword{scope}
\PMlinkescapeword{model}
\PMlinkescapeword{extension}
\PMlinkescapeword{type}
\PMlinkescapeword{relation}
\PMlinkescapeword{mean}
\PMlinkescapeword{one way}
\PMlinkescapeword{term}
\PMlinkescapeword{terms}
\PMlinkescapeword{series}
\PMlinkescapeword{contains}
Set theory is special among mathematical theories, in two ways: It plays
a central role in putting mathematics on a reliable axiomatic foundation,
and it provides the basic language and apparatus in which most of
mathematics is expressed.

\section{Axiomatic set theory}
I will informally list the undefined notions, the axioms, and two of the
``schemes'' of set theory, along the lines of Bourbaki's account. The
axioms are closer to the von Neumann-Bernays-G\"{o}del model than to the
equivalent ZFC model. (But some of the axioms are identical to some in ZFC;
see the entry \PMlinkname{ZermeloFraenkelAxioms}{ZermeloFraenkelAxioms}.) The intention here is just to
give an idea of the level and scope of these fundamental things.

There are three undefined notions:

1. the relation of equality of two sets

2. the relation of membership of one set in another ($x\in y$)

3. the notion of an ordered pair, which is a set comprised from two other
sets, in a specific order.

Most of the eight schemes belong more properly to logic than to
set theory, but they, or something on the same level, are
needed in the work of formalizing any theory that uses the notion of
equality, or uses quantifiers such as $\exists$.
Because of their formal nature, let me just (informally)
state two of the schemes:

S6. If $A$ and $B$ are sets, and $A=B$, then anything true of $A$ is true of
$B$, and conversely.

S7. If two properties $F(x)$ and $G(x)$ of a set $x$ are equivalent,
then the ``generic'' set having the property $F$, is the same as the
generic set having the property $G$.

(The notion of a generic set having a given property, is formalized
with the help of the Hilbert $\tau$ symbol; this is one way,
but not the only way, to incorporate what is called the Axiom of Choice.)

Finally come the five axioms in this axiomatization of set theory. (Some
are identical to axioms in ZFC, q.v.)

A1. Two sets $A$ and $B$ are equal iff they have the same elements, i.e.
iff the relation $x\in A$ implies $x\in B$ and vice versa.

A2. For any two sets $A$ and $B$, there is a set $C$ such that the
$x\in C$ is equivalent to $x=A$ or $x=B$.

A3. Two ordered pairs $(A,B)$ and $(C,D)$ are equal iff $A=C$ and $B=D$.

A4. For any set $A$, there exists a set $B$ such that $x\in B$ is
equivalent to $x\subset A$; in other words, there is a set of all
subsets of $A$, for any given set $A$.

A5. There exists an infinite set.

The word ``infinite'' is defined in terms of Axioms A1-A4. But to formulate
the definition, one must first build up some definitions and results about
functions and ordered sets, which we haven't done here.

\section{Product sets, relations, functions, etc.}
Moving away from foundations and toward applications, all the more complex
structures and relations of set theory are built up
out of the three undefined notions. (See the entry ``Set''.) For instance,
the relation $A\subset B$ between two sets, means simply
``if $x\in A$ then $x\in B$''.

Using the notion of ordered pair, we soon get the very important structure
called the product $A\times B$ of two sets $A$ and $B$. Next, we can get such
things as equivalence relations and order relations on a set $A$, for they
are subsets of $A\times A$. And we get the critical notion of a function
$A\to B$, as a subset of $A\times B$. Using functions, we get such things
as the product $\prod_{i\in I}A_i$ of a family of sets. (``Family'' is a
variation of the notion of function.)

To be strictly formal, we should distinguish between a function and the
graph of that function, and between a relation and its graph, but the
distinction is rarely necessary in practice.

\section{Some structures defined in terms of sets}
The natural numbers provide the first example. Peano, Zermelo and Fraenkel,
and others have given axiom-lists for the set $\mathbb{N}$, with its
addition, multiplication, and order relation; but
nowadays the custom is to define even the natural numbers in terms of
sets. In more detail, a natural number is the order-type of a finite
well-ordered set.
The relation $m\le n$ between $m,n\in \mathbb{N}$
is defined with the aid of a certain theorem which says, roughly, that
for any two well-ordered sets, one is a segment of the other.
The sum or product of two natural numbers is defined as the cardinal
of the sum or product, respectively, of two sets. (For an extension of
this idea, see surreal numbers.)

(The term ``cardinal'' takes some work to define.
The ``type'' of an ordered set, or any other kind of structure, is the
``generic'' structure of that kind, which is defined using $\tau$.)

Groups provide another simple example of a structure defined in terms of sets
and ordered pairs. A group is a pair $(G,f)$ in which $G$ is just a set, and
$f$ is a mapping $G\times G\to G$ satisfying certain axioms; the axioms
(associativity etc.) can all be spelled out in terms of sets and ordered
pairs, although in practice one uses algebraic notation to do it. When we
speak of (e.g.) ``the'' group $S_3$ of permutations of
a 3-element set, we mean the ``type'' of such a group.

Topological spaces provide another example of how mathematical structures
can be defined in terms of, ultimately, the sets and ordered pairs in set
theory. A topological space is a pair $(S,U)$, where the set $S$ is
arbitrary, but $U$ has these properties:

-- any element of $U$ is a subset of $S$

-- the union of any family (or set) of elements of $U$ is also an element of $U$

-- the intersection of any \emph{finite} family of elements of $U$ is an element of $U$.

Many special kinds of topological spaces are defined by enlarging this list
of restrictions on $U$.

Finally, many kinds of structure are based on more than one set. E.g. a
left module is a commutative group $M$ together with a ring $R$,
plus a mapping $R\times M\to M$ which satisfies a specific set of
restrictions.
\section{Categories, homological algebra}
Although set theory provides some of the language and apparatus used
in mathematics generally, that language and apparatus have expanded
over time, and now include what are called ``categories'' and ``functors''.
A category is not a set, and a functor is not a mapping, despite
similarities in both cases. A category comprises all the structured
sets of the same kind, e.g. the groups, and contains also a
definition of the notion of a morphism from one such structured
set to another of the same kind. A functor is similar to a morphism but
compares one category to another, not one structured set to another.
The classic examples are certain functors from the category of topological
spaces to the category of groups.

``Homological algebra'' is concerned with sequences of morphisms
within a category, plus functors from one category to another.
One of its aims is to get structure theories for specific categories;
the homology of groups and the cohomology of Lie algebras are examples.
For more details on the categories and functors of homological algebra, I
recommend a search for ``Eilenberg-Steenrod axioms''.
%%%%%
%%%%%
\end{document}
