\documentclass[12pt]{article}
\usepackage{pmmeta}
\pmcanonicalname{A11TypeUniverses}
\pmcreated{2013-11-09 4:43:13}
\pmmodified{2013-11-09 4:43:13}
\pmowner{PMBookProject}{1000683}
\pmmodifier{PMBookProject}{1000683}
\pmtitle{A.1.1 Type universes}
\pmrecord{1}{}
\pmprivacy{1}
\pmauthor{PMBookProject}{1000683}
\pmtype{Feature}
\pmclassification{msc}{03B15}

\usepackage{xspace}
\usepackage{amssyb}
\usepackage{amsmath}
\usepackage{amsfonts}
\usepackage{amsthm}
\newcommand{\ctx}{\ensuremath{\mathsf{ctx}}}
\newcommand{\define}[1]{\textbf{#1}}
\newcommand{\jdeq}{\equiv}      
\newcommand{\UU}{\ensuremath{\mathcal{U}}\xspace}
\begin{document}
We postulate a hierarchy of \define{universes} denoted by primitive constants
\index{type!universe}
%
\begin{equation*}
  \UU_0, \quad \UU_1, \quad  \UU_2, \quad \ldots
\end{equation*}
%
The first two rules for universes say that they form a cumulative hierarchy of types:
%
\begin{itemize}
\item $\UU_m : \UU_n$ for $m < n$,
\item if $A:\UU_m$ and $m \le n$, then $A:\UU_n$,
\end{itemize}
%
and the third expresses the idea that an object of a universe can serve as a type and stand to the
right of a colon in judgments:
%
\begin{itemize}
\item if $\Gamma \vdash A : \UU_n$, and $x$ is a new variable,%
\footnote{By ``new'' we mean that it does not appear in $\Gamma$ or $A$.}
then $\vdash (\Gamma, x:A)\; \ctx$.
\end{itemize}
%
In the body of the book, an equality judgment $A \jdeq B : \UU_n$ between types
$A$ and $B$ is usually abbreviated to $A \jdeq B$. This is an instance of
typical ambiguity\index{typical ambiguity}, as we can always switch to a larger universe, which however does not affect the validity of the judgment.

The following conversion rule allows us to replace a type by one equal to it in a typing judgment:
%
\begin{itemize}
\item if $a:A$ and $A \jdeq B$ then $a:B$.
\end{itemize}


\end{document}
