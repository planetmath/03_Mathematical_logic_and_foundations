\documentclass[12pt]{article}
\usepackage{pmmeta}
\pmcanonicalname{83pikleNOfAnNconnectedSpaceAndpikNSn}
\pmcreated{2013-11-06 14:14:57}
\pmmodified{2013-11-06 14:14:57}
\pmowner{PMBookProject}{1000683}
\pmmodifier{rspuzio}{6075}
\pmtitle{8.3 $\pi_{k \le n}$ of an n-connected space and $\pi_{k < n}(S^n)$}
\pmrecord{1}{}
\pmprivacy{1}
\pmauthor{PMBookProject}{6075}
\pmtype{Feature}
\pmclassification{msc}{03B15}

\endmetadata

\usepackage{xspace}
\usepackage{amssyb}
\usepackage{amsmath}
\usepackage{amsfonts}
\usepackage{amsthm}
\newcommand{\blank}{\mathord{\hspace{1pt}\text{--}\hspace{1pt}}}
\newcommand{\N}{\ensuremath{\mathbb{N}}\xspace}
\newcommand{\narrowbreak}{}
\newcommand{\Sn}{\mathbb{S}}
\newcommand{\trunc}[2]{\mathopen{}\left\Vert #2\right\Vert_{#1}\mathclose{}}
\newcommand{\truncf}[1]{\Vert \blank \Vert_{#1}}
\newcommand{\unit}{\ensuremath{\mathbf{1}}\xspace}
\newcounter{mathcount}
\setcounter{mathcount}{1}
\newtheorem{precor}{Corollary}
\newenvironment{cor}{\begin{precor}}{\end{precor}\addtocounter{mathcount}{1}}
\renewcommand{\theprecor}{8.3.\arabic{mathcount}}
\newtheorem{prelem}{Lemma}
\newenvironment{lem}{\begin{prelem}}{\end{prelem}\addtocounter{mathcount}{1}}
\renewcommand{\theprelem}{8.3.\arabic{mathcount}}
\newenvironment{narrowmultline*}{\csname equation*\endcsname}{\csname endequation*\endcsname}
\let\autoref\cref

\begin{document}

Let $(A,a)$ be a pointed type and $n:\N$.  Recall from
\autoref{thm:homotopy-groups} that if $n>0$ the set $\pi_n(A,a)$ has a group
structure, and if $n>1$ the group is abelian\index{group!abelian}.

We can now say something about homotopy groups of $n$-truncated and
$n$-connected types.

\begin{lem}
  If $A$ is $n$-truncated and $a:A$, then $\pi_k(A,a)=\unit$ for all $k>n$.
\end{lem}

\begin{proof}
  The loop space of an $n$-type  is an
  $(n-1)$-type, hence $\Omega^k(A,a)$ is an $(n-k)$-type, and we have
  $(n-k)\le-1$ so $\Omega^k(A,a)$ is a mere proposition. But $\Omega^k(A,a)$ is inhabited,
  so it is actually contractible and
  $\pi_k(A,a)=\trunc0{\Omega^k(A,a)}=\trunc0{\unit}=\unit$.
\end{proof}

\begin{lem} \label{lem:pik-nconnected}
  If $A$ is $n$-connected and $a:A$, then $\pi_k(A,a)=\unit$ for all $k\le{}n$.
\end{lem}

\begin{proof}
  We have the following sequence of equalities:
  %
  \begin{narrowmultline*}
    \pi_k(A,a) = \trunc0{\Omega^k(A,a)}
    = \Omega^k(\trunc k{(A,a)})
    = \Omega^k(\trunc k{\trunc n{(A,a)}})
    = \narrowbreak
      \Omega^k(\trunc k{\unit})
    = \Omega^k(\unit)
    = \unit.
  \end{narrowmultline*}
  %
  The third equality uses the fact that $k\le{}n$ in order to use that
  $\truncf k\circ\truncf n=\truncf k$ and the fourth equality uses the fact that $A$ is
  $n$-connected.
\end{proof}

\begin{cor}
  $\pi_k(\Sn ^n) = \unit$ for $k < n$.
\end{cor}
\begin{proof}
  The sphere $\Sn^n$ is $(n-1)$-connected by \autoref{cor:sn-connected}, so
  we can apply \autoref{lem:pik-nconnected}.
\end{proof}


\end{document}
