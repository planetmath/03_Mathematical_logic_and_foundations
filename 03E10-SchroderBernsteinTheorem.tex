\documentclass[12pt]{article}
\usepackage{pmmeta}
\pmcanonicalname{SchroderBernsteinTheorem}
\pmcreated{2013-03-22 12:21:46}
\pmmodified{2013-03-22 12:21:46}
\pmowner{yark}{2760}
\pmmodifier{yark}{2760}
\pmtitle{Schr\"oder-Bernstein theorem}
\pmrecord{9}{32091}
\pmprivacy{1}
\pmauthor{yark}{2760}
\pmtype{Theorem}
\pmcomment{trigger rebuild}
\pmclassification{msc}{03E10}
\pmsynonym{Schroeder-Bernstein theorem}{SchroderBernsteinTheorem}
\pmsynonym{Cantor-Schroeder-Bernstein theorem}{SchroderBernsteinTheorem}
\pmsynonym{Cantor-Schr\"oder-Bernstein theorem}{SchroderBernsteinTheorem}
\pmsynonym{Cantor-Bernstein theorem}{SchroderBernsteinTheorem}
\pmrelated{AnInjectionBetweenTwoFiniteSetsOfTheSameCardinalityIsBijective}
\pmrelated{ProofOfSchroederBernsteinTheoremUsingTarskiKnasterTheorem}

\usepackage{amssymb}
\usepackage{amsmath}
\usepackage{amsfonts}
\begin{document}
\PMlinkescapeword{theorem}
\PMlinkescapeword{between}

Let $S$ and $T$ be sets.
If there are injections $S \to T$ and $T \to S$,
then there is a bijection $S\to T$.

The Schr\"oder-Bernstein theorem is useful
for proving many results about cardinality,
since it replaces one hard problem (finding a bijection between $S$ and $T$)
with two generally easier problems (finding two injections).
%%%%%
%%%%%
\end{document}
