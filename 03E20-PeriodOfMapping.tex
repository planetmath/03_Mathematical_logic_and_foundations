\documentclass[12pt]{article}
\usepackage{pmmeta}
\pmcanonicalname{PeriodOfMapping}
\pmcreated{2013-03-22 13:48:53}
\pmmodified{2013-03-22 13:48:53}
\pmowner{bwebste}{988}
\pmmodifier{bwebste}{988}
\pmtitle{period of mapping}
\pmrecord{12}{34539}
\pmprivacy{1}
\pmauthor{bwebste}{988}
\pmtype{Definition}
\pmcomment{trigger rebuild}
\pmclassification{msc}{03E20}
\pmrelated{Retract}
\pmrelated{Idempotency}

% this is the default PlanetMath preamble.  as your knowledge
% of TeX increases, you will probably want to edit this, but
% it should be fine as is for beginners.

% almost certainly you want these
\usepackage{amssymb}
\usepackage{amsmath}
\usepackage{amsfonts}

% used for TeXing text within eps files
%\usepackage{psfrag}
% need this for including graphics (\includegraphics)
%\usepackage{graphicx}
% for neatly defining theorems and propositions
%\usepackage{amsthm}
% making logically defined graphics
%%%\usepackage{xypic}

% there are many more packages, add them here as you need them

% define commands here

\newcommand{\sR}[0]{\mathbb{R}}
\newcommand{\sC}[0]{\mathbb{C}}
\newcommand{\sN}[0]{\mathbb{N}}
\newcommand{\sZ}[0]{\mathbb{Z}}
\begin{document}
\PMlinkescapeword{period}

{\bf Definition}
Suppose $X$ is a set and $f$ is a mapping $f:X\to X$. 
If  $f^n$ is the identity mapping on $X$ for some $n=1,2,\ldots$, then 
$f$ is said to be a {\bf mapping of period} $n$.
Here, the notation $f^n$ means the $n$-fold 
composition $f\circ\cdots \circ f$.


\subsubsection{Examples}
\begin{enumerate}
\item A mapping $f$ is of period $1$ if and only if $f$ is the identity 
mapping.
\item Suppose $V$ is a vector space. Then a linear involution $L:V\to V$ is 
a mapping of period $2$.
For example, the reflection mapping $x\mapsto -x$ is a mapping of period $2$.
\item In the complex plane, the mapping $z\mapsto e^{-2\pi i/n}z $ is a 
mapping of period $n$ for $n=1,2,\ldots$. 
\item Let us consider the  function space spanned by the 
trigonometric functions $\sin$ and $\cos$. On this space,
the derivative is a mapping of period $4$. 
\end{enumerate}


\subsubsection{Properties}
\begin{enumerate}
\item  Suppose $X$ is a set. 
Then a mapping $f:X\to X$ of period $n$ is  a bijection. 
\PMlinkname{(proof.)}{MappingOfDegreeNIsASurjection}

\item Suppose $X$ is a topological space. 
Then a continuous mapping $f:X\to X$ of period $n$ is  a homeomorphism. 
\end{enumerate}
%%%%%
%%%%%
\end{document}
