\documentclass[12pt]{article}
\usepackage{pmmeta}
\pmcanonicalname{91CategoriesAndPrecategories}
\pmcreated{2013-11-06 16:26:48}
\pmmodified{2013-11-06 16:26:48}
\pmowner{PMBookProject}{1000683}
\pmmodifier{PMBookProject}{1000683}
\pmtitle{9.1 Categories and precategories}
\pmrecord{1}{}
\pmprivacy{1}
\pmauthor{PMBookProject}{1000683}
\pmtype{Feature}
\pmclassification{msc}{03B15}

\usepackage{xspace}
\usepackage{amssyb}
\usepackage{amsmath}
\usepackage{amsfonts}
\usepackage{amsthm}
\makeatletter
\newcommand{\Brck}[1]{\Trunc{}{#1}}
\newcommand{\ct}{  \mathchoice{\mathbin{\raisebox{0.5ex}{$\displaystyle\centerdot$}}}             {\mathbin{\raisebox{0.5ex}{$\centerdot$}}}             {\mathbin{\raisebox{0.25ex}{$\scriptstyle\,\centerdot\,$}}}             {\mathbin{\raisebox{0.1ex}{$\scriptscriptstyle\,\centerdot\,$}}}}
\newcommand{\defeq}{\vcentcolon\equiv}  
\newcommand{\define}[1]{\textbf{#1}}
\def\@dprd#1{\prod_{(#1)}\,}
\def\@dprd@noparens#1{\prod_{#1}\,}
\def\@dsm#1{\sum_{(#1)}\,}
\def\@dsm@noparens#1{\sum_{#1}\,}
\def\@eatprd\prd{\prd@parens}
\def\@eatsm\sm{\sm@parens}
\newcommand{\id}[3][]{\ensuremath{#2 =_{#1} #3}\xspace}
\newcommand{\idtoeqv}{\ensuremath{\mathsf{idtoeqv}}\xspace}
\newcommand{\idtoiso}{\ensuremath{\mathsf{idtoiso}}\xspace}
\newcommand{\indexdef}[1]{\index{#1|defstyle}}   
\newcommand{\indexsee}[2]{\index{#1|see{#2}}}    
\newcommand{\inv}[1]{{#1}^{-1}}
\newcommand{\isotoid}{\ensuremath{\mathsf{isotoid}}\xspace}
\newcommand{\opp}[1]{\mathord{{#1}^{-1}}}
\def\pizero{\trunc0}
\def\prd#1{\@ifnextchar\bgroup{\prd@parens{#1}}{\@ifnextchar\sm{\prd@parens{#1}\@eatsm}{\prd@noparens{#1}}}}
\def\prd@noparens#1{\mathchoice{\@dprd@noparens{#1}}{\@tprd{#1}}{\@tprd{#1}}{\@tprd{#1}}}
\def\prd@parens#1{\@ifnextchar\bgroup  {\mathchoice{\@dprd{#1}}{\@tprd{#1}}{\@tprd{#1}}{\@tprd{#1}}\prd@parens}  {\@ifnextchar\sm    {\mathchoice{\@dprd{#1}}{\@tprd{#1}}{\@tprd{#1}}{\@tprd{#1}}\@eatsm}    {\mathchoice{\@dprd{#1}}{\@tprd{#1}}{\@tprd{#1}}{\@tprd{#1}}}}}
\newcommand{\prop}{\ensuremath{\mathsf{Prop}}\xspace}
\newcommand{\refl}[1]{\ensuremath{\mathsf{refl}_{#1}}\xspace}
\def\sm#1{\@ifnextchar\bgroup{\sm@parens{#1}}{\@ifnextchar\prd{\sm@parens{#1}\@eatprd}{\sm@noparens{#1}}}}
\def\sm@noparens#1{\mathchoice{\@dsm@noparens{#1}}{\@tsm{#1}}{\@tsm{#1}}{\@tsm{#1}}}
\def\sm@parens#1{\@ifnextchar\bgroup  {\mathchoice{\@dsm{#1}}{\@tsm{#1}}{\@tsm{#1}}{\@tsm{#1}}\sm@parens}  {\@ifnextchar\prd    {\mathchoice{\@dsm{#1}}{\@tsm{#1}}{\@tsm{#1}}{\@tsm{#1}}\@eatprd}    {\mathchoice{\@dsm{#1}}{\@tsm{#1}}{\@tsm{#1}}{\@tsm{#1}}}}}
\newcommand{\symlabel}[1]{\refstepcounter{symindex}\label{#1}}
\def\@tprd#1{\mathchoice{{\textstyle\prod_{(#1)}}}{\prod_{(#1)}}{\prod_{(#1)}}{\prod_{(#1)}}}
\newcommand{\trans}[2]{\ensuremath{{#1}_{*}\mathopen{}\left({#2}\right)\mathclose{}}\xspace}
\newcommand{\trunc}[2]{\mathopen{}\left\Vert #2\right\Vert_{#1}\mathclose{}}
\newcommand{\Trunc}[2]{\Bigl\Vert #2\Bigr\Vert_{#1}}
\def\@tsm#1{\mathchoice{{\textstyle\sum_{(#1)}}}{\sum_{(#1)}}{\sum_{(#1)}}{\sum_{(#1)}}}
\newcommand{\urel}{\ensuremath{\mathcal{R}el}\xspace}
\newcommand{\uset}{\ensuremath{\mathcal{S}et}\xspace}
\newcommand{\UU}{\ensuremath{\mathcal{U}}\xspace}
\newcommand{\vcentcolon}{:\!\!}
\newcounter{mathcount}
\setcounter{mathcount}{1}
\newtheorem{predefn}{Definition}
\newenvironment{defn}{\begin{predefn}}{\end{predefn}\addtocounter{mathcount}{1}}
\renewcommand{\thepredefn}{9.1.\arabic{mathcount}}
\newtheorem{preeg}{Example}
\newenvironment{eg}{\begin{preeg}}{\end{preeg}\addtocounter{mathcount}{1}}
\renewcommand{\thepreeg}{9.1.\arabic{mathcount}}
\newenvironment{myeqn}{\begin{equation}}{\end{equation}\addtocounter{mathcount}{1}}
\renewcommand{\theequation}{9.1.\arabic{mathcount}}
\newtheorem{prelem}{Lemma}
\newenvironment{lem}{\begin{prelem}}{\end{prelem}\addtocounter{mathcount}{1}}
\renewcommand{\theprelem}{9.1.\arabic{mathcount}}
\let\autoref\cref
\let\rev\opp
\let\type\UU
\makeatother

\begin{document}

In classical mathematics, there are many equivalent definitions of a category.
In our case, since we have dependent types, it is natural to choose the arrows to be a type family indexed by the objects.
This matches the way hom-types are always used in category theory: we never even consider comparing two arrows unless we know their domains and codomains agree.
Furthermore, it seems clear that for a theory of 1-categories, the hom-types should all be sets.
This leads us to the following definition.

\begin{defn}\label{ct:precategory}
  A \define{precategory}
  \indexdef{precategory}
  $A$ consists of the following.
  \begin{enumerate}
  \item A type $A_0$ of \define{objects}.%
    \indexdef{object!in a (pre)category}
    We write $a:A$ for $a:A_0$.
  \item For each $a,b:A$, a set $\hom_A(a,b)$ of \define{arrows} or \define{morphisms}.%
    \indexsee{arrow}{morphism}%
    \indexdef{morphism!in a (pre)category}%
    \indexdef{hom-set}%
  \item For each $a:A$, a morphism $1_a:\hom_A(a,a)$.%
    \indexdef{identity!morphism in a (pre)category}
  \item For each $a,b,c:A$, a function%
    \indexdef{composition!of morphisms in a (pre)category}
    \[  \hom_A(b,c) \to \hom_A(a,b) \to \hom_A(a,c) \]
    denoted infix by $g\mapsto f\mapsto g\circ f$, or sometimes simply by $gf$.
  \item For each $a,b:A$ and $f:\hom_A(a,b)$, we have $\id f {1_b\circ f}$ and $\id f {f\circ 1_a}$.
  \item For each $a,b,c,d:A$ and 
    \begin{equation*}
      f:\hom_A(a,b), \qquad
      g:\hom_A(b,c), \qquad
      h:\hom_A(c,d),
    \end{equation*}
    we have $\id {h\circ (g\circ f)}{(h\circ g)\circ f}$.
  \end{enumerate}
\end{defn}

The problem with the notion of precategory is that for objects $a,b:A$, we have two possibly-different notions of ``sameness''.
On the one hand, we have the type $(\id[A_0]{a}{b})$.
But on the other hand, there is the standard categorical notion of \emph{isomorphism}.

\begin{defn}\label{ct:isomorphism}
  A morphism $f:\hom_A(a,b)$ is an \define{isomorphism}
  \indexdef{isomorphism!in a (pre)category}%
  if there is a morphism $g:\hom_A(b,a)$ such that $\id{g\circ f}{1_a}$ and $\id{f\circ g}{1_b}$.
  We write $a\cong b$ for the type of such isomorphisms.
\end{defn}

\begin{lem}\label{ct:isoprop}
  For any $f:\hom_A(a,b)$, the type ``$f$ is an isomorphism'' is a mere proposition.
  Therefore, for any $a,b:A$ the type $a\cong b$ is a set.
\end{lem}
\begin{proof}
  Suppose given $g:\hom_A(b,a)$ and $\eta:(\id{1_a}{g\circ f})$ and $\epsilon:(\id{f\circ g}{1_b})$, and similarly $g'$, $\eta'$, and $\epsilon'$.
We must show $\id{(g,\eta,\epsilon)}{(g',\eta',\epsilon')}$.
  But since all hom-sets are sets, their identity types are mere propositions, so it suffices to show $\id g {g'}$.
  For this we have
  \[g' = 1_a\circ g' = (g\circ f)\circ g' = g\circ (f\circ g') = g\circ 1_b = g\]
  using $\eta$ and $\epsilon'$.
\end{proof}

\symlabel{ct:inv}
\index{inverse!in a (pre)category}%
If $f:a\cong b$, then we write $\inv f$ for its inverse, which by \autoref{ct:isoprop} is uniquely determined.

The only relationship between these two notions of sameness that we have in a precategory is the following.

\begin{lem}[\textsf{idtoiso}]\label{ct:idtoiso}
  If $A$ is a precategory and $a,b:A$, then
  \[(\id a b)\to (a \cong b).\]
\end{lem}
\begin{proof}
  By induction on identity, we may assume $a$ and $b$ are the same.
  But then we have $1_a:\hom_A(a,a)$, which is clearly an isomorphism.
\end{proof}

Evidently, this situation is analogous to the issue that motivated us to introduce the univalence axiom.
In fact, we have the following:

\begin{eg}\label{ct:precatset}
  \index{set}%
  There is a precategory \uset, whose type of objects is \set, and with $\hom_{\uset}(A,B) \defeq (A\to B)$.
  The identity morphisms are identity functions and the composition is function composition.
  For this precategory, \autoref{ct:idtoiso} is equal to (the restriction to sets of) the map $\idtoeqv$ from \autoref{sec:compute-universe}.

  Of course, to be more precise we should call this category $\uset_\UU$, since its objects are only the \emph{small sets}
  \index{small!set}%
  relative to a universe \UU.
\end{eg}

Thus, it is natural to make the following definition.

\begin{defn}\label{ct:category}
  A \define{category}
  \indexdef{category}
  is a precategory such that for all $a,b:A$, the function $\idtoiso_{a,b}$ from \autoref{ct:idtoiso} is an equivalence.
\end{defn}

In particular, in a category, if $a\cong b$, then $a=b$.

\begin{eg}\label{ct:eg:set}
  \index{univalence axiom}%
  The univalence axiom implies immediately that \uset is a category.
  One can also show, using univalence, that any precategory of set-level structures such as groups, rings, topological spaces, etc.\ is a category; see \autoref{sec:sip}.
\end{eg}

We also note the following.

\begin{lem}\label{ct:obj-1type}
  In a category, the type of objects is a 1-type.
\end{lem}
\begin{proof}
  It suffices to show that for any $a,b:A$, the type $\id a b$ is a set.
  But $\id a b$ is equivalent to $a \cong b$, which is a set.
\end{proof}

\symlabel{isotoid}
We write $\isotoid$ for the inverse $(a\cong b) \to (\id a b)$ of the map $\idtoiso$ from \autoref{ct:idtoiso}.
The following relationship between the two is important.

\begin{lem}\label{ct:idtoiso-trans}
  For $p:\id a a'$ and $q:\id b b'$ and $f:\hom_A(a,b)$, we have
  \begin{myeqn}\label{ct:idtoisocompute}
    \id{\trans{(p,q)}{f}}
    {\idtoiso(q)\circ f \circ \inv{\idtoiso(p)}}.
  \end{myeqn}
\end{lem}
\begin{proof}
  By induction, we may assume $p$ and $q$ are $\refl a$ and $\refl b$ respectively.
Then the left-hand side of~\eqref{ct:idtoisocompute} is simply $f$.
  But by definition, $\idtoiso(\refl a)$ is $1_a$, and $\idtoiso(\refl b)$ is $1_b$, so the right-hand side of~\eqref{ct:idtoisocompute} is $1_b\circ f\circ 1_a$, which is equal to $f$.
\end{proof}

Similarly, we can show
\begin{gather}
  \id{\idtoiso(\rev p)}{\inv {(\idtoiso(p))}}\\
  \id{\idtoiso(p\ct q)}{\idtoiso(q)\circ \idtoiso(p)}\\
  \id{\isotoid(f\circ e)}{\isotoid(e)\ct \isotoid(f)}
\end{gather}
and so on.

\begin{eg}\label{ct:orders}
  A precategory in which each set $\hom_A(a,b)$ is a mere proposition is equivalently a type $A_0$ equipped with a mere relation ``$\le$'' that is reflexive ($a\le a$) and transitive (if $a\le b$ and $b\le c$, then $a\le c$).
  We call this a \define{preorder}.
  \indexdef{preorder}

  In a preorder, a witness $f: a\le b$ is an isomorphism just when there exists some witness $g: b\le a$.
  Thus, $a\cong b$ is the mere proposition that $a\le b$ and $b\le a$.
  Therefore, a preorder $A$ is a category just when (1) each type $a=b$ is a mere proposition, and (2) for any $a,b:A_0$ there exists a function $(a\cong b) \to (a=b)$.
  In other words, $A_0$ must be a set, and $\le$ must be antisymmetric\index{relation!antisymmetric} (if $a\le b$ and $b\le a$, then $a=b$).
  We call this a \define{(partial) order} or a \define{poset}.
  \indexdef{partial order}%
  \indexdef{poset}%
\end{eg}

\begin{eg}\label{ct:gaunt}
  If $A$ is a category, then $A_0$ is a set if and only if for any $a,b:A_0$, the type $a\cong b$ is a mere proposition.
  This is equivalent to saying that every isomorphism in $A$ is an identity; thus it is rather stronger than the classical\index{mathematics!classical} notion of ``skeletal'' category.
  Categories of this sort are sometimes called \define{gaunt}~\cite{bsp12infncats}.
  \indexdef{category!gaunt}%
  \indexdef{gaunt category}%
  \index{skeletal category}%
  \index{category!skeletal}%
  There is not really any notion of ``skeletality'' for our categories, unless one considers \autoref{ct:category} itself to be such.
\end{eg}

\begin{eg}\label{ct:discrete}
  For any 1-type $X$, there is a category with $X$ as its type of objects and with $\hom(x,y) \defeq (x=y)$.
  If $X$ is a set, we call this the \define{discrete}
  \indexdef{category!discrete}%
  \indexdef{discrete!category}%
  category on $X$.
  In general, we call it a \define{groupoid}
  \indexdef{groupoid}
  (see \autoref{ct:groupoids}).
\end{eg}

\begin{eg}\label{ct:fundgpd}
  For \emph{any} type $X$, there is a precategory with $X$ as its type of objects and with $\hom(x,y) \defeq \pizero{x=y}$.
  The composition operation
  \[ \pizero{y=z} \to \pizero{x=y} \to \pizero{x=z} \]
  is defined by induction on truncation from concatenation $(y=z)\to(x=y)\to(x=z)$.
  We call this the \define{fundamental pregroupoid}
  \indexdef{fundamental!pregroupoid}%
  \indexsee{pregroupoid, fundamental}{fundamental pregroupoid}%
  of $X$.
  (In fact, we have met it already in \autoref{sec:van-kampen}; see also \autoref{ex:rezk-vankampen}.)
\end{eg}

\begin{eg}\label{ct:hoprecat}
  There is a precategory whose type of objects is \type and with $\hom(X,Y) \defeq \pizero{X\to Y}$, and composition defined by induction on truncation from ordinary composition $(Y\to Z) \to (X\to Y) \to (X\to Z)$.
  We call this the \define{homotopy precategory of types}.
  \indexdef{precategory!of types}%
  \index{homotopy!category of types@(pre)category of types}%
\end{eg}

\begin{eg}\label{ct:rel}
  Let \urel be the following precategory:
  \begin{itemize}
  \item Its objects are sets.
  \item $\hom_{\urel}(X,Y) = X\to Y\to \prop$.
  \item For a set $X$, we have $1_X(x,x') \defeq (x=x')$.
  \item For $R:\hom_{\urel}(X,Y)$ and $S:\hom_{\urel}(Y,Z)$, their composite is defined by
    \[ (S\circ R)(x,z) \defeq \Brck{\sm{y:Y} R(x,y) \times S(y,z)}.\]
  \end{itemize}
  Suppose $R:\hom_{\urel}(X,Y)$ is an isomorphism, with inverse $S$.
  We observe the following.
  \begin{enumerate}
  \item If $R(x,y)$ and $S(y',x)$, then $(R\circ S)(y',y)$, and hence $y'=y$.
    Similarly, if $R(x,y)$ and $S(y,x')$, then $x=x'$.\label{item:rel1}
  \item For any $x$, we have $x=x$, hence $(S\circ R)(x,x)$.
    Thus, there merely exists a $y:Y$ such that $R(x,y)$ and $S(y,x)$.\label{item:rel2}
  \item Suppose $R(x,y)$.
    By~\ref{item:rel2}, there merely exists a $y'$ with $R(x,y')$ and $S(y',x)$.
    But then by~\ref{item:rel1}, merely $y'=y$, and hence $y'=y$ since $Y$ is a set.
    Therefore, by transporting $S(y',x)$ along this equality, we have $S(y,x)$.
    In conclusion, $R(x,y)\to S(y,x)$.
    Similarly, $S(y,x) \to R(x,y)$.\label{item:rel3}
  \item If $R(x,y)$ and $R(x,y')$, then by~\ref{item:rel3}, $S(y',x)$, so that by~\ref{item:rel1}, $y=y'$.
    Thus, for any $x$ there is at most one $y$ such that $R(x,y)$.
    And by~\ref{item:rel2}, there merely exists such a $y$, hence there exists such a $y$.
  \end{enumerate}
  In conclusion, if $R:\hom_{\urel}(X,Y)$ is an isomorphism, then for each $x:X$ there is exactly one $y:Y$ such that $R(x,y)$, and dually.
  Thus, there is a function $f:X\to Y$ sending each $x$ to this $y$, which is an equivalence; hence $X=Y$.
  With a little more work, we conclude that \urel is a category.
\end{eg}

We might now restrict ourselves to considering categories rather than precategories.
Instead, we will develop many concepts for precategories as well as categories, in order to emphasize how much better-behaved categories are, as compared both to precategories and to ordinary categories in classical\index{mathematics!classical} mathematics.

We will also see in \crefrange{sec:strict-categories}{sec:dagger-categories} that in slightly more exotic contexts, there are uses for certain kinds of precategories other than categories, each of which ``fixes'' the equality of objects in different ways.
This emphasizes the ``pre''-ness of precategories: they are the raw material out of which multiple important categorical structures can be defined.



\end{document}
