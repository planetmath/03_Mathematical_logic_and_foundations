\documentclass[12pt]{article}
\usepackage{pmmeta}
\pmcanonicalname{AronszajnTree}
\pmcreated{2013-03-22 12:52:34}
\pmmodified{2013-03-22 12:52:34}
\pmowner{Henry}{455}
\pmmodifier{Henry}{455}
\pmtitle{Aronszajn tree}
\pmrecord{10}{33217}
\pmprivacy{1}
\pmauthor{Henry}{455}
\pmtype{Definition}
\pmcomment{trigger rebuild}
\pmclassification{msc}{03E05}
\pmclassification{msc}{05C05}
\pmrelated{TreeSetTheoretic}
\pmrelated{Antichain}
\pmrelated{SuslinTree}
\pmrelated{WeaklyCompactCardinalsAndTheTreeProperty}
\pmdefines{Aronszajn tree}
\pmdefines{$\kappa$-Aronszajn tree}
\pmdefines{tree property}

\endmetadata

% this is the default PlanetMath preamble.  as your knowledge
% of TeX increases, you will probably want to edit this, but
% it should be fine as is for beginners.

% almost certainly you want these
\usepackage{amssymb}
\usepackage{amsmath}
\usepackage{amsfonts}

% used for TeXing text within eps files
%\usepackage{psfrag}
% need this for including graphics (\includegraphics)
%\usepackage{graphicx}
% for neatly defining theorems and propositions
%\usepackage{amsthm}
% making logically defined graphics
%%%\usepackage{xypic}

% there are many more packages, add them here as you need them

% define commands here
%\PMlinkescapeword{theory}
\begin{document}
\PMlinkescapeword{tree}
\PMlinkescapeword{trees}
A $\kappa$-\PMlinkname{tree}{TreeSetTheoretic} $T$ for which $|T_\alpha|<\kappa$ for all $\alpha<\kappa$ and which has no cofinal branches is called a \emph{$\kappa$-Aronszajn tree}.  If $\kappa=\omega_1$ then it is referred to simply as an Aronszajn tree.

If there are no $\kappa$-Aronszajn trees for some $\kappa$ then we say $\kappa$ has the \emph{tree property}.  $\omega$ has the tree property, but no singular cardinal has the tree property.
%%%%%
%%%%%
\end{document}
