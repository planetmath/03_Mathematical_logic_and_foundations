\documentclass[12pt]{article}
\usepackage{pmmeta}
\pmcanonicalname{IdentityMap}
\pmcreated{2013-03-22 14:03:43}
\pmmodified{2013-03-22 14:03:43}
\pmowner{bwebste}{988}
\pmmodifier{bwebste}{988}
\pmtitle{identity map}
\pmrecord{7}{35418}
\pmprivacy{1}
\pmauthor{bwebste}{988}
\pmtype{Definition}
\pmcomment{trigger rebuild}
\pmclassification{msc}{03E20}
\pmsynonym{identity mapping}{IdentityMap}
\pmsynonym{identity operator}{IdentityMap}
\pmsynonym{identity function}{IdentityMap}
\pmrelated{ZeroMap}
\pmrelated{IdentityMatrix}

\endmetadata

% this is the default PlanetMath preamble.  as your knowledge
% of TeX increases, you will probably want to edit this, but
% it should be fine as is for beginners.

% almost certainly you want these
\usepackage{amssymb}
\usepackage{amsmath}
\usepackage{amsfonts}

% used for TeXing text within eps files
%\usepackage{psfrag}
% need this for including graphics (\includegraphics)
%\usepackage{graphicx}
% for neatly defining theorems and propositions
%\usepackage{amsthm}
% making logically defined graphics
%%%\usepackage{xypic}

% there are many more packages, add them here as you need them

% define commands here

\newcommand{\sR}[0]{\mathbb{R}}
\newcommand{\sC}[0]{\mathbb{C}}
\newcommand{\sN}[0]{\mathbb{N}}
\newcommand{\sZ}[0]{\mathbb{Z}}

% The below lines should work as the command
% \renewcommand{\bibname}{References}
% without creating havoc when rendering an entry in 
% the page-image mode.
\makeatletter
\@ifundefined{bibname}{}{\renewcommand{\bibname}{References}}
\makeatother

\newcommand*{\norm}[1]{\lVert #1 \rVert}
\newcommand*{\abs}[1]{| #1 |}
\begin{document}
{\bf Definition}
If $X$ is a set, then the {\bf identity map} in $X$ is the mapping 
that maps each element in $X$ to itself. 

\subsubsection{Properties}
\begin{enumerate}
\item An identity map is always a bijection. 
\item Suppose $X$ has two topologies $\tau_1$ and $\tau_2$. Then
the identity mapping $I:(X,\tau_1)\to (X,\tau_2)$ is continuous if and only if
$\tau_1$ is finer than $\tau_2$, i.e., $\tau_1\subset\tau_2$.
\item
 The identity map on the $n$-sphere, is 
 \PMlinkname{homotopic}{HomotopyOfMaps}
to the antipodal map $A:S^n\to S^n$ if $n$ is odd \cite{guillemin}.
 \end{enumerate}
 
 \begin{thebibliography}{9}
 \bibitem{guillemin} V. Guillemin, A. Pollack,
 \emph{Differential topology}, Prentice-Hall Inc., 1974.
 \end{thebibliography}
%%%%%
%%%%%
\end{document}
