\documentclass[12pt]{article}
\usepackage{pmmeta}
\pmcanonicalname{ExampleOfAntisymmetric}
\pmcreated{2013-03-22 16:00:36}
\pmmodified{2013-03-22 16:00:36}
\pmowner{Algeboy}{12884}
\pmmodifier{Algeboy}{12884}
\pmtitle{example of antisymmetric}
\pmrecord{8}{38044}
\pmprivacy{1}
\pmauthor{Algeboy}{12884}
\pmtype{Example}
\pmcomment{trigger rebuild}
\pmclassification{msc}{03E20}

\endmetadata

\usepackage{latexsym}
\usepackage{amssymb}
\usepackage{amsmath}
\usepackage{amsfonts}
\usepackage{amsthm}

%%\usepackage{xypic}

%-----------------------------------------------------

%       Standard theoremlike environments.

%       Stolen directly from AMSLaTeX sample

%-----------------------------------------------------

%% \theoremstyle{plain} %% This is the default

\newtheorem{thm}{Theorem}

\newtheorem{coro}[thm]{Corollary}

\newtheorem{lem}[thm]{Lemma}

\newtheorem{lemma}[thm]{Lemma}

\newtheorem{prop}[thm]{Proposition}

\newtheorem{conjecture}[thm]{Conjecture}

\newtheorem{conj}[thm]{Conjecture}

\newtheorem{defn}[thm]{Definition}

\newtheorem{remark}[thm]{Remark}

\newtheorem{ex}[thm]{Example}



%\countstyle[equation]{thm}



%--------------------------------------------------

%       Item references.

%--------------------------------------------------


\newcommand{\exref}[1]{Example-\ref{#1}}

\newcommand{\thmref}[1]{Theorem-\ref{#1}}

\newcommand{\defref}[1]{Definition-\ref{#1}}

\newcommand{\eqnref}[1]{(\ref{#1})}

\newcommand{\secref}[1]{Section-\ref{#1}}

\newcommand{\lemref}[1]{Lemma-\ref{#1}}

\newcommand{\propref}[1]{Prop\-o\-si\-tion-\ref{#1}}

\newcommand{\corref}[1]{Cor\-ol\-lary-\ref{#1}}

\newcommand{\figref}[1]{Fig\-ure-\ref{#1}}

\newcommand{\conjref}[1]{Conjecture-\ref{#1}}


% Normal subgroup or equal.

\providecommand{\normaleq}{\unlhd}

% Normal subgroup.

\providecommand{\normal}{\lhd}

\providecommand{\rnormal}{\rhd}
% Divides, does not divide.

\providecommand{\divides}{\mid}

\providecommand{\ndivides}{\nmid}


\providecommand{\union}{\cup}

\providecommand{\bigunion}{\bigcup}

\providecommand{\intersect}{\cap}

\providecommand{\bigintersect}{\bigcap}










\begin{document}
The axioms of a partial ordering demonstrate that every partial ordering is antisymmetric.  That is: the relation $\leq $ on a set $S$ forces 
\begin{quote}
$a\leq b$ and $b\leq a$ implies $a=b$
\end{quote}
for every $a,b\in S$.

For a concrete example consider the natural numbers $\mathbb{N}=\{0,1,2,\dots\}$ (as defined by the \PMlinkname{Peano postulates}{PeanoArithmetic}).  Take the relation set to be:
\[R=\{(a,a+n):a,n\in \mathbb{N}\}\subset \mathbb{N}\times \mathbb{N}.\]
Then we denote $a\leq b$ if $(a,b)\in R$.  That is, $5\leq 7$ because $(5,7)=(5,5+2)$ and both $5,2\in\mathbb{N}$.

We can prove this relation is antisymmetric as follows: Suppose $a\leq b$ and
$b\leq a$ for some $a,b\in\mathbb{N}$.  Then there exist $n,m\in\mathbb{N}$
such that $a+n=b$ and $b+m=a$.  Therefore
\[b=a+n=b+m+n\]
so by the cancellation property of the natural numbers, $0=m+n$.  But by the first piano postulate, 0 has no predecessor, meaning $0\neq m+n$ unless $m=n=0$.

%%%%%
%%%%%
\end{document}
