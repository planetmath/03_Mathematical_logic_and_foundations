\documentclass[12pt]{article}
\usepackage{pmmeta}
\pmcanonicalname{ProductOfNonemptySetOfNonemptySetsIsNonempty}
\pmcreated{2013-03-22 18:44:28}
\pmmodified{2013-03-22 18:44:28}
\pmowner{CWoo}{3771}
\pmmodifier{CWoo}{3771}
\pmtitle{product of non-empty set of non-empty sets is non-empty}
\pmrecord{7}{41514}
\pmprivacy{1}
\pmauthor{CWoo}{3771}
\pmtype{Derivation}
\pmcomment{trigger rebuild}
\pmclassification{msc}{03E20}

\endmetadata

\usepackage{amssymb,amscd}
\usepackage{amsmath}
\usepackage{amsfonts}
\usepackage{mathrsfs}

% used for TeXing text within eps files
%\usepackage{psfrag}
% need this for including graphics (\includegraphics)
%\usepackage{graphicx}
% for neatly defining theorems and propositions
\usepackage{amsthm}
% making logically defined graphics
%%\usepackage{xypic}
\usepackage{pst-plot}

% define commands here
\newcommand*{\abs}[1]{\left\lvert #1\right\rvert}
\newtheorem{prop}{Proposition}
\newtheorem{thm}{Theorem}
\newtheorem{ex}{Example}
\newcommand{\real}{\mathbb{R}}
\newcommand{\pdiff}[2]{\frac{\partial #1}{\partial #2}}
\newcommand{\mpdiff}[3]{\frac{\partial^#1 #2}{\partial #3^#1}}
\begin{document}
In this entry, we show that the statement: 
\begin{quote}
(*) the non-empty generalized cartesian product of non-empty sets is non-empty
\end{quote}
is equivalent to the axiom of choice (AC).

\begin{prop} AC implies (*). \end{prop}
\begin{proof}  Suppose $C=\lbrace A_j\mid j\in J\rbrace$ is a set of non-empty sets, with $J\ne \varnothing$.  We want to show that $$B:=\prod_{j\in J} A_j$$ is non-empty.  Let $A = \bigcup C$.  Then, by AC, there is a function $f:C\to A$ such that $f(X)\in X$ for every $X\in C$.  Define $g:J \to A$ by $g(j):=f(A_j)$.  Then $g\in B$ as a result, $B$ is non-empty.
\end{proof}

\textbf{Remark}.  The statement that if $J\ne \varnothing$, then $B\ne \varnothing$ implies $A_j\ne \varnothing$ does not require AC: if $B$ is non-empty, then there is a function $g:J\to A$, and, as $J\ne \varnothing$, $g\ne \varnothing$, which means $A\ne \varnothing$, or that $A_j\ne \varnothing$ for some $j\in J$.

\begin{prop} (*) implies AC. \end{prop}
\begin{proof}  Suppose $C$ is a set of non-empty sets.  If $C$ itself is empty, then the choice function is the empty set.  So suppose that $C$ is non-empty.  We want to find a (choice) function $f: C\to \bigcup C$, such that $f(x)\in x$ for every $x\in C$.  Index elements of $C$ by $C$ itself: $A_x:=x$ for each $x\in C$.  So $A_x\ne \varnothing$ by assumption.  Hence, by (*), the (non-empty) cartesian product $B$ of the $A_x$ is non-empty.  But an element of $B$ is just a function $f$ whose domain is $C$ and whose codomain is the union of the $A_x$, or $\bigcup C$, such that $f(A_x)\in A_x$, which is precisely $f(x)\in x$.
\end{proof}
%%%%%
%%%%%
\end{document}
