\documentclass[12pt]{article}
\usepackage{pmmeta}
\pmcanonicalname{Irreflexive}
\pmcreated{2013-03-22 15:41:45}
\pmmodified{2013-03-22 15:41:45}
\pmowner{CWoo}{3771}
\pmmodifier{CWoo}{3771}
\pmtitle{irreflexive}
\pmrecord{14}{37639}
\pmprivacy{1}
\pmauthor{CWoo}{3771}
\pmtype{Definition}
\pmcomment{trigger rebuild}
\pmclassification{msc}{03E20}
\pmsynonym{antireflexive}{Irreflexive}
\pmrelated{Reflexive}

% this is the default PlanetMath preamble.  as your knowledge
% of TeX increases, you will probably want to edit this, but
% it should be fine as is for beginners.

% almost certainly you want these
\usepackage{amssymb}
\usepackage{amsmath}
\usepackage{amsfonts}

% used for TeXing text within eps files
%\usepackage{psfrag}
% need this for including graphics (\includegraphics)
%\usepackage{graphicx}
% for neatly defining theorems and propositions
%\usepackage{amsthm}
% making logically defined graphics
%%%\usepackage{xypic}

% there are many more packages, add them here as you need them

% define commands here
\begin{document}
A binary relation $\mathcal{R}$ on a set $A$ is said to be \emph{irreflexive} (or \emph{antireflexive}) if $\forall a\in A$, $\neg a\mathcal{R} a$. In other words, ``no element is $\mathcal{R}$-related to itself."

For example, the relation $<$ (``less than") is an irreflexive relation on the set of natural numbers.

Note that ``irreflexive" is not simply the negation of ``\PMlinkname{reflexive}{Reflexive}
." Although it is impossible for a relation (on a nonempty set) to be both \PMlinkname{reflexive}{Reflexive}
 and irreflexive, there exist relations that are neither. For example, the relation $\{(a,a)\}$ on the two element set $\{a,b\}$ is neither reflexive nor irreflexive. 

Here is an example of a non-reflexive, non-irreflexive relation ``in nature." A subgroup in a group is said to be \emph{self-normalizing} if it is equal to its own normalizer. For a group $G$, define a relation $\mathcal{R}$ on the set of all subgroups of $G$ by declaring $H\mathcal{R}K$ if and only if $H$ is the normalizer of $K$. Notice that every nontrivial group has a subgroup that is not self-normalizing; namely, the trivial subgroup $\{e\}$ consisting of only the identity. Thus, in any nontrivial group $G$, there is a subgroup $H$ of $G$ such that $\neg H\mathcal{R} H$. So the relation $\mathcal{R}$ is non-reflexive. Moreover, since the normalizer of a group $G$ in $G$ is $G$ itself, we have $G\mathcal{R} G$. So $\mathcal{R}$ is non-irreflexive.
%%%%%
%%%%%
\end{document}
