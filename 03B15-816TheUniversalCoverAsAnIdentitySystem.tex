\documentclass[12pt]{article}
\usepackage{pmmeta}
\pmcanonicalname{816TheUniversalCoverAsAnIdentitySystem}
\pmcreated{2013-11-06 2:39:37}
\pmmodified{2013-11-06 2:39:37}
\pmowner{PMBookProject}{1000683}
\pmmodifier{rspuzio}{6075}
\pmtitle{8.1.6 The universal cover as an identity system}
\pmrecord{1}{}
\pmprivacy{1}
\pmauthor{PMBookProject}{6075}
\pmtype{Feature}
\pmclassification{msc}{03B15}

\usepackage{xspace}
\usepackage{amssyb}
\usepackage{amsmath}
\usepackage{amsfonts}
\usepackage{amsthm}
\makeatletter
\newcommand{\base}{\ensuremath{\mathsf{base}}\xspace}
\newcommand{\code}{\ensuremath{\mathsf{code}}\xspace}
\newcommand{\ct}{  \mathchoice{\mathbin{\raisebox{0.5ex}{$\displaystyle\centerdot$}}}             {\mathbin{\raisebox{0.5ex}{$\centerdot$}}}             {\mathbin{\raisebox{0.25ex}{$\scriptstyle\,\centerdot\,$}}}             {\mathbin{\raisebox{0.1ex}{$\scriptscriptstyle\,\centerdot\,$}}}}
\newcommand{\defeq}{\vcentcolon\equiv}  
\def\@dprd#1{\prod_{(#1)}\,}
\def\@dprd@noparens#1{\prod_{#1}\,}
\def\@dsm#1{\sum_{(#1)}\,}
\def\@dsm@noparens#1{\sum_{#1}\,}
\def\@eatprd\prd{\prd@parens}
\def\@eatsm\sm{\sm@parens}
\newcommand{\eqv}[2]{\ensuremath{#1 \simeq #2}\xspace}
\newcommand{\jdeq}{\equiv}      
\def\lam#1{{\lambda}\@lamarg#1:\@endlamarg\@ifnextchar\bgroup{.\,\lam}{.\,}}
\def\@lamarg#1:#2\@endlamarg{\if\relax\detokenize{#2}\relax #1\else\@lamvar{\@lameatcolon#2},#1\@endlamvar\fi}
\def\@lameatcolon#1:{#1}
\def\@lamvar#1,#2\@endlamvar{(#2\,{:}\,#1)}
\newcommand{\lloop}{\ensuremath{\mathsf{loop}}\xspace}
\newcommand{\narrowbreak}{}
\newcommand{\pairpath}{\ensuremath{\mathsf{pair}^{\mathord{=}}}\xspace}
\def\prd#1{\@ifnextchar\bgroup{\prd@parens{#1}}{\@ifnextchar\sm{\prd@parens{#1}\@eatsm}{\prd@noparens{#1}}}}
\def\prd@noparens#1{\mathchoice{\@dprd@noparens{#1}}{\@tprd{#1}}{\@tprd{#1}}{\@tprd{#1}}}
\def\prd@parens#1{\@ifnextchar\bgroup  {\mathchoice{\@dprd{#1}}{\@tprd{#1}}{\@tprd{#1}}{\@tprd{#1}}\prd@parens}  {\@ifnextchar\sm    {\mathchoice{\@dprd{#1}}{\@tprd{#1}}{\@tprd{#1}}{\@tprd{#1}}\@eatsm}    {\mathchoice{\@dprd{#1}}{\@tprd{#1}}{\@tprd{#1}}{\@tprd{#1}}}}}
\newcommand{\refl}[1]{\ensuremath{\mathsf{refl}_{#1}}\xspace}
\def\sm#1{\@ifnextchar\bgroup{\sm@parens{#1}}{\@ifnextchar\prd{\sm@parens{#1}\@eatprd}{\sm@noparens{#1}}}}
\def\sm@noparens#1{\mathchoice{\@dsm@noparens{#1}}{\@tsm{#1}}{\@tsm{#1}}{\@tsm{#1}}}
\def\sm@parens#1{\@ifnextchar\bgroup  {\mathchoice{\@dsm{#1}}{\@tsm{#1}}{\@tsm{#1}}{\@tsm{#1}}\sm@parens}  {\@ifnextchar\prd    {\mathchoice{\@dsm{#1}}{\@tsm{#1}}{\@tsm{#1}}{\@tsm{#1}}\@eatprd}    {\mathchoice{\@dsm{#1}}{\@tsm{#1}}{\@tsm{#1}}{\@tsm{#1}}}}}
\newcommand{\Sn}{\mathbb{S}}
\def\@tprd#1{\mathchoice{{\textstyle\prod_{(#1)}}}{\prod_{(#1)}}{\prod_{(#1)}}{\prod_{(#1)}}}
\newcommand{\trans}[2]{\ensuremath{{#1}_{*}\mathopen{}\left({#2}\right)\mathclose{}}\xspace}
\newcommand{\transfib}[3]{\ensuremath{\mathsf{transport}^{#1}(#2,#3)\xspace}}
\def\@tsm#1{\mathchoice{{\textstyle\sum_{(#1)}}}{\sum_{(#1)}}{\sum_{(#1)}}{\sum_{(#1)}}}
\newcommand{\UU}{\ensuremath{\mathcal{U}}\xspace}
\newcommand{\vcentcolon}{:\!\!}
\newcommand{\Z}{\ensuremath{\mathbb{Z}}\xspace}
\newcounter{mathcount}
\setcounter{mathcount}{1}
\newtheorem{precor}{Corollary}
\newenvironment{cor}{\begin{precor}}{\end{precor}\addtocounter{mathcount}{1}}
\renewcommand{\theprecor}{8.1.\arabic{mathcount}}
\newenvironment{narrowmultline*}{\csname equation*\endcsname}{\csname endequation*\endcsname}
\newtheorem{prethm}{Theorem}
\newenvironment{thm}{\begin{prethm}}{\end{prethm}\addtocounter{mathcount}{1}}
\renewcommand{\theprethm}{8.1.\arabic{mathcount}}
\let\autoref\cref
\let\type\UU
\makeatother

\begin{document}

Note that the fibration $\code:\Sn^1\to\type$ together with $0:\code(\base)$ is a \emph{pointed predicate} in the sense of \autoref{defn:identity-systems}.
From this point of view, we can see that the encode-decode proof in \autoref{subsec:pi1s1-encode-decode} consists of proving that \code satisfies \autoref{thm:identity-systems}\ref{item:identity-systems3}, while the homotopy-theoretic proof in \autoref{subsec:pi1s1-homotopy-theory} consists of proving that it satisfies \autoref{thm:identity-systems}\ref{item:identity-systems4}.
This suggests a third approach.

\begin{thm}
  The pair $(\code,0)$ is an identity system at $\base:\Sn^1$ in the sense of \autoref{defn:identity-systems}.
\end{thm}
\begin{proof}
  Let $D:\prd{x:\Sn^1} \code(x) \to \type$ and $d:D(\base,0)$ be given; we want to define a function $f:\prd{x:\Sn^1}{c:\code(x)} D(x,c)$.
  By circle induction, it suffices to specify $f(\base):\prd{c:\code(\base)} D(\base,c)$ and verify that $\trans{\lloop}{f(\base)} = f(\base)$.

  Of course, $\code(\base)\jdeq \Z$.
  By \autoref{lem:transport-s1-code} and induction on $n$, we may obtain a path $p_n : \transfib{\code}{\lloop^n}{0} = n$ for any integer $n$.
  Therefore, by paths in $\Sigma$-types, we have a path $\pairpath(\lloop^n,p_n) : (\base,0) = (\base,n)$ in $\sm{x:\Sn^1} \code(x)$.
  Transporting $d$ along this path in the fibration $\widehat{D}:(\sm{x:\Sn^1} \code(x)) \to\type$ associated to $D$, we obtain an element of $D(\base,n)$ for any $n:\Z$.
  We define this element to be $f(\base)(n)$:
  \[ f(\base)(n) \defeq \transfib{\widehat{D}}{\pairpath(\lloop^n,p_n)}{d}. \]
  %
  Now we need $\transfib{\lam{x} \prd{c:\code(x)} D(x,c)}{\lloop}{f(\base)} = f(\base)$.
  By \autoref{thm:dpath-forall}, this means we need to show that for any $n:\Z$,
  %
  \begin{narrowmultline*}
    \transfib{\widehat D}{\pairpath(\lloop,\refl{\trans\lloop n})}{f(\base)(n)} 
    =_{D(\base,\trans\lloop n)} \narrowbreak
    f(\base)(\trans\lloop n).
  \end{narrowmultline*}
  %
  Now we have a path $q:\trans\lloop n = n+1$, so transporting along this, it suffices to show
  \begin{multline*}
    \transfib{D(\base)}{q}{\transfib{\widehat D}{\pairpath(\lloop,\refl{\trans\lloop n})}{f(\base)(n)}}\\
    =_{D(\base,n+1)} \transfib{D(\base)}{q}{f(\base)(\trans\lloop n)}.
  \end{multline*}
  By a couple of lemmas about transport and dependent application, this is equivalent to
  \[ \transfib{\widehat D}{\pairpath(\lloop,q)}{f(\base)(n)} =_{D(\base,n+1)} f(\base)(n+1). \]
  However, expanding out the definition of $f(\base)$, we have
  \begin{narrowmultline*}
    \transfib{\widehat D}{\pairpath(\lloop,q)}{f(\base)(n)}
    \narrowbreak
    \begin{aligned}[t]
      &= \transfib{\widehat D}{\pairpath(\lloop,q)}{\transfib{\widehat{D}}{\pairpath(\lloop^n,p_n)}{d}}\\
      &= \transfib{\widehat D}{\pairpath(\lloop^n,p_n) \ct \pairpath(\lloop,q)}{d}\\
      &= \transfib{\widehat D}{\pairpath(\lloop^{n+1},p_{n+1})}{d}\\
      &= f(\base)(n+1).
    \end{aligned}
  \end{narrowmultline*}
  We have used the functoriality of transport, the characterization of composition in $\Sigma$-types (which was an exercise for the reader), and a lemma relating $p_n$ and $q$ to $p_{n+1}$ which we leave it to the reader to state and prove.
  
  This completes the construction of $f:\prd{x:\Sn^1}{c:\code(x)} D(x,c)$.
  Since
  \[f(\base,0) \jdeq \trans{\pairpath(\lloop^0,p_0)}{d} = \trans{\refl{\base}}{d} = d,\]
  we have shown that $(\code,0)$ is an identity system.
\end{proof}

\begin{cor}
  For any $x:\Sn^1$, we have $\eqv{(\base=x)}{\code(x)}$.
\end{cor}
\begin{proof}
  By \autoref{thm:identity-systems}.
\end{proof}

Of course, this proof also contains essentially the same elements as the previous two.
Roughly, we can say that it unifies the proofs of \autoref{thm:pi1s1-decode},\autoref{lem:s1-encode-decode}, performing the requisite inductive argument only once in a generic case.



\end{document}
