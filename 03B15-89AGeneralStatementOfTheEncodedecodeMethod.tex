\documentclass[12pt]{article}
\usepackage{pmmeta}
\pmcanonicalname{89AGeneralStatementOfTheEncodedecodeMethod}
\pmcreated{2013-11-06 15:36:45}
\pmmodified{2013-11-06 15:36:45}
\pmowner{PMBookProject}{1000683}
\pmmodifier{rspuzio}{6075}
\pmtitle{8.9 A general statement of the encode-decode method}
\pmrecord{1}{}
\pmprivacy{1}
\pmauthor{PMBookProject}{6075}
\pmtype{Feature}
\pmclassification{msc}{03B15}

\endmetadata

\usepackage{xspace}
\usepackage{amssyb}
\usepackage{amsmath}
\usepackage{amsfonts}
\usepackage{amsthm}
\makeatletter
\newcommand{\blank}{\mathord{\hspace{1pt}\text{--}\hspace{1pt}}}
\newcommand{\code}{\ensuremath{\mathsf{code}}\xspace}
\newcommand{\decode}{\ensuremath{\mathsf{decode}}\xspace}
\def\@dprd#1{\prod_{(#1)}\,}
\def\@dprd@noparens#1{\prod_{#1}\,}
\def\@dsm#1{\sum_{(#1)}\,}
\def\@dsm@noparens#1{\sum_{#1}\,}
\def\@eatprd\prd{\prd@parens}
\def\@eatsm\sm{\sm@parens}
\newcommand{\encode}{\ensuremath{\mathsf{encode}}\xspace}
\newcommand{\id}[3][]{\ensuremath{#2 =_{#1} #3}\xspace}
\newcommand{\indexdef}[1]{\index{#1|defstyle}}   
\def\prd#1{\@ifnextchar\bgroup{\prd@parens{#1}}{\@ifnextchar\sm{\prd@parens{#1}\@eatsm}{\prd@noparens{#1}}}}
\def\prd@noparens#1{\mathchoice{\@dprd@noparens{#1}}{\@tprd{#1}}{\@tprd{#1}}{\@tprd{#1}}}
\def\prd@parens#1{\@ifnextchar\bgroup  {\mathchoice{\@dprd{#1}}{\@tprd{#1}}{\@tprd{#1}}{\@tprd{#1}}\prd@parens}  {\@ifnextchar\sm    {\mathchoice{\@dprd{#1}}{\@tprd{#1}}{\@tprd{#1}}{\@tprd{#1}}\@eatsm}    {\mathchoice{\@dprd{#1}}{\@tprd{#1}}{\@tprd{#1}}{\@tprd{#1}}}}}
\newcommand{\refl}[1]{\ensuremath{\mathsf{refl}_{#1}}\xspace}
\def\sm#1{\@ifnextchar\bgroup{\sm@parens{#1}}{\@ifnextchar\prd{\sm@parens{#1}\@eatprd}{\sm@noparens{#1}}}}
\def\sm@noparens#1{\mathchoice{\@dsm@noparens{#1}}{\@tsm{#1}}{\@tsm{#1}}{\@tsm{#1}}}
\def\sm@parens#1{\@ifnextchar\bgroup  {\mathchoice{\@dsm{#1}}{\@tsm{#1}}{\@tsm{#1}}{\@tsm{#1}}\sm@parens}  {\@ifnextchar\prd    {\mathchoice{\@dsm{#1}}{\@tsm{#1}}{\@tsm{#1}}{\@tsm{#1}}\@eatprd}    {\mathchoice{\@dsm{#1}}{\@tsm{#1}}{\@tsm{#1}}{\@tsm{#1}}}}}
\newcommand{\Sn}{\mathbb{S}}
\def\@tprd#1{\mathchoice{{\textstyle\prod_{(#1)}}}{\prod_{(#1)}}{\prod_{(#1)}}{\prod_{(#1)}}}
\newcommand{\tproj}[3][]{\mathopen{}\left|#3\right|_{#2}^{#1}\mathclose{}}
\newcommand{\transfib}[3]{\ensuremath{\mathsf{transport}^{#1}(#2,#3)\xspace}}
\newcommand{\trunc}[2]{\mathopen{}\left\Vert #2\right\Vert_{#1}\mathclose{}}
\def\@tsm#1{\mathchoice{{\textstyle\sum_{(#1)}}}{\sum_{(#1)}}{\sum_{(#1)}}{\sum_{(#1)}}}
\newcommand{\UU}{\ensuremath{\mathcal{U}}\xspace}
\newcounter{mathcount}
\setcounter{mathcount}{1}
\newtheorem{prelem}{Lemma}
\newenvironment{lem}{\begin{prelem}}{\end{prelem}\addtocounter{mathcount}{1}}
\renewcommand{\theprelem}{8.9.\arabic{mathcount}}
\let\autoref\cref
\let\type\UU
\makeatother

\begin{document}
\indexdef{encode-decode method}

We have used the encode-decode method to characterize the path spaces
of various types, including coproducts (\cref{thm:path-coprod}), natural
numbers (\cref{thm:path-nat}), truncations (\cref{thm:path-truncation}),
the circle (\cref{{cor:omega-s1}}), suspensions (\autoref{thm:freudenthal}), and pushouts
(\cref{thm:van-Kampen}).  Variants of this technique are used in the
proofs of many of the other theorems mentioned in the introduction to
this chapter, such as a direct proof of $\pi_n(\Sn^n)$, the Blakers--Massey theorem, and the construction of Eilenberg--Mac Lane spaces.
While it is tempting to try to
abstract the method into a lemma, this is difficult because
slightly different variants are needed for different problems.  For
example, different variations on the same method  can be used to
characterize a loop space (as in \cref{thm:path-coprod,cor:omega-s1}) or
a whole path space (as in \cref{thm:path-nat}), to give a complete
characterization of a loop space (e.g.\ $\Omega^1(\Sn ^1)$) or only to
characterize some truncation of it (e.g.\ van Kampen), and to calculate
homotopy groups or to prove that a map is $n$-connected (e.g.\ Freudenthal and
Blakers--Massey).

However, we can state lemmas for specific variants of the method.
The proofs of these lemmas are almost trivial; the main point is to
clarify the method by stating them in generality.  The simplest
case is using an encode-decode method to characterize the loop space of a
type, as in \cref{thm:path-coprod} and \cref{{cor:omega-s1}}.

\begin{lem}[Encode-decode for Loop Spaces]
  \index{loop space}%
Given a pointed type $(A,a_0)$ and a fibration
$\code : A \to \type$, if 
\begin{enumerate}
\item $c_0 : \code(a_0)$,\label{item:ed1}
\item $\decode : \prd{x:A} \code(x) \to (\id{a_0}{x})$,\label{item:ed2}
\item for all $c : \code(a_0)$, \id{\transfib{\code}{\decode(c)}{c_0}}{c}, and\label{item:ed3}
\item $\id{\decode(c_0)}{\refl{}}$,\label{item:ed4}
\end{enumerate}
then $(\id{a_0}{a_0})$ is equivalent to $\code(a_0)$.
\end{lem}

\begin{proof}
Define
$\encode : \prd{x:A} (\id{a_0}{x}) \to \code(x)$ by
\[
\encode_x(\alpha) = \transfib{\code}{\alpha}{c_0}.
\]
We show that $\encode_{a_0}$ and $\decode_{a_0}$ are quasi-inverses.  
The composition $\encode_{a_0} \circ \decode_{a_0}$ is immediate by
assumption~\ref{item:ed3}.  For the other composition, we show
\[
\prd{x:A}{p : \id{a_0}{x}} \id{\decode_{x} (\encode_{x} p)}{p}.
\] 
By path induction, it suffices to show 
$\id{{\decode_{{a_0}} (\encode_{{a_o}} \refl{})}}{\refl{}}$.
After reducing the $\mathsf{transport}$, it suffices to show 
$\id{{\decode_{{a_0}} (c_0)}}{\refl{}}$, which is assumption~\ref{item:ed4}.
\end{proof}

If a fiberwise equivalence between $(\id{a_0}{\blank})$ and $\code$ is desired,
it suffices to strengthen condition (iii) to
\[
\prd{x:A}{c : \code(x)} \id{\encode_{x}(\decode_{x}(c))}{c}.a
\]
However, to calculate a loop space (e.g. $\Omega(\Sn ^1)$), this
stronger assumption is not necessary.  

Another variation, which comes up often when calculating homotopy
groups, characterizes the truncation of a loop space:

\begin{lem}[Encode-decode for Truncations of Loop Spaces]
Assume a pointed type $(A,a_0)$ and a fibration
$\code : A \to \type$, where for every $x$, $\code(x)$ is a $k$-type.
Define 
\[
\encode : \prd{x:A} \trunc{k}{\id{a_0}{x}} \to \code(x).
\]
by truncation recursion (using the fact
that $\code(x)$ is a $k$-type), mapping $\alpha : \id{a_0}{x}$ to 
\transfib{\code}{\alpha}{c_0}. Suppose:
\begin{enumerate}
\item $c_0 : \code(a_0)$,
\item $\decode : \prd{x:A} \code(x) \to \trunc{k}{\id{a_0}{x}}$,
\item \label{item:decode-encode-loop-iii}
  $\id{\encode_{a_0}(\decode_{a0}(c))}{c}$ for all $c : \code(a_0)$, and
\item \label{item:decode-encode-loop-iv}
  $\id{\decode(c_0)}{\tproj{}{\refl{}}}$.
\end{enumerate}
Then $\trunc{k}{\id{a_0}{a_0}}$ is equivalent to $\code(a_0)$.
\end{lem}

\begin{proof}
That $\decode \circ \encode$ is identity is immediate by \ref{item:decode-encode-loop-iii}.
%
To prove $\encode \circ \decode$, we first do a truncation induction, by
which it suffices to show
\[
\prd{x:A}{p : \id{a_0}{x}} \id{\decode_{x}(\encode_{x}(\tproj{k}{p}))}{\tproj{k}{p}}.
\] 
The truncation induction is allowed because paths in a $k$-type are a
$k$-type.  To show this type, we do a path induction, and after reducing
the \encode, use assumption~\ref{item:decode-encode-loop-iv}.
\end{proof}


\end{document}
