\documentclass[12pt]{article}
\usepackage{pmmeta}
\pmcanonicalname{Contradiction}
\pmcreated{2013-03-22 16:02:48}
\pmmodified{2013-03-22 16:02:48}
\pmowner{Wkbj79}{1863}
\pmmodifier{Wkbj79}{1863}
\pmtitle{contradiction}
\pmrecord{9}{38097}
\pmprivacy{1}
\pmauthor{Wkbj79}{1863}
\pmtype{Definition}
\pmcomment{trigger rebuild}
\pmclassification{msc}{03F07}
\pmclassification{msc}{03B05}
\pmrelated{ContradictoryStatement}
\pmdefines{proof by contradiction}
\pmdefines{reductio ad absurdum}

% this is the default PlanetMath preamble.  as your knowledge
% of TeX increases, you will probably want to edit this, but
% it should be fine as is for beginners.

% almost certainly you want these
\usepackage{amssymb}
\usepackage{amsmath}
\usepackage{amsfonts}

% used for TeXing text within eps files
%\usepackage{psfrag}
% need this for including graphics (\includegraphics)
%\usepackage{graphicx}
% for neatly defining theorems and propositions
%\usepackage{amsthm}
% making logically defined graphics
%%%\usepackage{xypic}

% there are many more packages, add them here as you need them

% define commands here

\begin{document}
A \emph{contradiction} occurs when the statements $p$ and $\neg p$ are shown to be true simultaneously.  This concept appears most often in a \emph{proof by contradiction} (also known as \emph{reductio ad absurdum}), which is proving a statement by supposing its negation is true and logically deducing an absurd statement.  That is, in attempting to prove $q$, one may assume $\neg q$ and attempt to obtain a statement of the form $\neg r$, where $r$ is a statement that is assumed or known to be true.

Proofs by contradiction can become confusing.  This is especially the case when such proofs are nested; \PMlinkname{i.e.}{Ie}, a proof by contradiction occurs within a proof by contradiction.  Some mathematicians prefer to use a direct proof whenever possible, as such \PMlinkescapetext{arguments} are easier to follow in general.  A small minority of mathematicians go so far as to reject proof by contradiction as a valid proof technique.  It should be pointed out that something good can be said for proof by contradiction:  If one wants to prove a statement of the form $p \implies q$, using the technique of proof by contradiction gives an additional hypothesis with which to work.
%%%%%
%%%%%
\end{document}
