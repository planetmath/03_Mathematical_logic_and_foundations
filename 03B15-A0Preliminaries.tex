\documentclass[12pt]{article}
\usepackage{pmmeta}
\pmcanonicalname{A0Preliminaries}
\pmcreated{2013-11-09 3:26:43}
\pmmodified{2013-11-09 3:26:43}
\pmowner{PMBookProject}{1000683}
\pmmodifier{PMBookProject}{1000683}
\pmtitle{A.0 Preliminaries}
\pmrecord{1}{}
\pmprivacy{1}
\pmauthor{PMBookProject}{1000683}
\pmtype{Feature}
\pmclassification{msc}{03B15}

\endmetadata

\usepackage{amssyb}
\usepackage{amsmath}
\usepackage{amsfonts}
\usepackage{amsthm}
\makeatletter
\newcommand{\ctx}{\ensuremath{\mathsf{ctx}}}
\newcommand{\define}[1]{\textbf{#1}}
\newcommand{\emptyctx}{\ensuremath{\cdot}}
\newcommand{\indexdef}[1]{\index{#1|defstyle}}   
\newcommand{\jdeq}{\equiv}      
\newcommand{\jdeqtp}[4]{#1 \vdash #2 \jdeq #3 : #4}
\def\lam#1{{\lambda}\@lamarg#1:\@endlamarg\@ifnextchar\bgroup{.\,\lam}{.\,}}
\def\@lamarg#1:#2\@endlamarg{\if\relax\detokenize{#2}\relax #1\else\@lamvar{\@lameatcolon#2},#1\@endlamvar\fi}
\def\@lameatcolon#1:{#1}
\def\@lamvar#1,#2\@endlamvar{(#2\,{:}\,#1)}
\newcommand{\oftp}[3]{#1 \vdash #2 : #3}
\newcommand{\wfctx}[1]{#1\ \ctx}
\let\autoref\cref
\makeatother
\begin{document}
In \PMlinkexternal{Chapter 1}{http://planetmath.org/node/87533}, we presented the two basic \define{judgments}
\index{judgment}
of type theory. The first, $a:A$, asserts that a term $a$ has type $A$. The second,
$a\jdeq b:A$, states that the two terms $a$ and $b$ are \define{judgmentally
equal}
\index{equality!judgmental}
\index{judgmental equality}
at type $A$. These judgments are inductively defined by a set of
inference rules described in \autoref{sec:syntax-more-formally}.

To construct an element $a$ of a type $A$ is to derive $a:A$; in the book, we
give informal arguments which describe the construction of $a$, but formally,
one must specify a precise term $a$ and a full derivation that $a:A$.

However, the main difference between the presentation of type theory in the book
and in this appendix is that here judgments are explicitly
formulated in an ambient \define{context},
\index{context}
or list of assumptions, of the form
\[
  x_1:A_1, x_2:A_2,\dots,x_n:A_n.
\]
An element $x_i : A_i$ of the context expresses the assumption that the
variable
\index{variable}%
$x_i$ has type $A_i$. The variables $x_1, \ldots, x_n$ appearing in
the context must be distinct. We abbreviate contexts with the letters $\Gamma$
and $\Delta$.

The judgment $a:A$ in context $\Gamma$ is written 
\[ \oftp\Gamma aA \]
and means that $a:A$ under the assumptions listed in $\Gamma$. When the list of
assumptions is empty, we write simply
\[ \oftp{}aA \]
or
\[ \oftp\emptyctx aA \]
where $\emptyctx$ denotes the empty context. The same applies to the equality
judgment
\[
  \jdeqtp\Gamma{a}{b}{A}
\]

However, such judgments are sensible only for \define{well-formed} contexts,
\index{context!well-formed}%
a notion captured by our third and final judgment
\[
  \wfctx{(x_1:A_1, x_2:A_2,\dots,x_n:A_n)}
\]
expressing that each $A_i$ is a type in the context $x_1:A_1,
x_2:A_2,\dots,x_{i-1}:A_{i-1}$.  In particular, therefore, if $\oftp\Gamma aA$ and
$\wfctx\Gamma$, then we know that each $A_i$ contains only the variables
$x_1,\dots,x_{i-1}$, and that $a$ and $A$ contain only the variables
$x_1,\dots,x_n$.
\index{variable!in context}

In informal mathematical presentations, the context is
implicit. At each point in a proof, the mathematician knows which
variables are available and what types they have, either by historical
convention ($n$ is usually a number, $f$ is a function, etc.) or
because variables are explicitly introduced with sentences such as
``let $x$ be a real number''. We discuss some benefits of using explicit
contexts in \autoref{sec:more-formal-pi},\autoref{sec:more-formal-sigma}.

We write $B[a/x]$ for the \define{substitution}
\index{substitution}%
of a term $a$ for free occurrences of
the variable~$x$ in the term $B$, with possible capture-avoiding
renaming of bound variables,
\index{variable!and substitution}%
as discussed in
\PMlinkname{\S 1.2}{12functiontypes}. The general form of substitution
%
\[
   B[a_1,\dots,a_n/x_1,\dots,x_n]
\]
%
substitutes expressions $a_1,\dots,a_n$ for the variables
$x_1,\dots,x_n$ simultaneously.

To \define{bind a variable $x$ in an expression $B$}
\indexdef{variable!bound}%
means to incorporate both of them into a larger expression, called an \define{abstraction},
\indexdef{abstraction}%
whose purpose is to express the fact that $x$ is ``local'' to $B$, i.e., it
is not to be confused with other occurrences of $x$ appearing
elsewhere. Bound variables are familiar to programmers, but less so to mathematicians.
Various notations are used for binding, such as $x \mapsto B$,
$\lam x B$, and $x \,.\, B$, depending on the situation. We may write $C[a]$ for the
substitution of a term $a$ for the variable in the abstracted expression, i.e.,
we may define $(x.B)[a]$ to be $B[a/x]$. As discussed in
\PMlinkname{\S 1.2}{12functiontypes}, changing the name of a bound variable everywhere within an expression (``$\alpha$-conversion'')
\index{alpha-conversion@$\alpha $-conversion}%
does not change the expression. Thus, to be very
precise, an expression is an equivalence class of syntactic forms
which differ in names of bound variables.

One may also regard each variable $x_i$ of a judgment
\[
  x_1:A_1, x_2:A_2,\dots,x_n:A_n \vdash a : A
\]
to be bound in its \define{scope},
\indexdef{variable!scope of}%
\index{scope}%
consisting of the expressions $A_{i+1},
\ldots, A_n$, $a$, and $A$.

\end{document}
