\documentclass[12pt]{article}
\usepackage{pmmeta}
\pmcanonicalname{ExampleOfAnOperator}
\pmcreated{2013-03-22 15:50:04}
\pmmodified{2013-03-22 15:50:04}
\pmowner{Mathprof}{13753}
\pmmodifier{Mathprof}{13753}
\pmtitle{example of an operator}
\pmrecord{8}{37808}
\pmprivacy{1}
\pmauthor{Mathprof}{13753}
\pmtype{Example}
\pmcomment{trigger rebuild}
\pmclassification{msc}{03-00}

\endmetadata

% this is the default PlanetMath preamble.  as your knowledge
% of TeX increases, you will probably want to edit this, but
% it should be fine as is for beginners.

% almost certainly you want these
\usepackage{amssymb}
\usepackage{amsmath}
\usepackage{amsfonts}

% used for TeXing text within eps files
%\usepackage{psfrag}
% need this for including graphics (\includegraphics)
%\usepackage{graphicx}
% for neatly defining theorems and propositions
%\usepackage{amsthm}
% making logically defined graphics
%%%\usepackage{xypic}

% there are many more packages, add them here as you need them

% define commands here
\begin{document}
While the \PMlinkescapetext{words} operator and function are  often used as synonyms, it is customary in some \PMlinkescapetext{branches} of mathematics to use the \PMlinkescapetext{word} operator in \PMlinkescapetext{place} of mapping, function or transformation. In particular, in linear algebra a transformation $T: V \to W$ is referred to as an operator on $V$ if $W = V$. In general, in the \PMlinkescapetext{branches} of Algebra and \PMlinkescapetext{Functional} Analysis the \PMlinkescapetext{term} operator is preferred for transformations whose codomain is a subset of its domain.
%%%%%
%%%%%
\end{document}
