\documentclass[12pt]{article}
\usepackage{pmmeta}
\pmcanonicalname{Transitive1}
\pmcreated{2013-03-22 12:14:16}
\pmmodified{2013-03-22 12:14:16}
\pmowner{akrowne}{2}
\pmmodifier{akrowne}{2}
\pmtitle{transitive}
\pmrecord{5}{31618}
\pmprivacy{1}
\pmauthor{akrowne}{2}
\pmtype{Definition}
\pmcomment{trigger rebuild}
\pmclassification{msc}{03B05}
\pmsynonym{transitive property}{Transitive1}

\endmetadata

\usepackage{amssymb}
\usepackage{amsmath}
\usepackage{amsfonts}

%\usepackage{psfrag}
%\usepackage{graphicx}
%%%%\usepackage{xypic}
\begin{document}
The \emph{transitive property} of logic is 

$$ (a \Rightarrow b) \land (b \Rightarrow c) \Rightarrow (a \Rightarrow c) $$

Where $\Rightarrow$ is the conditional truth function.  From this we can derive that

$$ (a = b) \land (b = c) \Rightarrow (a = c) $$
%%%%%
%%%%%
%%%%%
\end{document}
