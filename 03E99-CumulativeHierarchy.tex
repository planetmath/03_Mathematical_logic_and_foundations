\documentclass[12pt]{article}
\usepackage{pmmeta}
\pmcanonicalname{CumulativeHierarchy}
\pmcreated{2013-03-22 16:18:43}
\pmmodified{2013-03-22 16:18:43}
\pmowner{yark}{2760}
\pmmodifier{yark}{2760}
\pmtitle{cumulative hierarchy}
\pmrecord{8}{38436}
\pmprivacy{1}
\pmauthor{yark}{2760}
\pmtype{Definition}
\pmcomment{trigger rebuild}
\pmclassification{msc}{03E99}
\pmsynonym{iterative hierarchy}{CumulativeHierarchy}
\pmsynonym{Zermelo hierarchy}{CumulativeHierarchy}
\pmrelated{CriterionForASetToBeTransitive}
\pmrelated{ExampleOfUniverse}
\pmdefines{rank}
\pmdefines{rank of a set}

\usepackage{amssymb}
\begin{document}
The \emph{cumulative hierarchy} of sets
is defined by transfinite recursion as follows:
we define $V_0=\varnothing$
and for each ordinal $\alpha$ we define $V_{\alpha+1}=\mathcal{P}(V_\alpha)$
and for each limit ordinal $\delta$ we define
$V_\delta=\bigcup_{\alpha\in\delta}V_\alpha$.

Every set is a subset of $V_\alpha$ for some ordinal $\alpha$,
and the least such $\alpha$ is called the \emph{rank} of the set.
It can be shown that the rank of an ordinal is itself,
and in general the rank of a set $X$
is the least ordinal greater than the rank of every element of $X$.
For each ordinal $\alpha$,
the set $V_\alpha$ is the set of all sets of rank less than $\alpha$,
and $V_{\alpha+1}\setminus V_\alpha$ is the set of all sets of rank $\alpha$.

Note that the previous paragraph makes use of the Axiom of Foundation:
if this axiom fails,
then there are sets that are not subsets of any $V_\alpha$
and therefore have no rank.
The previous paragraph also assumes that we are using a set theory such as ZF,
in which elements of sets are themselves sets.

Each $V_\alpha$ is a transitive set.
Note that $V_0=0$, $V_1=1$ and $V_2=2$,
but for $\alpha>2$ the set $V_\alpha$ is never an ordinal,
because it has the element $\{1\}$, which is not an ordinal.
%%%%%
%%%%%
\end{document}
