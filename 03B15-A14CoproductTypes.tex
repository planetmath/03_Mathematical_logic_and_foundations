\documentclass[12pt]{article}
\usepackage{pmmeta}
\pmcanonicalname{A14CoproductTypes}
\pmcreated{2013-11-09 4:51:41}
\pmmodified{2013-11-09 4:51:41}
\pmowner{PMBookProject}{1000683}
\pmmodifier{PMBookProject}{1000683}
\pmtitle{A.1.4 Coproduct types}
\pmrecord{1}{}
\pmprivacy{1}
\pmauthor{PMBookProject}{1000683}
\pmtype{Application}
\pmclassification{msc}{03B15}

\endmetadata

\usepackage{xspace}
\usepackage{amssyb}
\usepackage{amsmath}
\usepackage{amsfonts}
\usepackage{amsthm}
\makeatletter
\newcommand{\defeq}{\vcentcolon\equiv}  
\newcommand{\inl}{\ensuremath\inlsym\xspace}
\newcommand{\inlsym}{{\mathsf{inl}}}
\newcommand{\inr}{\ensuremath\inrsym\xspace}
\newcommand{\inrsym}{{\mathsf{inr}}}
\def\tprd#1{\@tprd{#1}\@ifnextchar\bgroup{\tprd}{}}
\def\@tprd#1{\mathchoice{{\textstyle\prod_{(#1)}}}{\prod_{(#1)}}{\prod_{(#1)}}{\prod_{(#1)}}}
\newcommand{\UU}{\ensuremath{\mathcal{U}}\xspace}
\newcommand{\vcentcolon}{:\!\!}
\makeatother
\begin{document}
We introduce primitive constants $c_+$, $c_\inlsym$, and $c_\inrsym$.
We write $A+B$ instead of $c_+(A,B)$, $\inl(a)$ instead of
$c_\inlsym(a)$, and $\inr(a)$ instead of $c_\inrsym(a)$:
%
\begin{itemize}
\item if $A,B : \UU_n$ then $A + B : \UU_n$
\item moreover, $\inl: A \rightarrow A+B$ and $\inr: B \rightarrow A+B$
\end{itemize}
%
If we have $A$ and $B$ as above, $C : A+B \rightarrow \UU_m$, 
$d:\tprd{x:A} C(\inl(x))$, and $e:\tprd{y:B} C(\inr(y))$,
then we can introduce a defined constant $f:\tprd{z:A+B}C(z)$ with the defining equations
%
\begin{equation*}
  f(\inl(x)) \defeq d(x)
  \qquad\text{and}\qquad
  f(\inr(y)) \defeq e(y).
\end{equation*}
\end{document}
