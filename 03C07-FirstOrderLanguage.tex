\documentclass[12pt]{article}
\usepackage{pmmeta}
\pmcanonicalname{FirstOrderLanguage}
\pmcreated{2013-03-22 12:42:46}
\pmmodified{2013-03-22 12:42:46}
\pmowner{CWoo}{3771}
\pmmodifier{CWoo}{3771}
\pmtitle{first order language}
\pmrecord{28}{32999}
\pmprivacy{1}
\pmauthor{CWoo}{3771}
\pmtype{Definition}
\pmcomment{trigger rebuild}
\pmclassification{msc}{03C07}
\pmclassification{msc}{03B10}
\pmsynonym{auxiliary symbol}{FirstOrderLanguage}
\pmsynonym{first-order language}{FirstOrderLanguage}
\pmrelated{Type2}
\pmrelated{Language}
\pmrelated{AtomicFormula}
\pmdefines{first order language}
\pmdefines{term}
\pmdefines{formula}

% this is the default PlanetMath preamble.  as your knowledge
% of TeX increases, you will probably want to edit this, but
% it should be fine as is for beginners.

% almost certainly you want these
\usepackage{amssymb}
\usepackage{amsmath}
\usepackage{amsfonts}

% used for TeXing text within eps files
%\usepackage{psfrag}
% need this for including graphics (\includegraphics)
%\usepackage{graphicx}
% for neatly defining theorems and propositions
%\usepackage{amsthm}
% making logically defined graphics
\usepackage[arrow,curve,poly,arc,2cell,frame,web]{xypic}

% there are many more packages, add them here as you need them

% define commands here
\newcommand{\br}{[\![}
\newcommand{\rb}{]\!]}
\newcommand{\oq}{\text{``}}
\newcommand{\cq}{\text{''}}


\newcommand{\im}{\mathbf{Im}}
\newcommand{\dom}{\mathbf{Dom}}


\newcommand{\Or}{\vee}
\newcommand{\Implies}{\Rightarrow}
\newcommand{\Iff}{\Leftrightarrow}
\newcommand{\proves}{\vdash}
\renewcommand{\And}{\wedge}
\newcommand{\Sup}{\bigwedge}
\newcommand{\Inf}{\bigvee}
\newcommand{\Z}{\mathbb{Z}}
\newcommand{\F}{\mathbb{F}}
\newcommand{\Q}{\mathbb{Q}}
\newcommand{\R}{\mathbb{R}}
\newcommand{\C}{\mathbb{C}}
\newcommand{\Nat}{\mathbb{N}}
\newcommand{\M}{\mathfrak{M}}
\newcommand{\N}{\mathfrak{N}}
\newcommand{\A}{\mathfrak{A}}
\newcommand{\B}{\mathfrak{B}}
\newcommand{\K}{\mathfrak{K}}
\newcommand{\G}{\mathbb{G}}
\newcommand{\Def}{\overset{\operatorname{def}}{:=}}



\newcommand{\spec}{\text{{\bf Spec}}}
\newcommand{\stab}{\text{{\bf Stab}}}
\newcommand{\ann}{\text{{\bf Ann}}}
\newcommand{\irr}{\text{{\bf Irr}}}
\newcommand{\qt}{\text{{\bf Qt}}}
\newcommand{\st}{\mathcal{Qt}}
\newcommand{\ro}{\mathbf{r.o.}}


\newcommand{\Endo}{\text{{\bf End}}}
\newcommand{\mat}{\text{{\bf Mat}}}
\newcommand{\der}{\text{{\bf Der}}}
\newcommand{\rad}{\text{{\bf Rad}}}
\newcommand{\trd}{\text{{\bf tr.d.}}}
\newcommand{\cl}{\text{{\bf acl}}}
\newcommand{\Int}{\text{{\bf int}}}
\newcommand{\V}{\mathbb{V}}
\newcommand{\D}{\mathbf{D}}

\newcommand{\del}{\partial}
\renewcommand{\O}{\mathcal{O}}
\newcommand{\aut}{\mathbf{Aut}}
\newcommand{\height}{\text{\bf Height}}
\newcommand{\coheight}{\text{\bf Co-height}}

\newcommand{\lcm}{\operatorname{lcm}}

\newcommand{\Gal}{\operatorname{Gal}}
\newcommand{\x}{\mathbf{x}}
\newcommand{\y}{\mathbf{y}}
\newcommand{\inner}[2]{\langle #1|#2\rangle}
\renewcommand{\r}{{r}}
\renewcommand{\t}{{t}}

\newcommand{\restr}{\upharpoonright}
\newcommand{\Matrix}[4]{\left(\begin{array}{cc} #1 & #2 \\ #3 & #4 
\end{array}\right)}
\begin{document}
Let $\Sigma$ be a signature.  The \emph{first order language} $\operatorname{FO}(\Sigma)$ on $\Sigma$ contains the following:
\begin{enumerate}
\item the set $S(\Sigma)$ of \emph{symbols} of $\operatorname{FO}(\Sigma)$, which is the disjoint union of the following sets:
\begin{enumerate}
\item $\Sigma$ (the \emph{non-logical symbols}), 
\item a countably infinite set $V$ of variables, 
\item the set of logical symbols $\lbrace \And, \Or, \neg, \Implies, \Iff, \forall, \exists \rbrace$, 
\item the singleton consisting of the equality symbol $\lbrace =\rbrace$, and 
\item the set of parentheses (left and right) $\lbrace (, )\rbrace$;
\end{enumerate}
\item the set $T(\Sigma)$ of \emph{terms} of $\operatorname{FO}(\Sigma)$, which is built inductively from $S(\Sigma)$, as follows: 
\begin{enumerate}
\item Any variable $v\in V$ is a term;
\item Any constant symbol in $\Sigma$ is a term;
\item If $f$ is an $n$-ary function symbol in $\Sigma$, and $t_1,...,t_n$ are
terms, then $f(t_1,...,t_n)$ is a term.
\end{enumerate}
\item the set $F(\Sigma)$ of \emph{formulas} of $\operatorname{FO}(\Sigma)$, which is built inductively from $T(\Sigma)$, as follows:
\begin{enumerate}
\item If $t_1$ and $t_2$ are terms, then $(t_1=t_2)$ is a formula;
\item If $R$ is an $n$-ary relation symbol and $t_1,...,t_n$ are
terms, then $(R(t_1,...,t_n))$ is a formula;
\item If $\varphi$ is a formula, then so is $(\neg\varphi)$;
\item If $\varphi$ and $\psi$ are formulas, then so is $(\varphi\Or\psi)$;
\item If $\varphi$ is a formula, and $x$ is a variable, then $(\exists x(\varphi))$ is a formula.
\end{enumerate}
\end{enumerate}
In other words, $T(\Sigma)$ and $F(\Sigma)$ are the smallest sets, among all sets satisfying the conditions given for terms and formulas, respectively.

Formulas in 3(a) and 3(b), which do not contain any logical connectives, are called the \emph{atomic formulas}.

For example, in the first order language of partially ordered rings, expressions such as 
$$0,\quad x^2,\quad\mbox{ and } \quad y+zx$$ are terms, while 
$$(x=xy),\quad (x+y \le yz),\quad \mbox{ and }\quad (\exists x ((x\le 0) \Or (0\le x)))$$ are formulas, and the first two of which are atomic.

\textbf{Remarks}.
\begin{enumerate}
\item
Generally, one omits parentheses in formulas, when there is no ambiguity.  For example, a formula $(\varphi)$ can be simply written $\varphi$.  As such, the parentheses are also called the \emph{auxiliary symbols}.
\item
The other logical symbols are obtained in the following way :
\begin{alignat*}
\varphi\varphi\And\psi&\Def\neg(\neg\varphi\Or\neg\psi)&\qquad
\varphi\Implies\psi&\Def\neg\varphi\Or\psi\\
\varphi\Iff\psi&\Def(\varphi\Implies\psi)\And(\psi\Implies\varphi)&\qquad
\forall x(\varphi)&\Def\neg(\exists x(\neg\varphi))
\end{alignat*}
where $\varphi$ and $\psi$ are formulas.  All logical symbols are used when building formulas.
\item
In the literature, it is a common practice to write $\Sigma_{\omega \omega}$ for $\operatorname{FO}(\Sigma)$.  The first subscript means that every formula in $\operatorname{FO}(\Sigma)$ contains a finite number of $\vee$'s (less than $\omega$), while the second subscript signifies that every formula has a finite number of $\exists$'s.  In general, $\Sigma_{\alpha\beta}$ denotes a language built from $\Sigma$ such that, in any given formula, the number of occurrences of $\vee$ is less than $\alpha$ and the number of occurrences of $\exists$ is less than $\beta$.  When the number of occurrences of $\vee$ (or $\exists$) in a formula is not limited, we use the symbol $\infty$ in place of $\alpha$ (or $\beta$).  Clearly, if $\alpha$ and $\beta$ are not $\omega$, we get a language that is not first-order.
\end{enumerate}

\subsubsection*{First Order Languages as Formal Languages}

If the signature $\Sigma$ and the set $V$ of variables are countable, then $S(\Sigma), T(\Sigma)$, and $F(\Sigma)$ can be viewed as formal languages over a certain (finite) alphabet $\Gamma$.  The set $\Gamma$ should include all of the logical connectives, the equality symbol, and the parentheses, as well as the following symbols $$R,F,V,I,\#,$$ where they are used to form words for relation, formula, and variable symbols.  More precisely, 
\begin{itemize}
\item $VI^n\#$ stands for the variable $v_n$,
\item $RI^n\#I^m\#$ stands for the $m$-th relation symbol of arity $n$, and
\item $FI^n\#I^m\#$ stands for the $m$-th function symbol of arity $n$,
\end{itemize}
where $m,n\ge 0$ are integers.  The symbol $\#$ is used as a delimiter or separator.  Note that the constant symbols are then words of the form $F\#I^m\#$.  It can shown that $S(\Sigma), T(\Sigma)$ and $F(\Sigma)$ are context-free over $\Gamma$, and in fact unambiguous.

\begin{thebibliography}{99}
\bibitem{H}
W.~Hodges, {\it A Shorter Model Theory}, Cambridge University Press, (1997).
\bibitem{M}
D.~Marker, {\it Model Theory, An Introduction}, Springer, (2002).
\end{thebibliography}
%%%%%
%%%%%
\end{document}
