\documentclass[12pt]{article}
\usepackage{pmmeta}
\pmcanonicalname{GolombsSequence}
\pmcreated{2013-03-22 17:47:07}
\pmmodified{2013-03-22 17:47:07}
\pmowner{PrimeFan}{13766}
\pmmodifier{PrimeFan}{13766}
\pmtitle{Golomb's sequence}
\pmrecord{4}{40245}
\pmprivacy{1}
\pmauthor{PrimeFan}{13766}
\pmtype{Definition}
\pmcomment{trigger rebuild}
\pmclassification{msc}{03E10}
\pmsynonym{Golomb sequence}{GolombsSequence}
\pmsynonym{Silverman's sequence}{GolombsSequence}
\pmsynonym{Silverman sequence}{GolombsSequence}

% this is the default PlanetMath preamble.  as your knowledge
% of TeX increases, you will probably want to edit this, but
% it should be fine as is for beginners.

% almost certainly you want these
\usepackage{amssymb}
\usepackage{amsmath}
\usepackage{amsfonts}

% used for TeXing text within eps files
%\usepackage{psfrag}
% need this for including graphics (\includegraphics)
%\usepackage{graphicx}
% for neatly defining theorems and propositions
%\usepackage{amsthm}
% making logically defined graphics
%%%\usepackage{xypic}

% there are many more packages, add them here as you need them

% define commands here

\begin{document}
{\em Golomb's sequence} is a self-referential ascending order sequence of integers $a$ in which the number of times each number $n$ occurrs is given by $a_n$. The sequence begins 1, 2, 2, 3, 3, 4, 4, 4, 5, 5, 5, 6, 6, 6, 6, 7, 7, 7, 7, 8, 8, 8, 8, 9, 9, 9, 9, 9, 10, 10, 10, 10, 10, ... (see A001462 in Sloane's OEIS). For example, $a_3 = 2$ and indeed 3 occurs twice in the sequence. $a_4 = 3$, so 4 occurs thrice.

The recurrence relation for the sequence is $a_1 = 1$ and $a_n = a_{n - a_{a_{a_{n - 1}}}} + 1$. Sometimes recurrence relations for this sequence are given which also explicitly set $a_2 = 2$. The $n$th term of the sequence can be obtained by the calculating $\phi^{2 - \phi} n^{\phi - 1}$ (with $$\phi = \frac{1 + \sqrt{5}}{2}$$ being the golden ratio) and rounding off to the nearest integer. For example, with precision to five decimal places: 1.20178, 1.84448, 2.36975, 2.83087, 3.24948, 3.63706, 4.0006, 4.34477, 4.67284, 4.98724, 5.28984, 5.58209, 5.86518, 6.14005, 6.40753, 6.66827, 6.92286, 7.17178, 7.41548, etc.
%%%%%
%%%%%
\end{document}
