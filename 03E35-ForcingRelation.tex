\documentclass[12pt]{article}
\usepackage{pmmeta}
\pmcanonicalname{ForcingRelation}
\pmcreated{2013-03-22 12:53:28}
\pmmodified{2013-03-22 12:53:28}
\pmowner{Henry}{455}
\pmmodifier{Henry}{455}
\pmtitle{forcing relation}
\pmrecord{5}{33238}
\pmprivacy{1}
\pmauthor{Henry}{455}
\pmtype{Definition}
\pmcomment{trigger rebuild}
\pmclassification{msc}{03E35}
\pmclassification{msc}{03E40}
\pmrelated{Forcing}
\pmdefines{forcing relation}
\pmdefines{forces}

% this is the default PlanetMath preamble.  as your knowledge
% of TeX increases, you will probably want to edit this, but
% it should be fine as is for beginners.

% almost certainly you want these
\usepackage{amssymb}
\usepackage{amsmath}
\usepackage{amsfonts}

% used for TeXing text within eps files
%\usepackage{psfrag}
% need this for including graphics (\includegraphics)
%\usepackage{graphicx}
% for neatly defining theorems and propositions
%\usepackage{amsthm}
% making logically defined graphics
%%%\usepackage{xypic}

% there are many more packages, add them here as you need them

% define commands here
%\PMlinkescapeword{theory}
\begin{document}
If $\mathfrak{M}$ is a transitive model of set theory and $P$ is a partial order then we can define a \emph{forcing relation}:
$$p\Vdash_P \phi(\tau_1,\ldots,\tau_n)$$
($p$ \emph{forces} $\phi(\tau_1,\ldots,\tau_n)$)

for any $p\in P$, where $\tau_1,\ldots,\tau_n$ are $P$- names.

Specifically, the relation holds if for every generic filter $G$ over $P$ which contains $p$, $$\mathfrak{M}[G]\vDash \phi(\tau_1[G],\ldots,\tau_n[G])$$

That is, $p$ forces $\phi$ if every \PMlinkescapetext{extension} of $\mathfrak{M}$ by a generic filter over $P$ containing $p$ makes $\phi$ true.

If $p\Vdash_P \phi$ holds for every $p\in P$ then we can write $\Vdash_P\phi$ to mean that for any generic $G\subseteq P$, $\mathfrak{M}[G]\vDash\phi$.
%%%%%
%%%%%
\end{document}
