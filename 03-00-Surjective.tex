\documentclass[12pt]{article}
\usepackage{pmmeta}
\pmcanonicalname{Surjective}
\pmcreated{2013-03-22 12:32:48}
\pmmodified{2013-03-22 12:32:48}
\pmowner{drini}{3}
\pmmodifier{drini}{3}
\pmtitle{surjective}
\pmrecord{7}{32791}
\pmprivacy{1}
\pmauthor{drini}{3}
\pmtype{Definition}
\pmcomment{trigger rebuild}
\pmclassification{msc}{03-00}
\pmsynonym{onto}{Surjective}
\pmrelated{TypesOfHomomorphisms}
\pmrelated{InjectiveFunction}
\pmrelated{Bijection}
\pmrelated{Function}
\pmrelated{OneToOneFunctionFromOntoFunction}
\pmdefines{surjection}

%\usepackage{graphicx}
%%%\usepackage{xypic} 
\usepackage{bbm}
\newcommand{\Z}{\mathbbmss{Z}}
\newcommand{\C}{\mathbbmss{C}}
\newcommand{\R}{\mathbbmss{R}}
\newcommand{\Q}{\mathbbmss{Q}}
\newcommand{\mathbb}[1]{\mathbbmss{#1}}
\begin{document}
A function $f\colon X\to Y$ is called \emph{surjective} or \emph{onto} if, for every $y\in Y$, there is an $x\in X$ such that $f(x)=y$.

Equivalently, $f\colon X\to Y$ is onto when its image is all the codomain:
$$\mathrm{Im} f= Y.$$

\subsubsection*{Properties}
\begin{enumerate}
\item If $f\colon X\to Y$ is any function, then $f\colon X\to f(X)$ is
      a surjection. That is, by restricting the codomain, 
      any function induces a surjection. 
\item The composition of surjective functions (when defined) is 
      again a surjective function. 
\item If $f\colon X\to Y$ is a surjection and $B\subseteq Y$, then 
(see \PMlinkname{this page}{InverseImage})
$$
   f f^{-1}(B) = B.
$$
\end{enumerate}
%%%%%
%%%%%
\end{document}
