\documentclass[12pt]{article}
\usepackage{pmmeta}
\pmcanonicalname{JointEmbeddingProperty}
\pmcreated{2013-03-22 19:36:14}
\pmmodified{2013-03-22 19:36:14}
\pmowner{Naturman}{26369}
\pmmodifier{Naturman}{26369}
\pmtitle{joint embedding property}
\pmrecord{25}{42596}
\pmprivacy{1}
\pmauthor{Naturman}{26369}
\pmtype{Definition}
\pmcomment{trigger rebuild}
\pmclassification{msc}{03C52}
\pmsynonym{JEP}{JointEmbeddingProperty}
\pmsynonym{SJEP}{JointEmbeddingProperty}
%\pmkeywords{joint embedding}
%\pmkeywords{ultra-universal}
\pmrelated{ultrauniversal}
\pmrelated{amalgamationproperty}
\pmrelated{UltraUniversal}
\pmrelated{FactorEmbeddable}
\pmdefines{joint embedding property}
\pmdefines{strong joint embedding property}

\endmetadata

% this is the default PlanetMath preamble.  as your knowledge
% of TeX increases, you will probably want to edit this, but
% it should be fine as is for beginners.

% almost certainly you want these
\usepackage{amssymb}
\usepackage{amsmath}
\usepackage{amsfonts}

% used for TeXing text within eps files
%\usepackage{psfrag}
% need this for including graphics (\includegraphics)
%\usepackage{graphicx}
% for neatly defining theorems and propositions
%\usepackage{amsthm}
% making logically defined graphics
%%%\usepackage{xypic}

% there are many more packages, add them here as you need them

% define commands here

\begin{document}
Let $K$ be a class of models (structures) of a given signature. We say that $K$ has the \emph{joint embedding property} (abbreviated \emph{JEP}) iff for any models $A$ and $B$ in $K$ there exists a model $C$ in $K$ such that both $A$ and $B$ are embeddable in $C$. \cite{FM, UM}

\subsubsection{Examples}
Examples include \cite{UM}:

\begin{itemize}
\item The class of all groups.
\item The class of all monoids.
\item The class of all non-trivial Boolean algebras.
\end{itemize}

As is the case with the above examples, classes having the joint embedding property often satisfy an even stronger condition - for every indexed family of models in the class there is a model in the class into which each member of the family can be embedded. This is known as the \emph{strong joint embedding property} (abbreviated \emph{SJEP}). \cite{IT}

In general any factor embeddable class closed under products will have the strong joint embedding property. \cite{UM}

\subsubsection{Characterizations}
Elementary classes with the joint embedding property may be characterized syntactically and semantically:

Let $T$ be a first order theory in a language $L$ and let $K$ be the class of models of $T$ then:

\begin{enumerate}
\item $K$ has the joint embedding property iff for all universal sentences $\phi$ and $\psi$ in $L$, $T\vdash\phi\vee\psi$ implies either $T\vdash\phi$ or $T\vdash\psi$. \cite{FM}
\item If $T$ is consistent, then $K$ has the joint embedding property iff $T$ has an ultra-universal model. \cite{UM}
\end{enumerate}

\begin{thebibliography}{1}
\bibitem{FM} Abraham Robinson: \emph{Forcing in model theory}, Actes du Congr\`es International des Math\'ematiciens (Nice, 1970) Gauthier-Villars, Paris, 1971, pp. 245-250
\bibitem{UM} Colin Naturman, Henry Rose: \emph{Ultra-universal models}, Quaestiones Mathematicae, 15(2), 1992, 189-195
\bibitem{IT} Colin Naturman: \emph{Interior Algebras and Topology}, Ph.D. thesis, University of Cape Town Department of Mathematics, 1991
\end{thebibliography}

%%%%%
%%%%%
\end{document}
