\documentclass[12pt]{article}
\usepackage{pmmeta}
\pmcanonicalname{AxiomSchemaOfSeparation}
\pmcreated{2013-03-22 13:42:46}
\pmmodified{2013-03-22 13:42:46}
\pmowner{Sabean}{2546}
\pmmodifier{Sabean}{2546}
\pmtitle{axiom schema of separation}
\pmrecord{18}{34393}
\pmprivacy{1}
\pmauthor{Sabean}{2546}
\pmtype{Axiom}
\pmcomment{trigger rebuild}
\pmclassification{msc}{03E30}
\pmsynonym{separation schema}{AxiomSchemaOfSeparation}
\pmsynonym{separation}{AxiomSchemaOfSeparation}

\endmetadata

% this is the default PlanetMath preamble.  as your knowledge
% of TeX increases, you will probably want to edit this, but
% it should be fine as is for beginners.

% almost certainly you want these
\usepackage{amssymb}
\usepackage{amsmath}
\usepackage{amsfonts}

% used for TeXing text within eps files
%\usepackage{psfrag}
% need this for including graphics (\includegraphics)
%\usepackage{graphicx}
% for neatly defining theorems and propositions
%\usepackage{amsthm}
% making logically defined graphics
%%%\usepackage{xypic}

% there are many more packages, add them here as you need them

% define commands here
\begin{document}
Let $\phi(u, p)$ be a formula.  For any $X$ and $p$, there exists a set $Y = \{ u \in X : \phi(u, p) \}$.

The Axiom Schema of Separation is an axiom schema of Zermelo-Fraenkel set theory.  Note that it represents infinitely many individual axioms, one for each formula $\phi$.  In symbols, it reads:
\[
\forall X \forall p \exists Y \forall u(u \in Y \leftrightarrow u \in X \land \phi(u, p)).
\]
By Extensionality, the set $Y$ is unique.

The Axiom Schema of Separation implies that $\phi$ may depend on more than one parameter $p$.

We may show by induction that if $\phi(u, p_1, \ldots, p_n)$ is a formula, then
\[
\forall X \forall p_1 \cdots \forall p_n \exists Y \forall u(u \in Y \leftrightarrow u \in X \land \phi(u, p_1, \ldots, p_n))
\]
holds, using the Axiom Schema of Separation and the Axiom of Pairing.

Another consequence of the Axiom Schema of Separation is that a subclass of any set is a set.  To see this, let $\mathbf{C}$ be the class $\mathbf{C} = \{ u : \phi(u, p_1, \ldots, p_n) \}$.  Then
\[
\forall X \exists Y (\mathbf{C} \cap X = Y)
\]
holds, which means that the intersection of $\mathbf{C}$ with any set is a set.  Therefore, in particular, the intersection of two sets $X \cap Y = \{ x \in X : x \in Y \}$ is a set.  Furthermore the difference of two sets $X - Y = \{ x \in X : x \notin Y \}$ is a set and, provided there exists at least one set, which is guaranteed by the Axiom of Infinity, the empty set is a set.  For if $X$ is a set, then $\emptyset = \{ x \in X : x \neq x \}$ is a set.

Moreover, if $\mathbf{C}$ is a nonempty class, then $\bigcap \mathbf{C}$ is a set, by Separation.  $\bigcap \mathbf{C}$ is a subset of every $X \in \mathbf{C}$.

Lastly, we may use Separation to show that the class of all sets, $V$, is not a set, i.e., $V$ is a proper class.  For example, suppose $V$ is a set.  Then by Separation
\[
V' = \{ x \in V : x \notin x \}
\]
is a set and we have reached a Russell paradox.
%%%%%
%%%%%
\end{document}
