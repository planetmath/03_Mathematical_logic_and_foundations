\documentclass[12pt]{article}
\usepackage{pmmeta}
\pmcanonicalname{ProofOfZermelosWellorderingTheorem}
\pmcreated{2013-03-22 12:59:07}
\pmmodified{2013-03-22 12:59:07}
\pmowner{Henry}{455}
\pmmodifier{Henry}{455}
\pmtitle{proof of Zermelo's well-ordering theorem}
\pmrecord{9}{33359}
\pmprivacy{1}
\pmauthor{Henry}{455}
\pmtype{Proof}
\pmcomment{trigger rebuild}
\pmclassification{msc}{03E25}

% this is the default PlanetMath preamble.  as your knowledge
% of TeX increases, you will probably want to edit this, but
% it should be fine as is for beginners.

% almost certainly you want these
\usepackage{amssymb}
\usepackage{amsmath}
\usepackage{amsfonts}

% used for TeXing text within eps files
%\usepackage{psfrag}
% need this for including graphics (\includegraphics)
%\usepackage{graphicx}
% for neatly defining theorems and propositions
%\usepackage{amsthm}
% making logically defined graphics
%%%\usepackage{xypic}

% there are many more packages, add them here as you need them

% define commands here
%\PMlinkescapeword{theory}
\begin{document}
Let $X$ be any set and let $f$ be a choice function on $\mathcal{P}(X)\setminus\{\emptyset \} $.  Then define a function $i$ by transfinite recursion on the class of ordinals as follows:
$$i(\beta)=f(X-\bigcup_{\gamma<\beta} \{i(\gamma)\})\text{ unless } X-\bigcup_{\gamma<\beta} \{i(\gamma)\}=\emptyset\text{ or }i(\gamma)\text{ is undefined for some }\gamma<\beta$$

(the function is undefined if either of the unless clauses holds).

Thus $i(0)$ is just $f(X)$ (the least element of $X$), and $i(1)=f(X-\{i(0)\})$ (the least element of $X$ other than $i(0)$).

Define by the axiom of replacement $\beta=i^{-1}[X]=\{\gamma\mid i(\gamma)=x \text{ for some }x\in X\}$.  Since $\beta$ is a set of ordinals, it cannot contain all the ordinals (by the Burali-Forti paradox).

Since the ordinals are well ordered, there is a least ordinal $\alpha$ not in $\beta$, and therefore $i(\alpha)$ is undefined.  It cannot be that the second unless clause holds (since $\alpha$ is the least such ordinal) so it must be that $X-\bigcup_{\gamma<\alpha} \{i(\gamma)\}=\emptyset$, and therefore for every $x\in X$ there is some $\gamma<\alpha$ such that $i(\gamma)=x$.  Since we already know that $i$ is injective, it is a bijection between $\alpha$ and $X$, and therefore establishes a well-ordering of $X$ by $x<_Xy\leftrightarrow i^{-1}(x)<i^{-1}(y)$.

The reverse is simple.  If $C$ is a set of nonempty sets, select any well ordering of $\bigcup C$.  Then a choice function is just $f(a)=$\texttt{ the least member of }$a$\texttt{ under that well ordering}.
%%%%%
%%%%%
\end{document}
