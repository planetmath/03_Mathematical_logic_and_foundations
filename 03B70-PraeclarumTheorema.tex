\documentclass[12pt]{article}
\usepackage{pmmeta}
\pmcanonicalname{PraeclarumTheorema}
\pmcreated{2013-03-22 17:47:37}
\pmmodified{2013-03-22 17:47:37}
\pmowner{Jon Awbrey}{15246}
\pmmodifier{Jon Awbrey}{15246}
\pmtitle{praeclarum theorema}
\pmrecord{16}{40255}
\pmprivacy{1}
\pmauthor{Jon Awbrey}{15246}
\pmtype{Theorem}
\pmcomment{trigger rebuild}
\pmclassification{msc}{03B70}
\pmclassification{msc}{03B35}
\pmclassification{msc}{03B22}
\pmclassification{msc}{03B05}
\pmclassification{msc}{03-03}
\pmclassification{msc}{01A45}
\pmsynonym{splendid theorem}{PraeclarumTheorema}
\pmrelated{OrderedGroup}

\endmetadata

% this is the default PlanetMath preamble.  as your knowledge
% of TeX increases, you will probably want to edit this, but
% it should be fine as is for beginners.

% almost certainly you want these

\usepackage{amssymb}
\usepackage{amsmath}
\usepackage{amsfonts}
\usepackage{graphicx}

% used for TeXing text within eps files
%\usepackage{psfrag}
% need this for including graphics (\includegraphics)
%\usepackage{graphicx}
% for neatly defining theorems and propositions
%\usepackage{amsthm}
% making logically defined graphics
%%%\usepackage{xypic}

% there are many more packages, add them here as you need them

% define commands here

\begin{document}
\PMlinkescapeword{ad}
\PMlinkescapeword{AD}
\PMlinkescapeword{c}
\PMlinkescapeword{C}
\PMlinkescapeword{calculus}
\PMlinkescapeword{Calculus}
\PMlinkescapeword{qed}
\PMlinkescapeword{QED}
\PMlinkescapeword{reflect}
\PMlinkescapeword{Reflect}

The \textbf{praeclarum theorema}, or \textit{splendid theorem}, is a theorem of propositional calculus that was noted and named by G.W. Leibniz, who stated and proved it in the following manner:

\begin{quote}
If $a$ is $b$ and $d$ is $c$, then $ad$ will be $bc$.

This is a fine theorem, which is proved in this way:

$a$ is $b$, therefore $ad$ is $bd$ (by what precedes),

$d$ is $c$, therefore $bd$ is $bc$ (again by what precedes),

$ad$ is $bd$, and $bd$ is $bc$, therefore $ad$ is $bc$.  Q.E.D.

(Leibniz, \textit{Logical Papers}, p. 41).
\end{quote}

Expressed in contemporary logical notation, the praeclarum theorema (PT) may be written as follows:

\[ ((a \Rightarrow b) \land (d \Rightarrow c)) \Rightarrow ((a \land d) \Rightarrow (b \land c)) \]

Representing \PMlinkname{propositions}{PropositionalCalculus} as \PMlinkname{logical graphs}{LogicalGraph} under the \PMlinkname{existential interpretation}{LogicalGraphFormalDevelopment}, the praeclarum theorema is expressed by means of the following formal equation:

\begin{center}\begin{tabular}{cc}
\includegraphics[scale=0.8]{PraeclarumTheoremaFigure1} & (1) \\
\end{tabular}\end{center}

And here's a neat proof of that nice theorem.

\begin{center}\begin{tabular}{cc}
\includegraphics[scale=0.8]{PraeclarumTheoremaFigure2} & (2) \\
\end{tabular}\end{center}

\section{References}
\begin{itemize}
\item
Leibniz, Gottfried W. (1679--1686 ?), ``Addenda to the Specimen of the Universal Calculus", pp. 40--46 in G.H.R. Parkinson (ed., trans., 1966), \textit{Leibniz : Logical Papers}, Oxford University Press, London, UK.
\end{itemize}

\section{Readings}
\begin{itemize}
\item
Sowa, John F. (2002), ``Peirce's Rules of Inference", \PMlinkexternal{Online}{http://www.jfsowa.com/peirce/infrules.htm}.
\end{itemize}

\section{Resources}
\begin{itemize}
\item
Dau, Frithjof (2008), \PMlinkexternal{Computer Animated Proof of Leibniz's Praeclarum Theorema}{http://web.archive.org/web/20070706192257/http://dr-dau.net/pc.shtml}.
\item
Megill, Norman (2008), \PMlinkexternal{Praeclarum Theorema}{http://us.metamath.org/mpegif/prth.html} @ \PMlinkexternal{Metamath Proof Explorer}{http://us.metamath.org/mpegif/mmset.html}.
\end{itemize}

%%%%%
%%%%%
\end{document}
