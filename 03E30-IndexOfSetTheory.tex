\documentclass[12pt]{article}
\usepackage{pmmeta}
\pmcanonicalname{IndexOfSetTheory}
\pmcreated{2013-03-22 16:40:32}
\pmmodified{2013-03-22 16:40:32}
\pmowner{rspuzio}{6075}
\pmmodifier{rspuzio}{6075}
\pmtitle{index of set theory}
\pmrecord{20}{38883}
\pmprivacy{1}
\pmauthor{rspuzio}{6075}
\pmtype{Definition}
\pmcomment{trigger rebuild}
\pmclassification{msc}{03E30}

% this is the default PlanetMath preamble.  as your knowledge
% of TeX increases, you will probably want to edit this, but
% it should be fine as is for beginners.

% almost certainly you want these
\usepackage{amssymb}
\usepackage{amsmath}
\usepackage{amsfonts}
\usepackage{multicol}

% used for TeXing text within eps files
%\usepackage{psfrag}
% need this for including graphics (\includegraphics)
%\usepackage{graphicx}
% for neatly defining theorems and propositions
%\usepackage{amsthm}
% making logically defined graphics
%%%\usepackage{xypic}

% there are many more packages, add them here as you need them

% define commands here

\begin{document}
\begin{multicols}{2}

\section{Basic Notions}
\begin{enumerate}
\item set theory
\item set
\item subset
\item union
\item power set 
\item generalized Cartesian product
\item transitive set
\item criterion for a set to be transitive
\item Cartesian product
\item proof of the associativity of the symmetric difference operator
\item proper subset 
\item an example of mathematical induction
\item principle of finite induction
\item principle of finite induction proven from the well-ordering principle for natural numbers  
\item de Morgan's laws
\item de Morgan's laws for sets (proof)
\end{enumerate}

\section{Functions and Relations}
\begin{enumerate}
\item antisymmetric
\item example of antisymmetric
\item argument
\item constant function
\item equivalence class 
\item direct image
\item domain
\item fibre
\item fix (transformation action)
\item function
\item function graph
\item identity map
\item inclusion mapping
\item invariant
\item inverse image
\item irreflexive
\item left function notation
\item right function notation
\item level set
\item mapping
\item mapping of period $n$ is a bijection
\item operation
\item operations on relations
\item partial function
\item partial mapping
\item period of mapping
\item properties of a function
\item properties of functions
\item quasi-inverse of a function
\item range
\item reflexive relation 
\item relation
\item restriction of a function
\item set difference
\item symmetric difference
\item symmetric relation
\item the inverse image commutes with set operations
\item transformation
\item transitive 
\item transitive closure
\item transitive relation
\item choice function
\item one-to-one function from onto function
\end{enumerate}

\subsection{Order Relations}
\begin{enumerate}
\item poset
\item maximal element 
\item minimal element
\item visualizing maximal elements 
\item cofinality
\item another definition of cofinality
\item chain
\item antichain
\item branch
\item tree (set theoretic)
\item example of tree (set theoretic)
\item proof that $\Omega$ has the tree property
\item filtration
\item well ordered set
\end{enumerate}

\section{Cardinals and Ordinals}
\begin{enumerate}
\item $\kappa$-complete
\item additively indecomposable
\item aleph numbers
\item algebraic numbers are countable
\item all algebraic numbers in a sequence 
\item another proof of cardinality of the rationals
\item beth numbers
\item Cantor normal form
\item Cantor's diagonal argument
\item Cantor's theorem
\item cardinal arithmetic
\item cardinal exponentiation under GCH 
\item cardinal number
\item cardinal successor
\item cardinality
\item cardinality of a countable union
\item cardinality of disjoint union of finite sets 
\item cardinality of the continuum 
\item cardinality of the rationals
\item classes of ordinals and enumerating functions
\item club
\item club filter
\item countable
\item countably infinite
\item finite
\item finite character
\item fixed points of normal functions
\item Fodor's lemma
\item Hilbert's hotel
\item if $A$ is infinite and $B$ is a finite subset of $A\,\!,$ then $A\setminus B$ is infinite
\item K\"onig's theorem
\item limit cardinal 
\item natural number
\item normal (ordinal) function
\item open and closed intervals have the same cardinality
\item ordinal arithmetic
\item ordinal number 
\item pigeonhole principle 
\item proof of pigeonhole principle
\item another proof of pigeonhole principle
\item proof of Cantor's theorem
\item proof of fixed points of normal functions
\item proof of Fodor's lemma
\item proof of the existence of transcendental numbers
\item proof of theorems in additively indecomposable
\item proof that countable unions are countable
\item proof that the rationals are countable
\item proof of Schroeder-Bernstein theorem
\item Schroeder-Bernstein theorem
\item stationary set
\item subsets of countable sets are countable
\item thin set
\item successor
\item successor cardinal
\item the Cartesian product of a finite number of countable sets is countable
\item law of trichotomy 
\item partitions less than cofinality
\item Aronszajn tree
\item example of Aronszajn tree
\item Suslin tree
\item Erd\H{o}s-Rado theorem
\item uncountable owned by yark
\item uniqueness of cardinality
\item Veblen function 
\item von Neumann integer
\item von Neumann ordinal
\item weakly compact cardinal
\item weakly compact cardinals and the tree property
\item inductive set
\item inaccessible cardinals
\end{enumerate}

\section{Axiomatic Formulation}
\begin{enumerate}
\item axiom of choice
\item axiom of countable choice
\item axiom of determinacy
\item axiom of extensionality
\item axiom of infinity
\item axiom of pairing
\item axiom of power set
\item axiom of union 
\item axiom schema of separation
\item continuum hypothesis
\item generalized continuum hypothesis
\item equivalence of Zorn's lemma and the axiom of choice
\item Hausdorff's maximum principle
\item Kuratowski's lemma
\item maximality principle
\item permutation model
\item Tukey's lemma
\item $\mathcal{U}$-small
\item proof of Tukey's lemma
\item proof of Zermelo's postulate
\item proof of Zermelo's well-ordering theorem
\item proof that a relation is union of functions if and only if AC
\item relation as union of functions
\item Selector
\item well-ordering principle for natural numbers proven from the principle of finite induction 
\item well-ordering principle implies axiom of choice
\item Martin's axiom
\item Martin's axiom and the continuum hypothesis
\item Martin's axiom is consistent
\item a shorter proof: Martin's axiom and the continuum hypothesis
\item Zermelo's postulate 
\item Zermelo's well-ordering theorem
\item Zorn's lemma
\item example of universe 
\item example of universe of finite sets
\item proof of properties of universe
\item Tarski's axiom
\item universe
\item von Neumann-Bernays-Goedel set theory
\item chain condition 
\item composition of forcing notions
\item composition preserves chain condition
\item equivalence of forcing notions
\item forcing
\item forcing relation
\item forcings are equivalent if one is dense in the other
\item FS iterated forcing preserves chain condition
\item iterated forcing
\item iterated forcing and composition
\item partial order with chain condition does not collapse cardinals
\item proof of partial order with chain condition does not collapse cardinals 
\item proof that forcing notions are equivalent to their composition
\item Boolean valued model
\item complete partial orders do not add small subsets
\item proof of complete partial orders do not add small subsets
\item Levy collapse
\item $\Diamond$ is equivalent to $\clubsuit$ and continuum hypothesis
\item proof of $\Diamond$ is equivalent to $\clubsuit$ and continuum hypothesis
\item clubsuit
\item diamond
\item combinatorial principle
\end{enumerate}

\end{multicols}
%%%%%
%%%%%
\end{document}
