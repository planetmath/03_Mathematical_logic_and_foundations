\documentclass[12pt]{article}
\usepackage{pmmeta}
\pmcanonicalname{ProductOfCountableSets}
\pmcreated{2013-03-22 19:02:53}
\pmmodified{2013-03-22 19:02:53}
\pmowner{CWoo}{3771}
\pmmodifier{CWoo}{3771}
\pmtitle{product of countable sets}
\pmrecord{10}{41926}
\pmprivacy{1}
\pmauthor{CWoo}{3771}
\pmtype{Result}
\pmcomment{trigger rebuild}
\pmclassification{msc}{03E10}

\usepackage{amssymb,amscd}
\usepackage{amsmath}
\usepackage{amsfonts}
\usepackage{mathrsfs}
\usepackage{tabls}

% used for TeXing text within eps files
%\usepackage{psfrag}
% need this for including graphics (\includegraphics)
%\usepackage{graphicx}
% for neatly defining theorems and propositions
\usepackage{amsthm}
% making logically defined graphics
%%\usepackage{xypic}
\usepackage{pst-plot}

% define commands here
\newcommand*{\abs}[1]{\left\lvert #1\right\rvert}
\newtheorem{prop}{Proposition}
\newtheorem{thm}{Theorem}
\newtheorem{cor}{Corollary}
\newtheorem{ex}{Example}
\newcommand{\real}{\mathbb{R}}
\newcommand{\pdiff}[2]{\frac{\partial #1}{\partial #2}}
\newcommand{\mpdiff}[3]{\frac{\partial^#1 #2}{\partial #3^#1}}
\begin{document}
\PMlinkescapeword{c}

\begin{prop} $\mathbb{N}^2$ is countable. \end{prop}

This is actually proved in \PMlinkname{this entry}{AlternativeDefinitionsOfCountable}, by finding either a surjection $\mathbb{N}\to \mathbb{N}^2$, or an injection $\mathbb{N}^2 \to \mathbb{N}$.  In the following proof, we are going to get a bijection.

\begin{proof}
There are many ways to prove this.  One way is to place the integer pairs in a two-dimensional array indicated by the table on the left below:
\begin{center}
\begin{tabular}{ c|c c c c }
$i \backslash j$ & $1$ & $2$ & $3$ & $\cdots$ \\
\hline
$1$ & $(0,0)$ & $(0,1)$ & $(0,2)$ & $\cdots$ \\
$2$ & $(1,0)$ & $(1,1)$ & $(1,2)$ & $\cdots$ \\
$3$ & $(2,0)$ & $(2,1)$ & $(2,2)$ & $\cdots$ \\
$\vdots$ & $\vdots$ & $\vdots$ & $\vdots$ & $\ddots$ 
\end{tabular}
\hspace{1cm}
\begin{tabular}{ c|c c c c }
$i \backslash j$ & $1$ & $2$ & $3$ & $\cdots$ \\
\hline
$1$ & $1$ & $2$ & $4$ & $\cdots$ \\
$2$ & $3$ & $5$ & $8$ & $\cdots$ \\
$3$ & $6$ & $9$ & $13$ & $\cdots$ \\
$\vdots$ & $\vdots$ & $\vdots$ & $\vdots$ & $\ddots$ 
\end{tabular}
\end{center}
Let $C(i,j)$ be the content of cell $(i,j)$, located in the $i$-th row and $j$-th column.  For example, $C(1,1)=(0,0)$, and $C(3,2)=(2,1)$.

Now, let us construct a list of the pairs, which essentially amounts to constructing a bijection $h: \mathbb{N}^2 \to \mathbb{N}$ (the table on the right above).  We start at cell $(1,1)$.  If cell $(i,j)$ has been counted, the next cell to be counted is $(i+1,j-1)$ if $j>1$, or $(1,i+1)$ if $j=1$.  Thus, the first several pairs on the list are
$$(0,0),\; (0,1),\; (1,0),\; (0,2),\; (1,1),\; (2,0),\; (0,3),\; \ldots$$
We leave it to the reader to find the bijection $h$ (hint: see the entry on pairing function).  Therefore, $\mathbb{N}^2$ is countable.  
\end{proof}

\begin{prop} If $A$ and $B$ are countable, so is $A\times B$. \end{prop}
\begin{proof}  Suppose $f:A\to \mathbb{N}$ and $g:B\to \mathbb{N}$ are injections.  Then $h:=(f,g):A\times B\to \mathbb{N}^2$ is an injection.  Since $\mathbb{N}^2$ is countable, so is $A\times B$.
\end{proof}

\begin{prop}  Let $n$ be a positive integer, and $A_1,\ldots, A_n$ sets.  Then $A_1 \times \cdots \times A_n$ is countable iff each $A_i$ is. \end{prop}
\begin{proof}
Again, if one of $A_i$ is empty, so is the product, and vice versa.  The countability follows immediately.  So we assume that none of $A_i$ is empty.  Set $A:=A_1\times \cdots A_n$.

Suppose first that $A_1,\ldots, A_n$ are countable.  We do induction on $n$.  The case where $n=1$ is clear.  Suppose now that $n=k$ is true.  Then $A_1\times \cdots \times A_k \times A_{k+1}$ is just the product of two countable sets $A_1 \times \cdots \times A_k$ and $A_{k+1}$, which we know is countable by the proposition above.

Conversely, suppose $A$ is countable.  Let $g: A \to \mathbb{N}$ be an injection.  Since $A_i \ne \varnothing$, fix $a_i\in A_i$ for each $i = 1,\ldots, n$.  Now, for any $A_i$, define a map $e_i: A_i\to A$ so that the $i$-th component of $e_i(a)$ is $a$, and the $j$-th component is the fixed element $a_j \in A_j$, if $j\ne i$.  Clearly, $e_i: A_i\to A$ is an injection, so its composition with $g$ is also an injection from $A$ to $\mathbb{N}$, showing that $A_i$ is countable.
\end{proof}

\begin{cor} For any positive integer $n$, $A$ is countable iff $A^n$ is. \end{cor}

\textbf{Remark}.  However, infinite products of sets are in general uncountable, even if each of the sets is finite.  In particular, $\lbrace 0,1\rbrace^{\mathbb{N}}$ is uncountable.  The proof uses Cantor's diagonalization argument.

%%%%%
%%%%%
\end{document}
