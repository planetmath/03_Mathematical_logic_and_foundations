\documentclass[12pt]{article}
\usepackage{pmmeta}
\pmcanonicalname{FoundationsOfMathematicsOverview}
\pmcreated{2013-03-22 18:21:01}
\pmmodified{2013-03-22 18:21:01}
\pmowner{gribskoff}{21395}
\pmmodifier{gribskoff}{21395}
\pmtitle{foundations of mathematics: overview}
\pmrecord{13}{40986}
\pmprivacy{1}
\pmauthor{gribskoff}{21395}
\pmtype{Topic}
\pmcomment{trigger rebuild}
\pmclassification{msc}{03-01}
\pmclassification{msc}{03A05}
%\pmkeywords{Logicism}
%\pmkeywords{Formalism}
%\pmkeywords{Intuitionism}
%\pmkeywords{Hilbert's Program}
%\pmkeywords{Predicativism}
%\pmkeywords{Structuralism}
\pmrelated{AxiomaticAndCategoricalFoundationsOfMathematicsII2}
\pmrelated{FormalLogicsAndMetaMathematics}
\pmrelated{BeyondFormalism}
\pmrelated{AdjointFunctor}
\pmrelated{IntuitionisticLogic}
\pmrelated{ConsistencyOfClassicNumberTheory}
\pmrelated{InclusionOfClassicalIntoIntuitionisticLogic}
\pmrelated{InterpretationOfIntuitionisticLogicByM}
\pmdefines{Foundations of Mathematics}
\pmdefines{Philosophy of Mathematics}

% this is the default PlanetMath preamble.  as your knowledge
% of TeX increases, you will probably want to edit this, but
% it should be fine as is for beginners.

% almost certainly you want these
\usepackage{amssymb}
\usepackage{amsmath}
\usepackage{amsfonts}

% used for TeXing text within eps files
%\usepackage{psfrag}
% need this for including graphics (\includegraphics)
%\usepackage{graphicx}
% for neatly defining theorems and propositions
%\usepackage{amsthm}
% making logically defined graphics
%%%\usepackage{xypic}

% there are many more packages, add them here as you need them

% define commands here

\begin{document}
The term 'foundations of mathematics' denotes a set of theories which from the late XIX century onwards have tried to characterize the nature of mathematical reasoning. Common to all these theories is the use of the metaphor according to which knowledge is a building for which we must secure reliable foundations. The metaphor comes from Descartes VI Metaphysical Meditation and by the beginning of the XX century the foundations of mathematics were the single most interesting result obtained by the epistemological position known as foundationalism. 

In this period we can find three main theories which differ essentially as to what is to be properly considered a foundation for mathematical reasoning or for the knowledge that it generates. 

The first is logicism, the position held by Frege and Russell, according to which the analytical propositions of logic are the foundation upon which mathematical inference can be justified and the knowledge gained validated. The second is Hilbert's Program, improperly called formalism, a theory according to which the only foundation of mathematical knowledge is to be found in the synthetic character of combinatorial reasoning. The third is Brouwer's intuitionism a theory which maintains that mathematical knowledge needs no exogenous foundation, logical or otherwise, because the mathematician's knowledge has an evidence as imediate as our perception of time. 

In the course of the XX century these three main theories motivated some new mixed forms. Originating in the early version of Hilbert's Program is Wittgenstein's finitism, a conception according to which we have only reliable knowledge of finite objects and processes. Infinity is merely a \emph{façon de parler} with no real content and has to be eliminated in every context. 

Predicativism originates in some of Russell's and Poincaré's work but today is mainly associated with the names of Georg Kreisel and Solomon Feferman. The predicativist attitude is a minimal form of platonism: the only "given" is the set of natural numbers and all other mathematical objects and structures can be, at least in principle, obtained from them by means of suitable arithmetical definitions and predicates. 

In the second half of the XX century Paul Benaceraff and Michael Resnick discovered the philosophical significance of Dedekind's work and using the language of modern algebra proposed a rejection of the primacy of the concept of mathematical object in favour of the abstract notion of structure. A corollary of this position was that the consensus that set theory and its objects was the proper theory of mathematical inference was broken and a general theory of structures was going to be the new foundation. 

In spite of the fact that category theory goes back to 1945, Benaceraff and Resnick never went so far as to propose category theory as the new foundation. In the course of time Benaceraff came to reject to be defined as a structuralist but M. Resnick has expressed his affinity with the categorical point of view. Structuralism today has many forms and a separate entry would be necessary to account for all of them. 

There is a bridge that connects problems in the foundations of mathematics with the philosphy of mathematics: it is the dispute over what is to be considered a foundation or over the best definition of reliable inference. For example the structuralist thesis that the priority given to the concept of mathematical object is to be abandoned in favour of the concept of mathematical structure can not be evaluated without using the conceptual machinery of epistemology or ontology. 

It would be desirable to separate foundations from philosophy arguing that foundations is mathematical work usually carried out in mathematical logic and philosophy is a second order activity upon the first order data obtained in mathematical logic. But one is justified in entertaining a residual doubt, because in the end foundations will always have to do with understanding rather than classifying and it is possible that a general theory of all mathematical structures will not have a mathematical formulation.

\vspace{2mm}
\begin{thebibliography} {90}

\bibitem{SF} Feferman, S. "Systems of Predicative Analysis", \emph{JSL}, vol.29, I, 1964 
\bibitem{GF} Frege, G., \emph{Die Grundlagen der Arithmetik}, Breslau, 1884.
\bibitem{DHDB} Hilbert, D., Bernays, P., \emph{Die Grundlagen der Mathematik}, Berlin, 1939. 
\bibitem{AH} Heyting, A., \emph{Intuitionism, An Introduction}, Amsterdam, 1956. 
\bibitem{GK} Kreisel, G., "La Predicativité", \emph{Bulletin de la Societé Mathématique}, vol. 88, Paris, 1960.
\bibitem{MR} Resnick, M., \emph{Mathematics as a Science of Structures}, Oxford, 1997. 
\bibitem{BRAW} Russell, B., Whitehead, A., \emph{Principia Mathematica}, Cambridge, 1910. 
\bibitem{LW} Wittgenstein, L., \emph{Bemerkungen über die Grundlagen der Mathematik}, Oxford, 1956.
\end{thebibliography}
\end{document}
%%%%%
%%%%%
\end{document}
