\documentclass[12pt]{article}
\usepackage{pmmeta}
\pmcanonicalname{TwoSimpleFactsAboutWellfoundedRelations}
\pmcreated{2013-03-22 18:25:43}
\pmmodified{2013-03-22 18:25:43}
\pmowner{yesitis}{13730}
\pmmodifier{yesitis}{13730}
\pmtitle{two simple facts about well-founded relations}
\pmrecord{10}{41084}
\pmprivacy{1}
\pmauthor{yesitis}{13730}
\pmtype{Feature}
\pmcomment{trigger rebuild}
\pmclassification{msc}{03E20}
%\pmkeywords{well-founded relation}

\endmetadata

% this is the default PlanetMath preamble.  as your knowledge
% of TeX increases, you will probably want to edit this, but
% it should be fine as is for beginners.

% almost certainly you want these
\usepackage{amssymb}
\usepackage{amsmath}
\usepackage{amsfonts}

% used for TeXing text within eps files
%\usepackage{psfrag}
% need this for including graphics (\includegraphics)
%\usepackage{graphicx}
% for neatly defining theorems and propositions
%\usepackage{amsthm}
% making logically defined graphics
%%%\usepackage{xypic}

% there are many more packages, add them here as you need them

% define commands here

\begin{document}
The following are two simple facts about well-founded relation $R$ on $X$:

\begin{enumerate}
\item For each $x\in X$, $x\not R x$. (See the entry R-minimal element.)\\
\item The requirement for symmetry is absent, i.e., for each $x, y\in X$, either $xRy$ or $yRx$, but not both.
\end{enumerate}

Justifications for these two facts are simple. For 1, consider the subclass $\{x\}$. Then $\{x\}$ has an $R-\textrm{minimal}$ element, which can only be $x$ itself. For 2, consider $\{x, y\}$. It has an $R-\textrm{minimal}$ element, which is either $x$ or $y$, not both.

Fact 1 is provided here for easy reference. Keeping these two facts in mind is helpful when dealing with (proving) basic theorems about the relation.
%%%%%
%%%%%
\end{document}
