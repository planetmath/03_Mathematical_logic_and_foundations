\documentclass[12pt]{article}
\usepackage{pmmeta}
\pmcanonicalname{54InductiveTypesAreInitialAlgebras}
\pmcreated{2013-11-20 2:21:42}
\pmmodified{2013-11-20 2:21:42}
\pmowner{PMBookProject}{1000683}
\pmmodifier{rspuzio}{6075}
\pmtitle{5.4 Inductive types are initial algebras}
\pmrecord{3}{87677}
\pmprivacy{1}
\pmauthor{PMBookProject}{6075}
\pmtype{Feature}
\pmclassification{msc}{03B15}

\endmetadata

\usepackage{xspace}
\usepackage{amssyb}
\usepackage{amsmath}
\usepackage{amsfonts}
\usepackage{amsthm}
\makeatletter
\newcommand{\defeq}{\vcentcolon\equiv}  
\newcommand{\define}[1]{\textbf{#1}}
\def\@dprd#1{\prod_{(#1)}\,}
\def\@dprd@noparens#1{\prod_{#1}\,}
\def\dsm#1{\@dsm{#1}\@ifnextchar\bgroup{\dsm}{}}
\def\@dsm#1{\sum_{(#1)}\,}
\def\@dsm@noparens#1{\sum_{#1}\,}
\def\@eatprd\prd{\prd@parens}
\def\@eatsm\sm{\sm@parens}
\newcommand{\emptyt}{\ensuremath{\mathbf{0}}\xspace}
\newcommand{\eqv}[2]{\ensuremath{#1 \simeq #2}\xspace}
\newcommand{\htpy}{\sim}
\newcommand{\id}[3][]{\ensuremath{#2 =_{#1} #3}\xspace}
\newcommand{\idfunc}[1][]{\ensuremath{\mathsf{id}_{#1}}\xspace}
\newcommand{\indexdef}[1]{\index{#1|defstyle}}   
\newcommand{\indexsee}[2]{\index{#1|see{#2}}}    
\newcommand{\iscontr}{\ensuremath{\mathsf{isContr}}}
\newcommand{\ishinitn}{\mathsf{isHinit}_\nat}
\newcommand{\ishinitw}{\mathsf{isHinit}_{\mathsf{W}}}
\def\lam#1{{\lambda}\@lamarg#1:\@endlamarg\@ifnextchar\bgroup{.\,\lam}{.\,}}
\def\@lamarg#1:#2\@endlamarg{\if\relax\detokenize{#2}\relax #1\else\@lamvar{\@lameatcolon#2},#1\@endlamvar\fi}
\def\@lameatcolon#1:{#1}
\def\lamu#1{{\lambda}\@lamuarg#1:\@endlamuarg\@ifnextchar\bgroup{.\,\lamu}{.\,}}
\def\@lamuarg#1:#2\@endlamuarg{#1}
\def\@lamvar#1,#2\@endlamvar{(#2\,{:}\,#1)}
\newcommand{\N}{\ensuremath{\mathbb{N}}\xspace}
\newcommand{\nalg}{\nat\mathsf{Alg}}
\newcommand{\narrowbreak}{}
\newcommand{\nhom}{\nat\mathsf{Hom}}
\def\prd#1{\@ifnextchar\bgroup{\prd@parens{#1}}{\@ifnextchar\sm{\prd@parens{#1}\@eatsm}{\prd@noparens{#1}}}}
\def\prd@noparens#1{\mathchoice{\@dprd@noparens{#1}}{\@tprd{#1}}{\@tprd{#1}}{\@tprd{#1}}}
\def\prd@parens#1{\@ifnextchar\bgroup  {\mathchoice{\@dprd{#1}}{\@tprd{#1}}{\@tprd{#1}}{\@tprd{#1}}\prd@parens}  {\@ifnextchar\sm    {\mathchoice{\@dprd{#1}}{\@tprd{#1}}{\@tprd{#1}}{\@tprd{#1}}\@eatsm}    {\mathchoice{\@dprd{#1}}{\@tprd{#1}}{\@tprd{#1}}{\@tprd{#1}}}}}
\def\sm#1{\@ifnextchar\bgroup{\sm@parens{#1}}{\@ifnextchar\prd{\sm@parens{#1}\@eatprd}{\sm@noparens{#1}}}}
\def\sm@noparens#1{\mathchoice{\@dsm@noparens{#1}}{\@tsm{#1}}{\@tsm{#1}}{\@tsm{#1}}}
\def\sm@parens#1{\@ifnextchar\bgroup  {\mathchoice{\@dsm{#1}}{\@tsm{#1}}{\@tsm{#1}}{\@tsm{#1}}\sm@parens}  {\@ifnextchar\prd    {\mathchoice{\@dsm{#1}}{\@tsm{#1}}{\@tsm{#1}}{\@tsm{#1}}\@eatprd}    {\mathchoice{\@dsm{#1}}{\@tsm{#1}}{\@tsm{#1}}{\@tsm{#1}}}}}
\newcommand{\suc}{\mathsf{succ}}
\newcommand{\supp}{\ensuremath\suppsym\xspace}
\newcommand{\suppsym}{{\mathsf{sup}}}
\newcommand{\symlabel}[1]{\refstepcounter{symindex}\label{#1}}
\def\tprd#1{\@tprd{#1}\@ifnextchar\bgroup{\tprd}{}}
\def\@tprd#1{\mathchoice{{\textstyle\prod_{(#1)}}}{\prod_{(#1)}}{\prod_{(#1)}}{\prod_{(#1)}}}
\def\@tsm#1{\mathchoice{{\textstyle\sum_{(#1)}}}{\sum_{(#1)}}{\sum_{(#1)}}{\sum_{(#1)}}}
\def\@twtype#1{\mathchoice{{\textstyle\mathsf{W}_{(#1)}}}{\mathsf{W}_{(#1)}}{\mathsf{W}_{(#1)}}{\mathsf{W}_{(#1)}}}
\newcommand{\unit}{\ensuremath{\mathbf{1}}\xspace}
\newcommand{\UU}{\ensuremath{\mathcal{U}}\xspace}
\newcommand{\vcentcolon}{:\!\!}
\newcommand{\w}{\mathsf{W}}
\newcommand{\walg}{\w\mathsf{Alg}}
\newcommand{\whom}{\w\mathsf{Hom}}
\def\wtype#1{\@ifnextchar\bgroup  {\mathchoice{\@twtype{#1}}{\@twtype{#1}}{\@twtype{#1}}{\@twtype{#1}}\wtype}  {\mathchoice{\@twtype{#1}}{\@twtype{#1}}{\@twtype{#1}}{\@twtype{#1}}}}
\newcounter{mathcount}
\setcounter{mathcount}{1}
\newtheorem{predefn}{Definition}
\newenvironment{defn}{\begin{predefn}}{\end{predefn}\addtocounter{mathcount}{1}}
\renewcommand{\thepredefn}{5.4.\arabic{mathcount}}
\newenvironment{myeqn}{\begin{equation}}{\end{equation}\addtocounter{mathcount}{1}}
\renewcommand{\theequation}{5.4.\arabic{mathcount}}
\newenvironment{narrowmultline*}{\csname equation*\endcsname}{\csname endequation*\endcsname}
\newtheorem{prethm}{Theorem}
\newenvironment{thm}{\begin{prethm}}{\end{prethm}\addtocounter{mathcount}{1}}
\renewcommand{\theprethm}{5.4.\arabic{mathcount}}
\let\autoref\cref
\let\nat\N
\let\type\UU
\makeatother

\begin{document}
\indexsee{initial!algebra characterization of inductive types}{homotopy-initial}%

As suggested earlier, inductive types also have a category-theoretic universal property.
They are \emph{homotopy-initial algebras}: initial objects (up to coherent homotopy) in a category of ``algebras'' determined by the specified constructors.
As a simple example, consider the natural numbers.
The appropriate sort of ``algebra'' here is a type equipped with the same structure that the constructors of $\nat$ give to it.

\index{natural numbers!as homotopy-initial algebra}
\begin{defn}\label{defn:nalg}
  A \define{$\nat$-algebra}
  \indexdef{N-algebra@$\nat$-algebra}%
  \indexdef{algebra!N-@$\nat$-}%
  is a type $C$ with two elements $c_0 : C$, $c_s : C \to C$. The type of such algebras is
\begin{equation*}
\nalg \defeq \sm {C : \type} C \times (C \to C).
\end{equation*}
\end{defn}

\begin{defn}\label{defn:nhom}
  A \define{$\nat$-homomorphism}
  \indexdef{N-homomorphism@$\nat$-homomorphism}%
  \indexdef{homomorphism!N-@$\nat$-}%
  between $\nat$-algebras $(C,c_0,c_s)$ and $(D,d_0,d_s)$ is a function $h : C \to D$ such that $h(c_0) = d_0$ and $h(c_s(c)) = d_s(h(c))$ for all $c : C$. The type of such homomorphisms is
\begin{narrowmultline*}
\nhom((C,c_0,c_s),(D,d_0,d_s)) \defeq \narrowbreak
 \dsm {h : C \to D} (\id{h(c_0)}{d_0}) \times \tprd{c:C} (\id{h(c_s(c))}{d_s(h(c))}).
\end{narrowmultline*}
\end{defn}

We thus have a category of $\nat$-algebras and $\nat$-homomorphisms, and the claim is that $\nat$ is the initial object of this category.
A category theorist will immediately recognize this as the definition of a \emph{natural numbers object} in a category.

Of course, since our types behave like $\infty$-groupoids\index{.infinity-groupoid@$\infty$-groupoid}, we actually have an $(\infty,1)$-category\index{.infinity1-category@$(\infty,1)$-category} of $\nat$-algebras, and we should ask $\nat$ to be initial in the appropriate $(\infty,1)$-categorical sense.
Fortunately, we can formulate this without needing to define $(\infty,1)$-categories.

\begin{defn}
  \index{universal!property!of natural numbers}%
  A $\nat$-algebra $I$ is called \define{homotopy-initial},
  \indexdef{homotopy-initial!N-algebra@$\nat$-algebra}%
  \indexdef{N-algebra@$\nat$-algebra!homotopy-initial (h-initial)}%
  or \define{h-initial}
  \indexsee{h-initial}{homotopy-initial}%
  for short, if for any other $\nat$-algebra $C$, the type of $\nat$-homomorphisms from $I$ to $C$ is contractible. Thus,
\begin{equation*}
\ishinitn(I) \defeq \prd{C : \nalg} \iscontr(\nhom(I,C)).
\end{equation*}
\end{defn}

When they exist, h-initial algebras are unique --- not just up to isomorphism, as usual in category theory, but up to equality, by the univalence axiom.

\begin{thm}
  Any two h-initial $\nat$-algebras are equal.
  Thus, the type of h-initial $\nat$-algebras is a mere proposition.
\end{thm}
\begin{proof}
  Suppose $I$ and $J$ are h-initial $\nat$-algebras.
  Then $\nhom(I,J)$ is contractible, hence inhabited by some $\nat$-homomorphism $f:I\to J$, and likewise we have an $\nat$-homomorphism $g:J\to I$.
  Now the composite $g\circ f$ is a $\nat$-homomorphism from $I$ to $I$, as is $\idfunc[I]$; but $\nhom(I,I)$ is contractible, so $g\circ f = \idfunc[I]$.
  Similarly, $f\circ g = \idfunc[J]$.
  Hence $\eqv IJ$, and so $I=J$. Since being contractible is a mere proposition and dependent products preserve mere propositions, it follows that being h-initial is itself a mere proposition. Thus any two proofs that $I$ (or $J$) is h-initial are necessarily equal, which finishes the proof. 
\end{proof}

We now have the following theorem.

\begin{thm}\label{thm:nat-hinitial}
The $\nat$-algebra $(\nat, \emptyt, \suc)$ is homotopy initial.
\end{thm}
\begin{proof}[Sketch of proof]
  Fix an arbitrary $\nat$-algebra $(C,c_0,c_s)$.
  The recursion principle of $\nat$ yields a function $f:\nat\to C$ defined by
  \begin{align*}
    f(0) &\defeq c_0\\
    f(\suc(n)) &\defeq c_s(f(n)).
  \end{align*}
  These two equalities make $f$ an $\nat$-homomorphism, which we can take as the center of contraction for $\nhom(\nat,C)$.
  The uniqueness theorem (\PMlinkname{Theorem 5.1.1}{51introductiontoinductivetypes#Thmprethm1}) then implies that any other $\nat$-homomorphism is equal to $f$.
\end{proof}

To place this in a more general context, it is useful to consider the notion of \emph{algebra for an endofunctor}. \index{algebra!for an endofunctor}
Note that to make a type $C$ into a $\nat$-algebra is the same as to give a function $c:C+\unit\to C$, and a function $f:C\to D$ is a $\nat$-homomorphism just when $f \circ c \htpy d \circ (f+\unit)$.
In categorical language, this means the \nat-algebras are the algebras for the endofunctor $F(X)\defeq X+1$ of the category of types.

\indexsee{functor!polynomial}{endofunctor, polynomial}%
\indexsee{polynomial functor}{endofunctor, polynomial}%
\indexdef{endofunctor!polynomial}%
\index{W-type@$\w$-type!as homotopy-initial algebra}
For a more generic case, consider the $\w$-type associated to $A : \type$ and $B : A \to \type$.
In this case we have an associated \define{polynomial functor}:
\begin{myeqn}
\label{eq:polyfunc}
P(X) = \sm{x : A} (B(x) \rightarrow X).
\end{myeqn}
Actually, this assignment is functorial only up to homotopy, but this makes no difference in what follows.
By definition, a \define{$P$-algebra}
\indexdef{algebra!for a polynomial functor}%
\indexdef{algebra!W-@$\w$-}%
is then a type $C$ equipped a function $s_C :  PC \rightarrow C$.
By the universal property of $\Sigma$-types, this is equivalent to giving a function $\prd{a:A} (B(a) \to C) \to C$.
We will also call such objects \define{$\w$-algebras}
\indexdef{W-algebra@$\w$-algebra}%
for $A$ and $B$, and we write
\symlabel{walg}
\begin{equation*}
\walg(A,B) \defeq \sm {C : \type} \prd{a:A} (B(a) \to C) \to C.
\end{equation*}

Similarly, for $P$-algebras $(C,s_C)$ and $(D,s_D)$, a \define{homomorphism}
\indexdef{homomorphism!of algebras for a functor}%
between them $(f, s_f) : (C, s_C) \rightarrow (D, s_D)$ consists of a function $f : C \rightarrow D$ and a homotopy between maps $PC \rightarrow D$
\[
s_f :  f \circ s_C \, = s_{D} \circ Pf,
\]
where $Pf : PC\rightarrow PD$ is the result of the easily-definable action of $P$ on $f: C \rightarrow D$. Such an algebra homomorphism can be represented suggestively in the form:
\begin{figure}
 \centering
 \includegraphics{HoTT_fig_5.4.1.png}
\end{figure}
%\[
%\xymatrix{
% PC \ar[d]_{s_C} \ar[r]^{Pf}  \ar@{}[dr]|{s_f} &  PD \ar[d]^{s_D}\\
%C \ar[r]_{f}   & D }
%\]
In terms of elements, $f$ is a $P$-homomorphism (or \define{$\w$-homomorphism}) if
\indexdef{W-homomorphism@$\w$-homomorphism}%
\indexdef{homomorphism!W-@$\w$-}%
\[f(s_C(a,h)) = s_D(a,\lam{b} f(h(b))).\]
We have the type of $\w$-homomorphisms:
\symlabel{whom}
\begin{equation*}
  \whom_{A,B}((C, c),(D,d)) \defeq \sm{h : C \to D} \prd{a:A}{f:B(a)\to C} \id{h(c(a,f))}{\lamu{b:B(a)} h(f(b))}
\end{equation*}

\index{universal!property!of $\w$-type}%
Finally, a $P$-algebra $(C, s_C)$ is said to be \define{homotopy-initial}
\indexdef{homotopy-initial!algebra for a functor}%
\indexdef{homotopy-initial!W-algebra@$\w$-algebra}%
if for every $P$-algebra $(D, s_D)$, the type of all algebra homomorphisms $(C, s_C) \rightarrow (D, s_D)$ is contractible.
That is,
\begin{equation*}
\ishinitw(A,B,I) \defeq \prd{C : \walg(A,B)} \iscontr(\whom_{A,B}(I,C)).
\end{equation*}
%
Now the analogous theorem to \PMlinkname{Theorem 5.4.5}{54inductivetypesareinitialalgebras#Thmprethm2} is:

\begin{thm}\label{thm:w-hinit}
For any type $A : \type$ and type family $B : A \to \type$, the $\w$-algebra $(\wtype{x:A}B(x), \supp)$ is h-initial.
\end{thm}

\begin{proof}[Sketch of proof]
Suppose we have $A : \type$ and $B : A \to \type$,
and consider the associated polynomial functor $P(X)\defeq\sm{x:A}(B(x)\to X)$.
Let $W \defeq \wtype{x:A}B(x)$.  Then using
the $\w$-introduction rule from \PMlinkname{\S 5.3}{53wtypes}, we have a structure map $s_W\defeq\supp: PW \rightarrow W$. 
We want to show that the algebra $(W, s_W)$ is
h-initial. So, let us consider another algebra $(C,s_C)$ and show that the type $T\defeq \whom_{A,B}((W, s_W),(C,s_C)) $ 
of  $\w$-homomorphisms from $(W, s_W)$ to $(C, s_C)$ is contractible. To do
so, observe that the $\w$-elimination rule and the $\w$-computation
rule allow us to define a $\w$-homomorphism $(f, s_f) : (W, s_W) \rightarrow (C, s_C)$, 
thus showing that $T$ is inhabited. It is furthermore necessary to show that for every $\w$-ho\-mo\-mor\-phism $(g, s_g) : (W, s_W) \rightarrow (C, s_C)$, there is an identity proof 
\begin{myeqn}
\label{equ:prequired}
p :  (f,s_f) = (g,s_g).
\end{myeqn}
This uses the fact that, in general, a type of the form $(f,s_f) = (g,s_g) $
is  equivalent to the type of what we call \define{algebra $2$-cells},
\indexdef{algebra!2-cell}%
whose canonical
elements are pairs of the form $(e, s_e)$, where $e : f=g$ and $s_e$ is a higher identity proof between the identity proofs represented by the following pasting diagrams:
\begin{figure}
 \centering
 \includegraphics{HoTT_fig_5.4.2.png}
\end{figure}
%\[
%\xymatrix{
%PW \ar@/^1pc/[r]^{Pg}   \ar[d]_{s_W} \ar@{}[r]_(.52){s_g}  & PD \ar[d]^{s_D}  \\
%W \ar@/^1pc/[r]^g  \ar@/_1pc/[r]_f  \ar@{}[r]|{e} & D } \qquad
%\xymatrix{
%PW \ar@/^1pc/[r]^{Pg}   \ar[d]_{s_W}   \ar@/_1pc/[r]_{Pf} \ar@{}[r]|{Pe}
%& PD \ar[d]^{s_D}  \\
%W  \ar@/_1pc/[r]_f  \ar@{}[r]^{s_f} & D }
%\]
In light of this fact, to prove that there exists an element as in~\eqref{equ:prequired}, it is 
sufficient to show that there is an algebra 2-cell 
\[
(e,s_e) : (f,s_f) = (g,s_g).
\]
The identity proof $e : f=g$ is now constructed by function extensionality and
$\w$-elimination so as to guarantee the existence of the required identity
proof $s_e$. 
\end{proof}

%%%%%%%%%%%%%%%%%%%%%%%%%%%%%%%%%%%%%%%%%%%%%%%%%%%%%%%%%%


\end{document}
