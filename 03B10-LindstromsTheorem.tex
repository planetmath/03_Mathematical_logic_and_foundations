\documentclass[12pt]{article}
\usepackage{pmmeta}
\pmcanonicalname{LindstromsTheorem}
\pmcreated{2013-03-22 13:49:30}
\pmmodified{2013-03-22 13:49:30}
\pmowner{mathcam}{2727}
\pmmodifier{mathcam}{2727}
\pmtitle{Lindstr\"om's theorem}
\pmrecord{8}{34556}
\pmprivacy{1}
\pmauthor{mathcam}{2727}
\pmtype{Theorem}
\pmcomment{trigger rebuild}
\pmclassification{msc}{03B10}

\endmetadata

% this is the default PlanetMath preamble.  as your knowledge
% of TeX increases, you will probably want to edit this, but
% it should be fine as is for beginners.

% almost certainly you want these
\usepackage{amssymb}
\usepackage{amsmath}
\usepackage{amsfonts}

% used for TeXing text within eps files
%\usepackage{psfrag}
% need this for including graphics (\includegraphics)
%\usepackage{graphicx}
% for neatly defining theorems and propositions
%\usepackage{amsthm}
% making logically defined graphics
%%%\usepackage{xypic}

% there are many more packages, add them here as you need them

% define commands here
\begin{document}
One of the very first results of the study of model theoretic logics is a characterization theorem due to Per Lindstr\"om. He showed that the classical first order logic is the strongest logic having the following properties

\begin{itemize}
 \item Being closed under contradictory negation
 \item Compactness
 \item L\"owenheim-Skolem theorem 
\end{itemize}

also, he showed that first order logic can be characterised as the strongest logic  for which the following hold

\begin{itemize}
 \item Completeness (r.e. axiomatisability)
 \item L\"owenheim-Skolem theorem
\end{itemize}

The notion of ``strength'' used here is as follows. A logic $\mathbf{L}'$ is stronger than $\mathbf{L}$ or as strong if every class of structures definable in $\mathbf{L}$ is also definable in $\mathbf{L}'$.
%%%%%
%%%%%
\end{document}
