\documentclass[12pt]{article}
\usepackage{pmmeta}
\pmcanonicalname{Cycle1}
\pmcreated{2013-03-22 12:24:23}
\pmmodified{2013-03-22 12:24:23}
\pmowner{yark}{2760}
\pmmodifier{yark}{2760}
\pmtitle{cycle}
\pmrecord{10}{32262}
\pmprivacy{1}
\pmauthor{yark}{2760}
\pmtype{Definition}
\pmcomment{trigger rebuild}
\pmclassification{msc}{03-00}
\pmclassification{msc}{05A05}
\pmclassification{msc}{20F55}
\pmrelated{Permutation}
\pmrelated{SymmetricGroup}
\pmrelated{Transposition}
\pmrelated{Group}
\pmrelated{Subgroup}
\pmrelated{DihedralGroup}
\pmrelated{CycleNotation}
\pmrelated{PermutationNotation}

\endmetadata

%\usepackage{graphicx}
%%%\usepackage{xypic} 
\usepackage{bbm}
\newcommand{\Z}{\mathbbmss{Z}}
\newcommand{\C}{\mathbbmss{C}}
\newcommand{\R}{\mathbbmss{R}}
\newcommand{\Q}{\mathbbmss{Q}}
\newcommand{\mathbb}[1]{\mathbbmss{#1}}
\begin{document}
Let $S$ be a set. A \emph{cycle} is a permutation
(bijective function of a set onto itself)
such that there exist distinct elements $a_1, a_2,\ldots,a_k$ of $S$
such that
$$f(a_i) = a_{i+1}\qquad \mbox{and}\qquad f(a_k)=a_1$$
that is
\begin{eqnarray*}
f(a_1)&=&a_{2}\\
f(a_{2})&=&a_{3}\\
&\vdots&\\
f(a_{k})&=&a_{1}\\
\end{eqnarray*}
and $f(x)=x$ for any other element of $S$.

This can also be pictured as
$$a_1\mapsto a_{2}\mapsto a_{3}\mapsto\cdots\mapsto a_{k}\mapsto a_{1}$$
and $$x\mapsto x$$ 
for any other element $x\in S$, where $\mapsto$ represents the action of $f$.

One of the basic results on symmetric groups
says that any finite permutation can be expressed as product of disjoint cycles.
%%%%%
%%%%%
\end{document}
