\documentclass[12pt]{article}
\usepackage{pmmeta}
\pmcanonicalname{AnotherProofOfCardinalityOfTheRationals}
\pmcreated{2013-03-22 16:01:49}
\pmmodified{2013-03-22 16:01:49}
\pmowner{Mathprof}{13753}
\pmmodifier{Mathprof}{13753}
\pmtitle{another proof of cardinality of the rationals}
\pmrecord{10}{38073}
\pmprivacy{1}
\pmauthor{Mathprof}{13753}
\pmtype{Proof}
\pmcomment{trigger rebuild}
\pmclassification{msc}{03E10}

\endmetadata

% this is the default PlanetMath preamble.  as your knowledge
% of TeX increases, you will probably want to edit this, but
% it should be fine as is for beginners.

% almost certainly you want these
\usepackage{amssymb}
\usepackage{amsmath}
\usepackage{amsfonts}

% used for TeXing text within eps files
%\usepackage{psfrag}
% need this for including graphics (\includegraphics)
%\usepackage{graphicx}
% for neatly defining theorems and propositions
%\usepackage{amsthm}
% making logically defined graphics
%%%\usepackage{xypic}

% there are many more packages, add them here as you need them

% define commands here

\begin{document}
If we have a rational number $p/q$ with $p$ and $q$ having no common factor,
and each expressed in base 10 then we can view  $p/q$ as a base 11 integer,
where the digits are $0,1,2,\ldots,9$ and $/$. That is, slash ($/$) is a symbol for a
digit. For example, the rational 3/2 corresponds to the integer $3\cdot 11^2 + 10\cdot 11 + 2$.
The rational $-3/2$ corresponds to the integer $-(3\cdot 11^2 + 10 \cdot 11 + 2)$.

This gives a one-to-one map into the 
integers so the cardinality of the rationals is at most the cardinality of
the integers. So the rationals are countable. 
%%%%%
%%%%%
\end{document}
