\documentclass[12pt]{article}
\usepackage{pmmeta}
\pmcanonicalname{WellfoundedRecursion}
\pmcreated{2013-03-22 17:54:52}
\pmmodified{2013-03-22 17:54:52}
\pmowner{CWoo}{3771}
\pmmodifier{CWoo}{3771}
\pmtitle{well-founded recursion}
\pmrecord{8}{40409}
\pmprivacy{1}
\pmauthor{CWoo}{3771}
\pmtype{Theorem}
\pmcomment{trigger rebuild}
\pmclassification{msc}{03E10}
\pmclassification{msc}{03E45}
\pmrelated{TransfiniteRecursion}

\usepackage{amssymb,amscd}
\usepackage{amsmath}
\usepackage{amsfonts}
\usepackage{mathrsfs}

% used for TeXing text within eps files
%\usepackage{psfrag}
% need this for including graphics (\includegraphics)
%\usepackage{graphicx}
% for neatly defining theorems and propositions
\usepackage{amsthm}
% making logically defined graphics
%%\usepackage{xypic}
\usepackage{pst-plot}

% define commands here
\newcommand*{\abs}[1]{\left\lvert #1\right\rvert}
\newtheorem{prop}{Proposition}
\newtheorem{thm}{Theorem}
\newtheorem{ex}{Example}
\newcommand{\real}{\mathbb{R}}
\newcommand{\pdiff}[2]{\frac{\partial #1}{\partial #2}}
\newcommand{\mpdiff}[3]{\frac{\partial^#1 #2}{\partial #3^#1}}
\begin{document}
\begin{thm} Let $G$ be a binary (class) function on $V$, the class of all sets.  Let $A$ be a well-founded set (with $R$ the well-founded relation).  Then there is a unique function $F$ such that $$F(x)=G(x,F|\operatorname{seg}(x)),$$
where $\operatorname{seg}(x):=\lbrace y\in A\mid yRx\rbrace$, the initial segment of $x$.
\end{thm}

\textbf{Remark}.  Since every well-ordered set is well-founded, the well-founded recursion theorem is a generalization of the transfinite recursion theorem.  Notice that the $G$ here is a function in two arguments, and that it is necessary to specify the element $x$ in the first argument (in contrast with the $G$ in the transfinite recursion theorem), since it is possible that $\operatorname{seg}(a)=\operatorname{seg}(b)$ for $a\ne b$ in a well-founded set. 

By converting $G$ into a formula ($\varphi(x,y,z)$ such that for all $x,y$, there is a unique $z$ such that $\varphi(x,y,z)$), then the above theorem can be proved in ZF (with the aid of the well-founded induction).  The proof is similar to the proof of the transfinite recursion theorem.
%%%%%
%%%%%
\end{document}
