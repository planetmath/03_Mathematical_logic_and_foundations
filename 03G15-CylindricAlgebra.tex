\documentclass[12pt]{article}
\usepackage{pmmeta}
\pmcanonicalname{CylindricAlgebra}
\pmcreated{2013-03-22 17:51:21}
\pmmodified{2013-03-22 17:51:21}
\pmowner{CWoo}{3771}
\pmmodifier{CWoo}{3771}
\pmtitle{cylindric algebra}
\pmrecord{10}{40331}
\pmprivacy{1}
\pmauthor{CWoo}{3771}
\pmtype{Definition}
\pmcomment{trigger rebuild}
\pmclassification{msc}{03G15}
\pmclassification{msc}{06E25}
\pmrelated{PolyadicAlgebra}
\pmrelated{PolyadicAlgebraWithEquality}

\endmetadata

\usepackage{amssymb,amscd}
\usepackage{amsmath}
\usepackage{amsfonts}
\usepackage{mathrsfs}

% used for TeXing text within eps files
%\usepackage{psfrag}
% need this for including graphics (\includegraphics)
%\usepackage{graphicx}
% for neatly defining theorems and propositions
\usepackage{amsthm}
% making logically defined graphics
%%\usepackage{xypic}
\usepackage{pst-plot}

% define commands here
\newcommand*{\abs}[1]{\left\lvert #1\right\rvert}
\newtheorem{prop}{Proposition}
\newtheorem{thm}{Theorem}
\newtheorem{ex}{Example}
\newcommand{\real}{\mathbb{R}}
\newcommand{\pdiff}[2]{\frac{\partial #1}{\partial #2}}
\newcommand{\mpdiff}[3]{\frac{\partial^#1 #2}{\partial #3^#1}}
\begin{document}
A \emph{cylindric algebra} is a quadruple $(B,V,\exists,d)$, where $B$ is a Boolean algebra, $V$ is a set whose elements we call \emph{variables}, $\exists$ and $d$ are functions
$$\exists:V\to B^B\qquad \mbox{ and } \qquad d:V\times V\to B$$
such that
\begin{enumerate}
\item $(B,\exists x)$ is a monadic algebra for each $x\in V$,
\item $\exists x\circ \exists y = \exists y\circ \exists x$ for any $x,y\in V$,
\item $d_{xx}=1$ for all $x\in V$,
\item for any $x,y\in V$ with $x\ne y$, and any $a \in B$, we have the equality $$\exists x(a\wedge d_{xy})\wedge \exists x\big(a'\wedge d_{xy}\big)=0$$
\item for any $x,y,z\in V$ with $x\ne y$ and $x\ne z$, we have the equality $$\exists x (d_{xy}\wedge d_{xz})=d_{yz}.$$
\end{enumerate}
where $\exists x$ and $d_{xy}$ are the abbreviations for $\exists(x)$ and $d(x,y)$ respectively.

Basically, the first two conditions say that the $(B,V,\exists)$ portion of the cylindric algebra is very similar to a quantifier algebra, except the domain is no longer the subsets of $V$, but the elements of $V$ instead.  The function $d$ is the algebraic abstraction of equality.  Condition 3 says that $x=x$ is always true, condition 4 says that the proposition $a$ and its complement $a'$, where any occurrences of the variable $x$ are replaced by the variable $y$, distinct from $x$, is always false, while condition 5 says $y=z$ iff there is an $x$ such that $x=y$ and $x=z$.

Below are some elementary properties of $d$:
\begin{itemize}
\item (symmetric property) $d_{xy}=d_{yx}$
\item (transitive property) $d_{xy}\wedge d_{yz}\le d_{xz}$
\item $\exists x(d_{xy})=1$
\item $\exists x(d_{yz})=d_{yz}$ provided that $x\notin \lbrace y,z\rbrace$
\item if $x\ne y$, then
\begin{enumerate}
\item $\exists x(d_{xy}\wedge a')= (\exists x(d_{xy}\wedge a))'$,
\item $d_{xy} \wedge a =d_{xy} \wedge \exists x(a\wedge d_{xy})$.
\end{enumerate}
\end{itemize}

\textbf{Remarks}
\begin{enumerate}
\item The \emph{dimension} of a cylindric algebra $(B,V,\exists,d)$ is the cardinality of $V$.
\item From the definition above, a cylindric algebra is a two-sorted structure, with $B$ and $V$ as the two distinct universes.  However, it is often useful to view a cylindric algebra as a one-sorted structure.  The way to do this is to dispense with $V$ and identify each $\exists x$ as a unary operator on $B$, and each $d_{xy}$ as a constant in $B$.  As a result, the cylindric algebra $(B,V,\exists,d)$ becomes a Boolean algebra with possibly infinitely many operators: $$(B,\exists x,d_{xy})_{x,y\in V}.$$
\item Let $L$ be a the language of a first order logic, and $S$ a set of sentences in $L$.  Define $\equiv$ on $L$ so that $$\varphi \equiv \psi\quad \mbox{ iff }\quad S \vdash (\varphi\leftrightarrow \psi).$$  Then $\equiv$ is an equivalence relation on $L$.  For each formula $\varphi\in L$, let $[\varphi]$ be the equivalence class containing $\varphi$.  Let $V$ be a countably infinite set of variables available to $L$.  Now, define operations $\vee,\wedge,',\exists x,d_{xy}$ as follows:
\begin{center}
\begin{eqnarray}
[\varphi] \vee [\psi] &:=& [\varphi \vee \psi], \\ 
\left[ \varphi \right] \wedge [\psi] &:=& [\varphi \wedge \psi], \\ 
\left[ \varphi \right]' &:=& [\neg \varphi], \\ 
0 &:=& [\neg  x=x], \\ 
1 &:=& [x=x], \\
\exists x[\varphi] &:=& [\exists x \varphi], \\
d_{xy} &:=& [x=y].
\end{eqnarray}
\end{center}
Then it can be shown that $(L/\equiv, V,\exists,d)$ is a cylindric algebra.  Thus a cylindric algebra can be thought of as an ``algebraization'' of first order logic (with equality), much the same way as a Boolean algebra (Lindenbaum-Tarski algebra) as the algebraic counterpart of propositional logic.
\end{enumerate}

\begin{thebibliography}{8}
\bibitem{ph} P. Halmos, \emph{Algebraic Logic}, Chelsea Publishing Co. New York (1962).
\bibitem{hmt} L. Henkin, J. D. Monk, A. Tarski, \emph{Cylindric Algebras, Part I.}, North-Holland, Amsterdam (1971).
\bibitem{dm} J. D. Monk, \emph{Mathematical Logic}, Springer, New York (1976).
\bibitem{bp} B. Plotkin, \emph{Universal Algebra, Algebraic Logic, and Databases}, Kluwer Academic Publishers (1994).
\end{thebibliography}
%%%%%
%%%%%
\end{document}
