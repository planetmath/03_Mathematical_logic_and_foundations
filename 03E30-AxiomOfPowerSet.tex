\documentclass[12pt]{article}
\usepackage{pmmeta}
\pmcanonicalname{AxiomOfPowerSet}
\pmcreated{2013-03-22 13:43:03}
\pmmodified{2013-03-22 13:43:03}
\pmowner{mathcam}{2727}
\pmmodifier{mathcam}{2727}
\pmtitle{axiom of power set}
\pmrecord{11}{34399}
\pmprivacy{1}
\pmauthor{mathcam}{2727}
\pmtype{Axiom}
\pmcomment{trigger rebuild}
\pmclassification{msc}{03E30}
\pmsynonym{power set axiom}{AxiomOfPowerSet}
\pmsynonym{powerset axiom}{AxiomOfPowerSet}
\pmsynonym{axiom of powerset}{AxiomOfPowerSet}

\endmetadata

% this is the default PlanetMath preamble.  as your knowledge
% of TeX increases, you will probably want to edit this, but
% it should be fine as is for beginners.

% almost certainly you want these
\usepackage{amssymb}
\usepackage{amsmath}
\usepackage{amsfonts}

% used for TeXing text within eps files
%\usepackage{psfrag}
% need this for including graphics (\includegraphics)
%\usepackage{graphicx}
% for neatly defining theorems and propositions
%\usepackage{amsthm}
% making logically defined graphics
%%%\usepackage{xypic}

% there are many more packages, add them here as you need them

% define commands
\begin{document}
The \emph{axiom of power set} is an axiom of Zermelo-Fraenkel set theory which postulates that for any set $X$ there exists a set $\mathcal{P}(X)$, called the \emph{power set} of $X$, consisting of all subsets of $X$.  In symbols, it reads:
\[
\forall X \exists \mathcal{P}(X) \forall u (u \in \mathcal{P}(X) \leftrightarrow u \subseteq X).
\]
In the above, $u \subseteq X$ is defined as $\forall z(z \in u \rightarrow z \in X)$.  By the extensionality axiom, the set $\mathcal{P}(X)$ is unique.

The Power Set Axiom allows us to define the Cartesian product of two sets $X$ and $Y$:
\[
X \times Y = \{ (x, y) : x \in X \land y \in Y \}.
\]

The Cartesian product is a set since
\[
X \times Y \subseteq \mathcal{P}(\mathcal{P}(X \cup Y)).
\]

We may define the Cartesian product of any finite collection of sets recursively:
\[
X_1 \times \cdots \times X_n = (X_1 \times \cdots \times X_{n-1}) \times X_n.
\]
%%%%%
%%%%%
\end{document}
