\documentclass[12pt]{article}
\usepackage{pmmeta}
\pmcanonicalname{ModusPonens}
\pmcreated{2013-03-22 16:50:48}
\pmmodified{2013-03-22 16:50:48}
\pmowner{CWoo}{3771}
\pmmodifier{CWoo}{3771}
\pmtitle{modus ponens}
\pmrecord{19}{39092}
\pmprivacy{1}
\pmauthor{CWoo}{3771}
\pmtype{Definition}
\pmcomment{trigger rebuild}
\pmclassification{msc}{03B22}
\pmclassification{msc}{03B05}
\pmclassification{msc}{03B35}
\pmsynonym{rule of detachment}{ModusPonens}
\pmsynonym{detachment}{ModusPonens}
\pmsynonym{modus ponendo ponens}{ModusPonens}

\endmetadata

\usepackage{amssymb,amscd}
\usepackage{amsmath}
\usepackage{amsfonts}

% used for TeXing text within eps files
%\usepackage{psfrag}
% need this for including graphics (\includegraphics)
%\usepackage{graphicx}
% for neatly defining theorems and propositions
\usepackage{amsthm}
% making logically defined graphics
%%\usepackage{xypic}
\usepackage{pst-plot}
\usepackage{psfrag}

% define commands here
\newtheorem{prop}{Proposition}
\newtheorem{thm}{Theorem}
\newtheorem{ex}{Example}
\newcommand{\real}{\mathbb{R}}
\begin{document}
\textbf{Modus ponens} is a rule of inference that is commonly found in many logics where the binary logical connective $\to$ (sometimes written $\Rightarrow$ or $\supset$) called logical implication are defined.  Informally, it states that 

\begin{center}
from $A$ and $A \to B$, we may infer $B$.
\end{center}

Modus ponens is also called the \emph{rule of detachment}: the theorem $b$ can be ``detached'' from the theorem $A \to B$ provided that $A$ is also a theorem.

An example of this rule is the following:  From the premisses ``It is raining'',
and ``If it rains, then my laundry will be soaked'', we may draw the conclusion 
``My laundry will be soaked''.

Two common ways of mathematically denoting modus ponens are the following:
$$\frac{A \quad A \to B}{B} \qquad \mbox{or} \quad \lbrace A, A \to B\rbrace \vdash B.$$

One formal way of looking at modus ponens is to define it as a partial function $\vdash : F \times F \to F,$ where $F$ is a set of formulas in a language $L$ where a binary operation $\to$ is defined, such that 
\begin{enumerate}
\item
$\vdash(A, B)$ is defined whenever $A, B \in F$ and $B \equiv (A \to C)$ for some $C \in L$, and
\item
when this is the case, $C \in F$ and $\vdash(A, B) := C$;
\item
$\vdash$ is not defined otherwise.
\end{enumerate}

\textbf{Remark}.  With modus ponens, one can easily prove the converse of the deduction theorem (see \PMlinkname{this link}{DeductionTheorem}).  Another easily proven fact is the following: 
\begin{quote}\begin{center}
If $\Delta \vdash A$ and $\Delta\vdash A\to B$, then $\Delta \vdash B$, where $\Delta$ is a set of formulas.
\end{center}\end{quote}
To see this, let $A_1,\ldots, A_n$ be a deduction of $A$ from $\Delta$, and $B_1,\ldots, B_m$ be a deduction of $A\to B$ from $\Delta$.  Then $A_1,\ldots, A_n, B_1,\ldots, B_m, B$ is a deduction of $B$ from $\Delta$, where $B$ is inferred from $A_n$ (which is $A$) and $B_m$ (which is $A\to B$) by modus ponens.
%%%%%
%%%%%
\end{document}
