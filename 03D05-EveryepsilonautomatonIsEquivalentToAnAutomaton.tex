\documentclass[12pt]{article}
\usepackage{pmmeta}
\pmcanonicalname{EveryepsilonautomatonIsEquivalentToAnAutomaton}
\pmcreated{2013-03-22 19:02:03}
\pmmodified{2013-03-22 19:02:03}
\pmowner{CWoo}{3771}
\pmmodifier{CWoo}{3771}
\pmtitle{every $\epsilon$-automaton is equivalent to an automaton}
\pmrecord{8}{41907}
\pmprivacy{1}
\pmauthor{CWoo}{3771}
\pmtype{Definition}
\pmcomment{trigger rebuild}
\pmclassification{msc}{03D05}
\pmclassification{msc}{68Q45}

\endmetadata

\usepackage{amssymb,amscd}
\usepackage{amsmath}
\usepackage{amsfonts}
\usepackage{mathrsfs}

% used for TeXing text within eps files
%\usepackage{psfrag}
% need this for including graphics (\includegraphics)
%\usepackage{graphicx}
% for neatly defining theorems and propositions
\usepackage{amsthm}
% making logically defined graphics
%%\usepackage{xypic}
\usepackage{pst-plot}

% define commands here
\newcommand*{\abs}[1]{\left\lvert #1\right\rvert}
\newtheorem{prop}{Proposition}
\newtheorem{thm}{Theorem}
\newtheorem{ex}{Example}
\newcommand{\real}{\mathbb{R}}
\newcommand{\pdiff}[2]{\frac{\partial #1}{\partial #2}}
\newcommand{\mpdiff}[3]{\frac{\partial^#1 #2}{\partial #3^#1}}
\begin{document}
In this entry, we show that an automaton with \PMlinkname{$\epsilon$-transitions}{EpsilonTransition} is no more power than one without.  Having $\epsilon$-transitions is purely a matter of convenience.

\begin{prop} Every \PMlinkname{$\epsilon$-automaton}{EpsilonAutomaton} is equivalent to an automaton. \end{prop}

For the proof, we use the following setup (see the parent entry for more detail):
\begin{itemize}
\item $E=(S,\Sigma,\delta,I,F,\epsilon)$ is an $\epsilon$-automaton, and $E_{\epsilon}$ is the automaton associated with $E$,
\item $h:(\Sigma\cup\lbrace \epsilon\rbrace)^* \to \Sigma^*$ is the homomorphism that erases $\epsilon$ (it takes $\epsilon$ to the empty word, also denoted by $\epsilon$).  From the parent entry, $L(E):=h(L(E_{\epsilon}))$.
\end{itemize}

\begin{proof}  Define a function $\delta_1: S\times \Sigma\to P(S)$, as follows: for each pair $(s,a)\in S\times \Sigma$, let $$\delta_1(s,a)= \bigcup \, \lbrace \, \delta(s,u)\mid h(u)=a \, \rbrace.$$

In other words, $\delta_1(s,a)$ is the set of all states reachable from $s$ by words of the form $\epsilon^m a\epsilon^n$.  As usual, we extend $\delta_1$ so its domain is $S\times \Sigma^*$.  By abuse of notation, we use $\delta_1$ again for this extension.  First, we set $\delta_1(s,\epsilon): = \lbrace s\rbrace$.  Then we inductively define $\delta_1(s,ua)= \delta_1(\delta_1(s,u),a)$.  Using induction, 
\begin{eqnarray*}
\delta_1(s,ua) &=& \delta_1(\delta_1(s,u),a) \\ 
&=& \delta_1(\bigcup_{h(v)=u} \delta(s,v),a) \\
&=& \bigcup_{h(v)=u} \delta_1(\delta(s,v),a) \\
&=& \bigcup_{h(v)=u} \; \bigcup_{t\in \delta(s,v)} \delta_1(t,a) \\
&=& \bigcup_{h(v)=u} \; \bigcup_{t\in \delta(s,v)} \; \bigcup_{h(w)=a} \delta(t,w) \\
&=& \bigcup_{h(v)=u} \; \bigcup_{h(w)=a} \; \bigcup_{t\in \delta(s,v)} \delta(t,w) \\
&=& \bigcup_{h(v)=u} \; \bigcup_{h(w)=a} \delta(\delta(s,v),w) \\
&=& \bigcup_{h(v)=u} \; \bigcup_{h(w)=a} \delta(s,vw) \\
&=& \bigcup \lbrace \delta(s,vw) \mid h(v)=u \mbox{ and } h(w)=a \rbrace \\
&=& \bigcup \lbrace \delta(s,x) \mid h(x)=ua \rbrace
\end{eqnarray*}
So for any non-empty word $u$, we have the following equation:
\begin{equation}
\delta_1(s,u) = \bigcup \, \lbrace \, \delta(s,v)\mid h(v)=u \, \rbrace.
\end{equation}

In other words, if $u=a_1a_2\cdots a_n$, then $\delta_1(s,u)$ is the set of all states reachable from $s$ by words of the form 
\begin{equation}
\epsilon^{i_0}a_1\epsilon^{i_1}a_2 \epsilon^{i_2}\cdots \epsilon^{i_{n-1}}a_n \epsilon^{i_n}.
\end{equation}

Now, define $A$ to be the automaton $(S,\Sigma,\delta_1,I,F)$.  Then, from equation (1) above, a word $$u=a_1 a_2\cdots a_n$$ is accepted by $A$ iff some word $v$ of the form (2) is accepted by $E_{\epsilon}$ iff $u=h(v)$ is accepted by $E$, proving the proposition.
\end{proof}

\textbf{Remark}.  Another approach is to use the concept of \PMlinkname{$\epsilon$-closure}{EpsilonClosure}.  The proof is very similar to the one given above, and the resulting equivalent automaton is a DFA.
%%%%%
%%%%%
\end{document}
