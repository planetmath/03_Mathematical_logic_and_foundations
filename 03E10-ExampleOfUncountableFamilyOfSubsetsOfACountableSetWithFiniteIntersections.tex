\documentclass[12pt]{article}
\usepackage{pmmeta}
\pmcanonicalname{ExampleOfUncountableFamilyOfSubsetsOfACountableSetWithFiniteIntersections}
\pmcreated{2013-03-22 19:16:29}
\pmmodified{2013-03-22 19:16:29}
\pmowner{joking}{16130}
\pmmodifier{joking}{16130}
\pmtitle{example of uncountable family of subsets of a countable set with finite intersections}
\pmrecord{4}{42206}
\pmprivacy{1}
\pmauthor{joking}{16130}
\pmtype{Example}
\pmcomment{trigger rebuild}
\pmclassification{msc}{03E10}

\endmetadata

% this is the default PlanetMath preamble.  as your knowledge
% of TeX increases, you will probably want to edit this, but
% it should be fine as is for beginners.

% almost certainly you want these
\usepackage{amssymb}
\usepackage{amsmath}
\usepackage{amsfonts}

% used for TeXing text within eps files
%\usepackage{psfrag}
% need this for including graphics (\includegraphics)
%\usepackage{graphicx}
% for neatly defining theorems and propositions
%\usepackage{amsthm}
% making logically defined graphics
%%%\usepackage{xypic}

% there are many more packages, add them here as you need them

% define commands here

\begin{document}
We wish to give an answer to the following:

\textbf{Problem.} Assume, that $X$ is a countable set. Is there a family $\{X_i\}_{i\in I}$ of subsets of $X$ such that $I$ is an uncountable set, but for any $i\neq j\in I$ the intersection $X_i\cap X_j$ is finite?

\textbf{Example.} Let $x\in [1,2)$ be a real number. Express $x$ using digits
$$x=1.x_1x_2x_3x_4\cdots = 1 + \sum_{i=1}^{\infty}x_i\cdot 10^{-i}$$
where each $x_i\in\{0,1,2,3,4,5,6,7,8,9\}$. With $x$ we associate the following natural numbers
$$\beta_n(x)=1x_1x_2x_3\cdots x_{n-1} x_n=10^{n+1} + \sum_{i=1}^{n}x_{i}\cdot 10^{n-i+1}.$$
Now define $A:[1,2)\to \mathrm{P}(\mathbb{N})$ (here $\mathrm{P}(X)$ stands for ,,the power set of $X$'') by $$A(x)=\{\beta_1(x),\beta_2(x),\beta_3(x),\ldots\}.$$
$A$ is injective. Indeed, note that for any $x,y\in[1,2)$ if $\beta_i(x)=\beta_j(y)$, then $i=j$ (this is because equal $\beta$ numbers have equal ,,length'' and this is because each $\beta$ has $1$ at the begining, zeros are not the problem). Therefore, if $A(x)=A(y)$ for some $x,y$, then it follows, that $\beta_i(x)=\beta_i(y)$ for each $i$, but this implies that corresponding digits of $x$ and $y$ are equal. Thus $x=y$.

This shows, that $\{A(x)\}_{x\in [1,2)}$ is an uncountable family of subsets of $\mathbb{N}$. Now in order to prove that $A(x)\cap A(y)$ is finite whenever $x\neq y$ it is enough to show that we can uniquely reconstruct $x$ from any infinite sequence of numbers from $A(x)$. This can be proved by using similar techniques as before and we leave it as a simple exercise.
%%%%%
%%%%%
\end{document}
