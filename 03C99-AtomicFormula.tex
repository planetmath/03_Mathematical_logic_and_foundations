\documentclass[12pt]{article}
\usepackage{pmmeta}
\pmcanonicalname{AtomicFormula}
\pmcreated{2013-03-22 12:42:54}
\pmmodified{2013-03-22 12:42:54}
\pmowner{CWoo}{3771}
\pmmodifier{CWoo}{3771}
\pmtitle{atomic formula}
\pmrecord{10}{33002}
\pmprivacy{1}
\pmauthor{CWoo}{3771}
\pmtype{Definition}
\pmcomment{trigger rebuild}
\pmclassification{msc}{03C99}
\pmclassification{msc}{03B10}
\pmsynonym{quantifier free formula}{AtomicFormula}
\pmrelated{TermsAndFormulas}
\pmrelated{CNF}
\pmrelated{DNF}
\pmdefines{literal}
\pmdefines{clause}
\pmdefines{quantifier-free formula}
\pmdefines{positive literal}
\pmdefines{negative literal}

% this is the default PlanetMath preamble.  as your knowledge
% of TeX increases, you will probably want to edit this, but
% it should be fine as is for beginners.

% almost certainly you want these
\usepackage{amssymb}
\usepackage{amsmath}
\usepackage{amsfonts}

% used for TeXing text within eps files
%\usepackage{psfrag}
% need this for including graphics (\includegraphics)
%\usepackage{graphicx}
% for neatly defining theorems and propositions
%\usepackage{amsthm}
% making logically defined graphics
\usepackage[arrow,curve,poly,arc,2cell,frame,web]{xypic}

% there are many more packages, add them here as you need them

% define commands here
\newcommand{\br}{[\![}
\newcommand{\rb}{]\!]}
\newcommand{\oq}{\text{``}}
\newcommand{\cq}{\text{''}}


\newcommand{\im}{\mathbf{Im}}
\newcommand{\dom}{\mathbf{Dom}}


\newcommand{\Or}{\vee}
\newcommand{\Implies}{\Rightarrow}
\newcommand{\Iff}{\Leftrightarrow}
\newcommand{\proves}{\vdash}
\renewcommand{\And}{\wedge}
\newcommand{\Sup}{\bigwedge}
\newcommand{\Inf}{\bigvee}
\newcommand{\Z}{\mathbb{Z}}
\newcommand{\F}{\mathbb{F}}
\newcommand{\Q}{\mathbb{Q}}
\newcommand{\R}{\mathbb{R}}
\newcommand{\C}{\mathbb{C}}
\newcommand{\Nat}{\mathbb{N}}
\newcommand{\M}{\mathfrak{M}}
\newcommand{\N}{\mathfrak{N}}
\newcommand{\A}{\mathfrak{A}}
\newcommand{\B}{\mathfrak{B}}
\newcommand{\K}{\mathfrak{K}}
\newcommand{\G}{\mathbb{G}}
\newcommand{\Def}{\overset{\operatorname{def}}{:=}}



\newcommand{\spec}{\text{{\bf Spec}}}
\newcommand{\stab}{\text{{\bf Stab}}}
\newcommand{\ann}{\text{{\bf Ann}}}
\newcommand{\irr}{\text{{\bf Irr}}}
\newcommand{\qt}{\text{{\bf Qt}}}
\newcommand{\st}{\mathcal{Qt}}
\newcommand{\ro}{\mathbf{r.o.}}


\newcommand{\Endo}{\text{{\bf End}}}
\newcommand{\mat}{\text{{\bf Mat}}}
\newcommand{\der}{\text{{\bf Der}}}
\newcommand{\rad}{\text{{\bf Rad}}}
\newcommand{\trd}{\text{{\bf tr.d.}}}
\newcommand{\cl}{\text{{\bf acl}}}
\newcommand{\Int}{\text{{\bf int}}}
\newcommand{\V}{\mathbb{V}}
\newcommand{\D}{\mathbf{D}}

\newcommand{\del}{\partial}
\renewcommand{\O}{\mathcal{O}}
\newcommand{\aut}{\mathbf{Aut}}
\newcommand{\height}{\text{\bf Height}}
\newcommand{\coheight}{\text{\bf Co-height}}

\newcommand{\lcm}{\operatorname{lcm}}

\newcommand{\Gal}{\operatorname{Gal}}
\newcommand{\x}{\mathbf{x}}
\newcommand{\y}{\mathbf{y}}
\newcommand{\inner}[2]{\langle #1|#2\rangle}
\renewcommand{\r}{{r}}
\renewcommand{\t}{{t}}

\newcommand{\restr}{\upharpoonright}
\newcommand{\Matrix}[4]{\left(\begin{array}{cc} #1 & #2 \\ #3 & #4 
\end{array}\right)}
\begin{document}
Let $\Sigma$ be a signature and $T(\Sigma)$ the set of terms over $\Sigma$.  The set $S$ of symbols for $T(\Sigma)$ is the disjoint union of $\Sigma$ and $V$, a countably infinite set whose elements are called \emph{variables}.  Now, adjoin $S$ the set $\lbrace =, (, )\rbrace$, assumed to be disjoint from $S$.  An \emph{atomic formula} $\varphi$ over $\Sigma$ is any one of the following:
\begin{enumerate}
\item either $(t_1=t_2)$, where $t_1$ and $t_2$ are terms in $T(\Sigma)$, 
\item or $(R(t_1,...,t_n))$, where $R\in\Sigma$ is an $n$-ary relation symbol, and $t_i\in T(\Sigma)$.
\end{enumerate}

\textbf{Remarks}.  
\begin{enumerate}
\item
Using atomic formulas, one can inductively build formulas using the logical connectives $\vee$, $\neg$, $\exists$, etc... In this sense, atomic formulas are formulas that can not be broken down into simpler formulas; they are the building blocks of formulas.
\item 
A \emph{literal} is a formula that is either atomic or of the form $\neg \varphi$ where $\varphi$ is atomic.  If a literal is atomic, it is called a \emph{positive literal}.  Otherwise, it is a \emph{negative literal}.
\item
A finite disjunction of literals is called a \emph{clause}.  In other words, a clause is a formula of the form $\varphi_1 \vee \varphi_2 \vee \cdots \vee \varphi_n$, where each $\varphi_i$ is a literal.
\item 
A \emph{qunatifier-free formula} is a formula that does not contain the symbols $\exists$ or $\forall$.  
\item
If we identify a formula $\varphi$ with its double negation $\neg (\neg \varphi)$, then it can be shown that any quantifier-free formula can be identified with a formula that is in conjunctive normal form, that is, a finite conjunction of clauses.  For a proof, see this \PMlinkname{link}{EveryPropositionIsEquivalentToAPropositionInDNF}
\end{enumerate}
%%%%%
%%%%%
\end{document}
