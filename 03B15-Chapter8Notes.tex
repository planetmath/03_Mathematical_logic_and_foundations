\documentclass[12pt]{article}
\usepackage{pmmeta}
\pmcanonicalname{Chapter8Notes}
\pmcreated{2013-11-06 15:41:37}
\pmmodified{2013-11-06 15:41:37}
\pmowner{PMBookProject}{1000683}
\pmmodifier{rspuzio}{6075}
\pmtitle{Chapter 8 Notes}
\pmrecord{1}{}
\pmprivacy{1}
\pmauthor{PMBookProject}{6075}
\pmtype{Feature}
\pmclassification{msc}{03B15}

\usepackage{amssyb}
\usepackage{amsmath}
\usepackage{amsfonts}
\usepackage{amsthm}
\makeatletter
\newcommand{\sectionNotes}{\phantomsection\section*{Notes}\addcontentsline{toc}{section}{Notes}\markright{\textsc{\@chapapp{} \thechapter{} Notes}}}
\newcommand{\Sn}{\mathbb{S}}
\let\autoref\cref
\makeatother

\begin{document}
%% For the spheres,  two
%% different definitions of the $n$-sphere $\Sn ^n$ have been used: the first as the
%% suspension of $\Sn ^ {n-1}$ (\autoref{sec:suspension}), and the second
%% as a higher inductive type with one base point and one loop in
%% $\Omega^n$ (\autoref{sec:circle}); we list the status for both
%% definitions.  The equivalence of the two can be deduced from the remarks
%% at the end of \autoref{sec:suspension}.

{
\newcommand{\humancheck}{\ding{52}}
\newcommand{\computercheck}{\ding{52}\kern-0.5em\ding{52}}
\begin{table}[htb]
  \centering
\begin{tabular}{lcc}
\toprule
Theorem         & Status \\
\midrule
$\pi_1(\Sn ^1)$                     & \computercheck \\
$\pi_{k<n}(\Sn ^n)$                  & \computercheck \\
%% $\pi_{k<n}(\Sn ^n)$ --- suspension  & \computercheck \\
%% $\pi_2(\Sn ^2)$ --- suspension      & \computercheck \\
long-exact-sequence of homotopy groups & \computercheck    \\
total space of Hopf fibration is $\Sn ^3$ & \humancheck    \\
$\pi_2(\Sn ^2)$                     & \computercheck \\
$\pi_3(\Sn ^2)$                     & \humancheck    \\
$\pi_n(\Sn ^n)$                     & \computercheck \\
%% $\pi_n(\Sn ^n)$ --- suspension      & \computercheck \\
$\pi_4(\Sn ^3)$                     & \humancheck    \\
Freudenthal suspension theorem      & \computercheck \\
Blakers--Massey theorem              & \computercheck \\
Eilenberg--Mac Lane spaces $K(G,n)$ & \computercheck \\
van Kampen theorem                & \computercheck \\
covering spaces                     & \computercheck \\
Whitehead's principle for $n$-types & \computercheck \\
\bottomrule
\end{tabular}
\caption{Theorems from homotopy theory proved by
  hand (\humancheck) and by computer (\computercheck).}
  \label{tab:theorems}
\end{table}
}

The theorems described in this chapter are standard results in classical
homotopy theory; many are described by \cite{hatcher02topology}.  In these
notes, we review the development of the new synthetic proofs of them in homotopy
type theory.  \autoref{tab:theorems} lists the homotopy-theoretic
theorems that have been proven in homotopy type theory, and whether they
have been computer-checked.
Almost all of these results were developed during the spring term at IAS
in 2013, as part of a significant collaborative effort.  Many people
contributed to these results, for example by being the principal author
of a proof, by suggesting problems to work on, by participating in many
discussions and seminars about these problems, or by giving feedback on
results.  The following people were the principal authors of the first
homotopy type theory proofs of the above theorems. Unless indicated otherwise, for the
theorems that have been computer-checked, the principal authors were
also the first ones to formalize\index{mathematics!formalized} the proof using a computer proof
assistant.
\begin{itemize}
\item 
Shulman gave the homotopy-theoretic calculation of $\pi_1(\Sn^1)$.  Licata later discovered the
encode-decode proof and the encode-decode method.

\item 
Brunerie calculated $\pi_{k<n}(\Sn ^ n)$.
Licata later gave an encode-decode version.

\item Voevodsky constructed the long exact sequence of homotopy groups. 

\item Lumsdaine constructed the Hopf fibration.
  Brunerie proved that its total space is $\Sn ^3$, thereby calculating $\pi_2(\Sn ^2)$ and
  $\pi_3(\Sn ^3)$.

\item Licata and Brunerie gave a direct calculation of
$\pi_n(\Sn ^n)$.  

\item 
  Lumsdaine proved the Freudenthal suspension theorem; Licata and
  Lumsdaine formalized this proof.
\item Lumsdaine, Finster, and Licata proved the Blakers--Massey theorem;
  Lumsdaine, Brunerie, Licata, and Hou formalized it.

\item 
Licata gave an encode-decode calculation of $\pi_2(\Sn ^2)$, and a
calculation of $\pi_n(\Sn ^n)$ using the Freudenthal suspension theorem; using similar
techniques, he constructed $K(G,n)$.

\item 
Shulman proved the van Kampen theorem; Hou formalized this proof.

\item 
Licata proved Whitehead's theorem for $n$-types.

\item Brunerie calculated $\pi_4(\Sn ^3)$.

\item 
Hou established the theory of covering spaces and formalized it.
\end{itemize}

The interplay between homotopy theory and type theory was crucial to the
development of these results.  For example, the first proof that
$\pi_1(\Sn ^1)=\mathbb{Z}$ was the one given in \autoref{subsec:pi1s1-homotopy-theory}, which follows a classical homotopy theoretic one.  A
type-theoretic analysis of this proof resulted in the development of the
encode-decode method.  The first calculation of $\pi_2(\Sn ^2)$ also followed
classical methods, but this led quickly to an encode-decode proof of the
result.  The encode-decode calculation generalized to $\pi_n(\Sn ^n)$, which
in turn led to the proof of the Freudenthal suspension theorem, by
combining an encode-decode argument with classical homotopy-theoretic
reasoning about connectedness, which in turn led to the Blakers--Massey
theorem and Eilenberg--Mac Lane spaces.  The rapid development of this
series of results illustrates the promise of our new understanding of
the connections between these two subjects.


\end{document}
