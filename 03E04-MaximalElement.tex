\documentclass[12pt]{article}
\usepackage{pmmeta}
\pmcanonicalname{MaximalElement}
\pmcreated{2013-03-22 12:30:44}
\pmmodified{2013-03-22 12:30:44}
\pmowner{akrowne}{2}
\pmmodifier{akrowne}{2}
\pmtitle{maximal element}
\pmrecord{9}{32749}
\pmprivacy{1}
\pmauthor{akrowne}{2}
\pmtype{Definition}
\pmcomment{trigger rebuild}
\pmclassification{msc}{03E04}
\pmdefines{greatest element}
\pmdefines{least element}
\pmdefines{minimal element}

\endmetadata

\usepackage{amssymb}
\usepackage{amsmath}
\usepackage{amsfonts}

%\usepackage{psfrag}
%\usepackage{graphicx}
%%%\usepackage{xypic}
\begin{document}
Let $\le$ be an ordering on a set $S$, and let $A \subseteq S$. Then, with respect  to the ordering $\le$, 

\begin{itemize}

\item $a \in A$ is the \emph{least} element of $A$ if $a \le x$, for all $x \in A$.
\item $a \in A$ is a \emph{minimal} element of $A$ if there exists no $x \in A$ such that $x \le a$ and $x \ne a$.
\item $a \in A$ is the \emph{greatest} element of $A$ if $x \le a$ for all $x \in A$.
\item $a \in A$ is a \emph{maximal} element of $A$ if there exists no $x \in A$ such that $a \le x$ and $x \ne a$.

\end{itemize}

\paragraph{Examples.}

\begin{itemize}
\item The natural numbers $\mathbb{N}$ ordered by divisibility ($\mid$) have a least element, $1$.  The natural numbers greater than 1 ($\mathbb{N} \setminus \{1\}$) have no least element, but infinitely many minimal elements (the primes.)  In neither case is there a greatest or maximal element.
\item  The negative integers ordered by the standard definition of $\le$ have a  maximal element which is also the greatest element, $-1$.  They have no minimal or least element.
\item The natural numbers $\mathbb{N}$ ordered by the standard $\le$ have a least element, $1$, which is also a minimal element.  They have no greatest or maximal element.  
\item The rationals greater than zero with the standard ordering $\le$ have no least element or minimal element, and no maximal or greatest element.
\end{itemize}
%%%%%
%%%%%
\end{document}
