\documentclass[12pt]{article}
\usepackage{pmmeta}
\pmcanonicalname{ParadoxOfTheBinaryTree}
\pmcreated{2013-03-22 17:02:37}
\pmmodified{2013-03-22 17:02:37}
\pmowner{WM}{16977}
\pmmodifier{WM}{16977}
\pmtitle{paradox of the binary tree}
\pmrecord{21}{39332}
\pmprivacy{1}
\pmauthor{WM}{16977}
\pmtype{Definition}
\pmcomment{trigger rebuild}
\pmclassification{msc}{03E75}
\pmclassification{msc}{03E15}
\pmsynonym{binary tree paradox}{ParadoxOfTheBinaryTree}
%\pmkeywords{set theory}
%\pmkeywords{Cantor's theorem}
%\pmkeywords{uncountability}
\pmdefines{complete binary tree}
\pmdefines{complete infinite binary tree}

\endmetadata

% this is the default PlanetMath preamble.  as your knowledge
% of TeX increases, you will probably want to edit this, but
% it should be fine as is for beginners.

% almost certainly you want these
\usepackage{amssymb}
\usepackage{amsmath}
\usepackage{amsfonts}

% used for TeXing text within eps files
%\usepackage{psfrag}
% need this for including graphics (\includegraphics)
%\usepackage{graphicx}
% for neatly defining theorems and propositions
%\usepackage{amsthm}
% making logically defined graphics
%%\usepackage{xypic}

% there are many more packages, add them here as you need them

% define commands here

\begin{document}
The complete infinite binary tree is a tree that consists of nodes (namely the numerals 0 and 1) such that every node has two children which are not children of any other node. The tree serves as binary representation of all real numbers of the interval [0,1] in form of paths, i.e., sequences of nodes.

Every finite binary tree with more than one level contains less paths than nodes. Up to level n there are 2^n paths and 2^(n+1) - 1 nodes.

Every finite binary tree can be represented as an ordered set of nodes, enumerated by natural numbers. The union of all finite binary trees is then identical with the infinite binary tree. The paradox is that, while the set of nodes
remains countable as is the set of paths of all finite trees, the set of paths in the infinite tree is uncountable by Cantor's theorem. (On the other hand, the paths are separated by the nodes. As no path can separate itself from another path without a node, the number of separated paths is the number of nodes.) 


Literature: W. M\"uckenheim: Die Mathematik des Unendlichen, Shaker-Verlag, Aachen 2006.

%%%%%
%%%%%
\end{document}
