\documentclass[12pt]{article}
\usepackage{pmmeta}
\pmcanonicalname{FormulasForStudentsOfClass8th}
\pmcreated{2013-03-11 19:14:42}
\pmmodified{2013-03-11 19:14:42}
\pmowner{Expert}{27887}
\pmmodifier{}{0}
\pmtitle{Formulas for Students of Class 8th.}
\pmrecord{4}{42616}
\pmprivacy{1}
\pmauthor{Expert}{0}
\pmtype{Theorem}
\pmclassification{msc}{03-00}
\pmsynonym{}{FormulasForStudentsOfClass8th}
%\pmkeywords{}
\pmrelated{}
\pmdefines{}


\begin{document}
1. Area of Trapezium -> 1/2*Sum of parallel sides*Distance between parallel sides.

2. Area of Rectangle -> Length*Breadth 

3. Perimeter of Rectangle -> 2(Length+Breadth) 

4. Area of Square -> Side*Side 

5. Perimeter of Square -> 4*Side 

6. Area of Genereal Quadrilateral -> 1/2*d(h1+h2) ,where "d" stands for diameter and "h1" and "h2" stands for height first and height sceond. 

7. Area of Triangle -> 1/2*breadth*height 

8. Perimeter of Triangle -> Sum of all 3 sides. 

9. Area of Quadrilateral -> 4*Side*Side 

10. Area of Rhombus -> 1/2*D1*D2 , where "D1 and D2" stands for Diameter first and Diameter second. 

I will post some more formulas later.

--Mrigendra

%%%%%
\end{document}
