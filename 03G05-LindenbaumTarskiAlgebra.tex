\documentclass[12pt]{article}
\usepackage{pmmeta}
\pmcanonicalname{LindenbaumTarskiAlgebra}
\pmcreated{2013-03-22 12:42:40}
\pmmodified{2013-03-22 12:42:40}
\pmowner{CWoo}{3771}
\pmmodifier{CWoo}{3771}
\pmtitle{Lindenbaum-Tarski algebra}
\pmrecord{31}{32997}
\pmprivacy{1}
\pmauthor{CWoo}{3771}
\pmtype{Definition}
\pmcomment{trigger rebuild}
\pmclassification{msc}{03G05}
\pmclassification{msc}{03B05}
\pmclassification{msc}{03B10}
\pmsynonym{Lindenbaum algebra}{LindenbaumTarskiAlgebra}

% this is the default PlanetMath preamble.  as your knowledge
% of TeX increases, you will probably want to edit this, but
% it should be fine as is for beginners.

% almost certainly you want these
\usepackage{amssymb}
\usepackage{amsmath}
\usepackage{color}
\usepackage{amsfonts}

% used for TeXing text within eps files
%\usepackage{psfrag}
% need this for including graphics (\includegraphics)
%\usepackage{graphicx}
% for neatly defining theorems and propositions
%\usepackage{amsthm}
% making logically defined graphics
\usepackage[arrow,curve,poly,arc,2cell,frame,web]{xypic}

% there are many more packages, add them here as you need them

% define commands here
\newcommand{\br}{[\![}
\newcommand{\rb}{]\!]}
\newcommand{\oq}{\text{``}}
\newcommand{\cq}{\text{''}}


\newcommand{\im}{\mathbf{Im}}
\newcommand{\dom}{\mathbf{Dom}}


\newcommand{\Or}{\vee}
\newcommand{\Implies}{\Rightarrow}
\newcommand{\Iff}{\Leftrightarrow}
\newcommand{\proves}{\vdash}
\renewcommand{\And}{\wedge}
\newcommand{\Sup}{\bigwedge}
\newcommand{\Inf}{\bigvee}
\newcommand{\Z}{\mathbb{Z}}
\newcommand{\F}{\mathbb{F}}
\newcommand{\Q}{\mathbb{Q}}
\newcommand{\R}{\mathbb{R}}
\newcommand{\C}{\mathbb{C}}
\newcommand{\Nat}{\mathbb{N}}
\newcommand{\M}{\mathfrak{M}}
\newcommand{\N}{\mathfrak{N}}
\newcommand{\A}{\mathfrak{A}}
\newcommand{\B}{\mathfrak{B}}
\newcommand{\K}{\mathfrak{K}}
\newcommand{\G}{\mathbb{G}}
\newcommand{\Def}{\overset{\operatorname{def}}{:=}}



\newcommand{\spec}{\text{{\bf Spec}}}
\newcommand{\stab}{\text{{\bf Stab}}}
\newcommand{\ann}{\text{{\bf Ann}}}
\newcommand{\irr}{\text{{\bf Irr}}}
\newcommand{\qt}{\text{{\bf Qt}}}
\newcommand{\st}{\mathcal{Qt}}
\newcommand{\ro}{\mathbf{r.o.}}


\newcommand{\Endo}{\text{{\bf End}}}
\newcommand{\mat}{\text{{\bf Mat}}}
\newcommand{\der}{\text{{\bf Der}}}
\newcommand{\rad}{\text{{\bf Rad}}}
\newcommand{\trd}{\text{{\bf tr.d.}}}
\newcommand{\cl}{\text{{\bf acl}}}
\newcommand{\Int}{\text{{\bf int}}}
\newcommand{\V}{\mathbb{V}}
\newcommand{\D}{\mathbf{D}}

\newcommand{\del}{\partial}
\renewcommand{\O}{\mathcal{O}}
\newcommand{\aut}{\mathbf{Aut}}
\newcommand{\height}{\text{\bf Height}}
\newcommand{\coheight}{\text{\bf Co-height}}

\newcommand{\lcm}{\operatorname{lcm}}

\newcommand{\Gal}{\operatorname{Gal}}
\newcommand{\x}{\mathbf{x}}
\newcommand{\y}{\mathbf{y}}
\newcommand{\inner}[2]{\langle #1|#2\rangle}
\renewcommand{\r}{{r}}
\renewcommand{\t}{{t}}

\newcommand{\restr}{\upharpoonright}
\newcommand{\Matrix}[4]{\left(\begin{array}{cc} #1 & #2 \\ #3 & #4 
\end{array}\right)}

\begin{document}
\subsubsection*{Lindenbaum-Tarski algebra of a propositional langauge}

Let $L$ be a classical propositional language.  We define the equivalence relation $\sim$ over formulas of $L$ by $\varphi\sim\psi$ if and only if $\proves\varphi\Iff\psi$.  Let $B=L/\sim$ be the set of equivalence classes.  We define the operations join $\vee$, meet $\wedge$, and complementation, denoted $[\varphi]'$ on $B$ by :
\begin{align*}
[\varphi]\vee [\psi]&:=[\varphi\Or\psi]\\
[\varphi]\wedge [\psi]&:=[\varphi\And\psi]\\
[\varphi]'&:=[\neg\varphi]
\end{align*}
We let $0=[\varphi\And\neg\varphi]$ and $1=[\varphi\Or\neg\varphi]$.  Then the structure $(B,\vee,\wedge,',0,1)$ is a Boolean algebra, called the \emph{Lindenbaum-Tarski algebra} of the propositional language $L$.

Intuitively, this algebra is an algebra of logical statements in which logically equivalent formulations of the same statement are not distinguished.  One can develop intuition for this algrebra by considering a simple case.  Suppose our language consists of a number of statement symbols $P_i$ and the connectives $\lor, \land, \neg$ and that $\proves$
denotes tautologies.  Then our algebra consists of statements formed from these connectives with tautologously equivalent satements reckoned as the same element of the algebra.  For instance, ``$\neg (P_1 \land P_2)$'' is
considered the same as ``$\neg P_1 \lor \neg P_2$''.  Furthermore, since any statement of propositional calculus may be recast in disjunctive normal form, we may view this particular Lindenbaum-Tarski algebra as a Boolean analogue of polynomials in the $P_i$'s and their negations.

It can be shown that the Lindenbaum-Tarski algebra of the propositional language $L$ is a free Boolean algebra freely generated by the set of all elements $[p]$, where each $p$ is a propositional variable of $L$

\subsubsection*{Lindenbaum-Tarski algebra of a first order langauge}

Now, let $L$ be a first order language.  As before, we define the equivalence relation $\sim$ over formulas of $L$ by $\varphi\sim\psi$ if and only if $\proves\varphi\Iff\psi$.  Let $B=L/\sim$ be the set of equivalence classes.  The operations $\vee$ and $\wedge$ and complementation on $B$ are defined exactly the same way as previously.  The resulting algebra is the Lindenbaum-Tarski algebra of the first order language $L$.  It may be shown that 
\begin{align*}
\bigvee_{t\in T}[\varphi(t)]&:=[\exists x \varphi(x)]\\
\bigwedge_{t\in T}[\varphi(t)]&:=[\forall x \varphi(x)]
\end{align*}
where $T$ is the set of all terms in the language $L$.  Basically, these results say that the statement $\exists x \varphi(x)$ is equivalent to taking the supremum of all statements $\varphi(x)$ where $x$ ranges over the entire set $V$ of variables.  In other words, if one of these statements is true (with truth value $1$, as opposed to $0$), then $\exists x\varphi(x)$ is true.  The statement $\forall x\varphi(x)$ can be similarly analyzed.

\textbf{Remark}.  It may possible to define the Lindenbaum-Tarski algebra on logical languages other than the classical ones mentioned above, as long as there is a notion of formal proof that can allow the definition of the equivalence relation.  For example, one may form the Lindenbaum-Tarski algebra of an intuitionistic propositional language (or predicate language) or a normal modal propositional language.  The resulting algebra is a Heyting algebra (or a complete Heyting algebra) for intuitionistic propositional language (or predicate language), or a Boolean algebras with an operator for normal modal propositional languages.

%%%%%
%%%%%
\end{document}
