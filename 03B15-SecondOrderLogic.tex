\documentclass[12pt]{article}
\usepackage{pmmeta}
\pmcanonicalname{SecondOrderLogic}
\pmcreated{2013-03-22 13:00:17}
\pmmodified{2013-03-22 13:00:17}
\pmowner{Henry}{455}
\pmmodifier{Henry}{455}
\pmtitle{second order logic}
\pmrecord{8}{33385}
\pmprivacy{1}
\pmauthor{Henry}{455}
\pmtype{Definition}
\pmcomment{trigger rebuild}
\pmclassification{msc}{03B15}
\pmsynonym{second-order logic}{SecondOrderLogic}
\pmsynonym{second order}{SecondOrderLogic}
\pmsynonym{second-order}{SecondOrderLogic}
\pmrelated{IFLogic}
\pmdefines{second order language}
\pmdefines{second-order language}
\pmdefines{second order theory}
\pmdefines{second-order theory}

\endmetadata

% this is the default PlanetMath preamble.  as your knowledge
% of TeX increases, you will probably want to edit this, but
% it should be fine as is for beginners.

% almost certainly you want these
\usepackage{amssymb}
\usepackage{amsmath}
\usepackage{amsfonts}

% used for TeXing text within eps files
%\usepackage{psfrag}
% need this for including graphics (\includegraphics)
%\usepackage{graphicx}
% for neatly defining theorems and propositions
%\usepackage{amsthm}
% making logically defined graphics
%%%\usepackage{xypic}

% there are many more packages, add them here as you need them

% define commands here
%\PMlinkescapeword{theory}
\begin{document}
\emph{Second order logic} refers to logics with two (or three) types where one type consists of the objects of interest and the second is either sets of those objects or functions on those objects (or both, in the three type case).  For instance, second order arithmetic has two types: the numbers and the sets of numbers.

Formally, second order logic usually has:

\begin{itemize}
\item the standard quantifiers (four of them, since each type needs its own universal and existential quantifiers)

\item the standard connectives

\item the relation $=$ with its normal semantics

\item if the second type represents sets, a relation $\in$ where the first argument is of the first type and the second argument is the second type

\item if the second type represents functions, a binary function which takes one argument of each type and results in an object of the first type, representing function application
\end{itemize}

Specific second order logics may deviate from this definition slightly.  In particular, there are some first order logics with additional quantifiers whose strength is comparable to that of second order logic.  Some mathematicians have argued that these should be considered second order logics, despite not precisely matching the definition above.

Some people, chiefly Quine, have raised philisophical objections to second order logic, centering on the question of whether models require fixing some set of sets or functions as the ``actual'' sets or functions for the purposes of that model.
%%%%%
%%%%%
\end{document}
