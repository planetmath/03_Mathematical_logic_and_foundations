\documentclass[12pt]{article}
\usepackage{pmmeta}
\pmcanonicalname{ContradictoryStatement}
\pmcreated{2013-03-22 16:27:07}
\pmmodified{2013-03-22 16:27:07}
\pmowner{pahio}{2872}
\pmmodifier{pahio}{2872}
\pmtitle{contradictory statement}
\pmrecord{9}{38608}
\pmprivacy{1}
\pmauthor{pahio}{2872}
\pmtype{Definition}
\pmcomment{trigger rebuild}
\pmclassification{msc}{03B05}
\pmsynonym{contradiction}{ContradictoryStatement}
%\pmkeywords{false}
\pmrelated{Tautology}
\pmrelated{LogicalConnective}
\pmrelated{Contradiction}

\endmetadata

% this is the default PlanetMath preamble.  as your knowledge
% of TeX increases, you will probably want to edit this, but
% it should be fine as is for beginners.

% almost certainly you want these
\usepackage{amssymb}
\usepackage{amsmath}
\usepackage{amsfonts}

% used for TeXing text within eps files
%\usepackage{psfrag}
% need this for including graphics (\includegraphics)
%\usepackage{graphicx}
% for neatly defining theorems and propositions
 \usepackage{amsthm}
% making logically defined graphics
%%%\usepackage{xypic}

% there are many more packages, add them here as you need them

% define commands here

\theoremstyle{definition}
\newtheorem*{thmplain}{Theorem}

\begin{document}
\PMlinkescapeword{terms} \PMlinkescapeword{simple} \PMlinkescapeword{column}

A contradictory statement is a statement (or form) which is false due to its logical form rather than because of the meaning of the terms employed.

In propositional logic, a {\em contradictory statement}, a.k.a. {\em contradiction}, is a statement which is false regardless of the truth values of the substatements which form it.\, According to G. Peano, one may generally denote a contradiction with the symbol $\curlywedge$.

For a simple example, the statement\, $P\!\wedge\!\lnot P$\, is a contradiction for any statement $P$.

The negation $\lnot Q$ of every contradiction $Q$ is a tautology, and vice versa:
 $$\lnot\curlywedge = \curlyvee, \;\;\; \lnot\curlyvee = \curlywedge$$

To test a given statement or form to see if it is a contradiction, one may construct its truth table.\, If it turns out that every value of the last column is ``F'', then the statement is a contradiction.

Cf. the entry ``\PMlinkname{contradiction}{Contradiction}''.
%%%%%
%%%%%
\end{document}
