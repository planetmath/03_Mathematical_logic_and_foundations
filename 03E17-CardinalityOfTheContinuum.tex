\documentclass[12pt]{article}
\usepackage{pmmeta}
\pmcanonicalname{CardinalityOfTheContinuum}
\pmcreated{2013-03-22 14:15:33}
\pmmodified{2013-03-22 14:15:33}
\pmowner{yark}{2760}
\pmmodifier{yark}{2760}
\pmtitle{cardinality of the continuum}
\pmrecord{19}{35708}
\pmprivacy{1}
\pmauthor{yark}{2760}
\pmtype{Definition}
\pmcomment{trigger rebuild}
\pmclassification{msc}{03E17}
\pmclassification{msc}{03E10}
\pmsynonym{cardinal of the continuum}{CardinalityOfTheContinuum}
\pmsynonym{cardinal number of the continuum}{CardinalityOfTheContinuum}
\pmrelated{CardinalNumber}
\pmrelated{CardinalArithmetic}
\pmdefines{continuum many}

\endmetadata

\usepackage{amssymb}
\usepackage{amsmath}
\usepackage{amsfonts}

\def\N{\mathbb{N}}
\def\R{\mathbb{R}}
\def\continuum{\mathfrak{c}}
\begin{document}
\PMlinkescapeword{independent}
\PMlinkescapeword{nor}
\PMlinkescapeword{properties}

The \emph{cardinality of the continuum}, often denoted by $\continuum$, is 
the cardinality of the set $\R$ of real numbers.
A set of cardinality $\continuum$ is said to have \emph{continuum many} elements.

Cantor's diagonal argument shows that $\continuum$ is uncountable.
Furthermore, it can be shown that 
$\R$ is equinumerous with the power set of $\N$, so $\continuum=2^{\aleph_0}$.
It can also be shown that $\continuum$ has uncountable cofinality.

It can also be shown that
$$\continuum=\continuum^{\aleph_0}=\aleph_0\continuum=\continuum\continuum
=\continuum+\kappa=\continuum^n$$
for all finite cardinals $n\ge1$ and all cardinals $\kappa\le\continuum$.
See the article on cardinal arithmetic
for some of the basic facts underlying these equalities.

There are many properties of $\continuum$ that independent of ZFC,
that is, they can neither be proved nor disproved in ZFC,
assuming that ZF is consistent.
For example, for every nonzero natural number $n$, 
the equality $\continuum=\aleph_n$ is independent of ZFC.
(The case $n=1$ is the well-known
\PMlinkname{Continuum Hypothesis}{ContinuumHypothesis}.)
The same is true for most other alephs, 
although in some cases equality can be ruled out on the grounds of cofinality,
e.g., $\continuum\neq\aleph_\omega$.
In particular,
$\continuum$ could be either $\aleph_1$ or $\aleph_{\omega_1}$,
so it could be either a successor cardinal or a limit cardinal,
and either a regular cardinal or a singular cardinal.
%%%%%
%%%%%
\end{document}
