\documentclass[12pt]{article}
\usepackage{pmmeta}
\pmcanonicalname{PsiIsSurjectiveIfAndOnlyIfPsiastIsInjective}
\pmcreated{2013-03-22 14:36:03}
\pmmodified{2013-03-22 14:36:03}
\pmowner{matte}{1858}
\pmmodifier{matte}{1858}
\pmtitle{$\Psi$ is surjective if and only if $\Psi^\ast$ is injective}
\pmrecord{6}{36170}
\pmprivacy{1}
\pmauthor{matte}{1858}
\pmtype{Theorem}
\pmcomment{trigger rebuild}
\pmclassification{msc}{03-00}

\endmetadata

% this is the default PlanetMath preamble.  as your knowledge
% of TeX increases, you will probably want to edit this, but
% it should be fine as is for beginners.

% almost certainly you want these
\usepackage{amssymb}
\usepackage{amsmath}
\usepackage{amsfonts}
\usepackage{amsthm}

% used for TeXing text within eps files
%\usepackage{psfrag}
% need this for including graphics (\includegraphics)
%\usepackage{graphicx}
% for neatly defining theorems and propositions
%
% making logically defined graphics
%%%\usepackage{xypic}

% there are many more packages, add them here as you need them

% define commands here

\newcommand{\sR}[0]{\mathbb{R}}
\newcommand{\sC}[0]{\mathbb{C}}
\newcommand{\sN}[0]{\mathbb{N}}
\newcommand{\sZ}[0]{\mathbb{Z}}

 \usepackage{bbm}
 \newcommand{\Z}{\mathbbmss{Z}}
 \newcommand{\C}{\mathbbmss{C}}
 \newcommand{\R}{\mathbbmss{R}}
 \newcommand{\Q}{\mathbbmss{Q}}



\newcommand*{\norm}[1]{\lVert #1 \rVert}
\newcommand*{\abs}[1]{| #1 |}



\newtheorem{thm}{Theorem}
\newtheorem{defn}{Definition}
\newtheorem{prop}{Proposition}
\newtheorem{lemma}{Lemma}
\newtheorem{cor}{Corollary}
\begin{document}
Suppose $X$ is a set and
   $V$ is a vector space over a field $F$.
Let us denote by $M(X,V)$ the set of mappings from $X$ to $V$.
Now $M(X,V)$ is again a vector space if we equip it with 
pointwise multiplication and addition. In detail, 
if $f,g\in M(X,V)$ and $\mu,\lambda\in F$, we set
\begin{eqnarray*}
\mu f+\lambda g\colon x&\mapsto& \mu f(x) + \lambda g(x).
\end{eqnarray*}

Next, let $Y$ be another set, let $\Psi\colon X\to Y$ is a mapping,
and let $\Psi^\ast\colon M(Y,V)\to M(X,V)$ be the pullback of 
$\Psi$ as defined in \PMlinkname{this}{Pullback2} entry. 

\begin{prop} $\ $
\begin{enumerate}
\item $\Psi^\ast$ is linear. 
\item If $V$ is not the zero vector space, then 
$\Psi$ is surjective if and only if $\Psi^\ast$ is injective.
\end{enumerate}
\end{prop}

\begin{proof} First, suppose $f,g\in M(Y,V)$, $\mu,\lambda \in F$, and
$x\in X$. Then
\begin{eqnarray*}
\Psi^\ast(\mu f+\lambda g)(x) &=&(\mu f+\lambda g)(\Psi(x)) \\
   &=&\mu f\circ \Psi(x)+\lambda g\circ\Psi(x) \\
   &=&\left(\mu \Psi^\ast(f)+\lambda \Psi^\ast(g)\right)(x),
\end{eqnarray*}
so $\Psi^\ast(\mu f+\lambda g) = \mu \Psi^\ast(f)+\lambda \Psi^\ast(g)$,
and $\Psi^\ast$ is linear. 
For the second claim, suppose $\Psi$ is surjective,
    $f\in M(Y,V)$, 
   and $\Psi^\ast(f)=0$. If $y\in Y$, then for some $x\in X$, we have
$\Psi(x)=y$, and $f(y)=f\circ\Psi(x)=\Psi^\ast(f)(x)=0$, so $f=0$. 
Hence, the kernel of $\Psi^\ast$ is zero, and $\Psi^\ast$ is an 
injection. 
On the other hand, suppose $\Psi^\ast$ is a injection, and 
    $\Psi$ is not a surjection. Then for some $y'\in Y$, we have 
$y'\notin \Psi(X)$. Also, as $V$ is not the zero vector space, we can 
find a non-zero vector $v\in V$, and define $f\in M(Y,V)$ as
$$
  f(y)= \begin{cases} v, & \mbox{if}\ y=y', \\
                      0, & \mbox{if}\ y\neq y', y\in Y. \end{cases}
$$
Now $f\circ\Psi(x)=0$ for all $x\in X$, so $\Psi^\ast f=0$,
but $f\neq 0$. \end{proof}
%%%%%
%%%%%
\end{document}
