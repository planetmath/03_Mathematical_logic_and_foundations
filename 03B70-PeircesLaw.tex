\documentclass[12pt]{article}
\usepackage{pmmeta}
\pmcanonicalname{PeircesLaw}
\pmcreated{2013-11-05 23:16:13}
\pmmodified{2013-11-05 23:16:13}
\pmowner{Jon Awbrey}{15246}
\pmmodifier{Jon Awbrey}{15246}
\pmtitle{Peirce's law}
\pmrecord{30}{40254}
\pmprivacy{1}
\pmauthor{Jon Awbrey}{15246}
\pmtype{Topic}
\pmcomment{trigger rebuild}
\pmclassification{msc}{03B70}
\pmclassification{msc}{03B35}
\pmclassification{msc}{03B22}
\pmclassification{msc}{03B05}
\pmclassification{msc}{03-03}
\pmclassification{msc}{01A55}

\endmetadata

% this is the default PlanetMath preamble.  as your knowledge
% of TeX increases, you will probably want to edit this, but
% it should be fine as is for beginners.

% almost certainly you want these

\usepackage{amssymb}
\usepackage{amsmath}
\usepackage{amsfonts}
\usepackage{graphicx}

% used for TeXing text within eps files
%\usepackage{psfrag}
% need this for including graphics (\includegraphics)
%\usepackage{graphicx}
% for neatly defining theorems and propositions
%\usepackage{amsthm}
% making logically defined graphics
%%%\usepackage{xypic}

% there are many more packages, add them here as you need them

% define commands here

\begin{document}
\textbf{Peirce's law} is a formula in propositional calculus that is commonly expressed in the following form:

\[ ((p \Rightarrow q) \Rightarrow p) \Rightarrow p \]

Peirce's law holds in classical propositional calculus, but not in intuitionistic propositional calculus.  The precise axiom system that one chooses for classical propositional calculus determines whether Peirce's law is taken as an axiom or proven as a theorem.

\tableofcontents

\section{History}

Here is Peirce's own statement and proof of the law:

\begin{quote}
A \textit{fifth icon} is required for the principle of excluded middle and other propositions connected with it.  One of the simplest formulae of this kind is:

\[ \{ (x \,-\!\!\!< y) \,-\!\!\!< x \} \,-\!\!\!< x. \]

This is hardly axiomatical.  That it is true appears as follows.  It can only be false by the final consequent $x$ being false while its antecedent $(x \,-\!\!\!< y) \,-\!\!\!< x$ is true.  If this is true, either its consequent, $x$, is true, when the whole formula would be true, or its antecedent $x \,-\!\!\!< y$ is false.  But in the last case the antecedent of $x \,-\!\!\!< y$, that is $x$, must be true.  (Peirce, CP 3.384).
\end{quote}

Peirce goes on to point out an immediate application of the law:

\begin{quote}
From the formula just given, we at once get:

\[ \{ (x \,-\!\!\!< y) \,-\!\!\!< a \} \,-\!\!\!< x, \]

where the $a$ is used in such a sense that $(x \,-\!\!\!< y) \,-\!\!\!< a$ means that from $(x \,-\!\!\!< y)$ every proposition follows.  With that understanding, the formula states the principle of excluded middle, that from the falsity of the denial of $x$ follows the truth of $x$.  (Peirce, CP 3.384).
\end{quote}

\textbf{Note.}  Peirce uses the ``sign of illation'' ($-\!\!\!<$) for implication.  In one place he explains it as a variant of the sign ($\le$) for ``less than or equal to''; in another place he suggests reading $A \,-\!\!\!< B$ as ``$A$, in every way that it can be, is $B$''.

\section{Graphical proof}

Representing \PMlinkname{propositions}{PropositionalCalculus} as \PMlinkname{logical graphs}{LogicalGraph} under the \PMlinkname{existential interpretation}{LogicalGraphFormalDevelopment}, Peirce's law is expressed by means of the following formal equation:

\begin{center}\begin{tabular}{cc}
\includegraphics[scale=0.8]{PeircesLawFigure1} & (1) \\
\end{tabular}\end{center}

\textbf{Proof.}  Using the axiom set given in the entry for \PMlinkname{logical graphs}{LogicalGraph}, Peirce's law may be proved in the following manner.

\begin{center}\begin{tabular}{cc}
\includegraphics[scale=0.8]{PeircesLawFigure2} & (2) \\
\end{tabular}\end{center}

\section{Equational form}

A stronger form of Peirce's law also holds, in which the final implication is observed to be reversible:

\[ ((p \Rightarrow q) \Rightarrow p) \Leftrightarrow p \]

\subsection{Proof 1}

Given what precedes, it remains to show that:

\[ p \Rightarrow ((p \Rightarrow q) \Rightarrow p) \]

But this is immediate, since $p \Rightarrow (r \Rightarrow p)$ for any proposition $r.$

\subsection{Proof 2}

Representing \PMlinkname{propositions}{PropositionalCalculus} as \PMlinkname{logical graphs}{LogicalGraph} under the \PMlinkname{existential interpretation}{LogicalGraphFormalDevelopment}, the strong form of Peirce's law is expressed by the following equation:

\begin{center}\begin{tabular}{cc}
\includegraphics[scale=0.8]{PeircesLawFigure3} & (3) \\
\end{tabular}\end{center}

Using the axioms and theorems listed in the entries on \PMlinkname{logical graphs}{LogicalGraph}, the equational form of Peirce's law may be proved in the following manner:

\begin{center}\begin{tabular}{cc}
\includegraphics[scale=0.8]{PeircesLawFigure4} & (4) \\
\end{tabular}\end{center}

\section{Bibliography}

\begin{itemize}
\item
Peirce, Charles Sanders (1885), ``On the Algebra of Logic : A Contribution to the Philosophy of Notation", \textit{American Journal of Mathematics} 7 (1885), 180--202.  Reprinted (CP 3.359--403), (CE 5, 162--190).
\item
Peirce, Charles Sanders (1931--1935, 1958), \textit{Collected Papers of Charles Sanders Peirce}, vols. 1--6, Charles Hartshorne and Paul Weiss (eds.), vols. 7--8, Arthur W. Burks (ed.), Harvard University Press, Cambridge, MA.  Cited as (CP volume.paragraph).
\item
Peirce, Charles Sanders (1981--), \textit{Writings of Charles S. Peirce : A Chronological Edition}, Peirce Edition Project (eds.), Indiana University Press, Bloomington and Indianapolis, IN.  Cited as (CE volume, page).
\end{itemize}

%%%%%
%%%%%
\end{document}
