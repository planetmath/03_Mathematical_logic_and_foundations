\documentclass[12pt]{article}
\usepackage{pmmeta}
\pmcanonicalname{IteratedForcingAndComposition}
\pmcreated{2013-03-22 12:54:51}
\pmmodified{2013-03-22 12:54:51}
\pmowner{Henry}{455}
\pmmodifier{Henry}{455}
\pmtitle{iterated forcing and composition}
\pmrecord{6}{33265}
\pmprivacy{1}
\pmauthor{Henry}{455}
\pmtype{Result}
\pmcomment{trigger rebuild}
\pmclassification{msc}{03E35}
\pmclassification{msc}{03E40}

\endmetadata

% this is the default PlanetMath preamble.  as your knowledge
% of TeX increases, you will probably want to edit this, but
% it should be fine as is for beginners.

% almost certainly you want these
\usepackage{amssymb}
\usepackage{amsmath}
\usepackage{amsfonts}

% used for TeXing text within eps files
%\usepackage{psfrag}
% need this for including graphics (\includegraphics)
%\usepackage{graphicx}
% for neatly defining theorems and propositions
%\usepackage{amsthm}
% making logically defined graphics
%%%\usepackage{xypic}

% there are many more packages, add them here as you need them

% define commands here
%\PMlinkescapeword{theory}
\begin{document}
There is a function satisfying forcings are equivalent if one is dense in the other $f:P_{\alpha}*Q_\alpha\rightarrow P_{\alpha+1}$.

\section*{Proof}

Let $f(\langle g,\hat{q}\rangle)=g\cup\{\langle \alpha,\hat{q}\rangle\}$.  This is obviously a member of $P_{\alpha+1}$, since it is a partial function from $\alpha+1$ (and if the domain of $g$ is less than $\alpha$ then so is the domain of $f(\langle g,\hat{q}\rangle)$), if $i<\alpha$ then obviously $f(\langle g,\hat{q}\rangle)$ applied to $i$ satisfies the definition of iterated forcing (since $g$ does), and if $i=\alpha$ then the definition is satisfied since $\hat{q}$ is a name in $P_i$ for a member of $Q_i$.

$f$ is order preserving, since if $\langle g_1,\hat{q}_1\rangle\leq \langle g_2,\hat{q}_2\rangle$, all the appropriate characteristics of a function carry over to the image, and $g_1\upharpoonright\alpha\Vdash_{P_i} \hat{q}_1\leq \hat{q}_2$ (by the definition of $\leq$ in $*$).

If $\langle g_1,\hat{q}_1\rangle$ and $\langle g_2,\hat{q}_2\rangle$ are incomparable then either $g_1$ and $g_2$ are incomparable, in which case whatever prevents them from being compared applies to their images as well, or $\hat{q}_1$ and $\hat{q}_2$ aren't compared appropriately, in which case again this prevents the images from being compared.

Finally, let $g$ be any element of $P_{\alpha+1}$.  Then $g\upharpoonright\alpha\in P_\alpha$.  If $\alpha\notin \operatorname{dom}(g)$ then this is just $g$, and $f(\langle g,\hat{q}\rangle)\leq g$ for any $\hat{q}$.  If $\alpha\in\operatorname{dom}(g)$ then $f(\langle g\upharpoonright\alpha,g(\alpha)\rangle)=g$.  Hence $f[P_\alpha*Q_\alpha]$ is dense in $P_{\alpha+1}$, and so these are equivalent.
%%%%%
%%%%%
\end{document}
