\documentclass[12pt]{article}
\usepackage{pmmeta}
\pmcanonicalname{IntersectionStructure}
\pmcreated{2013-03-22 17:06:28}
\pmmodified{2013-03-22 17:06:28}
\pmowner{CWoo}{3771}
\pmmodifier{CWoo}{3771}
\pmtitle{intersection structure}
\pmrecord{9}{39406}
\pmprivacy{1}
\pmauthor{CWoo}{3771}
\pmtype{Definition}
\pmcomment{trigger rebuild}
\pmclassification{msc}{03G10}
\pmclassification{msc}{06B23}
\pmsynonym{closure system}{IntersectionStructure}
\pmrelated{CriteriaForAPosetToBeACompleteLattice}
\pmdefines{topped intersection structure}
\pmdefines{algebraic intersection structure}
\pmdefines{algebraic closure system}

\endmetadata

\usepackage{amssymb,amscd}
\usepackage{amsmath}
\usepackage{amsfonts}
\usepackage{mathrsfs}

% used for TeXing text within eps files
%\usepackage{psfrag}
% need this for including graphics (\includegraphics)
%\usepackage{graphicx}
% for neatly defining theorems and propositions
\usepackage{amsthm}
% making logically defined graphics
%%\usepackage{xypic}
\usepackage{pst-plot}
\usepackage{psfrag}

% define commands here
\newtheorem{prop}{Proposition}
\newtheorem{thm}{Theorem}
\newtheorem{ex}{Example}
\newcommand{\real}{\mathbb{R}}
\newcommand{\pdiff}[2]{\frac{\partial #1}{\partial #2}}
\newcommand{\mpdiff}[3]{\frac{\partial^#1 #2}{\partial #3^#1}}

\begin{document}
\subsubsection*{Intersection structures}

An \emph{intersection structure} is a set $C$ such that 
\begin{enumerate}
\item $C$ is a subset of the powerset $P(A)$ of some set $A$, and
\item intersection of a non-empty family $\mathcal{F}$ of elements of $C$ is again in $C$.
\end{enumerate}

If order $C$ by set inclusion, then $C$ becomes a poset.

There are numerous examples of intersection structures.  In algebra, the set of all subgroups of a group, the set of all ideals of a ring, and the set of all subspaces of a vector space.  In topology, the set of all closed sets of a topological space is an intersection structure.  Finally, in functional analysis, the set of all convex subsets of a topological vector space is also an intersection structure.

The set of all partial orderings on a set is also an intersection structure.  A final example can be found in domain theory: let $C$ be the set of all partial functions from a non-empty set $X$ to a non-empty set $Y$.  Since each partial function is a subset of $X\times Y$, $C$ is a subset of $P(X\times Y)$.  Let $\mathcal{F}:=\lbrace f_i\mid i\in I\rbrace$ be an arbitrary collection of partial functions in $C$ and $f=\bigcap \mathcal{F}$.  $f$ is clearly a relation between $X$ and $Y$.  Suppose $x$ is in the domain of $f$.  Let $E=\lbrace y\in Y\mid xfy\rbrace$.  Then $xf_i y$ for each $f_i$ where $x$ is in the domain of $f_i$.  Since $f_i$ is a partial function, $y=f_i(x)$, so that $y$ is uniquely determined.  This means that $E$ is a singleton, hence $f$ is a  partial function, so that $\bigcap \mathcal{F}\in C$, meaning that $C$ is an intersection structure.

The main difference between the last two examples and the previous examples is that in the last two examples, $C$ is rarely a complete lattice.  For example, let $\le$ be a partial ordering on a set $P$.  Then its dual $\le^{\partial}$ is also a partial ordering on $P$.  But the join of $\le$ and $\le^{\partial}$ does not exist.  Here is another example: let $X=\lbrace 1\rbrace$ and $Y=\lbrace 2,3\rbrace$.  Then $C=\lbrace \varnothing, (1,2),(1,3)\rbrace$.  $(1,2)$ and $(1,3)$ are the maximal elements of $C$, but the join of these two elements does not exist.

\subsubsection*{Topped intersection strucutres}

If, in condition 2 above, we remove the requirement that $\mathcal{F}$ be non-empty, then we have an intersection structure called a \emph{topped intersection structure}.

The reason for calling them topped is because the top element of such an intersection structure always exists; it is the intersection of the empty family.  In addition, a topped intersection structure is always a complete lattice.  For a proof of this fact, see this \PMlinkname{link}{CriteriaForAPosetToBeACompleteLattice}.

As a result, for example, to show that the subgroups of a group form a complete lattice, it is enough to observe that arbitrary intersection of subgroups is again a subgroup.

\textbf{Remarks}.  
\begin{itemize}
\item
A topped intersection structure is also called a \emph{closure system}.  The reason for calling this is that every topped intersection structure $C\subseteq P(X)$ induces a closure operator $\operatorname{cl}$ on $P(X)$, making $X$ a closure space.  $\operatorname{cl}:P(X)\to P(X)$ given by $$\operatorname{cl}(A)=\bigcap \lbrace B\in C\mid A\subseteq B\rbrace$$ is well-defined.  
\item
Conversely, it is not hard to see that every closure space $(X,\operatorname{cl})$ gives rise to a closure system $C:=\lbrace \operatorname{cl}(A)\mid A\in P(X)\rbrace$.
\item
An intersection structure $C$ is said to be \emph{algebraic} if for every directed set $B\subseteq C$, we have that $\bigcup B\in C$.  All of the examples above, except the set of closed sets in a topological space, are algebraic intersection structures.  A topped intersection structure that is algebraic is called an \emph{algebraic closure system} if,   
\item
Every algebraic closure system is an algebraic lattice.
\end{itemize}

\begin{thebibliography}{6}
\bibitem{dp} B. A. Davey, H. A. Priestley, {\it Introduction to Lattices and Order}, 2nd Edition, Cambridge (2003)
\bibitem{gg} G. Gr\"{a}tzer: {\em Universal Algebra}, 2nd Edition, Springer, New York (1978).
\end{thebibliography}

%%%%%
%%%%%
\end{document}
