\documentclass[12pt]{article}
\usepackage{pmmeta}
\pmcanonicalname{LengthOfAString}
\pmcreated{2013-03-22 12:29:16}
\pmmodified{2013-03-22 12:29:16}
\pmowner{mathcam}{2727}
\pmmodifier{mathcam}{2727}
\pmtitle{length of a string}
\pmrecord{7}{32708}
\pmprivacy{1}
\pmauthor{mathcam}{2727}
\pmtype{Definition}
\pmcomment{trigger rebuild}
\pmclassification{msc}{03C07}
\pmdefines{equality of strings}
\pmdefines{empty string}

\endmetadata

\usepackage{amssymb}
\usepackage{amsmath}
\usepackage{amsfonts}
\begin{document}
Suppose we have a string $w$ on alphabet $\Sigma$. We can then represent
the string as $w = x_1x_2x_3\cdots x_{n-1}x_n$, where for all $x_i$
($1 \leq i \leq n$), $x_i \in \Sigma$ (this means that each $x_i$ must
be a ``letter'' from the alphabet). Then, the length of $w$ is $n$. The
\emph{length of a string} $w$ is represented as $ \Vert w \Vert $.

For example, if our alphabet is $\Sigma = \{ a, b, ca \} $
then the length of the string $w = bcaab $ is $ \Vert w \Vert = 4$, since
the string breaks down as follows: $ x_1 = b$, $x_2 = ca $, $x_3 = a$, $x_4 = b$.
So, our $x_n$ is $x_4$ and therefore $ n = 4$. Although you may think that
$ca$ is two separate symbols, our chosen alphabet in fact classifies it as
a single symbol.

A ``special case'' occurs when $\Vert w \Vert = 0$, i.e. it does not have any symbols
in it. This string is called the \emph{empty string}. Instead of saying
$ w = \ $, we use $\lambda$ to represent the empty string:
$w = \lambda$. This is similar to the practice of using $\beta$
to represent a space, even though a space is really blank.

If your alphabet contains $\lambda$ as a symbol, then
you must use something else to denote the empty string.

Suppose you also have a string $v$ on the same alphabet as $w$. We turn
$w$ into $x_1 \cdots x_n $ just as before, and similarly $v = y_1 \cdots y_m $.
We say $v$ is \emph{equal} to $w$ if and only if both $m = n$, and
for every $i$, $x_i = y_i$.

For example, suppose $ w = bba $ and $v = bab$, both strings on alphabet $\Sigma = \{ a,b \}$.
These strings are not equal because the second symbols do not match.
%%%%%
%%%%%
\end{document}
