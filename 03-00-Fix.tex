\documentclass[12pt]{article}
\usepackage{pmmeta}
\pmcanonicalname{Fix}
\pmcreated{2013-03-22 16:11:19}
\pmmodified{2013-03-22 16:11:19}
\pmowner{Wkbj79}{1863}
\pmmodifier{Wkbj79}{1863}
\pmtitle{fix}
\pmrecord{6}{38277}
\pmprivacy{1}
\pmauthor{Wkbj79}{1863}
\pmtype{Definition}
\pmcomment{trigger rebuild}
\pmclassification{msc}{03-00}
\pmclassification{msc}{03F07}
\pmsynonym{fixed}{Fix}
\pmrelated{Fixed}

\endmetadata

\usepackage{amssymb}
\usepackage{amsmath}
\usepackage{amsfonts}

\usepackage{psfrag}
\usepackage{graphicx}
\usepackage{amsthm}
%%\usepackage{xypic}

\begin{document}
\PMlinkescapeword{objects}
\PMlinkescapeword{object}
\PMlinkescapeword{states}
\PMlinkescapeword{word}

In mathematical statements, mathematical objects such as points and numbers are described as being \emph{fixed}.  A possible meaning for this usage is that the mathematical object in question is not allowed to vary throughout the statement or proof (or, in some cases, a portion thereof).  Although a fixed object typically does not vary, it is almost always arbitrary.  This may seem paradoxical, but it is quite logical:  An object is chosen arbitrarily, then it is never allowed to vary.  See the entry betweenness in rays for an example of this usage.

The usage of the \PMlinkescapetext{words} \emph{fix} and \emph{fixed} may also \PMlinkescapetext{mean} that a mapping sends the mathematical object to itself.  These two usages are technically not the same.  The former usage (described in the previous paragraph) states a property of the mathematical object in question and is always either part of an implication (as in ``If $x \in \mathbb{R}$ is fixed, then...'') or a command made by the author to the reader (as in ``Let $x \in \mathbb{R}$ be fixed.'' and ``Fix $x \in \mathbb{R}$.'').  The latter usage (described in this paragraph) states a property of a mapping and may or may not be part of a conditional statement or a command.  The word ``fixes'' \emph{always} refers to this usage (as in ``Note that $f$ fixes $x$.'').  See the entry \PMlinkname{fix (transformation actions)}{Fixed} for a further explanation of the latter usage.
%%%%%
%%%%%
\end{document}
