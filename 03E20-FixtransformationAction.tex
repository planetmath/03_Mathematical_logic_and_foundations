\documentclass[12pt]{article}
\usepackage{pmmeta}
\pmcanonicalname{FixtransformationAction}
\pmcreated{2013-03-22 12:26:12}
\pmmodified{2013-03-22 12:26:12}
\pmowner{rmilson}{146}
\pmmodifier{rmilson}{146}
\pmtitle{fix (transformation action)}
\pmrecord{15}{32510}
\pmprivacy{1}
\pmauthor{rmilson}{146}
\pmtype{Definition}
\pmcomment{trigger rebuild}
\pmclassification{msc}{03E20}
\pmsynonym{fix}{FixtransformationAction}
\pmsynonym{fixed}{FixtransformationAction}
\pmsynonym{fixes}{FixtransformationAction}
\pmrelated{Invariant}
\pmrelated{Transformation}
\pmrelated{Fix2}
\pmdefines{fixed set}

\usepackage{amsmath}
\usepackage{amsfonts}
\usepackage{amssymb}

\newcommand{\reals}{\mathbb{R}}
\newcommand{\natnums}{\mathbb{N}}
\newcommand{\cnums}{\mathbb{C}}

\newcommand{\lp}{\left(}
\newcommand{\rp}{\right)}
\newcommand{\lb}{\left[}
\newcommand{\rb}{\right]}

\newcommand{\supth}{^{\text{th}}}


\newtheorem{proposition}{Proposition}
\begin{document}
Let $A$ be a set, and   $T:A\rightarrow A$ a transformation of that
set.  We say that $x\in A$ is 
\emph{fixed} by $T$, or that $T$ \emph{fixes} $x$, whenever
$$T(x)=x.$$
The subset of fixed elements is called {\em the fixed set of $T$}, and is frequently denoted as $A^T$.

We say that a subset $B\subset A$ is
fixed by $T$ whenever all elements of $B$ are fixed by $T$,
i.e. $$B\subset A^T.$$   If this is so,  $T$ restricts to the identity
transformation on $B$.

The definition generalizes readily to a family of transformations with
common domain
$$T_i : A\rightarrow A,\quad i\in I$$
In this case we say that a subset $B\subset A$ is fixed, if it is fixed
by all the elements of the family, i.e. whenever
$$B\subset \bigcap_{i\in I} A^{T_i}.$$



%%%%%
%%%%%
\end{document}
