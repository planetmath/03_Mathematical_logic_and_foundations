\documentclass[12pt]{article}
\usepackage{pmmeta}
\pmcanonicalname{DeductionTheoremHoldsForClassicalPropositionalLogic}
\pmcreated{2013-03-22 19:13:42}
\pmmodified{2013-03-22 19:13:42}
\pmowner{CWoo}{3771}
\pmmodifier{CWoo}{3771}
\pmtitle{deduction theorem holds for classical propositional logic}
\pmrecord{23}{42151}
\pmprivacy{1}
\pmauthor{CWoo}{3771}
\pmtype{Theorem}
\pmcomment{trigger rebuild}
\pmclassification{msc}{03F03}
\pmclassification{msc}{03B99}
\pmclassification{msc}{03B22}
\pmrelated{AxiomSystemForPropositionalLogic}
\pmdefines{proof by contradiction}
\pmdefines{proof by contrapositive}

\usepackage{amssymb,amscd}
\usepackage{amsmath}
\usepackage{amsfonts}
\usepackage{mathrsfs}

% used for TeXing text within eps files
%\usepackage{psfrag}
% need this for including graphics (\includegraphics)
%\usepackage{graphicx}
% for neatly defining theorems and propositions
\usepackage{amsthm}
% making logically defined graphics
%%\usepackage{xypic}
\usepackage{pst-plot}

% define commands here
\newcommand*{\abs}[1]{\left\lvert #1\right\rvert}
\newtheorem{prop}{Proposition}
\newtheorem{thm}{Theorem}
\newtheorem{cor}{Corollary}
\newtheorem{ex}{Example}
\newcommand{\real}{\mathbb{R}}
\newcommand{\pdiff}[2]{\frac{\partial #1}{\partial #2}}
\newcommand{\mpdiff}[3]{\frac{\partial^#1 #2}{\partial #3^#1}}

\begin{document}
In this entry, we prove that the deduction theorem holds for classical propositional logic.  For the logic, we use the axiom system found in \PMlinkname{this entry}{AxiomSystemForPropositionalLogic}.  To prove the theorem, we use the theorem schema $A\to A$ (whose deduction can be found \PMlinkname{here}{AxiomSystemForPropositionalLogic}).

\begin{proof}  Suppose $A_1, \ldots, A_n$ is a deduction of $B$ from $\Delta \cup \lbrace A \rbrace$.  We want to find a deduction of $A\to B$ from $\Delta$.  There are two main cases to consider:
\begin{itemize}
\item
If $B$ is an axiom or in $\Delta\cup \lbrace A\rbrace$, then $$B, B\to (A\to B), A\to B$$ is a deduction of $A\to B$ from $\Delta$, where $A\to B$ is obtained by modus ponens applied to $B$ and the axiom $B\to (A\to B)$.  So $\Delta \vdash A\to B$.
\item
If $B$ is obtained from $A_i$ and $A_j$ by modus ponens.  Then $A_j$, say, is $A_i \to B$.  We use induction on the length $n$ of the deduction of $B$.  Note that $n\ge 3$.  If $n=3$, then the first two formulas are $C$ and $C\to B$.
\begin{itemize}
\item 
If $C$ is $A$, then $C\to B$ is either an axiom or in $\Delta$.  So $A\to B$, which is just $C\to B$, is a deduction of $A\to B$ from $\Delta$.
\item 
If $C\to B$ is $A$, then $C$ is either an axiom or in $\Delta$, and
$$\mathcal{E}_0,(A\to A)\to ((A\to C) \to (A\to B)),C\to(A\to C),C,A\to C, A\to B$$
is a deduction of $A\to B$ from $\Delta$, where $\mathcal{E}_0$ is a deduction of $A\to A$, followed by two axiom instances, followed by $C$, followed by results of two applications of modus ponens.
\item 
If both $C$ and $C\to B$ are either axioms are in $\Delta$, then
$$C, C\to B, B, B\to (A\to B), A\to B$$
is a deduction of $A\to B$ from $\Delta$.
\end{itemize}
Next, assume the deduction $\mathcal{E}$ of $B$ has length $n>3$.
A subsequence of $\mathcal{E}$ is a deduction of $A_i \to B$ from $\Delta \cup \lbrace A\rbrace$.  This deduction has length less than $n$, so by induction, $$\Delta \vdash A \to (A_i\to B),$$ and therefore by $(A \to (A_i\to B)) \to ((A\to A_i)\to (A\to B))$, an axiom instance, and modus ponens, $$\Delta \vdash (A\to A_i)\to (A\to B).$$
Likewise, a subsequence of $\mathcal{E}$ is a deduction of $A_i$, so by induction, $\Delta \vdash A \to A_i$.  Therefore, an application of modus ponens gives us $\Delta \vdash A\to B$.
\end{itemize}
In both cases, $\Delta \vdash A\to B$ and we are done.
\end{proof}

We record two corollaries:
\begin{cor} (Proof by Contradiction).  If $\Delta, A \vdash \perp$, then $\Delta \vdash \neg A$. \end{cor}
\begin{proof} From $\Delta, A \vdash \perp$, we have $\Delta \vdash A\to \perp$ by the deduction theorem.  Since $\neg A \leftrightarrow \perp$, the result follows.  \end{proof}

\begin{cor} (Proof by Contrapositive).  If $\Delta, A \vdash \neg B$, then $\Delta, B \vdash \neg A$. \end{cor}
\begin{proof} If $\Delta, A \vdash \neg B$, then $\Delta, A,B \vdash \perp$ by the deduction thereom, and therefore $\Delta, B \vdash \neg A$ by the deduction theorem again.  \end{proof}

\textbf{Remark}  The deduction theorem for the classical propositional logic can be used to prove the deduction theorem for the classical first and second order predicate logic.

\begin{thebibliography}{7}
\bibitem{JR} J. W. Robbin, {\it Mathematical Logic, A First Course}, Dover Publication (2006)
\end{thebibliography}

%%%%%
%%%%%
\end{document}
