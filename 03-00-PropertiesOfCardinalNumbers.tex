\documentclass[12pt]{article}
\usepackage{pmmeta}
\pmcanonicalname{PropertiesOfCardinalNumbers}
\pmcreated{2013-03-22 16:08:29}
\pmmodified{2013-03-22 16:08:29}
\pmowner{gilbert_51126}{14238}
\pmmodifier{gilbert_51126}{14238}
\pmtitle{properties of cardinal numbers}
\pmrecord{7}{38218}
\pmprivacy{1}
\pmauthor{gilbert_51126}{14238}
\pmtype{Theorem}
\pmcomment{trigger rebuild}
\pmclassification{msc}{03-00}

% this is the default PlanetMath preamble.  as your knowledge
% of TeX increases, you will probably want to edit this, but
% it should be fine as is for beginners.

% almost certainly you want these
\usepackage{amssymb}
\usepackage{amsmath}
\usepackage{amsfonts}

% used for TeXing text within eps files
%\usepackage{psfrag}
% need this for including graphics (\includegraphics)
%\usepackage{graphicx}
% for neatly defining theorems and propositions
%\usepackage{amsthm}
% making logically defined graphics
%%%\usepackage{xypic}

% there are many more packages, add them here as you need them

% define commands here

\begin{document}
\emph{Theorem.}
Let $\displaystyle \left(c_\gamma \right)_{\gamma \in \Gamma}$ be an indexed family of cardinal numbers indexed by a nonempty index set $\Gamma$. Also, let $\displaystyle \left({\Gamma}_\delta\right)_{\delta \in \Delta}$ be an arbitrary indexed partition of the index set. Then we have the following properties:

1. \emph{Associative Laws.}
\[   
\sum_{\gamma \in \Gamma}{c_\gamma} = \sum_{\delta \in \Delta} \sum_{\gamma \in {\Gamma}_\delta}{c_\gamma} 
\]
\begin{center}
and 
\end{center}
\[
\prod_{\gamma \in \Gamma}{c_\gamma} = \prod_{\delta \in \Delta} \prod_{\gamma \in {\Gamma}_\delta}{c_\gamma}. 
\]

2. \emph{Commutative Laws.} Let $\displaystyle \pi :\Gamma \to \Gamma$ be a partition. Then
\[
\sum_{\gamma \in \Gamma}{c_\gamma} =  \sum_{\gamma \in {\Gamma}}{c_{\pi(\gamma)}} 
\]
\begin{center}
and 
\end{center}
\[
\prod_{\gamma \in \Gamma}{c_\gamma} =  \prod_{\gamma \in {\Gamma}}{c_{\pi(\gamma)}}. 
\]

3. \emph{Distributive Laws.} Let $a$ be any arbitrary infinite cardinal number. Then
\[
a \left({\sum_{\gamma \in \Gamma}{c_\gamma}}\right) = \sum_{\gamma \in \Gamma}{a c_\gamma}
\]
%%%%%
%%%%%
\end{document}
