\documentclass[12pt]{article}
\usepackage{pmmeta}
\pmcanonicalname{FuzzySubset}
\pmcreated{2013-03-22 16:34:54}
\pmmodified{2013-03-22 16:34:54}
\pmowner{ggerla}{15808}
\pmmodifier{ggerla}{15808}
\pmtitle{fuzzy subset}
\pmrecord{11}{38773}
\pmprivacy{1}
\pmauthor{ggerla}{15808}
\pmtype{Definition}
\pmcomment{trigger rebuild}
\pmclassification{msc}{03E72}
\pmclassification{msc}{03G20}
\pmsynonym{fuzzy set}{FuzzySubset}
\pmsynonym{L-subset}{FuzzySubset}
%\pmkeywords{multi-valued logic}
%\pmkeywords{fuzzy control}
\pmrelated{Logic}
\pmrelated{FuzzyLogic2}

\endmetadata

% this is the default PlanetMath preamble.  as your knowledge
% of TeX increases, you will probably want to edit this, but
% it should be fine as is for beginners.

% almost certainly you want these
\usepackage{amssymb}
\usepackage{amsmath}
\usepackage{amsfonts}

% used for TeXing text within eps files
%\usepackage{psfrag}
% need this for including graphics (\includegraphics)
%\usepackage{graphicx}
% for neatly defining theorems and propositions
%\usepackage{amsthm}
% making logically defined graphics
%%%\usepackage{xypic}

% there are many more packages, add them here as you need them

% define commands here

\begin{document}
Fuzzy set theory is based on the idea that vague notions as ``big'', ``near'', ``hold'' can be modelled by ``fuzzy subsets''.  The idea of a fuzzy subset $T$ of a set $S$ is the following: each element $x\in S$, there is a number $p\in [0,1]$ such that $p_x$ is the ``probability'' that $x$ is in $T$.  

To formally define a fuzzy set, let us first recall a well-known fact about subsets: a subset $T$ of a set $S$ corresponds uniquely to the characteristic function $c_T:S\to \lbrace 0,1\rbrace$, such that $c_T(x)=1$ iff $x\in T$.  So if one were to replace $\lbrace 0,1\rbrace$ with the the closed unit interval $[0,1]$, one obtains a fuzzy subset:
\begin{quote}
A \emph{fuzzy subset} of a set $S$ is a map $s:S\rightarrow [0,1]$ from $S$ into the interval $[0,1]$. 
\end{quote}
More precisely, the interval $[0,1]$ is considered as a complete lattice with an involution $1-x$. 
We call \textit{fuzzy subset of } $S$ any element of the direct power $[0,1]^S$.  Whereas there are $2^{|S|}$ subsets of $S$, there are $\aleph_1^{|S|}$ fuzzy subsets of $S$.

The join and meet operations in the complete lattice $[0,1]^S$ are named \textit{union} and \textit{intersection}, respectively. The operation induced by the involution is called \textit{complement}. This means that if $s$ and $t$ are two fuzzy subsets, then the fuzzy subsets $s\cup t, s\cap t, -s$, are defined by the equations
$$(s\cup t)(x) = \max\{s(x), t(x)\} \,\,\, ; \,\,\,(s\cap t)(x) = \min\{s(x), t(x)\} \,\,\, ; \,\,\,-s(x) = 1-s(x).$$
It is also possible to consider any lattice $L$ instead of $[0,1]$. In such a case we call $L$\textit{-subset of } $S$ any element of the direct power $L^S$ and the union and the intersection are defined by setting
$$(s\cup t)(x) = s(x)\vee t(x) \,\,\, ; \,\,\,(s\cap t)(x) = s(x)\wedge t(x)$$
where $\vee$ and $\wedge$ denote the join and the meet operations in $L$, respectively. In the case an order reversing function $\neg : L \rightarrow L$ is defined in $L$, the \textit{complement} $-s$ of $s$ is defined by setting
$$-s(x) = \neg s(x).$$
Fuzzy set theory is devoted mainly to applications. The main success is perhaps \textit{fuzzy control}.

\begin{thebibliography}{9}
\bibitem{cdm} Cignoli R., D Ottaviano I. M. L. and Mundici D.,{\em Algebraic Foundations of Many-Valued Reasoning}. Kluwer, Dordrecht, (1999).
\bibitem{e} Elkan C., {\em The Paradoxical Success of Fuzzy Logic}. (November 1993). Available from \PMlinkexternal{http://www.cse.ucsd.edu/users/elkan/}{http://www.cse.ucsd.edu/users/elkan/} Elkan's home page.
\bibitem{gg} Gerla G., {\em Fuzzy logic: Mathematical tools for approximate reasoning}, Kluwer Academic Publishers, Dordrecht, (2001).
\bibitem{gj} Goguen J., The logic of inexact concepts,  {\em Synthese}, vol. 19 (1968/69)
\bibitem{gs} Gottwald S., {\em A treatise on many-valued logics}, Research Studies Press, Baldock (2000).
\bibitem{h} Hájek P., {\em Metamathematics of fuzzy logic}. Kluwer (1998).
\bibitem{k} Klir G. , UTE H. St.Clair and Bo Yuan, {\em Fuzzy Set Theory Foundations and Applications}, (1997).
\bibitem{z} Zimmermann H., {\em Fuzzy Set Theory and its Applications} (2001), ISBN 0-7923-7435-5.
\bibitem{z1} Zadeh L.A., Fuzzy Sets, {\em Information and Control}, 8 (1965) 338­-353.
\bibitem{z2} Zadeh L. A., The concept of a linguistic variable and its application to approximate reasoning I, II, III, {\em Information Sciences}, vol. 8, 9(1975), pp. 199-275, 301-357, 43-80.
\end{thebibliography}
%%%%%
%%%%%
\end{document}
