\documentclass[12pt]{article}
\usepackage{pmmeta}
\pmcanonicalname{Delta1Bootstrapping}
\pmcreated{2013-03-22 12:58:25}
\pmmodified{2013-03-22 12:58:25}
\pmowner{Henry}{455}
\pmmodifier{Henry}{455}
\pmtitle{$\Delta_1$ bootstrapping}
\pmrecord{10}{33344}
\pmprivacy{1}
\pmauthor{Henry}{455}
\pmtype{Example}
\pmcomment{trigger rebuild}
\pmclassification{msc}{03B10}

\endmetadata

% this is the default PlanetMath preamble.  as your knowledge
% of TeX increases, you will probably want to edit this, but
% it should be fine as is for beginners.

% almost certainly you want these
\usepackage{amssymb}
\usepackage{amsmath}
\usepackage{amsfonts}

% used for TeXing text within eps files
%\usepackage{psfrag}
% need this for including graphics (\includegraphics)
%\usepackage{graphicx}
% for neatly defining theorems and propositions
%\usepackage{amsthm}
% making logically defined graphics
%%%\usepackage{xypic}

% there are many more packages, add them here as you need them

% define commands here
%\PMlinkescapeword{theory}
\begin{document}
This proves that a number of useful relations and functions are $\Delta_1$ in first order arithmetic, providing a bootstrapping of parts of mathematical practice into any system including the $\Delta_1$ relations (since the $\Delta_1$ relations are exactly the recursive ones, this includes Turing machines).

First, we want to build a tupling relation which will allow a finite sets of numbers to be encoded by a single number.  To do this we first show that $R(a,b)\leftrightarrow a| b$ is $\Delta_1$.  This is true since $a| b\leftrightarrow \exists c\leq b (a\cdot c=b)$, a formula with only bounded quantifiers.

Next note that $P(x)\leftrightarrow$\texttt{x is prime} is $\Delta_1$ since $P(x)\leftrightarrow \neg\exists y<x (\neg y=1\wedge y| x)$. Also $A_P(x,y)\leftrightarrow P(x)\wedge P(y)\wedge \forall z<y(x<z\rightarrow \neg P(x))$.

These two can be used to define (the graph of) a primality function, $p(a)=a+1$\texttt{-th prime}.  Let $p(a,b)=\exists c\leq b^{a^2} (\neg[2|c]\wedge [\forall q<b\forall r\leq b (A_P(q,r)\rightarrow \forall j<c [q^j| c\leftrightarrow r^{j+1}| c])]\wedge [b^a| c]\wedge \neg[b^{a+1}|c])$.

This rather awkward looking formula is worth examining, since it illustrates a principle which will be used repeatedly.  $c$ is intended to be a function of the form $2^0\cdot 3^1\cdot 5^2\cdots$ and so on.  If it includes $b^a$ but not $b^{a+1}$ then we know that $b$ must be the $a+1$-th prime.  The definition is so complicated because we cannot just say, as we'd like to, $p(a+1)$ is the smallest prime greater than $p(a)$ (since we don't allow recursive definitions).  Instead we embed the series of values this recursion would take into a single number ($c$) and guarantee that the recursive relationship holds for at least $a$ terms; then we just check if the $a$-th value is $b$.

Finally, we can define our tupling relation.  Technically, since a given relation must have a fixed arity, we define for each $n$ a function $\langle x_0,\ldots,x_n\rangle=\sum_{\leq n} p_i^{x_i+1}$.  Then define $(x)_i$ to be the $i$-th element of $x$ when $x$ is interpreted as a tuple, so $\langle (x)_0,\ldots,(x)_n\rangle=x$.  Note that the tupling relation, even taken collectively, is not total.  For instance $5$ is not a tuple (although it is sometimes convenient to view it as a tuple with ``empty spaces'': $\langle \_,\_,5\rangle$).  In situations like this, and also when attempting to extract entries beyond the length, $(x)_i=0$ (for instance, $(5)_0=0$).  On the other hand there is a $0$-ary tupling relation, $\langle\rangle=1$.

Thanks to our definition of $p$, we have $\langle x_0,\ldots,x_n\rangle=x\leftrightarrow x=p(0)^{x_0+1}\cdot\cdots\cdot p(n)^{x_n+1}$.  This is clearly $\Delta_1$.  (Note that we don't use the $\sum$ as above, since we don't have that, but since we have a different tupling function for each $n$ this isn't a problem.)

For the reverse, $(x)_i=y\leftrightarrow \left([p(i)^{y+1}|x] \wedge \neg[p(i)^{y+2}|x]\right)\vee \left([y=0]\wedge \neg[p(i)|x]\right)$.

Also, define a length function by $\operatorname{len}(x)=y\leftrightarrow \neg [p(y+1)| x]\wedge\forall z\leq y[p(z)|x]$ and a membership relation by $\operatorname{in}(x,n)\leftrightarrow \exists i<\operatorname{len}(x) [(x)_i=n]$.

Armed with this, we can show that all primitive recursive functions are $\Delta_1$.  To see this, note that $x=0$, the zero function, is trivially recursive, as are $x=Sy$ and $p_{n,m}(x_1,\ldots,x_n)=x_m$.

The $\Delta_1$ functions are closed under composition, since if $\phi(\vec{x})$ and $\psi(\vec{x})$ both have no unbounded quantifiers, $\phi(\psi(\vec{x}))$ obviously doesn't either.

Finally, suppose we have functions $f(\vec{x})$ and $g(\vec{x},m,n)$ in $\Delta_1$.  Then define the primitive recursion $h(\vec{x},y)$ by first defining:
$$\bar{h}(\vec{x},y)=z\leftrightarrow ln(z)=y\wedge \forall i<y [(z)_{i+1}=g(\vec{x},i,(z)_i)]\wedge [ln(z)=0\vee (z)_0=f(\vec{x})$$

and then $h(\vec{x},y)=(\bar{h}(\vec{x},y))_y$.

$\Delta_1$ is also closed under minimization: if $R(\vec{x},y)$ is a $\Delta_1$ relation then $\mu y. f(\vec{x},y)$ is a function giving the least $y$ satisfying $R(\vec{x},y)$.  To see this, note that $\mu y. f(\vec{x},y)=z\leftrightarrow f(\vec{x},z)\wedge \forall m<z\neg f(\vec{x},m)$.

Finally, using primitive recursion it is possible to concatenate sequences.  First, to concatenate a single number, if $s=\langle x_0,\ldots x_n\rangle$ then $s*_1 y=t\cdot p(\operatorname{len}(s)+1)^{y+1}$.  Then we can define the concatenation of $s$ with $t=\langle y_0,\ldots,y_m\rangle$ by defining $f(s,t)=s$ and $g(s,t,j,i)=j*_1 (t)_i$, and by primitive recursion, there is a function $h(s,t,i)$ whose value is the first $j$ elements of $t$ appended to $s$.  Then $s*t=h(s,t,\operatorname{len}(t)$.

We can also define $*_u$, which concatenates only elements of $t$ not appearing in $s$.  This just requires defining the graph of $g$ to be $g(s,t,j,i,x)\leftrightarrow [\operatorname{in}(s,(t)_i)\wedge x=j] \vee [\neg\operatorname{in}(s,(t)_i)\wedge x=j*_1 (t)_i]$
%%%%%
%%%%%
\end{document}
