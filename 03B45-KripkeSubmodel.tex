\documentclass[12pt]{article}
\usepackage{pmmeta}
\pmcanonicalname{KripkeSubmodel}
\pmcreated{2013-03-22 19:34:50}
\pmmodified{2013-03-22 19:34:50}
\pmowner{CWoo}{3771}
\pmmodifier{CWoo}{3771}
\pmtitle{Kripke submodel}
\pmrecord{11}{42568}
\pmprivacy{1}
\pmauthor{CWoo}{3771}
\pmtype{Definition}
\pmcomment{trigger rebuild}
\pmclassification{msc}{03B45}
\pmdefines{generated submodel}

\endmetadata

\usepackage{amssymb,amscd}
\usepackage{amsmath}
\usepackage{amsfonts}
\usepackage{mathrsfs}
\usepackage{proof}
\usepackage{bussproofs}

% used for TeXing text within eps files
%\usepackage{psfrag}
% need this for including graphics (\includegraphics)
%\usepackage{graphicx}
% for neatly defining theorems and propositions
\usepackage{amsthm}
% making logically defined graphics
%%\usepackage{xypic}
\usepackage{pst-plot}
\usepackage{multicol}
\usepackage{enumerate}
\usepackage{tabls}

% define commands here
\newcommand*{\abs}[1]{\left\lvert #1\right\rvert}
\newtheorem{prop}{Proposition}
\newtheorem{thm}{Theorem}
\newtheorem{lem}{Lemma}
\newtheorem{cor}{Corollary}
\newtheorem{ex}{Example}

\begin{document}
Let $\mathcal{F}=(W,R)$ be a Kripke frame.  A \emph{subframe} of $\mathcal{F}$ is a pair $\mathcal{F}'=(W',R')$ such that $W'$ is a subset of $W$ and $R'=R\cap (W'\times W')$.  A \emph{submodel} of a Kripke model $M=(W,R,V)$ of a modal propositional logic PL$_M$ is a triple $(W',R',V')$ where $(W',R')$ is a subframe of $(W,R)$, and $V'(p)=V(p)\cap W'$ for each propositional variable $p$ in PL$_M$.

The most common submodel of a Kripke model $M=(W,R,V)$ is constructed by taking $W_w:= \lbrace u\mid w R^* u \rbrace$, where $R^*$ is the reflexive transitive closure of $R$, for any $w\in W$.  The submodel $M_w:=(W_w,R_w,V_w)$ is called the submodel of $M$ \emph{generated} by the world $w$, and $\mathcal{F}_w$ the subframe of $\mathcal{F}$ generated by $w$.  A submodel of $M$ is called a \emph{generated submodel} if it is generated by some world $w$ in $M$.

\begin{prop} For any wff $A$ and any $u\in W_w$, $M \models_u A$ iff $M_w \models_u A$. \end{prop}
\begin{proof}  We do induction on the number $n$ of connectives in $A$.  

If $n=0$, then $A$ is either $\perp$ or a propositional variable.  Clearly, $M \not \models_u \perp$ iff $M_w \not \models_u \perp$, or $M \models_u \perp$ iff $M_w \models_u \perp$.  If $A$ is some propositional variable $p$, then $M \models_u p$ iff $u\in V(p)$ iff $u\in V(p)$ iff $u\in V(p)$ and $u\in W_w$ (by assumption) iff $u\in V(p)\cap M_w$ iff $u\in V_w(p)$ iff $M_w \models_u p$.

Next, suppose $A$ is $B\to C$.  Then $M \models_u A$ iff $u \in V(B)^c \cup V(C)$ iff $u \in (W-V(B)) \cup V(C)$ and $u\in W_w$ (by assumption) iff $u\in W_w - V(B)$ or $u\in V_w(C)$ iff $u\in W_w - V_w(B)$ or $u\in V_w(C)$ iff $M_w \models_u A$.

Finally, suppose $A$ is $\square B$.  First let $M \models_u A$.  To show $M_w \models_u A$, pick any $v \in W_w$ such that $u R_w v$.  Then $u R v$, so that $M \models_v B$, or $v\in V(B)$.  Since $v\in W_w$, $v\in V_w(B)$, or $M_w \models_v B$, and thus $M_w \models_u A$.  Conversely, let $M_w \models_u A$.  To show $M \models_u A$, pick any $v \in W$ such that $u R v$.  Since $w R^* u$, $w R^* v$ so that $v\in W_w$.  Furthermore $(u,v) \in R \cap (W_w\times W_w)=R_w$.  So $M_w \models_u B$, or $v \in V_w(B)\subseteq V(B)$, which means $M \models_u A$.
\end{proof}

\begin{cor} $M\models A$ iff $M_w \models A$ for all $w\in W$. \end{cor}
\begin{proof}
If $M\models A$, then $M \models_u A$ for all $u\in W$, so that $M_w \models_u A$ for all $u\in W_w$ and all $w\in W$, or $M_w \models A$ for all $w\in W$.  Conversely, if $M_w \models A$ for all $w\in W$, then in particular $M_w \models_w A$ (since $R^*$ is reflexive) for all $w\in W$, or $M \models_w A$ for all $w\in W$, or $M \models A$.
\end{proof}

\begin{cor} $\mathcal{F} \models A$ iff $\mathcal{F}_w \models A$ for all $w\in W$. \end{cor}
\begin{proof}
If $\mathcal{F} \models A$, then $M \models A$ for all $M$ based on $\mathcal{F}$, or $M_w \models A$, where $M_w$ is based on $\mathcal{F}_w$, for all $w\in W$ by the last corollary.  Since any model based on $\mathcal{F}_w$ is of the form $M_w$, $\mathcal{F}_w \models A$.  Conversely, suppose $\mathcal{F}_w \models A$ for all $w\in W$.  Let $M$ be any model based on $\mathcal{F}$.  Then $M_w$ is based on $\mathcal{F}_w$, and therefore $M_w \models A$.  Since $w$ is arbitrary, $M\models A$ by the last corollary, so $\mathcal{F} \models A$.
\end{proof}

%%%%%
%%%%%
\end{document}
