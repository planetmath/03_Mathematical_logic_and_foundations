\documentclass[12pt]{article}
\usepackage{pmmeta}
\pmcanonicalname{Logic}
\pmcreated{2013-03-22 13:00:09}
\pmmodified{2013-03-22 13:00:09}
\pmowner{Henry}{455}
\pmmodifier{Henry}{455}
\pmtitle{logic}
\pmrecord{9}{33380}
\pmprivacy{1}
\pmauthor{Henry}{455}
\pmtype{Definition}
\pmcomment{trigger rebuild}
\pmclassification{msc}{03B15}
\pmclassification{msc}{03B10}
\pmrelated{FuzzySubset}
\pmdefines{syntax}
\pmdefines{semantics}
\pmdefines{type}
\pmdefines{sort}

% this is the default PlanetMath preamble.  as your knowledge
% of TeX increases, you will probably want to edit this, but
% it should be fine as is for beginners.

% almost certainly you want these
\usepackage{amssymb}
\usepackage{amsmath}
\usepackage{amsfonts}

% used for TeXing text within eps files
%\usepackage{psfrag}
% need this for including graphics (\includegraphics)
%\usepackage{graphicx}
% for neatly defining theorems and propositions
%\usepackage{amsthm}
% making logically defined graphics
%%%\usepackage{xypic}

% there are many more packages, add them here as you need them

% define commands here
%\PMlinkescapeword{theory}
\begin{document}
Generally, by logic, people mean first order logic, a formal set of rules for building mathematical statements out of symbols like $\neg$ (negation) and $\rightarrow$ (implication) along with quantifiers like $\forall$ (for every) and $\exists$ (there exists).

More generally, a \emph{logic} is any set of rules for forming sentences (the logic's \emph{syntax}) together with rules for assigning truth values to them (the logic's \emph{semantics}).  Normally it includes a (possibly empty) set of \emph{types} $T$ (also called \emph{sorts}), which represent the different kinds of objects that the theory discusses (typical examples might be sets, numbers, or sets of numbers).  In addition it specifies particular quantifiers, connectives, and variables.  Particular theories in the logic can then add relations and functions to fully specify a logical language.
%%%%%
%%%%%
\end{document}
