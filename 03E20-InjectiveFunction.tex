\documentclass[12pt]{article}
\usepackage{pmmeta}
\pmcanonicalname{InjectiveFunction}
\pmcreated{2013-03-22 11:51:38}
\pmmodified{2013-03-22 11:51:38}
\pmowner{drini}{3}
\pmmodifier{drini}{3}
\pmtitle{injective function}
\pmrecord{16}{30429}
\pmprivacy{1}
\pmauthor{drini}{3}
\pmtype{Definition}
\pmcomment{trigger rebuild}
\pmclassification{msc}{03E20}
\pmclassification{msc}{03E99}
\pmsynonym{one-to-one}{InjectiveFunction}
\pmsynonym{injection}{InjectiveFunction}
\pmsynonym{embedding}{InjectiveFunction}
\pmsynonym{injective}{InjectiveFunction}
\pmrelated{Bijection}
\pmrelated{Function}
\pmrelated{Surjective}

\usepackage{amssymb}
\usepackage{amsmath}
\usepackage{amsfonts}
\usepackage{graphicx}
%%%%\usepackage{xypic}
\begin{document}
We say that a function $f\colon A\to B$ is \emph{injective} or \emph{one-to-one} if $f(x)=f(y)$ implies $x=y$, or equivalently, whenever $x\neq y$, then $f(x)\neq f(y)$.

\subsubsection*{Properties}
\begin{enumerate}
\item Suppose $A,B,C$ are sets and $f\colon A\to B$, $g\colon B\to C$
are injective functions. Then the composition $g\circ f$ is an injection. 
\item Suppose $f\colon A\to B$ is an injection, and $C\subseteq A$. Then
the restriction $f|_C\colon C\to B$ is an injection.
\end{enumerate}

For a list of other \PMlinkescapetext{properties} of 
injective functions, see \cite{wiki}. 

\begin{thebibliography}{9}
\bibitem{wiki} Wikipedia, article on \PMlinkexternal{Injective function}{http://en.wikipedia.org/wiki/Injective_function}.
\end{thebibliography}
%%%%%
%%%%%
%%%%%
%%%%%
\end{document}
