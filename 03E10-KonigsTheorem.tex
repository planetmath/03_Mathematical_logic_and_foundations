\documentclass[12pt]{article}
\usepackage{pmmeta}
\pmcanonicalname{KonigsTheorem}
\pmcreated{2013-03-22 14:10:21}
\pmmodified{2013-03-22 14:10:21}
\pmowner{yark}{2760}
\pmmodifier{yark}{2760}
\pmtitle{K\"onig's theorem}
\pmrecord{15}{35598}
\pmprivacy{1}
\pmauthor{yark}{2760}
\pmtype{Theorem}
\pmcomment{trigger rebuild}
\pmclassification{msc}{03E10}
\pmsynonym{Koenig's theorem}{KonigsTheorem}
\pmsynonym{Konig's theorem}{KonigsTheorem}
\pmsynonym{K\"onig-Zermelo theorem}{KonigsTheorem}
\pmsynonym{Koenig-Zermelo theorem}{KonigsTheorem}
\pmsynonym{Konig-Zermelo theorem}{KonigsTheorem}
%\pmkeywords{inequality}
%\pmkeywords{cardinal}
\pmrelated{CantorsTheorem}

\usepackage{amssymb}
\usepackage{amsmath}
\usepackage{amsthm}

\newtheorem{theorem}{Theorem}
\begin{document}
\PMlinkescapeword{argument}
\PMlinkescapeword{case}
\PMlinkescapeword{cases}
\PMlinkescapeword{completes}
\PMlinkescapeword{equivalent}
\PMlinkescapeword{image}
\PMlinkescapeword{index}
\PMlinkescapeword{onto}
\PMlinkescapeword{similar}

\emph{K\"onig's Theorem} is a theorem of cardinal arithmetic.

\begin{theorem}
Let $\kappa_i$ and $\lambda_i$ be cardinals, for all $i$ in some index set $I$.
If $\kappa_i<\lambda_i$ for all $i\in I$, then
$$\sum_{i\in I}\kappa_i<\,\prod_{i\in I}\lambda_i.$$
\end{theorem}

The theorem can also be stated for arbitrary sets, as follows.
\begin{theorem}
Let $A_i$ and $B_i$ be sets, for all $i$ in some index set $I$.
If $|A_i|<|B_i|$ for all $i\in I$, then
$$\left|\,\bigcup_{i\in I}A_i\right|<\left|\,\prod_{i\in I}B_i\right|.$$
\end{theorem}
\begin{proof}
Let $\varphi\colon\bigcup_{i\in I}A_i\to\prod_{i\in I}B_i$ be a function.
For each $i\in I$ we have $|\varphi(A_i)|\le|A_i|<|B_i|$,
so there is some $x_i\in B_i$ 
that is not equal to $(\varphi(a))(i)$  for any $a\in A_i$.
Define $f\colon I\to\bigcup_{i\in I}B_i$ 
by $f(i)=x_i$ for all $i\in I$.
For any $i\in I$ and any $a\in A_i$,
we have $f(i)\ne(\varphi(a))(i)$, so $f\ne\varphi(a)$.
Therefore $f$ is not in the image of $\varphi$.
This shows that there is 
no surjection from $\bigcup_{i\in I}A_i$ onto $\prod_{i\in I}B_i$.
As $\prod_{i\in I}B_i$ is nonempty,
this also means that
there is no injection from $\prod_{i\in I}B_i$ into $\bigcup_{i\in I}A_i$.
This completes the proof of Theorem 2. 
Theorem 1 follows as an immediate corollary.
\end{proof}

Note that the above proof is a diagonal argument, 
similar to the proof of Cantor's Theorem.
In fact, Cantor's Theorem can be considered as a special case of K\"onig's Theorem, taking $\kappa_i=1$ and $\lambda_i=2$ for all $i$.

Also note that Theorem 2 is equivalent (in ZF) to the Axiom of Choice, as it implies that \PMlinkname{products}{GeneralizedCartesianProduct} of nonempty sets are nonempty. (Theorem 1, on the other hand, is not meaningful without the Axiom of Choice.)
%%%%%
%%%%%
\end{document}
