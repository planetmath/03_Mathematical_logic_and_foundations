\documentclass[12pt]{article}
\usepackage{pmmeta}
\pmcanonicalname{A13DependentPairTypesSigmatypes}
\pmcreated{2013-11-09 4:49:56}
\pmmodified{2013-11-09 4:49:56}
\pmowner{PMBookProject}{1000683}
\pmmodifier{PMBookProject}{1000683}
\pmtitle{A.1.3 Dependent pair types ($\Sigma$-types)}
\pmrecord{3}{87888}
\pmprivacy{1}
\pmauthor{PMBookProject}{1000683}
\pmtype{Feature}
\pmclassification{msc}{03B15}

\endmetadata

\usepackage{xspace}
\usepackage{amssyb}
\usepackage{amsmath}
\usepackage{amsfonts}
\usepackage{amsthm}
\makeatletter
\newcommand{\defeq}{\vcentcolon\equiv}  
\def\@dprd#1{\prod_{(#1)}\,}
\def\@dprd@noparens#1{\prod_{#1}\,}
\def\@dsm#1{\sum_{(#1)}\,}
\def\@dsm@noparens#1{\sum_{#1}\,}
\def\@eatprd\prd{\prd@parens}
\def\@eatsm\sm{\sm@parens}
\def\lam#1{{\lambda}\@lamarg#1:\@endlamarg\@ifnextchar\bgroup{.\,\lam}{.\,}}
\def\@lamarg#1:#2\@endlamarg{\if\relax\detokenize{#2}\relax #1\else\@lamvar{\@lameatcolon#2},#1\@endlamvar\fi}
\def\@lameatcolon#1:{#1}
\def\@lamvar#1,#2\@endlamvar{(#2\,{:}\,#1)}
\newcommand{\narrowbreak}{}
\def\prd#1{\@ifnextchar\bgroup{\prd@parens{#1}}{\@ifnextchar\sm{\prd@parens{#1}\@eatsm}{\prd@noparens{#1}}}}
\def\prd@noparens#1{\mathchoice{\@dprd@noparens{#1}}{\@tprd{#1}}{\@tprd{#1}}{\@tprd{#1}}}
\def\prd@parens#1{\@ifnextchar\bgroup  {\mathchoice{\@dprd{#1}}{\@tprd{#1}}{\@tprd{#1}}{\@tprd{#1}}\prd@parens}  {\@ifnextchar\sm    {\mathchoice{\@dprd{#1}}{\@tprd{#1}}{\@tprd{#1}}{\@tprd{#1}}\@eatsm}    {\mathchoice{\@dprd{#1}}{\@tprd{#1}}{\@tprd{#1}}{\@tprd{#1}}}}}
\def\sm#1{\@ifnextchar\bgroup{\sm@parens{#1}}{\@ifnextchar\prd{\sm@parens{#1}\@eatprd}{\sm@noparens{#1}}}}
\def\sm@noparens#1{\mathchoice{\@dsm@noparens{#1}}{\@tsm{#1}}{\@tsm{#1}}{\@tsm{#1}}}
\def\sm@parens#1{\@ifnextchar\bgroup  {\mathchoice{\@dsm{#1}}{\@tsm{#1}}{\@tsm{#1}}{\@tsm{#1}}\sm@parens}  {\@ifnextchar\prd    {\mathchoice{\@dsm{#1}}{\@tsm{#1}}{\@tsm{#1}}{\@tsm{#1}}\@eatprd}    {\mathchoice{\@dsm{#1}}{\@tsm{#1}}{\@tsm{#1}}{\@tsm{#1}}}}}
\def\tprd#1{\@tprd{#1}\@ifnextchar\bgroup{\tprd}{}}
\def\@tprd#1{\mathchoice{{\textstyle\prod_{(#1)}}}{\prod_{(#1)}}{\prod_{(#1)}}{\prod_{(#1)}}}
\def\@tsm#1{\mathchoice{{\textstyle\sum_{(#1)}}}{\sum_{(#1)}}{\sum_{(#1)}}{\sum_{(#1)}}}
\newcommand{\tup}[2]{(#1,#2)}
\newcommand{\UU}{\ensuremath{\mathcal{U}}\xspace}
\newcommand{\vcentcolon}{:\!\!}
\newenvironment{narrowmultline*}{\csname equation*\endcsname}{\csname endequation*\endcsname}
\makeatother

\begin{document}
We introduce primitive constants $c_\Sigma$ and $c_{\mathsf{pair}}$. An
expression of the form $c_\Sigma(A,\lam{a} B)$ is written as $\sm{a:A}B$,
and an expression of the form $c_{\mathsf{pair}}(a,b)$ is written as $\tup
a b$. We write $A\times B$ instead of $\sm{x:A} B$ if $x$ is not free in $B$.

Judgments concerning such expressions are introduced by the following
rules:
%
\begin{itemize}
\item if $A:\UU_n$ and $B: A \rightarrow \UU_n$, then $\sm{x:A}B(x) : \UU_n$
\item if, in addition, $a:A$ and $b:B(a)$, then $\tup a b:\sm{x:A}B(x)$
\end{itemize}
%
If we have $A$ and $B$ as above, $C : \sm{x:A}B(x) \rightarrow \UU_m$, and
\[
  d:\tprd{x:A}{y:B(x)} C(\tup x y)
\]
we can introduce a defined constant 
\[
  f:\tprd{p:\sm{x:A}B(x)} C(p)
\]
with the defining equation
\[
  f(\tup x y)\defeq d(x,y).
\]
%
Note that $C$, $d$, $x$, and $y$ may contain extra implicit parameters $x_1,\ldots,x_n$ if they were obtained in some non-empty context; therefore, the fully explicit recursion schema is
%
\begin{narrowmultline*}
 f(x_1,\dots,x_n,\tup{x(x_1,\dots,x_n)}{y(x_1,\dots,x_n)}) \defeq
 \narrowbreak
 d(x_1,\dots,x_n,\tup{x(x_1,\dots,x_n)}{y(x_1,\dots,x_n)}).
\end{narrowmultline*}

\end{document}
