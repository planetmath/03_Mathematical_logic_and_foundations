\documentclass[12pt]{article}
\usepackage{pmmeta}
\pmcanonicalname{Range}
\pmcreated{2013-03-22 12:00:51}
\pmmodified{2013-03-22 12:00:51}
\pmowner{akrowne}{2}
\pmmodifier{akrowne}{2}
\pmtitle{range}
\pmrecord{6}{30967}
\pmprivacy{1}
\pmauthor{akrowne}{2}
\pmtype{Definition}
\pmcomment{trigger rebuild}
\pmclassification{msc}{03E20}

\usepackage{amssymb}
\usepackage{amsmath}
\usepackage{amsfonts}
\usepackage{graphicx}
%%%\usepackage{xypic}
\begin{document}
Let $R$ be a binary relation.  Then the set of all $y$ such that $x R y$ for some $x$ is called the \emph{range} of $R$.  That is, the range of $R$ is the set of all second coordinates in the ordered pairs of $R$.

In \PMlinkescapetext{terms} of functions, this means that the range of a function is the full set of values it can take on (the outputs), given the full set of parameters (the inputs).  Note that the range is a subset of the codomain.
%%%%%
%%%%%
%%%%%
\end{document}
