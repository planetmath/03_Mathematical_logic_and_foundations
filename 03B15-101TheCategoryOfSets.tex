\documentclass[12pt]{article}
\usepackage{pmmeta}
\pmcanonicalname{101TheCategoryOfSets}
\pmcreated{2013-11-06 17:03:38}
\pmmodified{2013-11-06 17:03:38}
\pmowner{PMBookProject}{1000683}
\pmmodifier{PMBookProject}{1000683}
\pmtitle{10.1 The category of sets}
\pmrecord{1}{}
\pmprivacy{1}
\pmauthor{PMBookProject}{1000683}
\pmtype{Feature}
\pmclassification{msc}{03B15}

\endmetadata

\usepackage{xspace}
\usepackage{amssyb}
\usepackage{amsmath}
\usepackage{amsfonts}
\usepackage{amsthm}
\newcommand{\uset}{\ensuremath{\mathcal{S}et}\xspace}
\newcommand{\UU}{\ensuremath{\mathcal{U}}\xspace}
\let\autoref\cref
\begin{document}
Recall that in \autoref{cha:category-theory} we defined the category \uset to consist of all $0$-types (in some universe \UU) and maps between them, and observed that it is a category (not just a precategory).
We consider successively the levels of structure which \uset possesses.

\end{document}
