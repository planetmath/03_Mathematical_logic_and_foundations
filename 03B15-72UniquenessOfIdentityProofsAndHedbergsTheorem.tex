\documentclass[12pt]{article}
\usepackage{pmmeta}
\pmcanonicalname{72UniquenessOfIdentityProofsAndHedbergsTheorem}
\pmcreated{2013-11-18 22:13:00}
\pmmodified{2013-11-18 22:13:00}
\pmowner{PMBookProject}{1000683}
\pmmodifier{rspuzio}{6075}
\pmtitle{7.2 Uniqueness of identity proofs and Hedberg’s theorem}
\pmrecord{10}{87699}
\pmprivacy{1}
\pmauthor{PMBookProject}{6075}
\pmtype{Feature}
\pmclassification{msc}{03B15}

\usepackage{xspace}
\usepackage{amssyb}
\usepackage{amsmath}
\usepackage{amsfonts}
\usepackage{amsthm}
\makeatletter
\newcommand{\brck}[1]{\trunc{}{#1}}
\newcommand{\ct}{  \mathchoice{\mathbin{\raisebox{0.5ex}{$\displaystyle\centerdot$}}}             {\mathbin{\raisebox{0.5ex}{$\centerdot$}}}             {\mathbin{\raisebox{0.25ex}{$\scriptstyle\,\centerdot\,$}}}             {\mathbin{\raisebox{0.1ex}{$\scriptscriptstyle\,\centerdot\,$}}}}
\newcommand{\defeq}{\vcentcolon\equiv}  
\newcommand{\define}[1]{\textbf{#1}}
\def\@dprd#1{\prod_{(#1)}\,}
\def\@dprd@noparens#1{\prod_{#1}\,}
\def\@dsm#1{\sum_{(#1)}\,}
\def\@dsm@noparens#1{\sum_{#1}\,}
\def\@eatprd\prd{\prd@parens}
\def\@eatsm\sm{\sm@parens}
\newcommand{\emptyt}{\ensuremath{\mathbf{0}}\xspace}
\newcommand{\eqv}[2]{\ensuremath{#1 \simeq #2}\xspace}
\newcommand{\eqvsym}{\simeq}    
\newcommand{\id}[3][]{\ensuremath{#2 =_{#1} #3}\xspace}
\newcommand{\indexdef}[1]{\index{#1|defstyle}}   
\newcommand{\inl}{\ensuremath\inlsym\xspace}
\newcommand{\inlsym}{{\mathsf{inl}}}
\newcommand{\inr}{\ensuremath\inrsym\xspace}
\newcommand{\inrsym}{{\mathsf{inr}}}
\newcommand{\istype}[1]{\mathsf{is}\mbox{-}{#1}\mbox{-}\mathsf{type}}
\newcommand{\jdeq}{\equiv}      
\def\lam#1{{\lambda}\@lamarg#1:\@endlamarg\@ifnextchar\bgroup{.\,\lam}{.\,}}
\def\@lamarg#1:#2\@endlamarg{\if\relax\detokenize{#2}\relax #1\else\@lamvar{\@lameatcolon#2},#1\@endlamvar\fi}
\def\@lameatcolon#1:{#1}
\def\@lamvar#1,#2\@endlamvar{(#2\,{:}\,#1)}
\newcommand{\LEM}[1]{\ensuremath{\mathsf{LEM}_{#1}}\xspace}
\newcommand{\map}[2]{\ensuremath{{#1}\mathopen{}\left({#2}\right)\mathclose{}}\xspace}
\newcommand{\mapdepfunc}[1]{\ensuremath{\mathsf{apd}_{#1}}\xspace} 
\newcommand{\mapfunc}[1]{\ensuremath{\mathsf{ap}_{#1}}\xspace} 
\newcommand{\N}{\ensuremath{\mathbb{N}}\xspace}
\newcommand{\narrowequation}[1]{$#1$}
\newcommand{\nminusone}{\ensuremath{(n-1)}}
\newcommand{\nplusone}{\ensuremath{(n+1)}}
\newcommand{\opp}[1]{\mathord{{#1}^{-1}}}
\newcommand{\pairr}[1]{{\mathopen{}(#1)\mathclose{}}}
\newcommand{\Parens}[1]{\Bigl(#1\Bigr)}
\newcommand{\pointed}[1]{\ensuremath{#1_\bullet}}
\def\prd#1{\@ifnextchar\bgroup{\prd@parens{#1}}{\@ifnextchar\sm{\prd@parens{#1}\@eatsm}{\prd@noparens{#1}}}}
\def\prd@noparens#1{\mathchoice{\@dprd@noparens{#1}}{\@tprd{#1}}{\@tprd{#1}}{\@tprd{#1}}}
\def\prd@parens#1{\@ifnextchar\bgroup  {\mathchoice{\@dprd{#1}}{\@tprd{#1}}{\@tprd{#1}}{\@tprd{#1}}\prd@parens}  {\@ifnextchar\sm    {\mathchoice{\@dprd{#1}}{\@tprd{#1}}{\@tprd{#1}}{\@tprd{#1}}\@eatsm}    {\mathchoice{\@dprd{#1}}{\@tprd{#1}}{\@tprd{#1}}{\@tprd{#1}}}}}
\newcommand{\rec}[1]{\mathsf{rec}_{#1}}
\newcommand{\refl}[1]{\ensuremath{\mathsf{refl}_{#1}}\xspace}
\def\sm#1{\@ifnextchar\bgroup{\sm@parens{#1}}{\@ifnextchar\prd{\sm@parens{#1}\@eatprd}{\sm@noparens{#1}}}}
\def\sm@noparens#1{\mathchoice{\@dsm@noparens{#1}}{\@tsm{#1}}{\@tsm{#1}}{\@tsm{#1}}}
\def\sm@parens#1{\@ifnextchar\bgroup  {\mathchoice{\@dsm{#1}}{\@tsm{#1}}{\@tsm{#1}}{\@tsm{#1}}\sm@parens}  {\@ifnextchar\prd    {\mathchoice{\@dsm{#1}}{\@tsm{#1}}{\@tsm{#1}}{\@tsm{#1}}\@eatprd}    {\mathchoice{\@dsm{#1}}{\@tsm{#1}}{\@tsm{#1}}{\@tsm{#1}}}}}
\newcommand{\suc}{\mathsf{succ}}
\def\@tprd#1{\mathchoice{{\textstyle\prod_{(#1)}}}{\prod_{(#1)}}{\prod_{(#1)}}{\prod_{(#1)}}}
\newcommand{\trans}[2]{\ensuremath{{#1}_{*}\mathopen{}\left({#2}\right)\mathclose{}}\xspace}
\newcommand{\trunc}[2]{\mathopen{}\left\Vert #2\right\Vert_{#1}\mathclose{}}
\def\@tsm#1{\mathchoice{{\textstyle\sum_{(#1)}}}{\sum_{(#1)}}{\sum_{(#1)}}{\sum_{(#1)}}}
\newcommand{\UU}{\ensuremath{\mathcal{U}}\xspace}
\newcommand{\vcentcolon}{:\!\!}
\newcounter{mathcount}
\setcounter{mathcount}{1}
\newtheorem{precor}{Corollary}
\newenvironment{cor}{\begin{precor}}{\end{precor}\addtocounter{mathcount}{1}}
\renewcommand{\theprecor}{7.2.\arabic{mathcount}}
\newtheorem{prelem}{Lemma}
\newenvironment{lem}{\begin{prelem}}{\end{prelem}\addtocounter{mathcount}{1}}
\renewcommand{\theprelem}{7.2.\arabic{mathcount}}
\newtheorem{prethm}{Theorem}
\newenvironment{thm}{\begin{prethm}}{\end{prethm}\addtocounter{mathcount}{1}}
\renewcommand{\theprethm}{7.2.\arabic{mathcount}}
\let\ap\map
\let\apdfunc\mapdepfunc
\let\apfunc\mapfunc
\let\autoref\cref
\let\nat\N
\let\type\UU
\makeatother

\begin{document}
\index{set|(}%

In \PMlinkname{\S 3.1}{31setsandntypes} we defined a type $X$ to be a \emph{set} if for all $x, y : X$ and $p, q : x =_X y$ we have $p = q$.
In conventional type theory, this property goes by the name of \define{uniqueness of identity proofs (UIP)}.
\indexdef{uniqueness!of identity proofs}%
We have seen also that it is equivalent to being a $0$-type in the sense of the previous section.
Here is another equivalent characterization, involving Streicher's ``Axiom K'' \cite{Streicher93}:

\begin{thm}\label{thm:h-set-uip-K}
 A type $X$ is a set if and only if it satisfies \define{Axiom K}:
 \indexdef{axiom!Streicher's Axiom K}%
 for all $x : X$ and $p : (x =_A x)$ we have $p = \refl{x}$.
\end{thm}

\begin{proof}
  Clearly Axiom K is a special case of UIP.
  Conversely, if $X$ satisfies Axiom K, let $x, y : X$ and $p, q : (\id{x}{y})$; we want to show $p=q$.
  But induction on $q$ reduces this goal precisely to Axiom K.
\end{proof}

We stress that \emph{we} are not assuming UIP or the K principle as axioms!
They are simply properties which a particular type may or may not satisfy (which are equivalent to being a set).
Recall from \PMlinkname{Example 3.1.9}{31setsandntypes#Thmpreeg6} that \emph{not} all types are sets.

The following theorem is another useful way to show that types are sets.

\begin{thm}\label{thm:h-set-refrel-in-paths-sets}
  \index{relation!reflexive}%
  Suppose $R$ is a reflexive\index{reflexivity!of a relation} mere relation on a type $X$ implying identity.
  Then $X$ is a set, and $R(x,y)$ is equivalent to $\id[X]{x}{y}$ for all $x,y:X$.
\end{thm}

\begin{proof}
  Let $\rho : \prd{x:X} R(x,x)$ witness reflexivity of $R$, and let \narrowequation{f : \prd{x,y:X} R(x,y) \to (\id[X]{x}{y})} be a witness that $R$
implies identity.
  Note first that the two statements in the theorem are equivalent.
  For on one hand, if $X$ is a set, then $\id[X]xy$ is a mere proposition, and since it is logically equivalent to the mere proposition
$R(x,y)$ by hypothesis, it must also be equivalent to it.
  On the other hand, if $\id[X]xy$ is equivalent to $R(x,y)$, then like the latter it is a mere proposition for all $x,y:X$, and hence $X$
is a set.

  We give two proofs of this theorem.
  The first shows directly that $X$ is a set; the second shows directly that $R(x,y)\eqvsym (x=y)$.

  \emph{First proof:} we show that $X$ is a set.
  The idea is the same as that of \PMlinkname{Lemma 3.3.4}{33merepropositions#Thmprelem3}: the function $f$ must be continuous in its arguments $x$ and $y$.
  However, it is slightly more notationally complicated because we have to deal with the additional argument of type $R(x,y)$.

  Firstly, for any $x:X$ and $p:\id[X]xx$, consider $\apdfunc{f(x)}(p)$.
  This is a dependent path from $f(x,x)$ to itself.
  Since $f(x,x)$ is still a function $R(x,x) \to (\id[X]xy)$, by \PMlinkname{Lemma 2.9.6}{29pitypesandthefunctionextensionalityaxiom#Thmprelem1} this yields for any $r:R(x,x)$ a path
  \[\trans{p}{f(x,x,r)} = f(x,x,\trans{p}r).
  \]
  On the left-hand side, we have transport in an identity type, which is concatenation.
  And on the right-hand side, we have $\trans{p}r = r$, since both lie in the mere proposition $R(x,x)$.
  Thus, substituting $r\defeq \rho(x)$, we obtain
  \[ f(x,x,\rho(x)) \ct p = f(x,x,\rho(x)). \]
  By cancellation, $p=\refl{x}$.
  So $X$ satisfies Axiom K, and hence is a set.

  \emph{Second proof:} we show that each $f(x,y) : R(x,y) \to \id[X]{x}{y}$ is an equivalence.
  By \PMlinkname{Theorem 4.7.7}{47closurepropertiesofequivalences#Thmprethm4}, it suffices to show that $f$ induces an equivalence of total spaces:
  \begin{equation*}
    \eqv{\Parens{\sm{y:X}R(x,y)}}{\Parens{\sm{y:X}\id[X]{x}{y}}}.
  \end{equation*}
  By \PMlinkname{Lemma 3.11.8}{311contractibility#Thmprelem5}, the type on the right is contractible, so it
  suffices to show that the type on the left is contractible. As the center of
  contraction we take the pair $\pairr{x,\rho(x)}$.  It remains to show, for
  every ${y:X}$ and every ${H:R(x,y)}$ that
  \begin{equation*}
    \id{\pairr{x,\rho(x)}}{\pairr{y,H}}.
  \end{equation*}
  But since $R(x,y)$ is a mere proposition, by \PMlinkname{Theorem 2.7.2}{27sigmatypes#Thmprethm1} it suffices to show that
  $\id[X]{x}{y}$, which we get from $f(H)$.
\end{proof}

\begin{cor}\label{notnotstable-equality-to-set}
  If a type $X$ has the property that $\neg\neg(x=y)\to(x=y)$ for any $x,y:X$, then $X$ is a set.
\end{cor}

Another convenient way to show that a type is a set is the following.
Recall from \PMlinkname{\S 3.4}{34classicalvsintuitionisticlogic} that a type $X$ is said to have \emph{decidable equality}
\index{decidable!equality|(}%
if for all $x, y : X$ we have
\[(x =_X y) + \neg (x =_X y).\]
\index{continuity of functions in type theory@``continuity'' of functions in type theory}%
\index{functoriality of functions in type theory@``functoriality'' of functions in type theory}%
This is a very strong condition: it says that a path $x=y$ can be chosen, when it exists, continuously (or computably, or functorially) in $x$ and $y$.
This turns out to imply that $X$ is a set, by way of \PMlinkname{Theorem 7.2.2}{72uniquenessofidentityproofsandhedbergstheorem#Thmprethm2} and the following lemma.

\begin{lem}
For any type $A$ we have $(A+\neg A)\to(\neg\neg A\to A)$.
\end{lem}

\begin{proof}
Suppose $x:A+\neg A$. We have two cases to consider.
If $x$ is $\inl(a)$ for some $a:A$, then we have the constant function $\neg\neg A
\to A$ which maps everything to $a$. If $x$ is $\inr(f)$ for some $t:\neg A$,
we have $g(t):\emptyt$ for every $g:\neg\neg A$. Hence we may use
\emph{ex falso quodlibet}, that is $\rec{\emptyt}$, to obtain an element of $A$ for any $g:\neg\neg A$.
\end{proof}

\index{anger}
\begin{thm}[Hedberg]\label{thm:hedberg}
  \index{Hedberg's theorem}%
  \index{theorem!Hedberg's}%
  If $X$ has decidable equality, then $X$ is a set.
\end{thm}

\begin{proof}
If $X$ has decidable equality, it follows that $\neg\neg(x=y)\to(x=y)$ for any
$x,y:X$. Therefore, Hedberg's theorem follows from 
\PMlinkname{Corollary 7.2.3}{72UniquenessOfIdentityProofsAndHedbergsTheorem#Thmprecor1}.
\end{proof}

There is, of course, a strong connection between this theorem and \PMlinkname{Corollary 3.2.7}{32propositionsastypes#Thmprecor1}.
The statement \LEM{\infty} that is denied by \PMlinkname{Corollary 3.2.7}{32propositionsastypes#Thmprecor1} clearly implies that every type has decidable equality, and hence is a set, which we know is not the case.
\index{excluded middle}%
Note that the consistent axiom \LEM{} from \PMlinkname{\S 3.4}{34classicalvsintuitionisticlogic} implies only that every type has \emph{merely decidable equality}, i.e.\ that for any $A$ we have
\indexdef{equality!merely decidable}%
\indexdef{merely!decidable equality}%
\[ \prd{a,b:A} (\brck{a=b} + \neg\brck{a=b}). \]

\index{decidable!equality|)}%

As an example application of \PMlinkname{Theorem 7.2.5}{72uniquenessofidentityproofsandhedbergstheorem#Thmprethm3}, recall that in \PMlinkname{Example 3.1.4}{31setsandntypes#Thmpreeg3} we observed that $\nat$ is a set, using our characterization of its equality types in
\PMlinkname{\S 2.13}{213naturalnumbers}.
A more traditional proof of this theorem uses only~\PMlinkname{(2.13.2)}{213naturalnumbers#S0.E1} and~\PMlinkname{(2.13.2)}{213naturalnumbers#S0.E2}, rather than the full
characterization of \PMlinkname{Theorem 2.13.1}{213naturalnumbers#Thmprethm1}, with \PMlinkname{Theorem 7.2.5}{72uniquenessofidentityproofsandhedbergstheorem#Thmprethm3} to fill in the blanks.

\begin{thm}\label{prop:nat-is-set}
 The type $\nat$ of natural numbers has decidable equality, and hence is a set.
\end{thm}

\begin{proof}
  Let $x, y : \nat$ be given; we proceed by induction on $x$ and case analysis on $y$ to prove $(x=y)+\neg(x=y)$.
  If $x \jdeq 0$ and $y \jdeq 0$, we take $\inl(\refl{}(0))$.
  If $x \jdeq 0$ and $y \jdeq \suc(n)$, then by~\PMlinkname{(2.13.2)}{213naturalnumbers#S0.E1} we get $\neg (0 = \suc (n))$.

  For the inductive step, let $x \jdeq \suc (n)$.
  If $y \jdeq 0$, we use~\PMlinkname{(2.13.2)}{213naturalnumbers#S0.E1} again.
  Finally, if $y \jdeq \suc (m)$, the inductive hypothesis gives $(m = n)+\neg(m = n)$.
  In the first case, if $p:m=n$, then $\ap \suc p:\suc(m)=\suc(n)$.
  And in the second case,~\PMlinkname{(2.13.3)}{213naturalnumbers#S0.E2} yields $\neg(\suc(m)=\suc(n))$.
\end{proof}

\index{set|)}%

\index{axiom!Streicher's Axiom K!generalization to n-types@generalization to $n$-types}%
Although Hedberg's theorem appears rather special to sets ($0$-types), ``Axiom K'' generalizes naturally to $n$-types.
Note that the ordinary Axiom K (as a property of a type $X$) states that for all $x:X$, the loop space\index{loop space} $\Omega(X,x)$ (see \PMlinkname{Definition 2.1.8}{21typesarehighergroupoids#Thmpredefn2}) is contractible.
Since $\Omega(X,x)$ is always inhabited (by $\refl{x}$), this is equivalent to its being a mere proposition (a $(-1)$-type).
Since $0 = (-1)+1$, this suggests the following generalization.

\begin{thm}\label{thm:hlevel-loops}
  For any $n\geq -1$, a type $X$ is an $\nplusone$-type if and only if for all $x : X$, the type $\Omega(X, x)$ is an $n$-type.
\end{thm}

Before proving this, we prove an auxiliary lemma:

\begin{lem}\label{lem:hlevel-if-inhab-hlevel}
  Given $n \geq -1$ and $X : \type$.
  If, given any inhabitant of $X$ it follows that $X$ is an $n$-type, then $X$ is an $n$-type.
\end{lem}
\begin{proof}
  Let $f : X \to \istype{n}(X)$ be the given map.
  We need to show that for any $x, x' : X$, the type $\id{x}{x'}$ is an $\nminusone$-type.
  But then $f(x)$ shows that $X$ is an $n$-type, hence all its path spaces are $\nminusone$-types.
\end{proof}

\begin{proof}[Proof of \PMlinkname{Theorem 7.2.7}{72uniquenessofidentityproofsandhedbergstheorem#Thmprethm5}]
  The ``only if'' direction is obvious, since $\Omega(X,x)\defeq (\id[X]xx)$.
  Conversely, in order to show that $X$ is an $\nplusone$-type, we need to show that for any $x, x' : X$, the type $\id{x}{x'}$ is an
$n$-type.
  Following \PMlinkname{Lemma 7.2.8}{72uniquenessofidentityproofsandhedbergstheorem#Thmprelem2} it suffices to give a map
  \[ (\id{x}{x'}) \to \istype{n}(\id{x}{x'}). \]
  By path induction, it suffices to do this when $x\jdeq x'$, in which case it follows from the assumption that $\Omega(X, x)$ is an
$n$-type.
\end{proof}

\index{whiskering}
By induction and some slightly clever whiskering, we can obtain a generalization of the K property to $n>0$.

\begin{thm}\label{thm:ntype-nloop}
  \index{loop space!iterated}%
  For every $n\ge -1$, a type $A$ is an $n$-type if and only if $\Omega^{n+1}(A,a)$ is contractible for all $a:A$.
\end{thm}
\begin{proof}
  Recalling that $\Omega^0(A,a) = (A,a)$, the case $n=-1$ is \PMlinkexternal{Exercise 3.5}{http://planetmath.org/node/87838}.
  The case $n=0$ is \PMlinkname{Theorem 7.2.1}{72uniquenessofidentityproofsandhedbergstheorem#Thmprethm1}.
  Now we use induction; suppose the statement holds for $n:\N$.
  By \PMlinkname{Theorem 7.2.7}{72uniquenessofidentityproofsandhedbergstheorem#Thmprethm5}, $A$ is an $(n+1)$-type iff $\Omega(A,a)$ is an $n$-type for all $a:A$.
  By the inductive hypothesis, the latter is equivalent to saying that $\Omega^{n+1}(\Omega(A,a),p)$ is contractible for all $p:\Omega(A,a)$.

  Since $\Omega^{n+2}(A,a) \defeq \Omega^{n+1}(\Omega(A,a),\refl{a})$, and $\Omega^{n+1} = \Omega^n \circ \Omega$, it will suffice to show that $\Omega(\Omega(A,a),p)$ is equal to $\Omega(\Omega(A,a),\refl{a})$, in the type $\pointed\type$ of pointed types.
  For this, it suffices to give an equivalence
  \[ g : \Omega(\Omega(A,a),p) \eqvsym \Omega(\Omega(A,a),\refl{a}) \]
  which carries the basepoint $\refl{p}$ to the basepoint $\refl{\refl{a}}$.
  For $q:p=p$, define $g(q):\refl{a} = \refl{a}$ to be the following composite:
  \[ \refl{a} = p\ct \opp p \overset{q}{=} p\ct\opp p = \refl{a}, \]
  where the path labeled ``$q$'' is actually $\apfunc{\lam{r} r\ct\opp p} (q)$.
  Then $g$ is an equivalence because it is a composite of equivalences
  \[ (p=p) \xrightarrow{\apfunc{\lam{r} r\ct\opp p}} (p\ct \opp p = p\ct \opp p) \xrightarrow{i\ct - \ct \opp i} (\refl{a} = \refl{a}). \]
  using \PMlinkname{Example 2.4.8}{24homotopiesandequivalences#Thmpreeg2},\PMlinkname{Theorem 2.11.1}{211identitytype#Thmprethm1}, where $i:\refl{a} = p\ct \opp p$ is the canonical equality.
  And it is evident that $g(\refl{p}) = \refl{\refl{a}}$.
\end{proof}

\begin{thebibliography}{99}

\bibitem{Streicher93} {Thomas Streicher}. {Investigations into intensional type theory}, 1993. {Habilitationsschrift,Ludwig-Maximilians-Universit\"at M\"unchen}.

\end{thebibliography}

\end{document}
