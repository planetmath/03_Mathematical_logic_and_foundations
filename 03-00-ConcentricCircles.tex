\documentclass[12pt]{article}
\usepackage{pmmeta}
\pmcanonicalname{ConcentricCircles}
\pmcreated{2013-03-22 12:55:25}
\pmmodified{2013-03-22 12:55:25}
\pmowner{drini}{3}
\pmmodifier{drini}{3}
\pmtitle{concentric circles}
\pmrecord{5}{33277}
\pmprivacy{1}
\pmauthor{drini}{3}
\pmtype{Definition}
\pmcomment{trigger rebuild}
\pmclassification{msc}{03-00}
\pmdefines{annulus}

% this is the default PlanetMath preamble.  as your knowledge
% of TeX increases, you will probably want to edit this, but
% it should be fine as is for beginners.

% almost certainly you want these
\usepackage{amssymb}
\usepackage{amsmath}
\usepackage{amsfonts}
\usepackage{pstricks}

% used for TeXing text within eps files
%\usepackage{psfrag}
% need this for including graphics (\includegraphics)
%\usepackage{graphicx}
% for neatly defining theorems and propositions
%\usepackage{amsthm}
% making logically defined graphics
%%%\usepackage{xypic} 

% there are many more packages, add them here as you need them

% define commands here
\begin{document}
A collection of circles is said to be {\em concentric} if they have the same center. The region formed between two concentric circles is therefore an annulus.

In the following picture, the gray region is the annulus delimited by the two black concentric circles. 

\begin{center}
\begin{pspicture}(-2,-2)(2,2)
\psset{linewidth=0.5mm}
\pscircle[fillstyle=solid,fillcolor=lightgray](0,0){2}
\pscircle[fillstyle=solid,fillcolor=white](0,0){0.7}
\psdots(0,0)
\end{pspicture}
\end{center}
%%%%%
%%%%%
\end{document}
