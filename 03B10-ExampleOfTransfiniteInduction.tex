\documentclass[12pt]{article}
\usepackage{pmmeta}
\pmcanonicalname{ExampleOfTransfiniteInduction}
\pmcreated{2013-03-22 17:51:12}
\pmmodified{2013-03-22 17:51:12}
\pmowner{CWoo}{3771}
\pmmodifier{CWoo}{3771}
\pmtitle{example of transfinite induction}
\pmrecord{6}{40328}
\pmprivacy{1}
\pmauthor{CWoo}{3771}
\pmtype{Example}
\pmcomment{trigger rebuild}
\pmclassification{msc}{03B10}

\usepackage{amssymb,amscd}
\usepackage{amsmath}
\usepackage{amsfonts}
\usepackage{mathrsfs}

% used for TeXing text within eps files
%\usepackage{psfrag}
% need this for including graphics (\includegraphics)
%\usepackage{graphicx}
% for neatly defining theorems and propositions
\usepackage{amsthm}
% making logically defined graphics
%%\usepackage{xypic}
\usepackage{pst-plot}

% define commands here
\newcommand*{\abs}[1]{\left\lvert #1\right\rvert}
\newtheorem{prop}{Proposition}
\newtheorem{thm}{Theorem}
\newtheorem{ex}{Example}
\newcommand{\real}{\mathbb{R}}
\newcommand{\pdiff}[2]{\frac{\partial #1}{\partial #2}}
\newcommand{\mpdiff}[3]{\frac{\partial^#1 #2}{\partial #3^#1}}
\begin{document}
Suppose we are interested in showing the property $\Phi(\alpha)$ holds for all ordinals $\alpha$ using transfinite induction.  The proof basically involves three steps:
\begin{enumerate}
\item (first ordinal) show that $\Phi(0)$ holds;
\item (successor ordinal) if $\Phi(\beta)$ holds, then $\Phi(S\beta)$ holds;
\item (limit ordinal) if $\Phi(\gamma)$ holds for all $\gamma<\beta$ and $\beta=\sup\lbrace \gamma\mid \gamma<\beta\rbrace$, then $\Phi(\beta)$ holds.
\end{enumerate}

Below is an example of a proof using transfinite induction.

\begin{prop} $0+\alpha=\alpha$ for any ordinal $\alpha$.  \end{prop}

\begin{proof}
Let $\Phi(\alpha)$ be the property: $0+\alpha=\alpha$.  We follow the three steps outlined above.
\begin{enumerate}
\item Since $0+0=0$ by definition, $\Phi(0)$ holds.
\item Suppose $0+\beta=\beta$. By definition $0+S\beta=S(0+\beta)$, which is equal to $S\beta$ by the induction hypothesis, so $\Phi(S\beta)$ holds.
\item Suppose $\beta=\sup\lbrace\gamma\mid \gamma<\beta\rbrace$ and $0+\gamma=\gamma$ for all $\gamma<\beta$.  Then $$0+\beta = 0+\sup\lbrace\gamma\mid \gamma<\beta\rbrace = \sup \lbrace 0+\gamma\mid \gamma<\beta\rbrace.$$  The second equality follows from definition.  Furthermore, the last expression above is equal to $\sup \lbrace \gamma\mid \gamma<\beta\rbrace =\beta$ by the induction hypothesis.  So $\Phi(\beta)$ holds.
\end{enumerate}
Therefore $\Phi(\alpha)$ holds for every ordinal $\alpha$, which is the statement of the theorem, completing the proof.
\end{proof}
%%%%%
%%%%%
\end{document}
