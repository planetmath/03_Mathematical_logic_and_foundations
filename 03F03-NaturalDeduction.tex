\documentclass[12pt]{article}
\usepackage{pmmeta}
\pmcanonicalname{NaturalDeduction}
\pmcreated{2013-03-22 13:05:14}
\pmmodified{2013-03-22 13:05:14}
\pmowner{Henry}{455}
\pmmodifier{Henry}{455}
\pmtitle{natural deduction}
\pmrecord{5}{33503}
\pmprivacy{1}
\pmauthor{Henry}{455}
\pmtype{Definition}
\pmcomment{trigger rebuild}
\pmclassification{msc}{03F03}
\pmrelated{GentzenSystem}

% this is the default PlanetMath preamble.  as your knowledge
% of TeX increases, you will probably want to edit this, but
% it should be fine as is for beginners.

% almost certainly you want these
\usepackage{amssymb}
\usepackage{amsmath}
\usepackage{amsfonts}

% used for TeXing text within eps files
%\usepackage{psfrag}
% need this for including graphics (\includegraphics)
%\usepackage{graphicx}
% for neatly defining theorems and propositions
\usepackage{amsthm}
% making logically defined graphics
%%%\usepackage{xypic}

% there are many more packages, add them here as you need them

% define commands here
%\PMlinkescapeword{theory}
\begin{document}
Natural deduction refers to related proof systems for several different kinds of logic, intended to be similar to the way people actually reason. Unlike many other proof systems, it has many rules and few axioms.  Sequents in natural deduction have only one formula on the right side.

Typically the rules consist of one pair for each connective, one of which allows the introduction of that symbol and the other its elimination.

To give one example, the proof rules $\rightarrow I$ and $\rightarrow E$ are:

$$\frac{\Gamma,\alpha\Rightarrow\beta}{\Gamma\Rightarrow\alpha\rightarrow\beta}
(\rightarrow I)$$

and $$\frac{\begin{array}{cc}\Gamma\Rightarrow\alpha\rightarrow\beta &\Sigma\Rightarrow\alpha\end{array}}{[\Gamma,\Sigma]\Rightarrow\beta}
(\rightarrow E)$$
%%%%%
%%%%%
\end{document}
