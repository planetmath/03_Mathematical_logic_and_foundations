\documentclass[12pt]{article}
\usepackage{pmmeta}
\pmcanonicalname{BibliographyOfManyvaluedLogicsAndApplications}
\pmcreated{2013-03-22 18:19:11}
\pmmodified{2013-03-22 18:19:11}
\pmowner{bci1}{20947}
\pmmodifier{bci1}{20947}
\pmtitle{bibliography of many-valued logics and applications}
\pmrecord{20}{40949}
\pmprivacy{1}
\pmauthor{bci1}{20947}
\pmtype{Bibliography}
\pmcomment{trigger rebuild}
\pmclassification{msc}{03G30}
\pmclassification{msc}{03G12}
\pmclassification{msc}{03G10}
\pmclassification{msc}{03G20}
\pmclassification{msc}{03-00}
\pmsynonym{many-valued logic}{BibliographyOfManyvaluedLogicsAndApplications}
\pmsynonym{nonstandard logics}{BibliographyOfManyvaluedLogicsAndApplications}
\pmsynonym{N-valued logic}{BibliographyOfManyvaluedLogicsAndApplications}
%\pmkeywords{bibliography of many-valued logics and their applications in Quantum theories}
%\pmkeywords{quantum logic}
%\pmkeywords{categorical ontology}
%\pmkeywords{foundations of mathematics and mathematical biology}
\pmrelated{TopicEntryOnTheAlgebraicFoundationsOfMathematics}
\pmrelated{FormalLogicsAndMetaMathematics}

% this is the default PlanetMath preamble. 

% almost certainly you want these
\usepackage{amssymb}
\usepackage{amsmath}
\usepackage{amsfonts}

% used for TeXing text within eps files
%\usepackage{psfrag}
% need this for including graphics (\includegraphics)
%\usepackage{graphicx}
% for neatly defining theorems and propositions
%\usepackage{amsthm}
% making logically defined graphics
%%%\usepackage{xypic}


% define commands here
\usepackage{amsmath, amssymb, amsfonts, amsthm, amscd, latexsym,color,enumerate}
%%\usepackage{xypic}
\usepackage[mathscr]{eucal}
\usepackage{setspace}
\doublespacing
  \xyoption{curve}
\setlength{\textwidth}{6.5in}
%\setlength{\textwidth}{16cm}
\setlength{\textheight}{9.0in}
%\setlength{\textheight}{24cm}

\hoffset=-.75in        

\voffset=-.4in


\theoremstyle{plain}
\newtheorem{lemma}{Lemma}[section]
\newtheorem{proposition}{Proposition}[section]
\newtheorem{theorem}{Theorem}[section]
\newtheorem{corollary}{Corollary}[section]

\theoremstyle{definition}
\newtheorem{definition}{Definition}[section]
\newtheorem{example}{Example}[section]
%\theoremstyle{remark}
\newtheorem{remark}{Remark}[section]
\newtheorem*{notation}{Notation}
\newtheorem*{claim}{Claim}
\newtheorem{exe}{Example}[section]

\renewcommand{\thefootnote}{\ensuremath{\fnsymbol{footnote%%@
}}} \numberwithin{equation}{section}

\newcommand{\Ad}{{\rm Ad}}
\newcommand{\Aut}{{\rm Aut}}
\newcommand{\Cl}{{\rm Cl}}
\newcommand{\Co}{{\rm Co}}
\newcommand{\DES}{{\rm DES}}
\newcommand{\Diff}{{\rm Diff}}
\newcommand{\Dom}{{\rm Dom}}
\newcommand{\Hol}{{\rm Hol}}
\newcommand{\Mon}{{\rm Mon}}
\newcommand{\Hom}{{\rm Hom}}
\newcommand{\Ker}{{\rm Ker}}
\newcommand{\Ind}{{\rm Ind}}
\newcommand{\IM}{{\rm Im}}
\newcommand{\Is}{{\rm Is}}
\newcommand{\ID}{{\rm id}}
\newcommand{\GL}{{\rm GL}}
\newcommand{\Iso}{{\rm Iso}}
\newcommand{\Sem}{{\rm Sem}}
\newcommand{\St}{{\rm St}}
\newcommand{\Sym}{{\rm Sym}}
\newcommand{\SU}{{\rm SU}}
\newcommand{\Tor}{{\rm Tor}}
\newcommand{\U}{{\rm U}}

\newcommand{\A}{\mathcal A}
\newcommand{\Ce}{\mathcal C}
\newcommand{\D}{\mathcal D}
\newcommand{\E}{\mathcal E}
\newcommand{\F}{\mathcal F}
\newcommand{\G}{\mathcal G}
\newcommand{\Q}{\mathcal Q}
\newcommand{\R}{\mathcal R}
\newcommand{\cS}{\mathcal S}
\newcommand{\cU}{\mathcal U}
\newcommand{\W}{\mathcal W}

\newcommand{\bA}{\mathbb{A}}
\newcommand{\bB}{\mathbb{B}}
\newcommand{\bC}{\mathbb{C}}
\newcommand{\bD}{\mathbb{D}}
\newcommand{\bE}{\mathbb{E}}
\newcommand{\bF}{\mathbb{F}}
\newcommand{\bG}{\mathbb{G}}
\newcommand{\bK}{\mathbb{K}}
\newcommand{\bM}{\mathbb{M}}
\newcommand{\bN}{\mathbb{N}}
\newcommand{\bO}{\mathbb{O}}
\newcommand{\bP}{\mathbb{P}}
\newcommand{\bR}{\mathbb{R}}
\newcommand{\bV}{\mathbb{V}}
\newcommand{\bZ}{\mathbb{Z}}

\newcommand{\bfE}{\mathbf{E}}
\newcommand{\bfX}{\mathbf{X}}
\newcommand{\bfY}{\mathbf{Y}}
\newcommand{\bfZ}{\mathbf{Z}}

\renewcommand{\O}{\Omega}
\renewcommand{\o}{\omega}
\newcommand{\vp}{\varphi}
\newcommand{\vep}{\varepsilon}

\newcommand{\diag}{{\rm diag}}
\newcommand{\grp}{{\mathbb G}}
\newcommand{\dgrp}{{\mathbb D}}
\newcommand{\desp}{{\mathbb D^{\rm{es}}}}
\newcommand{\Geod}{{\rm Geod}}
\newcommand{\geod}{{\rm geod}}
\newcommand{\hgr}{{\mathbb H}}
\newcommand{\mgr}{{\mathbb M}}
\newcommand{\ob}{{\rm Ob}}
\newcommand{\obg}{{\rm Ob(\mathbb G)}}
\newcommand{\obgp}{{\rm Ob(\mathbb G')}}
\newcommand{\obh}{{\rm Ob(\mathbb H)}}
\newcommand{\Osmooth}{{\Omega^{\infty}(X,*)}}
\newcommand{\ghomotop}{{\rho_2^{\square}}}
\newcommand{\gcalp}{{\mathbb G(\mathcal P)}}

\newcommand{\rf}{{R_{\mathcal F}}}
\newcommand{\glob}{{\rm glob}}
\newcommand{\loc}{{\rm loc}}
\newcommand{\TOP}{{\rm TOP}}

\newcommand{\wti}{\widetilde}
\newcommand{\what}{\widehat}

\renewcommand{\a}{\alpha}
\newcommand{\be}{\beta}
\newcommand{\ga}{\gamma}
\newcommand{\Ga}{\Gamma}
\newcommand{\de}{\delta}
\newcommand{\del}{\partial}
\newcommand{\ka}{\kappa}
\newcommand{\si}{\sigma}
\newcommand{\ta}{\tau}
\def \C{\mathsf{C}}
\newcommand{\med}{\medbreak}
\newcommand{\medn}{\medbreak \noindent}
\newcommand{\bign}{\bigbreak \noindent}

%%George's macros%%


\newcommand{\B}{{\mathcal B}}

\newcommand{\MV}{{\mathcal MV}}

\newcommand{\LM}{{\mathcal LM}_n}

\newcommand{\CLM}{{\mathcal CLM}_n}

\newcommand{\Post}{{\mathcal P}ost_n}

\newcommand{\Ra}{\Rightarrow}

\newcommand{\GMn}{$NMV_n$}

\newcommand{\LMn}{$LM_n$}

\newcommand{\NMVn}{NMV_n}

\newtheorem{defi}{Definition}

\newtheorem{prop}{Proposition}

\newtheorem{rem}{Remark}

\newtheorem{coro}{Corollary}

\newtheorem{lema}{Lemma}

\newcommand{\lmn}{LM_{n}}

\newcommand{\lmt}{LM_{\Th}}

\newcommand{\ov}{\overline}

\newcommand{\sm}{\smile}

\newcommand{\lmrn}{LMR_{n}}

\renewcommand{\t}{\times}

\newcommand{\eu}{\equiv_{1}}

\newcommand{\lu}{\leq_{1}}

\newcommand{\ez}{\equiv_{0}}

\newcommand{\lz}{\leq_{0}}

\newcommand{\Su}{S_{1}}

\newcommand{\Sd}{S_{2}}

\newcommand{\Si}{S_{i}}

\newcommand{\pri}{pr_{i}}

\newcommand{\pru}{pr_{1}}

\newcommand{\prd}{pr_{2}}

\newcommand{\pii}{\pi_{i}}

\newcommand{\piu}{\pi_{1}}

\newcommand{\pid}{\pi_{2}}

\newcommand{\unudoi}{\{1,2\}}

\newcommand{\Li}{L_{i}}

\newcommand{\Lu}{L_{1}}

\newcommand{\Ld}{L_{2}}

\newcommand{\ioi}{\iota_{i}}

\newcommand{\iou}{\iota_{1}}

\newcommand{\iod}{\iota_{2}}

\newcommand{\iii}{\iou\t\iod}

\newcommand{\K}{{\cal K}}

\newcommand{\sta}{\stackrel}

\newcommand{\fu}{f_{1}}

\newcommand{\fn}{f_{n}}

\newcommand{\fk}{f_{k}}

\newcommand{\e}{\eta}

\newcommand{\eps}{\epsilon}

\newcommand{\SLI}{S^{[I]}}

\newcommand{\CSLI}{C(S^{[I]})}

\newcommand{\CLI}{C(L)^{[I]}}

\newcommand{\mSI}{\models_{\SI}}

\newcommand{\mS}{\models_{\S}}

\newcommand{\tu}{t_{1}}

\newcommand{\tn}{t_{n}}

\newcommand{\tk}{t_{k}}

\renewcommand{\S}{\Sigma}

\newcommand{\s}{\sigma}

\newcommand{\Sn}{\Sigma_{n}}

\newcommand{\Sw}{\Sigma_{w}}

\newcommand{\wu}{w_{1}}

\newcommand{\wn}{w_{n}}

\newcommand{\wi}{w_{i}}

\newcommand{\wj}{w_{j}}

\newcommand{\yu}{y_{1}}

\newcommand{\yn}{y_{n}}

\newcommand{\yi}{y_{i}}

\newcommand{\yj}{y{j}}

\newcommand{\zu}{z_{1}}

\newcommand{\zn}{z_{n}}

\newcommand{\zk}{z_{k}}

\newcommand{\zi}{z_{i}}

\newcommand{\xu}{x_{1}}

\newcommand{\xn}{x_{n}}

\renewcommand{\xi}{x_{i}}

\newcommand{\xj}{x_{j}}

\newcommand{\wk}{w_{k}}

\newcommand{\xk}{x_{k}}

\newcommand{\yk}{y_{k}}

\newcommand{\su}{s_{1}}

\newcommand{\sn}{s_{n}}



\newcommand{\sj}{s_{j}}

\newcommand{\sk}{s_{k}}



\newcommand{\Imp}{\Rightarrow}



\newcommand{\SI}{\S_{I}}

\newcommand{\EI}{E_{I}}

\renewcommand{\phi}{\varphi}



\newcommand{\bw}{\bigwedge}

\newcommand{\bv}{\bigvee}

\newcommand{\w}{\wedge}

\renewcommand{\v}{\vee}



\renewcommand{\phi}{\varphi}

\newcommand{\phii}{\phi_{i}}

\newcommand{\phij}{\phi_{j}}

\newcommand{\phik}{\phi_{k}}

\newcommand{\phiu}{\phi_{1}}

\newcommand{\phin}{\phi_{n-1}}



\newcommand{\unen}{\{1,\ldots,n\}}

\newcommand{\ffu}{f_{1}}

\newcommand{\ffn}{f_{n}}

\newcommand{\ffi}{f_{i}}

\newcommand{\ffj}{f_{j}}

\newcommand{\ffk}{f_{k}}



\newcommand{\orc}{\forall}

\newcommand{\exi}{\exists}



\newcommand{\au}{a_{1}}

\newcommand{\an}{a_{n}}

\newcommand{\ai}{a_{i}}

\newcommand{\aj}{a_{j}}

\newcommand{\ak}{a_{k}}

\newcommand{\bu}{b_{1}}

\newcommand{\bn}{b_{n}}

\newcommand{\bi}{b_{i}}

\newcommand{\bj}{b_{j}}

\newcommand{\bk}{b_{k}}

\newcommand{\ra}{\rightarrow}



\renewcommand{\P}{{\cal P}}

\newcommand{\N}{{I\!\!N}}



\newcommand{\p}{\oplus}

\newcommand{\cd}{\odot}

\newcommand{\unmu}{\{1,\ldots,n-1\}}



\newcommand{\Ss}{S_{\sigma}}

\newcommand{\Th}{\Theta}

\newcommand{\ThS}{\Theta_{\S}}



\newcommand{\Luk}{ {\cal L}uk_{n} }

\newcommand{\Gen}{{\cal NMVA}_{n}}

\newcommand{\CGen}{{\cal NMVA}_{n}}

\newcommand{\GR}{{\cal NLGU}_{n}}
\newcommand{\gr}{NLGU_{n}}

\newcommand{\NMVN}{{\cal NMVA}_n}



%%%%%%%%%%%%%%%%%%%%%%%%%%%%%%%%%%%%%%%%%%%%%%%%%%

\newcommand{\lra}{{\longrightarrow}}

\newcommand{\rat}{{\rightarrowtail}}
\newcommand{\oset}[1]{\overset {#1}{\ra}}
\newcommand{\osetl}[1]{\overset {#1}{\lra}}
\newcommand{\hr}{{\hookrightarrow}}
\newcommand{\labto}[1]{\overset{#1}{\lra}}
\newcommand{\midsq}[1]{\save\go[0,0];[1,1]:(0.5,0) \drop{#1}\restore}
%the following allows for longer arrows with long labels
%the first entry gives the additional length and the second gives the label
%e.g. \llabto{0.75}{\rm{long label}}
\newcommand{\llabto}[2]{\stackrel{#2}
{\rule[0.5ex]{#1 em}{0.05ex}\hspace{-0.4em}\longrightarrow}}

%the next gives two direction arrows at the top of a 2 x 2 matrix

\newcommand{\directs}[2]{\def\objectstyle{\scriptstyle}  \objectmargin={0pt}
\xy (0,4)*+{}="a",(0,-2)*+{\rule{0em}{1.5ex}#2}="b",(7,4)*+{\;#1}="c" \ar@{->} "a";"b"
\ar @{->}"a";"c" \endxy }
%the next gives two direction arrows at the middle of a 2 x 2 matrix

\newcommand{\xdirects}[2]{\def\objectstyle{\scriptstyle}  \objectmargin={0pt}
\xy (0,0)*+{}="a",(0,-6)*+{\rule{0em}{1.5ex}#2}="b",(7,0)*+{\;#1}="c" \ar@{->} "a";"b"
\ar @{->}"a";"c" \endxy }
%and this is smaller for a 1 x 1 matrix
\newcommand{\sdirects}[2]{\def\objectstyle{\scriptstyle}  \objectmargin={0pt}
\xy (0,2.2)*+{}="a",(0,-2.5)*+{\rule{0em}{1.5ex}#2}="b",(7,2.2)*+{\;#1}="c" \ar@{->}
"a";"b" \ar @{->}"a";"c" \endxy }

% the following are codes for the identities and connections
\newcommand{\bl}{\mbox{\rule{0.08em}{1.7ex}\hspace{-0.00em}\rule{0.7em}{0.2ex}}}

\newcommand{\br}{\mbox{\rule{0.7em}{0.2ex}\hspace{-0.04em}\rule{0.08em}{1.7ex}}}

\newcommand{\tr}{\mbox{\rule[1.5ex]{0.7em}{0.2ex}\hspace{-0.03em}\rule{0.08em}{1.7ex}}}

\newcommand{\tl}{\mbox{\rule{0.08em}{1.7ex}\rule[1.54ex]{0.7em}{0.2ex}}}

\newcommand{\hh}{\mbox{\rule{0.7em}{0.2ex}\hspace{-0.7em}\rule[1.5ex]{0.70em}{0.2ex}}}

\newcommand{\vv}{\mbox{\rule{0.08em}{1.7ex}\hspace{0.6em}\rule{0.08em}{1.7ex}}}

\newcommand{\sq}{\mbox{\rule{0.08em}{1.7ex}\hspace{-0.00em}\rule{0.7em}{0.2ex}\hspace{-0.7em}\rule[1.54ex]{0.7em}{0.2ex}\hspace{-0.03em}\rule{0.08em}{1.7ex}}}

\newcommand{\tsq}{\mbox{\rule{0.04em}{1.55ex}\hspace{-0.00em}\rule{0.7em}{0.1ex}\hspace{-0.7em}\rule[1.5ex]{0.7em}{0.1ex}\hspace{-0.03em}\rule{0.04em}{1.55ex}}}
\newcommand{\tssq}{\mbox{\rule{0.04em}{1.55ex}\hspace{-0.00em}\rule{0.7em}{0.1ex}\hspace{-0.7em}\rule[1.5ex]{0.7em}{0.1ex}\hspace{-0.03em}\rule{0.04em}{1.55ex}\hspace{-0.43em}\rule[0.7ex]{0.1em}{0.2ex}}}

\newcommand{\quads}[4]{\left( #1 \hspace{1em}
\overset{\textstyle{#2}}{\underset{\textstyle{#4}} {\rule{0mm}{1mm}}} \hspace{1em} #3
\right)}
\def\bu{\bullet}
\def\prt{\partial}
\def\eps{\varepsilon}
\def\red{\textcolor{red}}
\def\blue{\textcolor{blue}}

\def\leq{\leqslant}
\def\geq{\geqslant}
\def\le{\leqslant}
\def\ge{\geqslant}

\begin{document}
\subsection{A list of references for N-valued logics and their applications}

\begin{thebibliography} {99}

\bibitem{AS-BC2k}
Awodey, S. \& Butz, C., 2000, Topological Completeness for Higher Order Logic., Journal of Symbolic Logic, 65, 3, 1168--1182.

\bibitem{AS-RER2k2}
Awodey, S. \& Reck, E. R., 2002, Completeness and Categoricity II. Twentieth-Century Metalogic to Twenty-first-Century Semantics, \emph{History and Philosophy of Logic}, 23, (2): 77--94.

\bibitem{AS96}
Awodey, S., 1996, Structure in Mathematics and Logic: A Categorical Perspective,
\emph{Philosophia Mathematica}, 3: 209--237.

\bibitem{BAJ-DJ97}
Baez, J., 1997, An Introduction to n-Categories, in \emph{Category Theory and Computer Science, Lecture Notes in Computer Science}, 1290, Berlin: Springer-Verlag, 1--33.

\bibitem{BAJ-DJ98a}
Baez, J. \& Dolan, J., 1998a, Higher-Dimensional Algebra III. n-Categories and the Algebra of Opetopes,
in: \emph{Advances in Mathematics}, 135, 145--206.

\bibitem{IBRGB05}
Baianu, I. C., R. Brown , G. Georgescu and J. F. Glazebrook:  2006, Complex Nonlinear Biodynamics in
Categories, Higher Dimensional Algebra and \L ukasiewicz--Moisil Topos: Transformations of Neuronal, Genetic and
Neoplastic Networks., \emph{Axiomathes,}, 16: 82-165.

\bibitem{ICB77}
Baianu, I. C.: 1977, A Logical Model of Genetic Activities in \L{}ukasiewicz Algebras: The Non--linear Theory,
\emph{Bull. of Math. Biol}. \textbf{39}, 249--258.

\bibitem{Ba-We2k}
M.~Barr and C.~Wells. {\em Toposes, Triples and Theories}. Montreal: McGill University, 2000.

\bibitem{Ba-We85}
Barr, M. \& Wells, C., 1985, Toposes, Triples and Theories, New York: Springer-Verlag.

\bibitem{Birkhoff48}
Birkhoff, G.: 1948, Lattice Theory, {\em Amer. Math. Soc.}, New York.

\bibitem{bfgr:luk}
Boicescu, V., A. Filipoiu, G. Georgescu, and S. Rudeanu.: 1991, \emph{\L{}ukasiewicz-Moisil Algebras},
North-Holland, Amsterdam.

\bibitem{cha:alg58}
Chang, C. C.: 1958, Algebraic analysis of many valued logics. \emph{Trans. Amer. Math. Soc}., \textbf{88},
467--490.

\bibitem {cha:alg59}
Chang, C. C.: 1959, A new proof of the completeness of the \L{}ukasiewicz axioms, \emph{Transactions American
Mathematical Society} \textbf{93}, 74-80.

\bibitem{cig:moi}
Cignoli, R., Esteva, F., Godo, L. and Torrens, A. : 2000, Basic Fuzzy Logic is the logic of continuous t-norms
and their residua, \emph{Soft Computing} \textbf{4}, 106-112.

\bibitem{cig:moi}
Cignoli, R.: Moisil algebras, \emph{Notas de Logica Matematica}, Inst. Mat., Univ. Nacional del Sur,
Bahia-Blanca, No. 27.

\bibitem{NB64}
Bourbaki, N. : 1964. \emph{El\'ements de Math\'ematique, Livre II, Alg\`ebre}, \textbf{4}, Hermann, Editor,
Paris.

\bibitem{CR638}
Carnap, R.: 1938, \emph{The Logical Syntax of Language}, Harcourt, Brace and Co., New York.

\bibitem{Ehresmann65}
Ehresmann, C.: 1965, \emph{Cat\'egories et Structures}, Dunod, Paris.

\bibitem{EM45}
Eilenberg, S. and S. MacLane: 1945, The General Theory of Natural Equivalences, \emph{Trans. Amer. Math. Soc.}
\textbf{58}, 231--294.

\bibitem{GGDP68}
Georgescu, G. and D. Popescu: 1968, On Algebraic Categories, \emph{Rev. Roum. Math. Pures et Appl.}
\textbf{13}, 337--342.

\bibitem{geo:cen70}
Georgescu, G., and C. Vraciu.: 1970. On the characterization of centered \L{}ukasiewicz algebras. 
\emph{J. Algebra} \textbf{16}, 486-495.

\bibitem{geo:tow2k}
Georgescu, G., and I. Leu\c stean.: 2000. Towards a probability theory based on Moisil logic, \emph{Soft
Computing} \textbf{4}, 19-26.

\bibitem{grig77}
Grigolia, R.S.: 1977. Algebraic analysis of \L ukasiewicz-Tarski's logical systems, in W\'{o}jcicki, R.,
Malinowski, G. (Eds), \emph{Selected Papers on \L ukasiewicz Sentential Calculi}, Osolineum, Wroclaw, pp. 81-92.

\bibitem{Hilbert-Ack27}
Hilbert, D. and W. Ackerman: 1927, \emph{Grunduge der Theoretischen Logik}, Springer, Berlin.

\bibitem{Kan}
Kan, D.M.: 1958, Adjoint Functors, \emph{Trans Amer. Math. Soc.} \textbf{87}, 294-329.

\bibitem{lambek-scott86}
Lambek J. and P. J. Scott: 1986, \emph{Introduction to Higher Order Categorical Logic}, Cambridge University
Press, Cambridge, UK, 1986.

\bibitem{Lawvere}
Lawvere, F.W.: 1963, Functorial Semantics of Algebraic Theories, \emph{Proc. Natl. Acad. Sci. USA.}
\textbf{50}, 869--872.

\bibitem{Lof}
L\"{o}fgren, L.: 1968, An Axiomatic Explanation of Complete Self-Reproduction,\emph{ Bull. Math. Biophys.}
\textbf{30}, 317--348.

\bibitem{Luk70}
\L{}ukasiewicz, J.: 1970, \emph{Selected Works}, (ed.: L. Borkowski), North-Holland Publ. Co., Amsterdam and PWN,
Warsaw.

\bibitem{MM}
MacLane, S. and I. Moerdijk: 1992, \emph{Sheaves in Geometry and Logic - A first Introduction to Topos Theory},
Springer Verlag, New York.

\bibitem{MP43}
McCulloch, W. and W. Pitts: 1943, `A Logical Calculus of Ideas Immanent in Nervous Activity', \emph{Bull. Math.
Biophys}. \textbf{5}, 115--133.

\bibitem{Mn51}
McNaughton, R.: 1951, A theorem about infinite-valued sentential logic, \emph{Journal Symbolic Logic}
\textbf{16}, 1-13.

\bibitem{moi:ess}
Moisil, Gr. C.: 1972, \emph{Essai sur les logiques non-chrysippiennes}. Ed. Academiei, Bucharest.

\bibitem{mundici86}
Mundici, D.: 1986, Interpretation of AF C*-algebras in \L{}ukasiewicz sentential calculus, \emph{J. Functional
Analysis} \textbf{65}, 15-63.

\bibitem{Rose56}
Rose, A.: 1956, Formalisation du calcul propositionnel implicatif \`a $\aleph_0$ valeurs de \L{}ukasiewicz,
\emph{C. R. Acad. Sci. Paris} \textbf{243},1183-1185.

\bibitem{RRs58}
Rose, A. and Rosser, J.B.: 1958, Fragments of many-valued statement calculi, \emph{Transactions American
Mathematical Society} \textbf{87}, 1-53.

\bibitem{Rose62}
Rose, A.: 1962, Extensions of Some Theorems of Anderson and Belnap, \emph{J. Symbolic Logic}, \textbf{27},
(4), 423--425.

\bibitem{Rose78}
Rose, A.: 1978, `Formalisations of Further $\aleph_0$--Valued \L{}ukasiewicz Propositional Calculi'. \emph{J. Symbolic
Logic}, \textbf{43}(2): 207-210

\bibitem{RR58A}
Rosen, R.: 1958a, A Relational Theory of Biological Systems, \emph{Bull. Math. Biophys.} \textbf{20},
245--260.

\bibitem{RR58B}
Rosen, R.: 1958b, ``The Representation of Biological Systems from the Standpoint of the Theory of Categories.'',
\emph{Bull. Math. Biophys.} \textbf{20}, 317-341.

\bibitem {RR91}
Rosen, R.: 1991, \emph{Life Itself}, Columbia University Press, New York.

\bibitem{RR99}
Rosen, R.: 1999, \emph{Essays on Life Itself}, Columbia University Press, New York.

\bibitem{RP50}
Rosenbloom, Paul.: 1950, \emph{The Elements of Mathematical Logic}, Dover, New York.

\bibitem{RP62}
Rosenbloom, Paul.:1962, \emph{ibid.}, Prentice Hall, Englewood Cliffs, N.J.

\bibitem{RJTA52}
Rosser, J.B. and Turquette, A.R.: 1952, \emph{Many-Valued Logics}. North-Holland Publ. Co., Amsterdam.

\end{thebibliography}

%%%%%
%%%%%
\end{document}
