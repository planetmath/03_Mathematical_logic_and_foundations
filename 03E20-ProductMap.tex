\documentclass[12pt]{article}
\usepackage{pmmeta}
\pmcanonicalname{ProductMap}
\pmcreated{2013-03-22 17:48:28}
\pmmodified{2013-03-22 17:48:28}
\pmowner{asteroid}{17536}
\pmmodifier{asteroid}{17536}
\pmtitle{product map}
\pmrecord{6}{40270}
\pmprivacy{1}
\pmauthor{asteroid}{17536}
\pmtype{Definition}
\pmcomment{trigger rebuild}
\pmclassification{msc}{03E20}

% this is the default PlanetMath preamble.  as your knowledge
% of TeX increases, you will probably want to edit this, but
% it should be fine as is for beginners.

% almost certainly you want these
\usepackage{amssymb}
\usepackage{amsmath}
\usepackage{amsfonts}

% used for TeXing text within eps files
%\usepackage{psfrag}
% need this for including graphics (\includegraphics)
%\usepackage{graphicx}
% for neatly defining theorems and propositions
%\usepackage{amsthm}
% making logically defined graphics
%%%\usepackage{xypic}

% there are many more packages, add them here as you need them

% define commands here

\begin{document}
\PMlinkescapeword{continuous}

{\bf Notation:} If $\{X_i\}_{i \in I}$ is a collection of sets (indexed by $I$) then $\displaystyle \prod_{i \in I} X_i$ denotes the generalized Cartesian product of $\{X_i\}_{\i \in I}$.

Let $\{A_i\}_{i\in I}$ and $\{B_i\}_{i\in I}$ be collections of sets indexed by the same set $I$ and $f_i:A_i\longrightarrow B_i$ a collection of functions.

The {\bf product map} is the function
\begin{align*}
\prod_{i \in I} f_i : \prod_{i \in I} A_i \longrightarrow \prod_{i \in I} B_i\\
\Big( \prod_{i \in I} f_i \Big) (a_i)_{i \in I} := (f_i(a_i))_{i \in I}
\end{align*}

\subsection{Properties:}
\begin{itemize}
\item If $f_i:A_i\longrightarrow B_i$ and $g_i:B_i\longrightarrow C_i$ are collections of functions then 
\begin{displaymath}
\prod_{i \in I} g_i \circ \prod_{i \in I} f_i = \prod_{i \in I} g_i \circ f_i
\end{displaymath}
\item $\displaystyle \prod_{i \in I} f_i$ is injective if and only if each $f_i$ is injective.
\item $\displaystyle \prod_{i \in I} f_i$ is surjective if and only if each $f_i$ is surjective.
\item Suppose $\{A_i\}_{i \in I}$ and $\{B_i\}_{i \in I}$ are topological spaces. Then $\displaystyle \prod_{i \in I} f_i$ is \PMlinkname{continuous}{Continuous} (in the product topology) if and only if each $f_i$ is continuous.
\item Suppose $\{A_i\}_{i \in I}$ and $\{B_i\}_{i \in I}$ are groups, or rings or algebras. Then $\displaystyle \prod_{i \in I} f_i$ is a group (ring or \PMlinkescapetext{algebra}) homomorphism if and only if each $f_i$ is a group (ring or \PMlinkescapetext{algebra}) homomorphism.
\end{itemize}
%%%%%
%%%%%
\end{document}
