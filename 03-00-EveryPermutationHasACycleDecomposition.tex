\documentclass[12pt]{article}
\usepackage{pmmeta}
\pmcanonicalname{EveryPermutationHasACycleDecomposition}
\pmcreated{2013-03-22 16:48:39}
\pmmodified{2013-03-22 16:48:39}
\pmowner{rspuzio}{6075}
\pmmodifier{rspuzio}{6075}
\pmtitle{every permutation has a cycle decomposition}
\pmrecord{10}{39045}
\pmprivacy{1}
\pmauthor{rspuzio}{6075}
\pmtype{Proof}
\pmcomment{trigger rebuild}
\pmclassification{msc}{03-00}
\pmclassification{msc}{05A05}
\pmclassification{msc}{20F55}

\endmetadata

% this is the default PlanetMath preamble.  as your knowledge
% of TeX increases, you will probably want to edit this, but
% it should be fine as is for beginners.

% almost certainly you want these
\usepackage{amssymb}
\usepackage{amsmath}
\usepackage{amsfonts}

% used for TeXing text within eps files
%\usepackage{psfrag}
% need this for including graphics (\includegraphics)
%\usepackage{graphicx}
% for neatly defining theorems and propositions
%\usepackage{amsthm}
% making logically defined graphics
%%%\usepackage{xypic}

% there are many more packages, add them here as you need them

% define commands here

\begin{document}
In this entry, we shall show that every permutation of a 
finite set can be factored into a product of disjoint cycles.  
To accomplish this, we shall proceed in two steps.

We begin by showing that, if $f$ is a non-trivial 
permutation of a set $\{ x_i \mid 1 \le i \le n \}$,
then there exists a cycle $(y_1, \ldots y_m)$ where
\[
\{ y_i \mid 1 \le i \le m \} \subseteq
\{ x_i \mid 1 \le i \le n \}
\]
and a permutation $g$ of $\{ x_i \mid 1 \le i \le n \}$
such that $f = (y_1, \ldots y_m) \circ g$ and $g(y_i) = y_i$.

Since the permutation is not trivial, there exists
$z$ such that $f(z) \neq z$.  Define a sequence
inductively as follows:
\begin{align*}
z_1 &= z \\
z_{k+1} &= f(z_k)
\end{align*}

Note that we cannot have $z_{k+1} = z_k$ for any $k$.
This follows from a simple induction argument.  By
definition $f(z_1) = f(z) \neq z = z_1$.  Suppose 
that $f(z_k) \neq z_k$ but that $f(z_{k+1}) = z_{k+1}$.
By definition, $f(z_k) = z_{k+1}$.  Since $f$ is a 
permutation, $f(z_k) = z_{k+1}$ and $f(z_{k+1}) = z_{k+1}$
imply that $z_k = z_{k+1}$, so $z_k = f(z_k)$, which
contradicts a hypothesis.  Hence, if $f(z_k) \neq z_k$,
then $f(z_{k+1}) \neq z_{k+1}$ so, by induction,
$f(z_k) \neq z_k$ for all $k$. 

By the pigeonhole principle, there must exist $p$ and $q$
such that $p < q$ but $f(z_p) = f(z_q)$.  Let $m$ be the
least integer such that $f(z_p) = f(z_{p+m})$ but
$f(z_p) \neq f(s_{p+k})$ when $k < m$.  Set $y_k = z_{k+p}$.
Then we have that $(y_1, \ldots y_m)$ is a cycle.

Since $f$ is a permutation and $\{ y_i \mid 1 \le i \le m \}$
is closed under $f$, it follows that 
\[
\{ x_i \mid 1 \le i \le n \} \setminus 
\{ y_i \mid 1 \le i \le m \}
\]
is also closed under $f$.  Define $g$ as follows:
\[
g(z) =
\begin{cases}
z    & z \in \{ y_i \mid 1 \le i \le m \} \\
f(z) & z \in \{ x_i \mid 1 \le i \le n \} \setminus 
\{ y_i \mid 1 \le i \le m \}
\end{cases}
\]
Then it is easily verified that 
$f = (y_1, \ldots y_m) \circ g$.

We are now in a position to finish the proof that
every permutation can be decomposed into cycles.
Trivially, a permutation of a set with one element
can be decomposed into cycles because the only
permutation of a set with one element is the
identity permutation, which requires no cycles
to decompose.  Next, suppose that any set with less
than $n$ elements can be decomposed into cycles.
Let $f$ be a permutation on a set with $n$ elements.
Then, by what we have shown, $f$ can be written as
the product of a cycle and a permutation $g$ which
fixes the elements of the cycle.  The restriction of
$g$ to those elements $z$ such that $g(z) \neq z$
is a permutation on less than $n$ elements and
hence, by our supposition, can be decomposed into 
cycles.  Thus, $f$ can also be decomposed into
cycles.  By induction, we conclude that any
permutation of a finite set can be decomposed into cycles.
%%%%%
%%%%%
\end{document}
