\documentclass[12pt]{article}
\usepackage{pmmeta}
\pmcanonicalname{NeutrosophicLogic}
\pmcreated{2013-03-22 14:21:45}
\pmmodified{2013-03-22 14:21:45}
\pmowner{para0doxa}{5174}
\pmmodifier{para0doxa}{5174}
\pmtitle{neutrosophic logic}
\pmrecord{18}{35843}
\pmprivacy{1}
\pmauthor{para0doxa}{5174}
\pmtype{Definition}
\pmcomment{trigger rebuild}
\pmclassification{msc}{03B50}
\pmclassification{msc}{03B53}
\pmclassification{msc}{03B60}
%\pmkeywords{neutrosophy}
%\pmkeywords{neutrosophic set}
%\pmkeywords{neutrosophic probability}
%\pmkeywords{neutrosophic statistics}
%\pmkeywords{fuzzy logic}
%\pmkeywords{intuitionistic logic}
%\pmkeywords{paraconsistent logic}
\pmrelated{FlorentinSmarandache}

% this is the default PlanetMath preamble.  as your knowledge
% of TeX increases, you will probably want to edit this, but
% it should be fine as is for beginners.

% almost certainly you want these
\usepackage{amssymb}
\usepackage{amsmath}
\usepackage{amsfonts}

% used for TeXing text within eps files
%\usepackage{psfrag}
% need this for including graphics (\includegraphics)
%\usepackage{graphicx}
% for neatly defining theorems and propositions
%\usepackage{amsthm}
% making logically defined graphics
%%%\usepackage{xypic}

% there are many more packages, add them here as you need them

% define commands here
\begin{document}
A logic, in which each proposition is estimated to have the degree of truth in $T$, the degree of indeterminacy (neither true nor false) in $I$, and the degree of false in $F$, is called \emph{neutrosophic logic}, where $T, I, F$ are standard or non-standard real subsets of the non-standard unit interval $]^-0, 1^+[$.

$T, I, F$ are called \emph{neutrosophic components}.

Now let's explain the previous notations:
\newline A number $\varepsilon$ is said to be \emph{infinitesimal} if and only if for all positive integers $n$ one has $|\varepsilon| < \frac{1}{n}$.  Let $\varepsilon > 0$ be a such infinitesimal number.  The \emph{hyper-real number set} is an  extension of the real number set, which includes classes of infinite numbers and classes of infinitesimal numbers.  
\newline Generally, for any real number $a$ one defines $^-a$ which signifies a \emph{monad}, i.e. a set of hyper-real numbers in non-standard analysis, as follows:
\newline $^-a = \{a-\varepsilon: \varepsilon \in R^*, \varepsilon$ is infinitesimal $\}$,
\newline and similarly one defines $a^+$, which is also a monad, as:
\newline $a^+ = \{a+\varepsilon: \varepsilon \in R^*, \varepsilon$ is infinitesimal $\}$.
\newline A \emph{binad} $^-a^+$ is a union of the above two monads, i.e.
\newline $ ^-a^+ = ^-a \cup a^+$.
\newline For example: The non-standard finite number $1^+ = 1+\varepsilon$, where $1$ is its \emph{standard part} and $\varepsilon$ its \emph{non-standard part}, and similarly the non-standard finite number $^-0 = 0-\varepsilon$, where $0$ is its standard part and $\varepsilon$ its \emph{non-standard part}.
\newline Similarly for $3^+ = 3+ \varepsilon$, etc.
\newline Note that $] ^-0, 1^+ [$ is called the \emph{non-standard unit interval}.  
\PMlinkexternal{ More information on non-standard intervals is available.} {http://www.gallup.unm.edu/~smarandache/Introduction.pdf}

The superior sum of the neutrosophic components is defined as:
$$n_{sup} = sup(T) + sup(I) + sup(F) \in ]^-0, 3^+[$$
may be as high as 3 or $3^+$. 
\newline While the inferior sum of the neutrosophic components is defined as:
$$n_{inf} = inf(T) + inf(I) + inf(F) \in ]^-0, 3^+[$$
may be as low as 0 or $^-0$.  

Neutrosophic logic was introduced by Florentin Smarandache in 1995 as a generalization of \emph{fuzzy logic} (especially of intuitionistic fuzzy logic) when $n_{sup} = 1$, of \emph{intuitionistic logic} when $n_{sup} < 1$, and of \emph{paraconsistent logic} when $n_{sup} > 1$.

The main distinctions between the neutrosophic logic (NL) and intuitionistic fuzzy logic (IFL) are the facts that (a) the sum of neutrosophic components (or of their superior limits when they are subsets) in NL is not necessarily 1 as in IFL but any number from $^-0$ to $3^+$ in order to allow the characterization of incomplete or paraconsistent information, and (b) in NL one uses a non-standard interval $]^-0, 1^+[$ in order to make a difference in philosophy between \emph{absolute truth}, denoted by $1^+$, and \emph{relative truth}, denoted by $1$, and similarly distinctions between \emph {absolute falsehood} and \emph {relative falsehood} or between \emph {absolute indeterminacy} and \emph {relative indeterminacy} respectively, while in IFL one has a standard interval $[0, 1]$.

{\bf  Examples}:
\newline - One uses a subset of truth (or indeterminacy, or falsity), instead of a number, because in many cases we are not able to exactly determine the degrees of truth and of false but to approximate them.
\newline - In technical applications, where there is no need for distinctions between absolute truth and relative truth, we can use standard subsets instead of non-standard subsets and respectively the unit interval $[0,1]$ instead of the non-standard unit interval $]^-0, 1^+[$.
\newline - Let say the proposition \textit{One month from today it will be raining} can be between 0.30-0.40 or 0.45-0.50 true (according to various analyzers), 0.10 or 0.20 indeterminate (neither true nor false, but unknown, due to possible hidden parameters that might influence the raining), and 0.60 or between 0.66-0.70 false.
\newline - The subsets are not necessary intervals, but any sets (discrete, continuous, open or closed or half-open/half-closed interval, intersections or unions of the previous sets, etc.) in accordance with the given proposition.
\newline - A subset may also have only one element, which is the easiest particular case of this logic.

\begin{thebibliography}{9}
\bibitem{smarandache} Florentin Smarandache, {\em A Unifying Field in Logics: Neutrosophic Logic. Neutrosophy, Neutrosophic Set, Neutrosophic Probability and Statistics}, third edition, Xiquan, Phoenix, 2003.
\PMlinkexternal{Also online.}{http://www.gallup.unm.edu/~smarandache/eBook-Neutrosophics2.pdf}
\bibitem{smarandache2} F. Smarandache, J. Dezert, A. Buller, M. Khoshnevisan, S. Bhattacharya, S. Singh, F. Liu, Gh. C. Dinulescu-Campina, C. Lucas, C. Gershenson, {\em Proceedings of the First International Conference on Neutrosophy, Neutrosophic Logic, Neutrosophic Set, Neutrosophic Probability and Statistics}, The University of New Mexico, Gallup Campus, 1-3 December 2001.
\htmladdnormallink{Also online.}{http://arxiv.org/pdf/math.GM/0306384}
\end{thebibliography}
%%%%%
%%%%%
\end{document}
