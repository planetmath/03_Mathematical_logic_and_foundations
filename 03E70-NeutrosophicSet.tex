\documentclass[12pt]{article}
\usepackage{pmmeta}
\pmcanonicalname{NeutrosophicSet}
\pmcreated{2013-03-22 15:21:49}
\pmmodified{2013-03-22 15:21:49}
\pmowner{para0doxa}{5174}
\pmmodifier{para0doxa}{5174}
\pmtitle{neutrosophic set}
\pmrecord{9}{37187}
\pmprivacy{1}
\pmauthor{para0doxa}{5174}
\pmtype{Definition}
\pmcomment{trigger rebuild}
\pmclassification{msc}{03E70}
%\pmkeywords{neutrosophy}
%\pmkeywords{neutrosophic logic}

\endmetadata

% this is the default PlanetMath preamble.  as your knowledge
% of TeX increases, you will probably want to edit this, but
% it should be fine as is for beginners.

% almost certainly you want these
\usepackage{amssymb}
\usepackage{amsmath}
\usepackage{amsfonts}

% used for TeXing text within eps files
%\usepackage{psfrag}
% need this for including graphics (\includegraphics)
%\usepackage{graphicx}
% for neatly defining theorems and propositions
%\usepackage{amsthm}
% making logically defined graphics
%%%\usepackage{xypic}

% there are many more packages, add them here as you need them

% define commands here
\begin{document}
Let $M$ be a subset of a universe of discourse $U$. Each element 
$x\in U$ has degrees of membership, indeterminacy, and non-membership 
in $M$, which are subsets of the hyperreal interval $]^-0,1^+[$. The 
notation $x(T, I, F) \in M$ means that
 \begin{itemize}
 \item the degree of membership of $x$ in $M$ is $T$;
 \item the degree of indeterminacy of $x$ in $M$ is $I$; and
 \item the degree of non-membership of $x$ in $M$ is $F$.
 \end{itemize}

$M$ is called \emph{neutrosophic set}, whereas $T, I, F$ are called \emph{neutrosophic components} of the element $x$ with respect to $M$.

Now let's explain the previous notations:
\newline A number $\varepsilon$ is said to be \emph{infinitesimal} if and only if for all positive integers $n$ one has $|\varepsilon| < \frac{1}{n}$.  Let $\varepsilon > 0$ be a such infinitesimal number.  The \emph{hyper-real number set} is an  extension of the real number set, which includes classes of infinite numbers and classes of infinitesimal numbers.  
\newline Generally, for any real number $a$ one defines $^-a$ which signifies a \emph{monad}, i.e. a set of hyper-real numbers in non-standard analysis, as follows:
\newline $^-a = \{a-\varepsilon: \varepsilon \in R^*, \varepsilon$ is infinitesimal $\}$,
\newline and similarly one defines $a^+$, which is also a monad, as:
\newline $a^+ = \{a+\varepsilon: \varepsilon \in R^*, \varepsilon$ is infinitesimal $\}$.
\newline A \emph{binad} $^-a^+$ is a union of the above two monads, i.e.
\newline $ ^-a^+ = ^-a \cup a^+$.
\newline For example: The non-standard finite number $1^+ = 1+\varepsilon$, where $1$ is its \emph{standard part} and $\varepsilon$ its \emph{non-standard part}, and similarly the non-standard finite number $^-0 = 0-\varepsilon$, where $0$ is its standard part and $\varepsilon$ its \emph{non-standard part}.
\newline Similarly for $3^+ = 3+ \varepsilon$, etc.
\newline Note that $] ^-0, 1^+ [$ is called the \emph{non-standard unit interval}.  
\newline More information on hyperreal intervals \PMlinkexternal{is available}{http://www.gallup.unm.edu/~smarandache/Introduction.pdf}.

The superior sum of the neutrosophic components is defined as 
$$n_{sup} = sup(T) + sup(I) + sup(F) \in ] ^-0, 3^+[$$
which may be as high as 3 or $3^+$. 
\newline While the inferior sum of the neutrosophic components is defined as 
$$n_{inf} = inf(T) + inf(I) + inf(F) \in ] ^-0, 3^+[$$
\newline which may be as low as 0 or $^-0$.  

The notion of neutrosophic set was introduced by Florentin Smarandache in 1995 as a generalization of \emph{fuzzy set} (especially of intuitionistic fuzzy set) when $n_{sup} = 1$, of \emph{intuitionistic set} when $n_{sup} < 1$, and of \emph{paraconsistent set} when $n_{sup} > 1$.

The main distinctions between the neutrosophic set (NS) and intuitionistic fuzzy set (IFS) are the facts that (a) the sum of the scalar neutrosophic components (or their superior sum, $n_{sup}$, if the neutrosophic components are subsets) in NS is not necessarily 1 as in IFS but any number from $^-0$ to $3^+$ in order to allow the characterization of incomplete or paraconsistent information, and (b) in NS one uses the non-standard interval $]^-0, 1^+[$ in order to make a difference between \emph{absolute membership}, denoted by $1^+$, and \emph{relative membership}, denoted by $1$, while in IFS one only uses the standard interval $[0, 1]$.

{\bf An example}:
\newline Let $A$ be a neutrosophic set.
\newline One can say, by abuse of language, that any element \emph{neutrosophically} belongs to any set, due to the flexibility of degrees of truth/indeterminacy/falsity involved, which each varies between $^-0$ and $1^+$.
\newline Thus the element $x(0.1, 0.2, 0.3) \in A$ means, the degree of membership of $x$ in $A$ is 0.1, the degree on indeterminacy (undecidability) is 0.2, and the degree of non-membership is 0.3 (as one sees, the sum of components is < 1). 
\newline Similarly the element $y(0.6, 0.2, 0.5) \in A$, with the sum of components > 1.
\newline Or the element $z(0.7, 0.1, 0.2) \in A$, with the sum of components = 1.
\newline  More general, the element $w( (0.20-0.30), (0.40-0.45) \cup [0.50-0.51], \{0.20, 0.24, 0.28\} ) \in A$, means:
\newline - the degree of membership is between 0.20-0.30 (one cannot find an exact approximation because of various sources used);
\newline - the degree of indeterminacy related to the appurtenance of $w$ to $A$ is between 0.40-0.45 or between 0.50-0.51 (limits included);
\newline - the degree of non-membership is 0.20 or 0.24 or 0.28.

{\bf A remark}:
\newline - In technical applications, where there is no need for distinctions between absolute membership and relative membership, we can use standard subsets instead of non-standard subsets and respectively the unit interval $[0,1]$ instead of the non-standard unit interval $]^-0, 1^+[$.

\begin{thebibliography}{9}
\bibitem{smarandache} F. Smarandache, {\em A Unifying Field in Logics: Neutrosophic Logic. Neutrosophy, Neutrosophic Set, Neutrosophic Probability and Statistics}, third edition, Xiquan, Phoenix, 2003.
\PMlinkexternal{The whole book is also online and can be downloaded here. }{http://www.gallup.unm.edu/~smarandache/eBook-Neutrosophics2.pdf}.
\bibitem{smarandache2} F. Smarandache, J. Dezert, A. Buller, M. Khoshnevisan, S. Bhattacharya, S. Singh, F. Liu, Gh. C. Dinulescu-Campina, C. Lucas, C. Gershenson, {\em Proceedings of the First International Conference on Neutrosophy, Neutrosophic Logic, Neutrosophic Set, Neutrosophic Probability and Statistics}, The University of New Mexico, Gallup Campus, 1-3 December 2001.
\htmladdnormallink{The Proceedings are also online and can be downloaded here.}{http://arxiv.org/pdf/math.GM/0306384}
\bibitem{wang} Haibin Wang, Praveen Madiraju, Yanqing Zhang, Rajshekhar Sunderraman, {\em Interval Neutrosophic Sets}, International Journal of Applied Mathematics and Statistics, Vol. 3, No. M05, 1-18, 2005.
\end{thebibliography}
%%%%%
%%%%%
\end{document}
