\documentclass[12pt]{article}
\usepackage{pmmeta}
\pmcanonicalname{IndexSetTheorem}
\pmcreated{2013-03-22 18:09:51}
\pmmodified{2013-03-22 18:09:51}
\pmowner{yesitis}{13730}
\pmmodifier{yesitis}{13730}
\pmtitle{index set theorem}
\pmrecord{5}{40724}
\pmprivacy{1}
\pmauthor{yesitis}{13730}
\pmtype{Theorem}
\pmcomment{trigger rebuild}
\pmclassification{msc}{03D25}

\endmetadata

% this is the default PlanetMath preamble.  as your knowledge
% of TeX increases, you will probably want to edit this, but
% it should be fine as is for beginners.

% almost certainly you want these
\usepackage{amssymb}
\usepackage{amsmath}
\usepackage{amsfonts}

% used for TeXing text within eps files
%\usepackage{psfrag}
% need this for including graphics (\includegraphics)
%\usepackage{graphicx}
% for neatly defining theorems and propositions
%\usepackage{amsthm}
% making logically defined graphics
%%%\usepackage{xypic}

% there are many more packages, add them here as you need them

% define commands here

\begin{document}
Index Set Theorem: \emph{If $A$ is an index set and $A\neq\varnothing, \omega$, then either $K\leq_1 A$ or $K\leq_1 A^{\complement}$.}

In the statement of the theorem, $K$ is the halting set $\{x : \varphi_x(x)\: converges \}$, $\leq_1$ is the one-one reducibility (or 1-reducibility) relation symbol, and $A^{\complement}$ stands for the complement of the set $A$ (relative to $\omega$).
%%%%%
%%%%%
\end{document}
