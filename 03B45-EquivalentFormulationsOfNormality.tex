\documentclass[12pt]{article}
\usepackage{pmmeta}
\pmcanonicalname{EquivalentFormulationsOfNormality}
\pmcreated{2013-03-22 19:34:42}
\pmmodified{2013-03-22 19:34:42}
\pmowner{CWoo}{3771}
\pmmodifier{CWoo}{3771}
\pmtitle{equivalent formulations of normality}
\pmrecord{16}{42565}
\pmprivacy{1}
\pmauthor{CWoo}{3771}
\pmtype{Feature}
\pmcomment{trigger rebuild}
\pmclassification{msc}{03B45}

\usepackage{amssymb,amscd}
\usepackage{amsmath}
\usepackage{amsfonts}
\usepackage{mathrsfs}
\usepackage{multicol}

% used for TeXing text within eps files
%\usepackage{psfrag}
% need this for including graphics (\includegraphics)
%\usepackage{graphicx}
% for neatly defining theorems and propositions
\usepackage{amsthm}
% making logically defined graphics
%%\usepackage{xypic}
\usepackage{pst-plot}

% define commands here
\newcommand*{\abs}[1]{\left\lvert #1\right\rvert}
\newtheorem{prop}{Proposition}
\newtheorem{thm}{Theorem}
\newtheorem{lem}{Lemma}
\newtheorem{ex}{Example}
\newcommand{\real}{\mathbb{R}}
\newcommand{\pdiff}[2]{\frac{\partial #1}{\partial #2}}
\newcommand{\mpdiff}[3]{\frac{\partial^#1 #2}{\partial #3^#1}}

\begin{document}
Recall that a logic in a modal propositional language is a set of wff's containing all the tautologies and is closed under modus ponens.  Depending on the intended meaning of the modal connective $\square$, one may extend the logic by adding wff's and/or imposing additional inference rules.  Among the more common modal inference rules are the following: 
\begin{multicols}{2}{
\begin{enumerate}
\item [RK]: $\displaystyle{\frac{(A_1 \wedge \cdots \wedge A_n) \to A}{(\square A_1 \wedge \cdots \wedge \square A_n)\to \square A}}$, $n\ge 0$
\item [RR]: $\displaystyle{\frac{(A \wedge B) \to C}{(\square A \wedge \square B)\to \square C}}$
\item [RM]: $\displaystyle{\frac{A \to B}{\square A \to \square B}}$
\item [RN](necessitation rule): $\displaystyle{\frac{A}{\square A}}$
\end{enumerate}
}
\end{multicols}
The most common modification is what is known as the normal modal logic.  It is obtained from a logic by adding the schema K: $$\square (A\to B)\to (\square A \to \square B)$$
and the rule of necessitation RN.

As it turns out, this is not the only way to formulate the notion of normality, as the following proposition illustrates:
\begin{prop}  Let $L$ be a logic.  Then the following are equivalent:
\begin{enumerate}
\item $L$ is normal
\item $L$ is closed under RK
\item $L$ contains $\square \top$ and is closed under RM and RR, where $\top$ is $\neg \perp$.
\item $L$ contains schemas $\square (A\to A)$, $\square A \land \square B \to \square (A\land B)$, and is closed under RM
\item $L$ contains schemas $\square (A\to A)$, K, and is closed under RM
\end{enumerate}
\end{prop}

First, three quick observations:
\begin{enumerate}
\item If $L$ is closed under RK, then it is closed under RR, RM, and RN.
\begin{proof} RR, RM, and RN are RK with $n=2,1$, and $0$ respectively.  \end{proof}
\item If $L$ is closed under RN, then $L$ contains \square (A\to A)$, and in particular $\square \top$.
\begin{proof} Since $A \to A$ is a tautology, we have $\vdash \square (A\to A)$ by RN.  Letting $A$ be $\perp$ gives us $\vdash \top$.  \end{proof}
\item If $L$ is closed under RM, and contains $\square B$ for some wff $B$, then $L$ is closed under RN.
\begin{proof} For any wff $A$ in $L$, by modus ponens on tautology $A\to (B\to A)$, we have $\vdash B\to A$, and so by RM, $\vdash \square B\to \square A$.  But $\vdash \square B$ by assumption, $\vdash \square A$ by modus ponens. \end{proof}
\end{enumerate}

Now, we prove the proposition.
\begin{proof}  We will prove the following implications $1 \Rightarrow 2 \Rightarrow 3 \Rightarrow 4 \Rightarrow 5 \Rightarrow 1$.
\begin{enumerate}
\item[$1 \Rightarrow 2$.] See this \PMlinkID{entry}{12558}.
\item[$2 \Rightarrow 3$.] By the first observation, $L$ is closed under RM and RR, and RN, and therefore contains $\square \top$ based on the second observation.
\item[$3 \Rightarrow 4$.] Apply RR to $A\land B \to A\land B$, we get $\vdash \square A \land \square B \to \square (A\land B)$.  Apply the third observation to $\square \top$, $L$ is closed under RN, and therefore contains $\square (A\to A)$ by the second observation.
\item[$4 \Rightarrow 5$.]  Apply the third observation to $\vdash \square (A\to A)$, we see that $L$ is closed under RN.  

Next, we show that $L$ contains K.  From the tautologies $A \land (A\to B)\to A\land B$ and $A\land B\to B$, we get the tautology $A \land (A\to B)\to B$ by the law of syllogism, so that $\vdash \square (A\land (A\to B)) \to \square B$ by RM.  Now, $\vdash \square A \land \square (A\to B) \to \square (A\land (A\to B))$ by assumption, $\vdash \square A \land \square (A\to B) \to \square B$ by the law of syllogism.  From the tautologies $X\land Y \leftrightarrow Y\land X$ and $(X\land Y \to Z) \to (X \to (Y \to Z))$, we get $\vdash $K by the substitution theorem and modus ponens.
\item[$5 \Rightarrow 1$.]  Similarly, $L$ is closed under RN.  Since $L$ contains K, it is normal.
\end{enumerate}
\end{proof}

%%%%%
%%%%%
\end{document}
