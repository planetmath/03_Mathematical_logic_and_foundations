\documentclass[12pt]{article}
\usepackage{pmmeta}
\pmcanonicalname{PartialMapping}
\pmcreated{2013-03-22 13:59:31}
\pmmodified{2013-03-22 13:59:31}
\pmowner{mathcam}{2727}
\pmmodifier{mathcam}{2727}
\pmtitle{partial mapping}
\pmrecord{5}{34768}
\pmprivacy{1}
\pmauthor{mathcam}{2727}
\pmtype{Definition}
\pmcomment{trigger rebuild}
\pmclassification{msc}{03E20}

% this is the default PlanetMath preamble.  as your knowledge
% of TeX increases, you will probably want to edit this, but
% it should be fine as is for beginners.

% almost certainly you want these
\usepackage{amssymb}
\usepackage{amsmath}
\usepackage{amsfonts}
\usepackage{amsthm}

% used for TeXing text within eps files
%\usepackage{psfrag}
% need this for including graphics (\includegraphics)
%\usepackage{graphicx}
% for neatly defining theorems and propositions
%\usepackage{amsthm}
% making logically defined graphics
%%%\usepackage{xypic}

% there are many more packages, add them here as you need them

% define commands here

\newcommand{\mc}{\mathcal}
\newcommand{\mb}{\mathbb}
\newcommand{\mf}{\mathfrak}
\newcommand{\ol}{\overline}
\newcommand{\ra}{\rightarrow}
\newcommand{\la}{\leftarrow}
\newcommand{\La}{\Leftarrow}
\newcommand{\Ra}{\Rightarrow}
\newcommand{\nor}{\vartriangleleft}
\newcommand{\Gal}{\text{Gal}}
\newcommand{\GL}{\text{GL}}
\newcommand{\Z}{\mb{Z}}
\newcommand{\R}{\mb{R}}
\newcommand{\Q}{\mb{Q}}
\newcommand{\C}{\mb{C}}
\newcommand{\<}{\langle}
\renewcommand{\>}{\rangle}
\begin{document}
Let $X_1, \cdots, X_n$ and $Y$ be sets, and let $f$ be a function of $n$ variables:  $f:X_1\times X_2\times\cdots\times X_n\to Y$.  \PMlinkescapetext{Fix} $x_i\in X_i$ for $2\leq i\leq n$.  The induced mapping $a\mapsto f(a,x_2,\ldots,x_n)$ is called the \emph{partial mapping} determined by $f$ corresponding to the first variable.

In the case where $n=2$, the map defined by $a\mapsto f(a,x)$ is often denoted $f(\cdot,x)$.  Further, any function $f:X_1\times X_2\to Y$ determines a mapping
from $X_1$ into the set of mappings of $X_2$ into $Y$, namely
$\overline{f}:x\mapsto(y\mapsto f(x,y))$.
The converse holds too, and it is customary to identify $f$ with
$\overline{f}$.  Many of the ``canonical isomorphisms'' that we come across (e.g. in multilinear algebra) are illustrations of this kind of identification.
%%%%%
%%%%%
\end{document}
