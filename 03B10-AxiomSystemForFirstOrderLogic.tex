\documentclass[12pt]{article}
\usepackage{pmmeta}
\pmcanonicalname{AxiomSystemForFirstOrderLogic}
\pmcreated{2013-03-22 19:32:22}
\pmmodified{2013-03-22 19:32:22}
\pmowner{CWoo}{3771}
\pmmodifier{CWoo}{3771}
\pmtitle{axiom system for first order logic}
\pmrecord{24}{42518}
\pmprivacy{1}
\pmauthor{CWoo}{3771}
\pmtype{Definition}
\pmcomment{trigger rebuild}
\pmclassification{msc}{03B10}
\pmsynonym{universal generalization}{AxiomSystemForFirstOrderLogic}
\pmdefines{generalization}

\usepackage{amssymb,amscd}
\usepackage{amsmath}
\usepackage{amsfonts}
\usepackage{mathrsfs}

% used for TeXing text within eps files
%\usepackage{psfrag}
% need this for including graphics (\includegraphics)
%\usepackage{graphicx}
% for neatly defining theorems and propositions
\usepackage{amsthm}
% making logically defined graphics
%%\usepackage{xypic}
\usepackage{pst-plot}

% define commands here
\newcommand*{\abs}[1]{\left\lvert #1\right\rvert}
\newtheorem{prop}{Proposition}
\newtheorem{thm}{Theorem}
\newtheorem{ex}{Example}
\newcommand{\real}{\mathbb{R}}
\newcommand{\pdiff}[2]{\frac{\partial #1}{\partial #2}}
\newcommand{\mpdiff}[3]{\frac{\partial^#1 #2}{\partial #3^#1}}

\begin{document}
Let FO$(\Sigma)$ be a first order language over signature $\Sigma$.  Furthermore, let $V$ be a countably infinite set of variables and $V(\Sigma)$ the set extending $V$ to include the constants in $\Sigma$.  Before describing the axiom system for FO$(\Sigma)$, we need the following definition:

\textbf{Definition}.  Let $A$ be a wff.  A \emph{generalization} of $A$ is a wff having the form $\forall x_1 \forall x_2 \cdots \forall x_n A$, for any $n\ge 0$.  Thus, $A$ is a generalization of itself.

The axiom system of FO$(\Sigma)$ consists of any wff that is a generalization of a wff belonging to any one of the following six schemas:
\begin{enumerate}
\item $A\to (B\to A)$
\item $(A\to (B\to C)) \to ((A\to B)\to (A\to C))$
\item $\neg \neg A \to A$
\item $\forall x (A\to B) \to (\forall x A \to \forall x B)$, where $x\in V$
\item $A\to \forall x A$, where $x\in V$ is not free in $A$
\item $\forall x A \to A[a/x]$, where $x\in V$, $a\in V(\Sigma)$, and $a$ is free for $x$ in $A$
\end{enumerate}
and modus ponens (from $A$ and $A\to B$ we may infer $B$) as its only rule of inference.

In schema 6, the wff $A[a/x]$ is obtained by replacing all free occurrences of $x$ in $A$ by $a$.  

Logical symbols $\land,\lor$, and $\exists$ are derived: $A\lor B:=\neg A\to B$; $A\land B:=\neg (\neg A\lor \neg B)$; and $\exists x A:=\neg \forall x \neg A$.

\textbf{Remark}.  Again, in schema 6, the reason why we want $a$ free for $x$ in $A$ is to keep the ``intended meaning'' of $A$ intact.  For example, suppose $A$ is $\forall y (x < y)$.  Then $A[y/x]$ is $\forall y (y < y)$.  Obviously the two do not have the same ``intended meaning''.  In fact, $A[y/x]$ is not valid in any model.  Similarly, in the schema 5, we want $x$ not occur free in $A$ to maintain its ``intended meaning''.  For example, if $A$ is the formula $x=0$, then $\forall x A$ is $\forall x (x=0)$, which is not true in any model with at least two elements.

The concepts of deductions and theorems are exactly the same as those found in propositional logic.  Here is an example: if $\vdash A$, then $\vdash \forall x A$.  To see this, we induct on the length $n$ of the deduction of $A$.  If $n=1$, then $A$ is an axiom, and therefore so is its generalization $\forall x A$, so that $\vdash \forall x A$. Suppose now that the length is $n+1$.  If $A$ is an axiom, we are back to the previous argument.  Otherwise, $A$ is obtained from earlier formulas $B$ and $B\to A$ via modus ponens.  Since the deduction of $B$ and $B\to A$ are at most $n$, by induction, we have $\vdash \forall x B$ and $\vdash \forall x (B\to A)$.  Since $\forall x (B\to A) \to (\forall x B \to \forall x A)$ is an axiom, we have $\vdash \forall x B \to \forall x A$ by modus ponens, and then $\vdash \forall x A$ by modus ponens again.

\textbf{Remark}.  Another popular axiom system for first order logic has 1, 2, 3, 6 above as its axiom schemas, plus the following schema
\begin{itemize}
\item $\forall x (A\to B) \to (A \to \forall x B)$, where $x \in V$ is not free in $A$
\end{itemize}
and two rules of inferences: modus ponens, and \emph{universal generalization}:
\begin{itemize}
\item from $A$ we may infer $\forall x A$, where $x \in V$
\end{itemize}
Note the similarity between the rule of generalization and axiom schema 5 above.  However, the important difference is that $x$ is not permitted to be free in $A$ in the axiom schema.

With this change, the concept of deductions needs to be modified: we say that a wff $A$ is \emph{deducible} from a set $\Gamma$ of wff's, if there is a finite sequence $$A_1,\ldots, A_n$$
where $A=A_n$, such that for each $i$, $i=1,\ldots, n$
\begin{enumerate}
\item $A_i$ is either an axiom or in $\Gamma$
\item $A_i$ is obtained by an application of modus ponens
\item $A_i$ is obtained by an application of generalization: $A_i$ is $\forall x A_j$ for some $j<i$.
\end{enumerate}

%%%%%
%%%%%
\end{document}
