\documentclass[12pt]{article}
\usepackage{pmmeta}
\pmcanonicalname{FuzzyLogic}
\pmcreated{2013-03-22 16:33:32}
\pmmodified{2013-03-22 16:33:32}
\pmowner{ggerla}{15808}
\pmmodifier{ggerla}{15808}
\pmtitle{fuzzy logic}
\pmrecord{53}{38746}
\pmprivacy{1}
\pmauthor{ggerla}{15808}
\pmtype{Topic}
\pmcomment{trigger rebuild}
\pmclassification{msc}{03B52}
\pmclassification{msc}{03B15}
\pmclassification{msc}{03B10}
\pmsynonym{multi-valued logic}{FuzzyLogic}
%\pmkeywords{logic}
%\pmkeywords{vagueness}
%\pmkeywords{approximate reasoning}
\pmrelated{FuzzySubset}
\pmrelated{MinimalAndMaximalNumber}
\pmrelated{FuzzyLogicsOfLivingSystems}

% this is the default PlanetMath preamble.  as your knowledge
% of TeX increases, you will probably want to edit this, but
% it should be fine as is for beginners.

% almost certainly you want these
\usepackage{amssymb}
\usepackage{amsmath}
\usepackage{amsfonts}

% used for TeXing text within eps files
\usepackage{psfrag}
% need this for including graphics (\includegraphics)
\usepackage{graphicx}
% for neatly defining theorems and propositions
\usepackage{amsthm}
% making logically defined graphics
%%\usepackage{xypic}

% there are many more packages, add them here as you need them

% define commands here

\begin{document}
First order fuzzy logic is a new chapter of logic which originates from the notion of fuzzy subset proposed by L. A. Zadeh. From a semantical point of view, fuzzy logic is not different in nature from first-order multi-valued logic. Indeed in both the logics one refers to "worlds with graded properties". Instead, if we refer to the managment of the information on these worlds, and therefore to the deduction apparatus, fuzzy logic is a totally different and new topic. In fact it is based on the notion of approximate reasoning as suggested by Zadeh, Goguen, Pavelka and other authors. This means that if $F$ denotes the set of sentences of the considered first order language, the available information (system of proper axioms) is represented by a fuzzy subet $s : F \rightarrow [0,1]$ of formulas. Such a fuzzy subset gives constraints on the possible truth degree of the formulas. Namely it says that, for every formula $\alpha$, the truth degree of $\alpha$ is greater or equal to $s(\alpha)$. The managment of such an information is obtained by a deduction apparatus enabling us to define the fuzzy subset $D(s): F \rightarrow [0,1]$ of logical consequences of $s$. Again $D(s)$ is a constraint on the truth degree of the formulas but it is the best constraint we can obtain given $s$. We can define such an apparatus by fixing a suitable set of fuzzy inference rules and a suitable fuzzy subset of logical axioms. This gives a notion of proof $\pi$ and a way to calculate the  degree of validity $Valid(\pi,s)$ of $\pi$ given $s$. Then, $D(s)$ is obtained by setting

$D(s)(\alpha) = \sup\{Valid(\pi,s) : \pi \hbox{ is a proof of }\alpha \}.$

Precise definitions and completeness theorems can be found in [7] and [8]. Notice that the so defined notion of approximate reasoning enables us to give an interesting solution of the famous heap paradox (see [2] and[3]). 

An alternative and very important approach is obtained by introducing in the language propositional constants to denote truth values. In such a way it is possible to reduce the question of the deduction in fuzzy logic to the classical paradigm based on logical axioms and crisp inference rules (see the basic book of P. Hájek).

{\bf BIBLIOGRAPHY } 

1. Cignoli R., D Ottaviano I. M. L. and Mundici D.,{\em Algebraic Foundations of Many-Valued Reasoning}. Kluwer, Dordrecht, 1999.

2. Gerla G., {\em Fuzzy logic: Mathematical tools for approximate reasoning}, Kluwer Academic Publishers, Dordrecht,2001.

3. Goguen J., The logic of inexact concepts,  {\em Synthese}, vol. 19 (1968/69)

4. Gottwald S., {\em A treatise on many-valued logics}, Research Studies Press, Baldock 2000.

5. Gottwald, S., Mathematical fuzzy logic, {\em The Bulletin of Symbolic Logic}, vol. 14, 2008, pp. 210-239.

6. Hájek P., {\em Metamathematics of fuzzy logic}. Kluwer 1998.

7. Novak V., Perfilieva I., Mockor J., {\em Mathematical Principles of Fuzzy Logic}, Kluwer Academic Publ., 1999.

8. Pavelka J.,  On fuzzy logic I-III, {\em Zeitschrift f$\ddot{u}$r Mathemathische Logik
  und Grundlagen der Mathematik}, vol. 25 (1979), pp. 45--52; 119--134; 447--464.

9. Yager R. and Filev D., {\em Essentials of Fuzzy Modeling and Control} (1994), ISBN 0-471-01761-2 

10. Zimmermann H., {\em Fuzzy Set Theory and its Applications} (2001), ISBN 0-7923-7435-5.

11. Zadeh L.A., Fuzzy Sets, {\em Information and Control}, 8 (1965) 338­-353.

12. Zadeh L. A., The concept of a linguistic variable and its application to approximate reasoning I-III, {\em Information Sciences}, vol. 8, 9(1975), pp. 199-275, 301-357, 43-80.

%%%%%
%%%%%
\end{document}
