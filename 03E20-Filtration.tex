\documentclass[12pt]{article}
\usepackage{pmmeta}
\pmcanonicalname{Filtration}
\pmcreated{2013-03-22 12:08:38}
\pmmodified{2013-03-22 12:08:38}
\pmowner{CWoo}{3771}
\pmmodifier{CWoo}{3771}
\pmtitle{filtration}
\pmrecord{9}{31331}
\pmprivacy{1}
\pmauthor{CWoo}{3771}
\pmtype{Definition}
\pmcomment{trigger rebuild}
\pmclassification{msc}{03E20}
\pmrelated{FiltrationOfSigmaAlgebras}

\endmetadata

\usepackage{amssymb}
\usepackage{amsmath}
\usepackage{amsfonts}
\usepackage{graphicx}
%%%\usepackage{xypic}
\begin{document}
A {\em filtration} is a sequence of sets $A_1, A_2, \dots, A_n$ with
$$
A_1 \subset A_2 \subset \cdots \subset A_n.
$$
If one considers the sets $A_1, \dots, A_n$ as elements of a larger set which are partially ordered by inclusion, then a filtration is simply a finite chain with respect to this partial ordering. It should be noted that in some contexts the word ``filtration'' may also be employed to describe an infinite chain.
%%%%%
%%%%%
%%%%%
\end{document}
