\documentclass[12pt]{article}
\usepackage{pmmeta}
\pmcanonicalname{ExamplesOfContrapositive}
\pmcreated{2013-03-22 16:23:05}
\pmmodified{2013-03-22 16:23:05}
\pmowner{alozano}{2414}
\pmmodifier{alozano}{2414}
\pmtitle{examples of contrapositive}
\pmrecord{6}{38529}
\pmprivacy{1}
\pmauthor{alozano}{2414}
\pmtype{Example}
\pmcomment{trigger rebuild}
\pmclassification{msc}{03B05}
\pmrelated{ConverseTheorem}

\endmetadata

% this is the default PlanetMath preamble.  as your knowledge
% of TeX increases, you will probably want to edit this, but
% it should be fine as is for beginners.

% almost certainly you want these
\usepackage{amssymb}
\usepackage{amsmath}
\usepackage{amsthm}
\usepackage{amsfonts}

% used for TeXing text within eps files
%\usepackage{psfrag}
% need this for including graphics (\includegraphics)
%\usepackage{graphicx}
% for neatly defining theorems and propositions
%\usepackage{amsthm}
% making logically defined graphics
%%%\usepackage{xypic}

% there are many more packages, add them here as you need them

% define commands here

\newtheorem{thm}{Theorem}
\newtheorem{defn}{Definition}
\newtheorem{prop}{Proposition}
\newtheorem{lemma}{Lemma}
\newtheorem{cor}{Corollary}

\theoremstyle{definition}
\newtheorem{exa}{Example}

% Some sets
\newcommand{\Nats}{\mathbb{N}}
\newcommand{\Ints}{\mathbb{Z}}
\newcommand{\Reals}{\mathbb{R}}
\newcommand{\Complex}{\mathbb{C}}
\newcommand{\Rats}{\mathbb{Q}}
\newcommand{\Gal}{\operatorname{Gal}}
\newcommand{\Cl}{\operatorname{Cl}}
\begin{document}
Recall that the contrapositive of an implication $p \implies q$ is the equivalent implication $\neg q \implies \neg p$, which is read: ``not $q$ implies not $p$''. The following are examples of the contrapositive and converse of a logical statement:

\begin{enumerate}
\item Let $p$ be the statement ``it is raining'' and let $q$ be ``the ground is getting wet''. Then the statement ``if it is raining then the ground is getting wet'' is equivalent to ``if the ground is not getting wet then it is not raining''.  Notice that these are both true statements.  Notice also that the converse would be ``if the ground is getting wet then it is raining'' (which is not necessarily true!).

\item Let $f:S\to T$ be a function of sets and let $S$ be finite. The contrapositive statement of ``if $f$ is surjective then $T$ is finite'' (a true statement) would be the implication ``if $T$ is not finite then $f$ is not surjective'' (also a true statement). The converse would be ``if $T$ is finite then $f$ is surjective'' (a false statement).
\end{enumerate}
%%%%%
%%%%%
\end{document}
