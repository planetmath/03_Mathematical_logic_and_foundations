\documentclass[12pt]{article}
\usepackage{pmmeta}
\pmcanonicalname{Finite}
\pmcreated{2013-03-22 11:53:25}
\pmmodified{2013-03-22 11:53:25}
\pmowner{djao}{24}
\pmmodifier{djao}{24}
\pmtitle{finite}
\pmrecord{9}{30500}
\pmprivacy{1}
\pmauthor{djao}{24}
\pmtype{Definition}
\pmcomment{trigger rebuild}
\pmclassification{msc}{03E10}
\pmclassification{msc}{92C05}
\pmclassification{msc}{92B05}
\pmclassification{msc}{18-00}
\pmclassification{msc}{92C40}
\pmclassification{msc}{18-02}
\pmrelated{Infinite}
\pmdefines{finite set}

\endmetadata

\usepackage{amssymb}
\usepackage{amsmath}
\usepackage{amsfonts}
\usepackage{graphicx}
%%%%\usepackage{xypic}
\begin{document}
A set $S$ is \emph{finite} if there exists a natural number $n$ and a bijection from $S$ to $n$. Note that we are using the set theoretic definition of natural number, under which the natural number $n$ equals the set $\{0,1,2,\ldots,n-1\}$.  If there exists such an $n$, then it is unique, and we call $n$ the \emph{cardinality} of $S$.

Equivalently, a set $S$ is finite if and only if there is no bijection between $S$ and any proper subset of $S$.

%%%%%
%%%%%
%%%%%
%%%%%
\end{document}
