\documentclass[12pt]{article}
\usepackage{pmmeta}
\pmcanonicalname{Club}
\pmcreated{2013-03-22 12:53:01}
\pmmodified{2013-03-22 12:53:01}
\pmowner{Henry}{455}
\pmmodifier{Henry}{455}
\pmtitle{club}
\pmrecord{5}{33227}
\pmprivacy{1}
\pmauthor{Henry}{455}
\pmtype{Definition}
\pmcomment{trigger rebuild}
\pmclassification{msc}{03E10}
\pmdefines{club}
\pmdefines{closed}
\pmdefines{unbounded}
\pmdefines{closed unbounded}
\pmdefines{closed set}
\pmdefines{unbounded set}
\pmdefines{closed unbounded set}
\pmdefines{club set}

\endmetadata

% this is the default PlanetMath preamble.  as your knowledge
% of TeX increases, you will probably want to edit this, but
% it should be fine as is for beginners.

% almost certainly you want these
\usepackage{amssymb}
\usepackage{amsmath}
\usepackage{amsfonts}

% used for TeXing text within eps files
%\usepackage{psfrag}
% need this for including graphics (\includegraphics)
%\usepackage{graphicx}
% for neatly defining theorems and propositions
%\usepackage{amsthm}
% making logically defined graphics
%%%\usepackage{xypic}

% there are many more packages, add them here as you need them

% define commands here
%\PMlinkescapeword{theory}
\begin{document}
If $\kappa$ is a cardinal then a set $C\subseteq\kappa$ is \emph{closed} iff for any $S\subseteq C$ and $\alpha<\kappa$, $\sup(S\cap \alpha)=\alpha$ then $\alpha\in C$.  (That is, if the limit of some sequence in $C$ is less than $\kappa$ then the limit is also in $C$.)

If $\kappa$ is a cardinal and $C\subseteq\kappa$ then $C$ is \emph{unbounded} if, for any $\alpha<\kappa$, there is some $\beta\in C$ such that $\alpha<\beta$.

If a set is both closed and unbounded then it is a \emph{club} set.
%%%%%
%%%%%
\end{document}
