\documentclass[12pt]{article}
\usepackage{pmmeta}
\pmcanonicalname{TruthTable}
\pmcreated{2013-03-22 11:54:35}
\pmmodified{2013-03-22 11:54:35}
\pmowner{rspuzio}{6075}
\pmmodifier{rspuzio}{6075}
\pmtitle{truth table}
\pmrecord{16}{30532}
\pmprivacy{1}
\pmauthor{rspuzio}{6075}
\pmtype{Definition}
\pmcomment{trigger rebuild}
\pmclassification{msc}{03-00}
\pmclassification{msc}{34C29}
\pmrelated{ZerothOrderLogic}
\pmrelated{PropositionalCalculus}

\endmetadata

\usepackage{amssymb}
\usepackage{amsmath}
\usepackage{amsfonts}
\usepackage{graphicx}
%%%%\usepackage{xypic}
\begin{document}
A \emph{truth table} is a tabular listing of all possible input value combinations for a logical function and their corresponding output values.  Similarly, the truth table of a logical proposition is the truth table of the corresponding logical function. 

For instance, the truth table of the connective ``or'' is as follows:
\begin{center}
\begin{tabular}{ccc}
$a$ & $b$ & $a \lor b$ \\
\hline 
F & F & F \\
F & T & T \\
T & F & T \\
T & T & T 
\end{tabular}
\end{center}

For $n$ input variables, there will always be $2^n$ rows in the truth table.  
A sample truth table for ``$(a \land b) \rightarrow c$'' would be

\begin{center}
\begin{tabular}{cccc}
$a$ & $b$ & $c$ & $(a \land b) \rightarrow c$ \\
\hline 
F & F & F & T \\
F & F & T & F \\
F & T & F & T \\
F & T & T & F \\
T & F & F & T \\
T & F & T & F \\
T & T & F & T \\
T & T & T & T 
\end{tabular}
\end{center}

(Note that $\land$ represents logical and, while $\rightarrow$ represents the conditional truth function).

To compute truth tables of expressions, one often proceeds in steps.  for instance, 
to compute a truth table for ``$\neg p \lor (p \land q)$, one might proceed as follows:

\begin{center}
\begin{tabular}{ccccc}
$p$ & $q$ & $\neg p$ & $(p \land q)$ & $\neg p \lor (p \land q)$ \\
\hline 
F & F & T & F & T \\
F & T & T & F & T \\
T & F & F & F & F \\
T & T & F & T & T
\end{tabular}
\end{center}

For reference, here is a truth table of some popular connectives:

\begin{center}
\begin{tabular}{ccccccc}
$p$ & $q$ & $p \lor q$ & $p \land q$ & $p \veebar q$ & $p \rightarrow q$ & $p \leftrightarrow q$ \\
\hline 
F & F & F & F & F & T & T \\
F & T & T & F & T & T & F \\
T & F & T & F & T & F & F \\
T & T & T & T & F & T & T
\end{tabular}
\end{center}

For completeness, here are the remaining connectives, excluding trivial connectives which
depend on only one or none of their arguments:

\begin{center}
\begin{tabular}{ccccccccc}
$p$ & $q$ & $p \not\!\!\land q$ & $p \not\!\lor q$ & $p \leftarrow q$ & $p \not\rightarrow q$ & $p \not\!\leftarrow q$ \\
\hline 
F & F & T & T & T & F & F \\
F & T & T & F & F & F & T \\
T & F & T & F & T & T & F \\
T & T & F & F & T & F & F
\end{tabular}
\end{center}
%%%%%
%%%%%
%%%%%
%%%%%
\end{document}
