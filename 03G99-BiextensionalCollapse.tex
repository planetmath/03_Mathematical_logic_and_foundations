\documentclass[12pt]{article}
\usepackage{pmmeta}
\pmcanonicalname{BiextensionalCollapse}
\pmcreated{2013-03-22 13:04:54}
\pmmodified{2013-03-22 13:04:54}
\pmowner{Henry}{455}
\pmmodifier{Henry}{455}
\pmtitle{biextensional collapse}
\pmrecord{6}{33496}
\pmprivacy{1}
\pmauthor{Henry}{455}
\pmtype{Definition}
\pmcomment{trigger rebuild}
\pmclassification{msc}{03G99}
\pmdefines{equivalent Chu space}

\endmetadata

% this is the default PlanetMath preamble.  as your knowledge
% of TeX increases, you will probably want to edit this, but
% it should be fine as is for beginners.

% almost certainly you want these
\usepackage{amssymb}
\usepackage{amsmath}
\usepackage{amsfonts}

% used for TeXing text within eps files
%\usepackage{psfrag}
% need this for including graphics (\includegraphics)
%\usepackage{graphicx}
% for neatly defining theorems and propositions
%\usepackage{amsthm}
% making logically defined graphics
%%%\usepackage{xypic}

% there are many more packages, add them here as you need them

% define commands here
%\PMlinkescapeword{theory}
\begin{document}
If $\mathcal{C}=(\mathcal{A},r,\mathcal{X})$ is a Chu space, we can define the \emph{biextensional collapse} of $\mathcal{C}$ to be $(\hat{r}[A],r^\prime,\check{r}[X])$ where $r^\prime(\hat{r}(a),\check{r}(x))=r(a,x)$.

That is, to name the rows of the biextensional collapse, we just use functions representing the actual rows of the original Chu space (and similarly for the columns).  The effect is to merge indistinguishable rows and columns.

We say that two Chu spaces are \emph{equivalent} if their biextensional collapses are isomorphic.
%%%%%
%%%%%
\end{document}
