\documentclass[12pt]{article}
\usepackage{pmmeta}
\pmcanonicalname{A4BasicMetatheory}
\pmcreated{2013-11-09 3:42:57}
\pmmodified{2013-11-09 3:42:57}
\pmowner{PMBookProject}{1000683}
\pmmodifier{PMBookProject}{1000683}
\pmtitle{A.4 Basic metatheory}
\pmrecord{1}{}
\pmprivacy{1}
\pmauthor{PMBookProject}{1000683}
\pmtype{Feature}
\pmclassification{msc}{03B15}

\usepackage{xspace}
\usepackage{amssyb}
\usepackage{amsmath}
\usepackage{amsfonts}
\usepackage{amsthm}
\makeatletter
\newcommand{\conv}{\downarrow}
\newcommand{\defeq}{\vcentcolon\equiv}  
\newcommand{\define}[1]{\textbf{#1}}
\newcommand{\emptyt}{\ensuremath{\mathbf{0}}\xspace}
\newcommand{\indexdef}[1]{\index{#1|defstyle}}   
\newcommand{\jdeq}{\equiv}      
\def\lam#1{{\lambda}\@lamarg#1:\@endlamarg\@ifnextchar\bgroup{.\,\lam}{.\,}}
\def\@lamarg#1:#2\@endlamarg{\if\relax\detokenize{#2}\relax #1\else\@lamvar{\@lameatcolon#2},#1\@endlamvar\fi}
\def\@lameatcolon#1:{#1}
\def\@lamvar#1,#2\@endlamvar{(#2\,{:}\,#1)}
\newcommand{\mentalpause}{\medskip} 
\newcommand{\N}{\ensuremath{\mathbb{N}}\xspace}
\newcommand{\production}{\vcentcolon\vcentcolon=}
\newcommand{\suc}{\mathsf{succ}}
\newcommand{\UU}{\ensuremath{\mathcal{U}}\xspace}
\newcommand{\vcentcolon}{:\!\!}
\newcounter{mathcount}
\setcounter{mathcount}{1}
\newtheorem{precor}{Corollary}
\newenvironment{cor}{\begin{precor}}{\end{precor}\addtocounter{mathcount}{1}}
\renewcommand{\theprecor}{12.5.\arabic{mathcount}}
\newtheorem{prelem}{Lemma}
\newenvironment{lem}{\begin{prelem}}{\end{prelem}\addtocounter{mathcount}{1}}
\renewcommand{\theprelem}{12.5.\arabic{mathcount}}
\newtheorem{prethm}{Theorem}
\newenvironment{thm}{\begin{prethm}}{\end{prethm}\addtocounter{mathcount}{1}}
\renewcommand{\theprethm}{12.5.\arabic{mathcount}}
\let\autoref\cref
\makeatother

\begin{document}
This section discusses the meta-theoretic properties of the type theory presented in 
\autoref{sec:syntax-informally}, and similar results hold for \autoref{sec:syntax-more-formally}. Figuring out which of these still hold when we add the features from \autoref{sec:hott-features} quickly leads to open questions,\index{open!problem} as discussed at the end of this section.

Recall that \autoref{sec:syntax-informally} defines the terms of type theory as
an extension of the untyped $\lambda$-calculus. The $\lambda$-calculus 
has its own notion of computation, namely the computation rule\index{computation rule!for function types}: 
\[
  (\lam{x} t)(u) \defeq t[u/x]
\]
This rule, together with the defining equations for the defined constants form
\emph{rewriting rules}\index{rewriting rule}\index{rule!rewriting} that determine reduction steps for a rewriting 
system. These steps yield a notion of computation in the sense that each rule
has a natural direction: one simplifies $(\lam{x} t)(u)$ by evaluating the
function at its argument.

Moreover, this system is \emph{confluent}\index{confluence}, that is, if $a$ simplifies in some
number of steps to both $a'$ and $a''$, there is some $b$ to which both $a'$ and
$a''$ eventually simplify. Thus we can define $t\conv u$ to mean that $t$ and
$u$ simplify to the same term.

(The situation is similar in \autoref{sec:syntax-more-formally}: Although there
we presented the computation rules as undirected equalities $\jdeq$, we can give
an operational semantics by saying that the application of an eliminator to an
introductory form simplifies to its equal, not the other way around.)

It is straightforward to show that the system in \autoref{sec:syntax-informally}
has the following properties:

\begin{thm}\label{thm:conversion-preserves-typing}
If $A : \UU$ and $A \conv A'$ then $A' : \UU$.
If $t:A$ and $t \conv t'$ then $t':A$.
\end{thm}

We say that a term is \define{normalizable}
\indexdef{term!normalizable}%
\index{normalization}%
\indexdef{normalizable term}%
(respectively, \define{strongly
normalizable})
\indexdef{term!strongly normalizable}%
\index{normalization!strong}%
\index{strong!normalization}%
if some (respectively, every), sequence of rewriting steps from the term
terminates.

\begin{thm}\label{thm:strong-normalization}
If $A : \UU$ then $A$ is strongly normalizable.
If $t:A$ then $A$ and $t$ are strongly normalizable.
\end{thm}

We say that a term is in \define{normal form}
\index{normal form}%
\index{term!normal form of}%
if it cannot be further
simplified, and that a term is \define{closed}
\index{closed!term}%
\index{term!closed}%
if no variable occurs freely in
it. A closed normal type has to be a primitive type, i.e., of the form
$c(\vec{v})$ for some primitive constant $c$ (where the list $\vec{v}$ of closed
normal terms may be omitted if empty, for instance, as with $\N$). In fact, we
can explicitly describe all normal forms:

\begin{lem}\label{lem:normal-forms}
  The terms in normal form can be described by the following syntax:
  % 
  \begin{align*}
    v & \production  k \mid \lam{x} v \mid c(\vec{v}) \mid f(\vec{v}), \\
    k &\production x \mid k(v) \mid f(\vec{v})(k),
  \end{align*}
  % 
  where $f(\vec{v})$ represents a partial application of the defined function $f$.
  In particular, a type in normal form is of the form $k$ or $c(\vec{v})$.
\end{lem}

\begin{thm}
  If $A$ is in normal form then the 
  judgment $A : \UU$ is decidable. If $A : \UU$ and $t$ is in normal form then the judgment
  $t:A$ is decidable.
\end{thm}

Logical consistency\index{consistency} (of the system in \autoref{sec:syntax-informally}) follows
immediately: if we had $a:\emptyt$ in the empty context, then by
\autoref{thm:conversion-preserves-typing},\autoref{thm:strong-normalization}, $a$
simplifies to a normal term $a':\emptyt$. But by
\autoref{lem:normal-forms} no such term exists.

\begin{cor}
 The system in \autoref{sec:syntax-informally} is logically consistent.
\end{cor}

Similarly, we have the \emph{canonicity}\indexdef{canonicity} property that if $a:\N$ in the empty
context, then $a$ simplifies to a normal term $\suc^k(0)$ for some numeral $k$.

\begin{cor}
 The system in \autoref{sec:syntax-informally} has the canonicity property.
\end{cor}

Finally, if $a,A$ are in normal form, it is \emph{decidable} whether $a:A$; in
other words, because type-checking amounts to verifying the correctness of a
proof, this means we can always ``recognize a correct proof when we see one.''

\begin{cor}
The property of being a proof in the system in \autoref{sec:syntax-informally} is decidable.
\end{cor}

\mentalpause

The above results do not apply to the extended system of homotopy type
theory (i.e., the above system extended by \autoref{sec:hott-features}), since
occurrences of the univalence axiom and constructors of higher inductive types
never simplify, breaking \autoref{lem:normal-forms}. It is an open question\index{open!problem}
whether one can simplify applications of these constants in order to restore
canonicity. We also do not have a schema describing all permissible higher
inductive types, nor are we certain how to correctly formulate their rules
(e.g., whether the computation rules on higher constructors should be judgmental
equalities).

The consistency\index{consistency} of Martin-L\"{o}f type theory extended with univalence and higher
inductive types could be shown by inventing an appropriate normalization procedure, but currently
the only proofs that these systems are consistent are via semantic models---for
univalence, a model in Kan\index{Kan complex} complexes due to Voevodsky \cite{klv:ssetmodel}, and
for higher inductive types, a model due to Lumsdaine and Shulman \cite{ls:hits}.

Other metatheoretic issues, and a summary of our current results, are discussed
in greater length in the ``Constructivity'' and ``Open problems'' sections of
the introduction to this book.

\index{metatheory|)}%

\end{document}
