\documentclass[12pt]{article}
\usepackage{pmmeta}
\pmcanonicalname{84FiberSequencesAndTheLongExactSequence}
\pmcreated{2013-11-06 14:17:27}
\pmmodified{2013-11-06 14:17:27}
\pmowner{PMBookProject}{1000683}
\pmmodifier{rspuzio}{6075}
\pmtitle{8.4 Fiber sequences and the long exact sequence}
\pmrecord{1}{}
\pmprivacy{1}
\pmauthor{PMBookProject}{6075}
\pmtype{Feature}
\pmclassification{msc}{03B15}

\usepackage{xspace}
\usepackage{amssyb}
\usepackage{amsmath}
\usepackage{amsfonts}
\usepackage{amsthm}
\makeatletter
\newcommand{\blank}{\mathord{\hspace{1pt}\text{--}\hspace{1pt}}}
\newcommand{\ct}{  \mathchoice{\mathbin{\raisebox{0.5ex}{$\displaystyle\centerdot$}}}             {\mathbin{\raisebox{0.5ex}{$\centerdot$}}}             {\mathbin{\raisebox{0.25ex}{$\scriptstyle\,\centerdot\,$}}}             {\mathbin{\raisebox{0.1ex}{$\scriptscriptstyle\,\centerdot\,$}}}}
\newcommand{\defeq}{\vcentcolon\equiv}  
\newcommand{\define}[1]{\textbf{#1}}
\def\@dprd#1{\prod_{(#1)}\,}
\def\@dprd@noparens#1{\prod_{#1}\,}
\def\@dsm#1{\sum_{(#1)}\,}
\def\@dsm@noparens#1{\sum_{#1}\,}
\def\@eatprd\prd{\prd@parens}
\def\@eatsm\sm{\sm@parens}
\newcommand{\eqvsym}{\simeq}    
\def\exis#1{\exists (#1)\@ifnextchar\bgroup{.\,\exis}{.\,}}
\newcommand{\hfib}[2]{{\mathsf{fib}}_{#1}(#2)}
\newcommand{\im}{\ensuremath{\mathsf{im}}} 
\newcommand{\indexdef}[1]{\index{#1|defstyle}}   
\newcommand{\indexsee}[2]{\index{#1|see{#2}}}    
\newcommand{\jdeq}{\equiv}      
\def\lam#1{{\lambda}\@lamarg#1:\@endlamarg\@ifnextchar\bgroup{.\,\lam}{.\,}}
\def\@lamarg#1:#2\@endlamarg{\if\relax\detokenize{#2}\relax #1\else\@lamvar{\@lameatcolon#2},#1\@endlamvar\fi}
\def\@lameatcolon#1:{#1}
\def\@lamvar#1,#2\@endlamvar{(#2\,{:}\,#1)}
\newcommand{\map}[2]{\ensuremath{{#1}\mathopen{}\left({#2}\right)\mathclose{}}\xspace}
\newcommand{\opp}[1]{\mathord{{#1}^{-1}}}
\def\prd#1{\@ifnextchar\bgroup{\prd@parens{#1}}{\@ifnextchar\sm{\prd@parens{#1}\@eatsm}{\prd@noparens{#1}}}}
\def\prd@noparens#1{\mathchoice{\@dprd@noparens{#1}}{\@tprd{#1}}{\@tprd{#1}}{\@tprd{#1}}}
\def\prd@parens#1{\@ifnextchar\bgroup  {\mathchoice{\@dprd{#1}}{\@tprd{#1}}{\@tprd{#1}}{\@tprd{#1}}\prd@parens}  {\@ifnextchar\sm    {\mathchoice{\@dprd{#1}}{\@tprd{#1}}{\@tprd{#1}}{\@tprd{#1}}\@eatsm}    {\mathchoice{\@dprd{#1}}{\@tprd{#1}}{\@tprd{#1}}{\@tprd{#1}}}}}
\newcommand{\proj}[1]{\ensuremath{\mathsf{pr}_{#1}}\xspace}
\newcommand{\refl}[1]{\ensuremath{\mathsf{refl}_{#1}}\xspace}
\def\sm#1{\@ifnextchar\bgroup{\sm@parens{#1}}{\@ifnextchar\prd{\sm@parens{#1}\@eatprd}{\sm@noparens{#1}}}}
\def\sm@noparens#1{\mathchoice{\@dsm@noparens{#1}}{\@tsm{#1}}{\@tsm{#1}}{\@tsm{#1}}}
\def\sm@parens#1{\@ifnextchar\bgroup  {\mathchoice{\@dsm{#1}}{\@tsm{#1}}{\@tsm{#1}}{\@tsm{#1}}\sm@parens}  {\@ifnextchar\prd    {\mathchoice{\@dsm{#1}}{\@tsm{#1}}{\@tsm{#1}}{\@tsm{#1}}\@eatprd}    {\mathchoice{\@dsm{#1}}{\@tsm{#1}}{\@tsm{#1}}{\@tsm{#1}}}}}
\newcommand{\symlabel}[1]{\refstepcounter{symindex}\label{#1}}
\def\@tprd#1{\mathchoice{{\textstyle\prod_{(#1)}}}{\prod_{(#1)}}{\prod_{(#1)}}{\prod_{(#1)}}}
\newcommand{\tproj}[3][]{\mathopen{}\left|#3\right|_{#2}^{#1}\mathclose{}}
\newcommand{\trunc}[2]{\mathopen{}\left\Vert #2\right\Vert_{#1}\mathclose{}}
\newcommand{\truncf}[1]{\Vert \blank \Vert_{#1}}
\def\tsm#1{\@tsm{#1}\@ifnextchar\bgroup{\tsm}{}}
\def\@tsm#1{\mathchoice{{\textstyle\sum_{(#1)}}}{\sum_{(#1)}}{\sum_{(#1)}}{\sum_{(#1)}}}
\newcommand{\vcentcolon}{:\!\!}
\newcounter{mathcount}
\setcounter{mathcount}{1}
\newtheorem{precor}{Corollary}
\newenvironment{cor}{\begin{precor}}{\end{precor}\addtocounter{mathcount}{1}}
\renewcommand{\theprecor}{8.4.\arabic{mathcount}}
\newtheorem{predefn}{Definition}
\newenvironment{defn}{\begin{predefn}}{\end{predefn}\addtocounter{mathcount}{1}}
\renewcommand{\thepredefn}{8.4.\arabic{mathcount}}
\newtheorem{prelem}{Lemma}
\newenvironment{lem}{\begin{prelem}}{\end{prelem}\addtocounter{mathcount}{1}}
\renewcommand{\theprelem}{8.4.\arabic{mathcount}}
\newtheorem{prethm}{Theorem}
\newenvironment{thm}{\begin{prethm}}{\end{prethm}\addtocounter{mathcount}{1}}
\renewcommand{\theprethm}{8.4.\arabic{mathcount}}
\let\ap\map
\let\autoref\cref
\let\rev\opp
\let\setof\Set    
\makeatother

\begin{document}

\index{fiber sequence|(}%
\index{sequence!fiber|(}%

If the codomain of a function $f:X\to Y$ is equipped with a basepoint $y_0:Y$, then we refer to the fiber $F\defeq \hfib f {y_0}$ of $f$ over $y_0$ as \define{the fiber of $f$}\index{fiber}.
(If $Y$ is connected, then $F$ is determined up to mere equivalence; see \autoref{ex:unique-fiber}.)
We now show that if $X$ is also pointed and $f$ preserves basepoints, then there is a relation between the homotopy groups of $F$, $X$, and $Y$ in the form of a \emph{long exact sequence}.
We derive this by way of the \emph{fiber sequence} associated to such an $f$.

\begin{defn}\label{def:pointedmap}
  A \define{pointed map}
  \indexdef{pointed!map}%
  \indexsee{function!pointed}{pointed map}%
  between pointed types $(X,x_0)$ and $(Y,y_0)$ is a
  map $f:X\to Y$ together with a path $f_0:f(x_0)=y_0$.
\end{defn}

For any pointed types $(X,x_0)$ and $(Y,y_0)$, there is a pointed map $(\lam{x} y_0) : X\to Y$ which is constant at the basepoint.
We call this the \define{zero map}\indexdef{zero!map}\indexdef{function!zero} and sometimes write it as $0:X\to Y$.

Recall that every pointed type $(X,x_0)$ has a loop space\index{loop space} $\Omega (X,x_0)$.
We now note that this operation is functorial on pointed maps.\index{loop space!functoriality of}\index{functor!loop space}

\begin{defn}
  Given a pointed map between pointed types $f:X \to Y$, we define a pointed
  map $\Omega f:\Omega X
  \to \Omega Y$ by
  \[(\Omega f)(p) \defeq \rev{f_0}\ct\ap{f}{p}\ct f_0.\]
  The path $(\Omega f)_0 : (\Omega f) (\refl{x_0}) = \refl{y_0}$, which exhibits $\Omega f$ as a pointed map, is the obvious path of type
  \[\rev{f_0}\ct\ap{f}{\refl{x_0}}\ct f_0=\refl{y_0}.\]
\end{defn}

There is another functor on pointed maps, which takes $f:X\to Y$ to $\proj1 : \hfib f {y_0} \to X$.
When $f$ is pointed, we always consider $\hfib f {y_0}$ to be pointed with basepoint $(x_0,f_0)$, in which case $\proj1$ is also a pointed map, with witness $(\proj1)_0 \defeq \refl{x_0}$.
Thus, this operation can be iterated.

\begin{defn}
  The \define{fiber sequence}
  \indexdef{fiber sequence}%
  \indexdef{sequence!fiber}%
  of a pointed map $f:X\to Y$ is the infinite sequence of pointed types and pointed maps
  \[\xymatrix{\dots \ar[r]^{f^{(n+1)}} & X^{(n+1)} \ar[r]^{f^{(n)}} & X^{(n)} \ar^-{f^{(n-1)}}[r] & \dots \ar[r] & X^{(2)} \ar^-{f^{(1)}}[r] & X^{(1)} \ar[r]^{f^{(0)}} & X^{(0)}}\]
  defined recursively by
  \[ X^{(0)} \defeq Y\qquad
    X^{(1)} \defeq X\qquad
    f^{(0)} \defeq f\qquad
    \]    
  and
  \begin{alignat*}{2}
    X^{(n+1)} &\defeq \hfib {f^{(n-1)}}{x^{(n-1)}_0}\\
    f^{(n)} &\defeq \proj1 &: X^{(n+1)} \to X^{(n)}.
  \end{alignat*}
  where $x^{(n)}_0$ denotes the basepoint of $X^{(n)}$, chosen recursively as above.
\end{defn}

Thus, any adjacent pair of maps in this fiber sequence is of the form
\[ \xymatrix{ X^{(n+1)} \jdeq \hfib{f^{(n-1)}}{x^{(n-1)}_0} \ar[rr]^-{f^{(n)}\jdeq \proj1} && X^{(n)} \ar[r]^{f^{(n-1)}} & X^{(n-1)}. } \]
In particular, we have $f^{(n-1)} \circ f^{(n)} = 0$.
We now observe that the types occurring in this sequence are the iterated loop spaces\index{loop space!iterated} of the base
space $Y$, the total space\index{total!space} $X$, and the fiber $F\defeq \hfib f {y_0}$, and similarly for the maps.

\begin{lem}\label{thm:fiber-of-the-fiber}
  Let $f:X\to Y$ be a pointed map of pointed spaces.  Then:
  \begin{enumerate}
  \item The fiber of $f^{(1)}\defeq \proj1 : \hfib f {y_0} \to X$ is equivalent to $\Omega Y$.\label{item:fibseq1}
  \item Similarly, the fiber of $f^{(2)} : \Omega Y \to \hfib f {y_0}$ is equivalent to $\Omega X$.\label{item:fibseq2}
  \item Under these equivalences, the map $f^{(3)} : \Omega X\to \Omega Y$ is identified with $\Omega f\circ\rev{(\blank)}$.\label{item:fibseq3}
  \end{enumerate}
\end{lem}
\begin{proof}
  For~\ref{item:fibseq1}, we have
  \begin{align*}
    \hfib{f^{(1)}}{x_0}
    &\defeq \sm{z:\hfib{f}{y_0}} (\proj{1}(z) = x_0)\\
    &\eqvsym \sm{x:A}{p:f(x)=y_0} (x = x_0) &\text{(by \autoref{ex:sigma-assoc})}\\
    &\eqvsym (f(x_0) = y_0) &\text{(as $\tsm{x:A} (x=x_0)$ is contractible)}\\
    &\eqvsym (y_0 = y_0) &\text{(by $(f_0 \ct \blank)$)}\\
    &\jdeq \Omega Y.
  \end{align*}
  Tracing through, we see that this equivalence sends $((x,p),q)$ to $\opp{f_0} \ct \ap{f}{\opp q} \ct p$, while its inverse sends $r:y_0=y_0$ to $((x_0, f_0 \ct r), \refl{x_0})$.
  In particular, the basepoint $((x_0,f_0),\refl{x_0})$ of $\hfib{f^{(1)}}{x_0}$ is sent to $\opp{f_0} \ct \ap{f}{\opp {\refl{x_0}}} \ct f_0$, which equals $\refl{y_0}$.
  Hence this equivalence is a pointed map (see \autoref{ex:pointed-equivalences}).
  Moreover, under this equivalence, $f^{(2)}$ is identified with $\lam{r} (x_0, f_0 \ct r): \Omega Y \to \hfib f {y_0}$.

  \cref{item:fibseq2} follows immediately by applying~\ref{item:fibseq1} to $f^{(1)}$ in place of $f$.
  % The resulting equivalence sends $(((x,p),p'),q') : \hfib{f^{(2)}}{(x_0,f_0)}$ to $\ap{f}{\opp {q'}} \ct p'$, while its inverse sends $s:x_0=x_0$ to $(((x_0,f_0), s), \refl{(x_0,f_0)})$.
  Since $(f^{(1)})_0 \defeq \refl{x_0}$, under this equivalence $f^{(3)}$ is identified with the map $\Omega X \to \hfib {f^{(1)}}{x_0}$ defined by $s \mapsto ((x_0,f_0),s)$.
  Thus, when we compose with the previous equivalence $\hfib {f^{(1)}}{x_0} \eqvsym \Omega Y$, we see that $s$ maps to $\opp{f_0} \ct \ap{f}{\opp s} \ct f_0$, which is by definition $(\Omega f)(\opp s)$, giving~\ref{item:fibseq3}.
\end{proof}

Thus, the fiber sequence of $f:X\to Y$ can be pictured as:
\[\xymatrix@C=1.5pc{
  \dots \ar[r] &
  \Omega^2 X \ar[r]^-{\Omega^2 f} &
  \Omega^2 Y \ar[r]^-{-\Omega \partial} &
  \Omega F \ar[r]^-{-\Omega i} &
  \Omega X \ar[r]^-{-\Omega f} &
  \Omega Y \ar[r]^-{\partial} &
  F \ar[r]^-{i} &
  X \ar[r]^{f} & Y.}\]
where the minus signs denote composition with path inversion $\rev{(\blank)}$.
Note that by \autoref{ex:ap-path-inversion}, we have
\[ \Omega\left(\Omega f\circ \opp{(\blank)}\right) \circ \opp{(\blank)}
= \Omega^2 f \circ \opp{(\blank)} \circ \opp{(\blank)}
= \Omega^2 f.
\]
Thus, there are minus signs on the $k$-fold loop maps whenever $k$ is odd.

From this fiber sequence we will deduce an \emph{exact sequence of pointed sets}.
\index{image}%
Let $A$ and $B$ be sets and $f:A\to B$ a function, and recall from \autoref{defn:modal-image} the definition of the \emph{image} $\im(f)$, which can be regarded as a subset of $B$:
\[\im(f) \defeq \setof{b:B | \exis{a:A} f(a)=b}. \]
If $A$ and $B$ are moreover pointed with basepoints $a_0$ and $b_0$, and $f$ is a pointed map, we define the \define{kernel}
\indexdef{kernel}%
\indexdef{pointed!map!kernel of}%
of $f$ to be the following subset of $A$:
\symlabel{kernel}
\[\ker(f) \defeq \setof{x:A | f(x) = b_0}. \]
Of course, this is just the fiber of $f$ over the basepoint $b_0$; it a subset of $A$ because $B$ is a set.

Note that any group is a pointed set, with its unit element as basepoint, and any group homomorphism is a pointed map.
In this case, the kernel and image agree with the usual notions from group theory.

\begin{defn}
  An \define{exact sequence of pointed sets}
  \indexdef{exact sequence}%
  \indexdef{sequence!exact}%
  is a (possibly bounded) sequence of pointed sets and pointed maps:
  \[\xymatrix{\dots \ar[r] & A^{(n+1)} \ar[r]^-{f^{(n)}} & A^{(n)} \ar[r]^{f^{(n-1)}} & A^{(n-1)} \ar[r] &
    \dots}\]
  such that for every $n$, the image of $f^{(n)}$ is equal, as a subset of $A^{(n)}$, to the kernel of $f^{(n-1)}$.
  In other words, for all $a:A^{(n)}$ we have
  \[ (f^{(n-1)}(a) = a^{(n-1)}_0) \iff \exis{b:A^{(n+1)}} (f^{(n)}(b)=a). \]
  where $a^{(n)}_0$ denotes the basepoint of $A^{(n)}$.
\end{defn}

Usually, most or all of the pointed sets in an exact sequence are groups, and often abelian groups.
When we speak of an \define{exact sequence of groups}, it is assumed moreover that the maps are group homomorphisms and not just pointed maps.

\index{exact sequence}
\begin{thm}\label{thm:les}
  Let $f:X \to Y$ be a pointed map between pointed spaces with fiber $F\defeq \hfib f {y_0}$.
  Then we have the following long exact sequence, which consists of groups except for the last three terms, and abelian groups except for the last six.

  \[
  \xymatrix@R=1.2pc@C=3pc{
    \vdots & \vdots & \vdots \ar[lld] \\
    \pi_k(F) \ar[r] & \pi_k(X) \ar[r] & \pi_k(Y) \ar[lld] \\
    \vdots & \vdots & \vdots \ar[lld] \\
    \pi_2(F) \ar[r] & \pi_2(X) \ar[r] & \pi_2(Y) \ar[lld] \\
    \pi_1(F) \ar[r] & \pi_1(X) \ar[r] & \pi_1(Y) \ar[lld]\\
    \pi_0(F) \ar[r] & \pi_0(X) \ar[r] & \pi_0(Y)}
  \]
\end{thm}

% In US trade we might want a page break here, and extra stretch, otherwise the
% whole page looks ugly.
\vspace*{0pt plus 10ex}
\goodbreak

\begin{proof}
  We begin by showing that the 0-truncation of a fiber sequence is an exact sequence of pointed sets.
  Thus, we need to show that for any adjacent pair of maps in a fiber sequence:
  %
  \[\xymatrix{\hfib{f}{z_0} \ar^-g[r] & W \ar^-f[r] & Z}\]
  %
  with $g\defeq \proj1$, the sequence
  %
  \[\xymatrix{\trunc{0}{\hfib{f}{z_0}} \ar^-{\trunc0g}[r] & \trunc0{W}
    \ar^-{\trunc0{f}}[r] & \trunc0{Z}}\]
  %
  is exact, i.e.\ that $\im(\trunc0g)\subseteq\ker(\trunc0f)$ and $\ker(\trunc0f)\subseteq\im(\trunc0g)$.

  The first inclusion is equivalent to $\trunc0g\circ\trunc0f=0$, which holds by functoriality of $\truncf0$ and the fact that $g\circ f=0$.
  For the second, we assume $w':\trunc0W$ and $p':\trunc0f(w')=\tproj0{z_0}$ and show there merely exists ${t:\hfib{f}{z_0}}$ such that $g(t)=w'$.
  Since our goal is a mere proposition, we can assume that $w'$ is of the
  form $\tproj0w$ for some $w:W$.
  Now by \autoref{thm:path-truncation}, $p' : \tproj0{f(w)}=\tproj0{z_0}$ yields $p'': \trunc{-1}{f(w)=z_0}$, so by a further truncation induction we may assume some $p:f(w)=z_0$.
  But now we have $\tproj0{(w,p)}:\tproj0{\hfib{f}{z_0}}$ whose image under $\trunc0g$ is $\tproj0w\jdeq w'$, as desired.

  Thus, applying $\truncf0$ to the fiber sequence of $f$, we obtain a long exact sequence involving the pointed sets $\pi_k(F)$, $\pi_k(X)$, and $\pi_k(Y)$ in the desired order.
  And of course, $\pi_k$ is a group for $k\ge1$, being the 0-truncation of a loop space, and an abelian group for $k\ge 2$ by the Eckmann--Hilton argument
  \index{Eckmann--Hilton argument} (\autoref{thm:EckmannHilton}).
  Moreover, \autoref{thm:fiber-of-the-fiber} allows us to identify the maps $\pi_k(F) \to \pi_k(X)$ and $\pi_k(X) \to \pi_k(Y)$ in this exact sequence as $(-1)^k \pi_k(i)$ and $(-1)^k \pi_k(f)$ respectively.

  More generally, every map in this long exact sequence except the last three is of the form $\trunc0{\Omega h}$ or $\trunc0{-\Omega h}$ for some $h$.
  In the former case it is a group homomorphism, while in the latter case it is a homomorphism if the groups are abelian; otherwise it is an ``anti-homomorphism''.
  However, the kernel and image of a group homomorphism are unchanged when we replace it by its negative, and hence so is the exactness of any sequence involving it.
  Thus, we can modify our long exact sequence to obtain one involving $\pi_k(i)$ and $\pi_k(f)$ directly and in which all the maps are group homomorphisms (except the last three).
\end{proof}

The usual properties of exact sequences of abelian groups\index{group!abelian!exact sequence of} can be proved as
usual. In particular we have:
\begin{lem}\label{thm:ses}
  Suppose given an exact sequence of abelian groups:
  %
  \[\xymatrix{K \ar[r]& G \ar^f[r] & H \ar[r] & Q.}\]
  %
  \begin{enumerate}
  \item If $K=0$, then $f$ is injective.\label{item:sesinj}
  \item If $Q=0$, then $f$ is surjective.\label{item:sessurj}
  \item If $K=Q=0$, then $f$ is an isomorphism.\label{item:sesiso}
  \end{enumerate}
\end{lem}
\begin{proof}
  Since the kernel of $f$ is the image of $K\to G$, if $K=0$ then the kernel of $f$ is $\{0\}$;
  hence $f$ is injective because it's a group morphism.
  Similarly, since the image of $f$ is the kernel of $H\to Q$, if $Q=0$ then the image of $f$ is all of $H$, so $f$ is surjective.
  Finally,~\ref{item:sesiso} follows from~\ref{item:sesinj} and~\ref{item:sessurj} by \autoref{thm:mono-surj-equiv}.
\end{proof}

As an immediate application, we can now quantify in what way $n$-connectedness of a map is stronger than inducing an equivalence on $n$-truncations.

\begin{cor}\label{thm:conn-pik}
  Let $f:A\to B$ be $n$-connected and $a:A$, and define $b\defeq f(a)$.  Then:
  \begin{enumerate}
  \item If $k\le n$, then $\pi_k(f):\pi_k(A,a) \to \pi_k(B,b)$ is an isomorphism.
  \item If $k=n+1$, then $\pi_k(f):\pi_k(A,a) \to \pi_k(B,b)$ is surjective.
  \end{enumerate}
\end{cor}
\begin{proof}
  As part of the long exact sequence, for each $k$ we have an exact sequence
  \[\xymatrix{\pi_k(\hfib f b) \ar[r]& \pi_k(A,a) \ar^f[r] & \pi_k(B,b) \ar[r] & \pi_{k-1}(\hfib f b).}\]
  Now since $f$ is $n$-connected, $\trunc n{\hfib f b}$ is contractible.
  Therefore, if $k\le n$, then $\pi_k(\hfib f b) = \trunc0{\Omega^k(\hfib f b)} = \Omega^k(\trunc k{\hfib f b})$ is also contractible.
  Thus, $\pi_k(f)$ is an isomorphism for $k\le n$ by \autoref{thm:ses}\ref{item:sesiso}, while for $k=n+1$ it is surjective by \autoref{thm:ses}\ref{item:sessurj}.
\end{proof}

In \autoref{sec:whitehead} we will see that the converse of \autoref{thm:conn-pik} also holds.

\index{fiber sequence|)}%
\index{sequence!fiber|)}%


\end{document}
