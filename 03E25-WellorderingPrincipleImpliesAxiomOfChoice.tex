\documentclass[12pt]{article}
\usepackage{pmmeta}
\pmcanonicalname{WellorderingPrincipleImpliesAxiomOfChoice}
\pmcreated{2013-03-22 16:07:46}
\pmmodified{2013-03-22 16:07:46}
\pmowner{Wkbj79}{1863}
\pmmodifier{Wkbj79}{1863}
\pmtitle{well-ordering principle implies axiom of choice}
\pmrecord{7}{38200}
\pmprivacy{1}
\pmauthor{Wkbj79}{1863}
\pmtype{Theorem}
\pmcomment{trigger rebuild}
\pmclassification{msc}{03E25}
\pmrelated{AxiomOfChoice}
\pmrelated{ZermelosWellOrderingTheorem}

\usepackage{amssymb}
\usepackage{amsmath}
\usepackage{amsfonts}

\usepackage{psfrag}
\usepackage{graphicx}
\usepackage{amsthm}
%%\usepackage{xypic}

\newtheorem*{thm*}{Theorem}

\begin{document}
\begin{thm*}
The well-ordering principle implies the axiom of choice.
\end{thm*}

\begin{proof}
Let $C$ be a collection of nonempty sets.  Then $\displaystyle \bigcup_{S \in C} S$ is a set.  By the well-ordering principle, $\displaystyle \bigcup_{S \in C} S$ is well-ordered under some relation $<$.  Since each $S$ is a nonempty subset of $\displaystyle \bigcup_{S \in C} S$, each $S$ has a least member $m_S$ with respect to the relation $<$.

Define $\displaystyle f \colon C \to \bigcup_{S \in C} S$ by $f(S)=m_S$.  Then $f$ is a choice function.  Hence, the axiom of choice holds.
\end{proof}
%%%%%
%%%%%
\end{document}
