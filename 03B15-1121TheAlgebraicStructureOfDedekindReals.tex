\documentclass[12pt]{article}
\usepackage{pmmeta}
\pmcanonicalname{1121TheAlgebraicStructureOfDedekindReals}
\pmcreated{2013-11-06 17:47:07}
\pmmodified{2013-11-06 17:47:07}
\pmowner{PMBookProject}{1000683}
\pmmodifier{PMBookProject}{1000683}
\pmtitle{11.2.1 The algebraic structure of Dedekind reals}
\pmrecord{1}{}
\pmprivacy{1}
\pmauthor{PMBookProject}{1000683}
\pmtype{Feature}
\pmclassification{msc}{03B15}

\usepackage{xspace}
\usepackage{amssyb}
\usepackage{amsmath}
\usepackage{amsfonts}
\usepackage{amsthm}
\makeatletter
\newcommand{\apart}{\mathrel{\#}}  
\newcommand{\defeq}{\vcentcolon\equiv}  
\newcommand{\define}[1]{\textbf{#1}}
\def\exis#1{\exists (#1)\@ifnextchar\bgroup{.\,\exis}{.\,}}
\def\fall#1{\forall (#1)\@ifnextchar\bgroup{.\,\fall}{.\,}}
\newcommand{\indexdef}[1]{\index{#1|defstyle}}   
\newcommand{\indexsee}[2]{\index{#1|see{#2}}}    
\newcommand{\jdeq}{\equiv}      
\newcommand{\Q}{\ensuremath{\mathbb{Q}}\xspace}
\newcommand{\RD}{\ensuremath{\mathbb{R}_\mathsf{d}}\xspace} 
\newcommand{\symlabel}[1]{\refstepcounter{symindex}\label{#1}}
\newcommand{\vcentcolon}{:\!\!}
\newcounter{mathcount}
\setcounter{mathcount}{1}
\newtheorem{predefn}{Definition}
\newenvironment{defn}{\begin{predefn}}{\end{predefn}\addtocounter{mathcount}{1}}
\renewcommand{\thepredefn}{11.2.\arabic{mathcount}}
\newenvironment{myeqn}{\begin{equation}}{\end{equation}\addtocounter{mathcount}{1}}
\renewcommand{\theequation}{11.2.\arabic{mathcount}}
\newtheorem{prelem}{Lemma}
\newenvironment{lem}{\begin{prelem}}{\end{prelem}\addtocounter{mathcount}{1}}
\renewcommand{\theprelem}{11.2.\arabic{mathcount}}
\newtheorem{prermk}{Remark}
\newenvironment{rmk}{\begin{prermk}}{\end{prermk}\addtocounter{mathcount}{1}}
\renewcommand{\theprermk}{11.2.\arabic{mathcount}}
\newtheorem{prethm}{Theorem}
\newenvironment{thm}{\begin{prethm}}{\end{prethm}\addtocounter{mathcount}{1}}
\renewcommand{\theprethm}{11.2.\arabic{mathcount}}
\let\autoref\cref
\makeatother

\begin{document}

The construction of the algebraic and order-theoretic structure of Dedekind reals proceeds
as usual in intuitionistic logic. Rather than dwelling on details we point out the
differences between the classical\index{mathematics!classical} and intuitionistic setup. Writing $L_x$ and $U_x$ for
the lower and upper cut of a real number $x : \RD$, we define addition as%
%
\indexdef{addition!of Dedekind reals}%
\begin{align*}
  L_{x + y}(q) &\defeq \exis{r, s : \Q} L_x(r) \land L_y(s) \land q = r + s, \\
  U_{x + y}(q) &\defeq \exis{r, s : \Q} U_x(r) \land U_y(s) \land q = r + s,
\end{align*}
%
and the additive inverse by
%
\begin{align*}
  L_{-x}(q) &\defeq \exis{r : \Q} U_x(r) \land q = - r, \\
  U_{-x}(q) &\defeq \exis{r : \Q} L_x(r) \land q = - r.
\end{align*}
%
With these operations $(\RD, 0, {+}, {-})$ is an abelian\index{group!abelian} group. Multiplication is a bit
more cumbersome:
%
\indexdef{multiplication!of Dedekind reals}%
\begin{align*}
  L_{x \cdot y}(q) &\defeq
  \begin{aligned}[t]
    \exis{a, b, c, d : \Q} & L_x(a) \land U_x(b) \land L_y(c) \land U_y(d) \land {}\\
                           & \qquad q < \min (a \cdot c, a \cdot d, b \cdot c, b \cdot d),
  \end{aligned} \\
  U_{x \cdot y}(q) &\defeq
  \begin{aligned}[t]
    \exis{a, b, c, d : \Q} & L_x(a) \land U_x(b) \land L_y(c) \land U_y(d) \land {}\\
                           & \qquad \max (a \cdot c, a \cdot d, b \cdot c, b \cdot d) < q.
  \end{aligned}
\end{align*}
%
\index{interval!arithmetic}%
These formulas are related to multiplication of intervals in interval arithmetic, where
intervals $[a,b]$ and $[c,d]$ with rational endpoints multiply to the interval
%
\begin{equation*}
  [a,b] \cdot [c,d] =
  [\min(a c, a d, b c, b d), \max(a c, a d, b c, b d)].
\end{equation*}
%
For instance, the formula for the lower cut can be read as saying that $q < x \cdot y$
when there are intervals $[a,b]$ and $[c,d]$ containing $x$ and $y$, respectively, such
that $q$ is to the left of $[a,b] \cdot [c,d]$. It is generally useful to think of an
interval $[a,b]$ such that $L_x(a)$ and $U_x(b)$ as an approximation of~$x$, see
\autoref{ex:RD-interval-arithmetic}.

We now have a commutative ring\index{ring} with unit
\index{unit!of a ring}%
$(\RD, 0, 1, {+}, {-}, {\cdot})$. To treat
multiplicative inverses, we must first introduce order. Define $\leq$ and $<$ as
%
\begin{align*}
  (x \leq y) &\ \defeq \ \fall{q : \Q} L_x(q) \Rightarrow L_y(q), \\
  (x < y)    &\ \defeq \ \exis{q : \Q} U_x(q) \land L_y(q).
\end{align*}

\begin{lem} \label{dedekind-in-cut-as-le}
  For all $x : \RD$ and $q : \Q$, $L_x(q) \Leftrightarrow (q < x)$ and $U_x(q)
  \Leftrightarrow (x < q)$.
\end{lem}

\begin{proof}
  If $L_x(q)$ then by roundedness there merely is $r > q$ such that $L_x(r)$, and since
  $U_q(r)$ it follows that $q < x$. Conversely, if $q < x$ then there is $r : \Q$ such
  that $U_q(r)$ and $L_x(r)$, hence $L_x(q)$ because $L_x$ is a lower set. The other half
  of the proof is symmetric.
\end{proof}

\index{partial order}%
\index{transitivity!of . for reals@of $<$ for reals}
\index{transitivity!of . for reals@of $\leq$ for reals}
\index{relation!irreflexive}
\index{irreflexivity!of . for reals@of $<$ for reals}
The relation $\leq$ is a partial order, and $<$ is transitive and irreflexive. Linearity
\index{order!linear}%
\index{linear order}%
%
\begin{equation*}
  (x < y) \lor (y \leq x)
\end{equation*}
%
is valid if we assume excluded middle, but without it we get weak linearity
%
\index{order!weakly linear}
\index{weakly linear order}
\begin{myeqn} \label{eq:RD-linear-order}
  (x < y) \Rightarrow (x < z) \lor (z < y).
\end{myeqn}
%
At first sight it might not be clear what~\eqref{eq:RD-linear-order} has to do with
linear order. But if we take $x \jdeq u - \epsilon$ and $y \jdeq u + \epsilon$ for
$\epsilon > 0$, then we get
%
\begin{equation*}
  (u - \epsilon < z) \lor (z < u + \epsilon).
\end{equation*}
%
This is linearity ``up to a small numerical error'', i.e., since it is unreasonable to
expect that we can actually compute with infinite precision, we should not be surprised
that we can decide~$<$ only up to whatever finite precision we have computed.

To see that~\eqref{eq:RD-linear-order} holds, suppose $x < y$. Then there merely exists $q : \Q$ such that $U_x(q)$ and
$L_y(q)$. By roundedness there merely exist $r, s : \Q$ such that $r < q < s$, $U_x(r)$
and $L_y(s)$. Then, by locatedness $L_z(r)$ or $U_z(s)$. In the first case we get $x < z$
and in the second $z < y$. 

Classically, multiplicative inverses exist for all numbers which are different from zero.
However, without excluded middle, a stronger condition is required. Say that $x, y : \RD$
are \define{apart}
\indexdef{apartness}%
from each other, written $x \apart y$, when $(x < y) \lor (y < x)$:
%
\symlabel{apart}
\begin{equation*}
  (x \apart y) \defeq (x < y) \lor (y < x).
\end{equation*}
%
If $x \apart y$, then $\lnot (x = y)$.
The converse is true if we assume excluded middle, but is not provable constructively.
\index{mathematics!constructive}%
Indeed, if $\lnot (x = y)$ implies $x\apart y$, then a little bit of excluded middle follows; see \autoref{ex:reals-apart-neq-MP}.

\begin{thm} \label{RD-inverse-apart-0}
  A real is invertible if, and only if, it is apart from $0$.
\end{thm}

\begin{rmk}
  We observe that a real is invertible if, and only if, it is merely
  invertible.  Indeed, the same is true in any ring,\index{ring} since a ring is a set, and
  multiplicative inverses are unique if they exist.  See the discussion
  following \autoref{cor:UC}.
\end{rmk}

\begin{proof}
  Suppose $x \cdot y = 1$. Then there merely exist $a, b, c, d : \Q$ such that
  $a < x < b$, $c < y < d$ and $0 < \min (a c, a d, b c, b d)$. From $0 < a c$ and $0 < b c$ it follows
  that $a$, $b$, and $c$ are either all positive or all negative.
  Hence either $0 < a < x$ or $x < b < 0$, so that $x \apart 0$.

  Conversely, if $x \apart 0$ then
  %
  \begin{align*}
    L_{x^{-1}}(q) &\defeq
    \exis{r : \Q} U_x(r) \land ((0 < r \land q r < 1) \lor (r < 0 \land 1 < q r))
    \\
    U_{x^{-1}}(q) &\defeq
    \exis{r : \Q} L_x(r) \land ((0 < r \land q r > 1) \lor (r < 0 \land 1 > q r))
  \end{align*}
  %
  defines the desired inverse. Indeed, $L_{x^{-1}}$ and $U_{x^{-1}}$ are inhabited because
  $x \apart 0$.
\end{proof}

\index{ordered field!archimedean}%
\index{dense}%
\indexsee{order-dense}{dense}%
The archimedean principle can be stated in several ways. We find it most illuminating in the
form which says that $\Q$ is dense in $\RD$.

\begin{thm}[Archimedean principle for $\RD$] \label{RD-archimedean}
  %
  For all $x, y : \RD$ if $x < y$ then there merely exists $q : \Q$ such that
  $x < q < y$.
\end{thm}

\begin{proof}
  By definition of $<$.
\end{proof}

Before tackling completeness of Dedekind reals, let us state precisely what algebraic
structure they possess. In the following definition we are not aiming at a minimal
axiomatization, but rather at a useful amount of structure and properties.

\begin{defn} \label{ordered-field} An \define{ordered field}
  \indexdef{ordered field}%
  \indexsee{field!ordered}{ordered field}%
  is a set $F$ together with
  constants $0$, $1$, operations $+$, $-$, $\cdot$, $\min$, $\max$, and mere relations
  $\leq$, $<$, $\apart$ such that:
  %
  \begin{enumerate}
  \item $(F, 0, 1, {+}, {-}, {\cdot})$ is a commutative ring with unit;
    \index{unit!of a ring}%
    \index{ring}%
  \item $x : F$ is invertible if, and only if, $x \apart 0$;
  \item $(F, {\leq}, {\min}, {\max})$ is a lattice;
  \item the strict order $<$ is transitive, irreflexive,
    \index{relation!irreflexive}
    \index{irreflexivity!of . in a field@of $<$ in a field}%
    and weakly linear ($x < y \Rightarrow x < z \lor z < y$);\index{transitivity!of . in a field@of $<$ in a field}
    \index{order!weakly linear}
    \index{weakly linear order}
    \index{strict!order}%
    \index{order!strict}%
  \item apartness $\apart$ is irreflexive, symmetric and cotransitive ($x \apart y \Rightarrow x \apart z \lor y \apart z$);
    \index{relation!irreflexive}
    \index{irreflexivity!of apartness}%
    \indexdef{relation!cotransitive}%
    \index{cotransitivity of apartness}%
  \item for all $x, y, z : F$:
    %
    \begin{align*}
      x \leq y &\Leftrightarrow \lnot (y < x), &
      x < y \leq z &\Rightarrow x < z, \\
      x \apart y &\Leftrightarrow (x < y) \lor (y < x), &
      x \leq y < z &\Rightarrow x < z, \\
      x \leq y &\Leftrightarrow x + z \leq y + z, &
      x \leq y \land 0 \leq z &\Rightarrow x z \leq y z, \\
      x < y &\Leftrightarrow x + z < y + z, &
      0 < z \Rightarrow (x < y &\Leftrightarrow x z < y z), \\
      0 < x + y &\Rightarrow 0 < x \lor 0 < y, &
      0 &< 1.
    \end{align*}
  \end{enumerate}
  %
  Every such field has a canonical embedding $\Q \to F$. An ordered field is
  \define{archimedean}
  \indexdef{ordered field!archimedean}%
  \indexsee{archimedean property}{ordered field, archi\-mede\-an}%
  when for all $x, y : F$, if $x < y$ then there merely exists $q :
  \Q$ such that $x < q < y$.
\end{defn}

\begin{thm} \label{RD-archimedean-ordered-field}
  The Dedekind reals form an ordered archimedean field.
\end{thm}

\begin{proof}
  We omit the proof in the hope that what we have demonstrated so far makes the theorem
  plausible.
\end{proof}


\end{document}
