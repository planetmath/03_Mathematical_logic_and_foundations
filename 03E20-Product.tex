\documentclass[12pt]{article}
\usepackage{pmmeta}
\pmcanonicalname{Product}
\pmcreated{2013-03-22 15:45:22}
\pmmodified{2013-03-22 15:45:22}
\pmowner{pahio}{2872}
\pmmodifier{pahio}{2872}
\pmtitle{product}
\pmrecord{10}{37710}
\pmprivacy{1}
\pmauthor{pahio}{2872}
\pmtype{Definition}
\pmcomment{trigger rebuild}
\pmclassification{msc}{03E20}
\pmrelated{Multiplication}
\pmrelated{RuleOfProduct}
\pmrelated{FrobeniusProduct}
\pmdefines{factor}
\pmdefines{factor of a product}

\endmetadata

% this is the default PlanetMath preamble.  as your knowledge
% of TeX increases, you will probably want to edit this, but
% it should be fine as is for beginners.

% almost certainly you want these
\usepackage{amssymb}
\usepackage{amsmath}
\usepackage{amsfonts}

% used for TeXing text within eps files
%\usepackage{psfrag}
% need this for including graphics (\includegraphics)
%\usepackage{graphicx}
% for neatly defining theorems and propositions
 \usepackage{amsthm}
% making logically defined graphics
%%%\usepackage{xypic}

% there are many more packages, add them here as you need them

% define commands here

\theoremstyle{definition}
\newtheorem*{thmplain}{Theorem}
\begin{document}
\PMlinkescapeword{word}

The word {\em product}\, in mathematics generally means the result of some \PMlinkescapetext{type} of {\em multiplication} operation, \PMlinkname{i.e.}{Ie} of certain \PMlinkescapetext{types} of mapping\, $X\!\times\!X \rightarrow Y$; such operations are commonly distributive over the {\em addition} operation of $X$ if it is defined.

If $x_1$ and $x_2$ are two elements of the set $X$, giving the product\, $y\in Y$, then $x_1$ and $x_2$ are in general called the {\em factors} of this product.\\

Some most usual products are
\begin{itemize}
\item the \PMlinkname{ring product}{Ring} (also in fields), especially the product of numbers and the product of square matrices;
\item on vectors the scalar product, the vector product, the dyad product and the Hadamard product; on ideals the product of ideals;
\item the Cartesian product, the direct products of various systems (not in \PMlinkescapetext{connection} with any additions).\\
\end{itemize}

Such kinds of product that are associative, allow to form a product of more than two factors, which is justified by the theorem in the entry general associativity.\, E.g. the usual product of the integers from 1 to $n$ is the factorial of $n$.


%%%%%
%%%%%
\end{document}
