\documentclass[12pt]{article}
\usepackage{pmmeta}
\pmcanonicalname{ModelsConstructedFromConstants}
\pmcreated{2013-03-22 12:44:42}
\pmmodified{2013-03-22 12:44:42}
\pmowner{ratboy}{4018}
\pmmodifier{ratboy}{4018}
\pmtitle{models constructed from constants}
\pmrecord{17}{33049}
\pmprivacy{1}
\pmauthor{ratboy}{4018}
\pmtype{Definition}
\pmcomment{trigger rebuild}
\pmclassification{msc}{03C07}
\pmsynonym{completeness theorem}{ModelsConstructedFromConstants}
\pmsynonym{G\"odel completeness theorem}{ModelsConstructedFromConstants}
%\pmkeywords{completeness theorem}
%\pmkeywords{compactness theorem}
%\pmkeywords{model}
\pmrelated{UpwardsSkolemLowenheimTheorem}
\pmdefines{set of witnesses}

\usepackage{amssymb}
\usepackage{amsmath}
\usepackage{amsfonts}
%\usepackage{amsthm}

\newcommand{\br}{[\![}
\newcommand{\rb}{]\!]}
\newcommand{\oq}{\text{``}}
\newcommand{\cq}{\text{''}}


\newcommand{\im}{\mathbf{Im}}
\newcommand{\dom}{\mathbf{Dom}}


\newcommand{\Or}{\vee}
\newcommand{\Implies}{\Rightarrow}
\newcommand{\Iff}{\Leftrightarrow}
\newcommand{\proves}{\vdash}
\renewcommand{\And}{\wedge}
\newcommand{\Sup}{\bigwedge}
\newcommand{\Inf}{\bigvee}
\newcommand{\Z}{\mathbb{Z}}
\newcommand{\F}{\mathbb{F}}
\newcommand{\Q}{\mathbb{Q}}
\newcommand{\R}{\mathbb{R}}
\newcommand{\C}{\mathbb{C}}
\newcommand{\Nat}{\mathbb{N}}
\newcommand{\M}{\mathfrak{M}}
\newcommand{\N}{\mathfrak{N}}
\newcommand{\A}{\mathfrak{A}}
\newcommand{\B}{\mathfrak{B}}
\newcommand{\K}{\mathfrak{K}}
\newcommand{\G}{\mathbb{G}}
\newcommand{\Def}{\overset{\operatorname{def}}{:=}}



\newcommand{\spec}{\text{{\bf Spec}}}
\newcommand{\stab}{\text{{\bf Stab}}}
\newcommand{\ann}{\text{{\bf Ann}}}
\newcommand{\irr}{\text{{\bf Irr}}}
\newcommand{\qt}{\text{{\bf Qt}}}
\newcommand{\st}{\mathcal{Qt}}
\newcommand{\ro}{\mathbf{r.o.}}


\newcommand{\Endo}{\text{{\bf End}}}
\newcommand{\mat}{\text{{\bf Mat}}}
\newcommand{\der}{\text{{\bf Der}}}
\newcommand{\rad}{\text{{\bf Rad}}}
\newcommand{\trd}{\text{{\bf tr.d.}}}
\newcommand{\cl}{\text{{\bf acl}}}
\newcommand{\Int}{\text{{\bf int}}}
\newcommand{\V}{\mathbb{V}}
\newcommand{\D}{\mathbf{D}}

\newcommand{\del}{\partial}
\renewcommand{\O}{\mathcal{O}}
\newcommand{\aut}{\mathbf{Aut}}
\newcommand{\height}{\text{\bf Height}}
\newcommand{\coheight}{\text{\bf Co-height}}

\newcommand{\lcm}{\operatorname{lcm}}

\newcommand{\Gal}{\operatorname{Gal}}
\newcommand{\x}{\mathbf{x}}
\newcommand{\y}{\mathbf{y}}
\newcommand{\inner}[2]{\langle #1|#2\rangle}
\renewcommand{\r}{{r}}
\renewcommand{\t}{{t}}

\newcommand{\restr}{\upharpoonright}
\newcommand{\Matrix}[4]{\left(\begin{array}{cc} #1 & #2 \\ #3 & #4 
\end{array}\right)}
\begin{document}
\PMlinkescapeword{fixed}
\PMlinkescapeword{lemma}
\PMlinkescapeword{simple}
\PMlinkescapeword{witnesses}

The definition of a structure and of the satisfaction relation is
nice, but it raises the following question : how do we get models in
the first place?  The most basic construction for models of
first-order theory is the construction that uses constants. Throughout
this entry, $L$ is a fixed first-order language.

Let $C$ be a set of constant symbols of $L$, and $T$ be a theory in
$L$.  Then we say $C$ is a {\em set of witnesses} for $T$ if and only
if for every formula $\varphi$ with at most one free variable $x$, 
we have $T\proves\exists x(\varphi)\Implies\varphi(c)$ for some $c\in C$.

{\bf Lemma.}
Let $T$ is any consistent 
set of sentences of $L$, and $C$ is a set of new symbols
such that $|C|=|L|$.  Let $L'=L\cup C$.  Then there is a consistent
set $T'\subseteq L'$ extending $T$ and which has $C$ as set of
witnesses.

{\bf Lemma.}
If $T$ is a consistent theory in $L$, and $C$ is a set of witnesses
for $T$ in $L$, then $T$ has a model whose elements are the constants
in $C$.

{\bf Proof:}
Let $\Sigma$ be the signature for $L$.  If $T$ is a consistent set of
sentences of $L$, then there is a maximal consistent $T'\supseteq T$.
Note that $T'$ and $T$ have the same sets of witnesses.  As every
model of $T'$ is also a model of $T$, we may assume $T$ is maximal
consistent.

We let the universe of $\M$ be the set of equivalence classes
$C/\sim$, where $a\sim b$ if and only if $\oq a=b\cq\in T$.  As $T$ is
maximal consistent, this is an equivalence relation.  We interpret the
non-logical symbols as follows :
\begin{enumerate}
\item $[a]=^\M[b]$ if and only if $a\sim b$;
\item Constant symbols are interpreted in the obvious way, i.e. if
  $c\in \Sigma$ is a constant symbol, then $c^\M=[c]$;
\item If $R\in\Sigma$ is an $n$-ary relation symbol, then
  $([a_1],...,[a_n])\in R^\M$ if and only if $R(a_1,...,a_n)\in T$;
\item If $F\in\Sigma$ is an $n$-any function symbol, then
  $F^\M([a_0],...,[a_n])=[b]$ if and only if $\oq
  F(a_1,...,a_n)=b\cq\in T$.
\end{enumerate}
From the fact that $T$ is maximal consistent, and $\sim$ is an
equivalence relation, we get that the operations are well-defined (it
is not so simple, i'll write it out later).
The proof that $\M\models T$ is a straightforward induction on the
complexity of the formulas of $T$.
\hfill$\diamondsuit$

{\bf Corollary.}
(The extended completeness theorem) A set $T$ of formulas of $L$ is
consistent if and only if it has a model (regardless of whether or not
$L$ has witnesses for $T$).

{\bf Proof:}
First add a set $C$ of new constants to $L$, and expand $T$ to $T'$ in
such a way that $C$ is a set of witnesses for $T'$.  Then expand $T'$
to a maximal consistent set $T''$.  This set has a model $\M$ consisting of
the constants in $C$, and $\M$ is also a model of $T$.
\hfill$\diamondsuit$

{\bf Corollary.}
(Compactness theorem) A set $T$ of sentences of $L$ has a model if
and only if every finite subset of $T$ has a model.

{\bf Proof:}
Replace ``has a model'' by ``is consistent'', and apply the syntactic
compactness theorem.
\hfill$\diamondsuit$

{\bf Corollary.}
(G\"odel's completeness theorem)
Let $T$ be a consistent set of formulas of $L$.  Then
A sentence $\varphi$ is a theorem of $T$ if and only if it is true in
every model of $T$.

{\bf Proof:}
If $\varphi$ is not a theorem of $T$, then $\neg\varphi$ is consistent
with $T$, so $T\cup\{\neg\varphi\}$ has a model $\M$, in which
$\varphi$ cannot be true.
\hfill$\diamondsuit$

{\bf Corollary.}
(Downward L\"owenheim-Skolem theorem)  If $T\subseteq L$ has a model,
then it has a model of power at most $|L|$.

{\bf Proof:}
If $T$ has a model, then it is consistent.  The model constructed from
constants has power at most $|L|$ (because we must add at most $|L|$
many new constants).
\hfill$\diamondsuit$

Most of the treatment found in this entry can be read in more details
in Chang and Keisler's book {\em Model Theory}.
%%%%%
%%%%%
\end{document}
