\documentclass[12pt]{article}
\usepackage{pmmeta}
\pmcanonicalname{Argument}
\pmcreated{2013-03-22 16:07:06}
\pmmodified{2013-03-22 16:07:06}
\pmowner{Wkbj79}{1863}
\pmmodifier{Wkbj79}{1863}
\pmtitle{argument}
\pmrecord{8}{38185}
\pmprivacy{1}
\pmauthor{Wkbj79}{1863}
\pmtype{Definition}
\pmcomment{trigger rebuild}
\pmclassification{msc}{03E20}
\pmclassification{msc}{97D70}

\endmetadata

% this is the default PlanetMath preamble.  as your knowledge
% of TeX increases, you will probably want to edit this, but
% it should be fine as is for beginners.

% almost certainly you want these
\usepackage{amssymb}
\usepackage{amsmath}
\usepackage{amsfonts}

% used for TeXing text within eps files
%\usepackage{psfrag}
% need this for including graphics (\includegraphics)
%\usepackage{graphicx}
% for neatly defining theorems and propositions
%\usepackage{amsthm}
% making logically defined graphics
%%%\usepackage{xypic}

% there are many more packages, add them here as you need them

% define commands here

\begin{document}
The {\sl argument\/} of a function is its input.  For example, in the expression $f(x)$, $x$ is the argument of $f$.

A common error for those who are unfamiliar with mathematics is to treat a function and its argument as two separate entities.  For example, in solving the equation $\ln x=5$ for $x$, people who are unfamiliar with mathematics may give the erroneous answer $\displaystyle x=\frac{5}{\ln}$.  This error might be circumvented by stressing that a function and its argument are not multiplied, but rather that a function \PMlinkescapetext{acts on} its argument.

Another common error is to try to separate the argument of a function.  This error is most common when the argument consists of at least two \PMlinkescapetext{terms}.  For example, students may write $f(x+5)=f(x)+f(5)$ regardless of what the function $f$ is.
%%%%%
%%%%%
\end{document}
