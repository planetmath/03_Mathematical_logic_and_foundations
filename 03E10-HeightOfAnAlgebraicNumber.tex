\documentclass[12pt]{article}
\usepackage{pmmeta}
\pmcanonicalname{HeightOfAnAlgebraicNumber}
\pmcreated{2013-03-22 13:24:34}
\pmmodified{2013-03-22 13:24:34}
\pmowner{kidburla2003}{1480}
\pmmodifier{kidburla2003}{1480}
\pmtitle{height of an algebraic number}
\pmrecord{17}{33954}
\pmprivacy{1}
\pmauthor{kidburla2003}{1480}
\pmtype{Definition}
\pmcomment{trigger rebuild}
\pmclassification{msc}{03E10}
\pmsynonym{height}{HeightOfAnAlgebraicNumber}
\pmrelated{AlgebraicNumbersAreCountable}

% this is the default PlanetMath preamble.  as your knowledge
% of TeX increases, you will probably want to edit this, but
% it should be fine as is for beginners.

% almost certainly you want these
\usepackage{amssymb}
\usepackage{amsmath}
\usepackage{amsfonts}

% used for TeXing text within eps files
%\usepackage{psfrag}
% need this for including graphics (\includegraphics)
%\usepackage{graphicx}
% for neatly defining theorems and propositions
%\usepackage{amsthm}
% making logically defined graphics
%%%\usepackage{xypic}

% there are many more packages, add them here as you need them

% define commands here
\begin{document}
Suppose we have an algebraic number such that the polynomial of smallest degree it is a root of (with the co-efficients relatively prime) is given by:

$$
\sum_{i=0}^n a_i x^i .
$$

Then the height $h$ of the algebraic number is given by:

$$
h = n + \sum_{i=0}^n |a_i| .
$$

This is a quantity which is used in the proof of the existence of transcendental numbers.

\begin{thebibliography}{99}
\bibitem{shaw} Shaw, R. Mathematics Society Notes, 1st edition. King's School Chester, 2003.
\bibitem{stewart} Stewart, I. Galois Theory, 3rd edition. Chapman and Hall, 2003.
\bibitem{baker} Baker, A. Transcendental Number Theory, 1st edition. Cambridge University Press, 1975.
\end{thebibliography}
%%%%%
%%%%%
\end{document}
