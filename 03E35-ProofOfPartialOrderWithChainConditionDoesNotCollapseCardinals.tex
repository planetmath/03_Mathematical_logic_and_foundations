\documentclass[12pt]{article}
\usepackage{pmmeta}
\pmcanonicalname{ProofOfPartialOrderWithChainConditionDoesNotCollapseCardinals}
\pmcreated{2013-03-22 12:53:43}
\pmmodified{2013-03-22 12:53:43}
\pmowner{Henry}{455}
\pmmodifier{Henry}{455}
\pmtitle{proof of partial order with chain condition does not collapse cardinals}
\pmrecord{4}{33243}
\pmprivacy{1}
\pmauthor{Henry}{455}
\pmtype{Proof}
\pmcomment{trigger rebuild}
\pmclassification{msc}{03E35}

\endmetadata

% this is the default PlanetMath preamble.  as your knowledge
% of TeX increases, you will probably want to edit this, but
% it should be fine as is for beginners.

% almost certainly you want these
\usepackage{amssymb}
\usepackage{amsmath}
\usepackage{amsfonts}

% used for TeXing text within eps files
%\usepackage{psfrag}
% need this for including graphics (\includegraphics)
%\usepackage{graphicx}
% for neatly defining theorems and propositions
%\usepackage{amsthm}
% making logically defined graphics
%%%\usepackage{xypic}

% there are many more packages, add them here as you need them

% define commands here
%\PMlinkescapeword{theory}
\begin{document}
\emph{Outline:}

Given any function $f$ purporting to violate the theorem by being surjective (or cofinal) on $\lambda$, we show that there are fewer than $\kappa$ possible values of $f(\alpha)$, and therefore only $\max(\alpha,\kappa)$ possible elements in the entire range of $f$, so $f$ is not surjective (or cofinal).

\emph{Details:}

Suppose $\lambda>\kappa$ is a cardinal of $\mathfrak{M}$ that is not a cardinal in $\mathfrak{M}[G]$.  

There is some function $f\in\mathfrak{M}[G]$ and some cardinal $\alpha<\lambda$ such that $f:\alpha\rightarrow\lambda$ is surjective.  This has a name, $\hat{f}$.  For each $\beta<\alpha$, consider 
$$F_\beta=\{\gamma<\lambda\mid p\Vdash \hat{f}(\beta)=\gamma\}\text{ for some }p\in P$$

$|F_\beta|<\kappa$, since any two $p\in P$ which force different values for $\hat{f}(\beta)$ are incompatible and $P$ has no sets of incompatible elements of size $\kappa$.

Notice that $F_\beta$ is definable in $\mathfrak{M}$.  Then the range of $f$ must be contained in $F=\bigcup_{i<\alpha} F_i$.  But $|F|\leq\alpha\cdot\kappa=\max(\alpha,\kappa)<\lambda$.  So $f$ cannot possibly be surjective, and therefore $\lambda$ is not collapsed.

Now suppose that for some $\alpha\geq\lambda>\kappa$, $\operatorname{cf}(\alpha)=\lambda$ in $\mathfrak{M}$ and for some $\eta<\lambda$ there is a cofinal function $f:\eta\rightarrow\alpha$.

We can construct $F_\beta$ as above, and again the range of $f$ is contained in $F=\bigcup_{i<\eta} F_i$.  But then $|\operatorname{range}(f)|\leq|F|\leq\eta\cdot\kappa<\lambda$.  So there is some $\gamma<\alpha$ such that $f(\beta)<\gamma$ for any $\beta<\eta$, and therefore $f$ is not cofinal in $\alpha$.
%%%%%
%%%%%
\end{document}
