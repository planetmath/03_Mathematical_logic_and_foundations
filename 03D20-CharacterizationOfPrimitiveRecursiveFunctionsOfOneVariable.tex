\documentclass[12pt]{article}
\usepackage{pmmeta}
\pmcanonicalname{CharacterizationOfPrimitiveRecursiveFunctionsOfOneVariable}
\pmcreated{2013-03-22 16:45:25}
\pmmodified{2013-03-22 16:45:25}
\pmowner{rspuzio}{6075}
\pmmodifier{rspuzio}{6075}
\pmtitle{characterization of primitive recursive functions of one variable}
\pmrecord{9}{38983}
\pmprivacy{1}
\pmauthor{rspuzio}{6075}
\pmtype{Theorem}
\pmcomment{trigger rebuild}
\pmclassification{msc}{03D20}

\endmetadata

% this is the default PlanetMath preamble.  as your knowledge
% of TeX increases, you will probably want to edit this, but
% it should be fine as is for beginners.

% almost certainly you want these
\usepackage{amssymb}
\usepackage{amsmath}
\usepackage{amsfonts}

% used for TeXing text within eps files
%\usepackage{psfrag}
% need this for including graphics (\includegraphics)
%\usepackage{graphicx}
% for neatly defining theorems and propositions
\usepackage{amsthm}
% making logically defined graphics
%%%\usepackage{xypic}

% there are many more packages, add them here as you need them

% define commands here
\newtheorem{theorem}{Theorem}
\newtheorem{definition}{Definition}
\begin{document}
It is possible to characterize primitive recursive functions of one
variable in terms of operations involving only functions of a single
variable.  To describe how this goes, it is useful to first define
some notation.

\begin{definition}
Define the constant function $K \colon 
\mathbb{N} \to \mathbb{N}$ by $K(n) = 1$ for all $n$.  
\end{definition}

\begin{definition}
Define the identity function $I \colon 
\mathbb{N} \to \mathbb{N}$ by $I(n) = n$ for all $n$.  
\end{definition}

\begin{definition}
Define the excess over square function 
$E \colon \mathbb{N} \to \mathbb{N}$ by 
$E(n) = n - m^2$, where $m$ is the largest integer such that
$m^2 \le n$. 
\end{definition}

\begin{definition}
Given a function $f \colon \mathbb{N} \to \mathbb{N}$, define
the function $R(f) \colon \mathbb{N} \to \mathbb{N}$ by the
following conditions:
\begin{itemize}
\item $R(f)(0) = 0$
\item $R(f)(n+1) = f(R(f)(n))$ for all integers $n \ge 0$.
\end{itemize}
\end{definition}

\begin{theorem}
The class of primitive recursive functions of a single variable is
the smallest class $X$ of functions which contains the functions 
$E$ and $K$ defined above and is closed under the following
three operations:
\begin{enumerate}
\item If $f \in X$ and $g \in X$, then $f \circ g \in X$.
\item If $f \in X$, then $f + I \in X$.
\footnote{Here $f + I$ has the usual meaning of pointwise addition ---
$(f + I)(x) = f(x) + I(x)$}
\item If $f \in X$, then $R(f) \in X$.
\end{enumerate}
\end{theorem}
%%%%%
%%%%%
\end{document}
