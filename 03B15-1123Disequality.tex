\documentclass[12pt]{article}
\usepackage{pmmeta}
\pmcanonicalname{1123Disequality}
\pmcreated{2013-11-13 21:16:07}
\pmmodified{2013-11-13 21:16:07}
\pmowner{PMBookProject}{1000683}
\pmmodifier{rspuzio}{6075}
\pmtitle{1.12.3 Disequality}
\pmrecord{3}{87630}
\pmprivacy{1}
\pmauthor{PMBookProject}{6075}
\pmtype{Feature}
\pmclassification{msc}{03B15}

\endmetadata

\usepackage{xspace}
\usepackage{amssyb}
\usepackage{amsmath}
\usepackage{amsfonts}
\usepackage{amsthm}
\newcommand{\defeq}{\vcentcolon\equiv}  
\newcommand{\define}[1]{\textbf{#1}}
\newcommand{\id}[3][]{\ensuremath{#2 =_{#1} #3}\xspace}
\newcommand{\indexdef}[1]{\index{#1|defstyle}}   
\newcommand{\jdeq}{\equiv}      
\newcommand{\vcentcolon}{:\!\!}
\let\autoref\cref
\begin{document}
Finally, let us also say something about \define{disequality},
\indexdef{disequality}%
which is negation of equality:%
\footnote{We use ``inequality''
  to refer to $<$ and $\leq$. Also, note that this is negation of the \emph{propositional} identity type.
Of course, it makes no sense to negate judgmental equality $\jdeq$, because judgments are not subject to logical operations.}
%
\begin{equation*}
  (x \neq_A y) \ \defeq\ \lnot (\id[A]{x}{y}).
\end{equation*}
If $x\neq y$, we say that $x$ and $y$ are \define{unequal}
\indexdef{unequal}%
or \define{not equal}.
%
Just like negation, disequality plays a less important role here than it does in classical\index{mathematics!classical}
mathematics. For example, we cannot prove that two things are equal by proving that they
are not unequal: that would be an application of the classical law of double negation, see \PMlinkname{\S 3.4}{34classicalvsintuitionisticlogic}.


Sometimes it is useful to phrase disequality in a positive way. For example,
in~\PMlinkname{Theorem 11.2.4}{1122dedekindrealsarecauchycomplete#Thmprethm1} we shall prove that a real number $x$ has an inverse if,
and only if, its distance from~$0$ is positive, which is a stronger requirement than $x
\neq 0$.

\index{type!identity|)}%

\end{document}
