\documentclass[12pt]{article}
\usepackage{pmmeta}
\pmcanonicalname{RingOfSets}
\pmcreated{2013-03-22 15:47:46}
\pmmodified{2013-03-22 15:47:46}
\pmowner{CWoo}{3771}
\pmmodifier{CWoo}{3771}
\pmtitle{ring of sets}
\pmrecord{18}{37757}
\pmprivacy{1}
\pmauthor{CWoo}{3771}
\pmtype{Definition}
\pmcomment{trigger rebuild}
\pmclassification{msc}{03E20}
\pmclassification{msc}{28A05}
\pmsynonym{lattice of sets}{RingOfSets}
\pmsynonym{algebra of sets}{RingOfSets}
\pmrelated{SigmaAlgebra}
\pmrelated{AbsorbingElement}
\pmrelated{RepresentingADistributiveLatticeByRingOfSets}
\pmrelated{RepresentingABooleanLatticeByFieldOfSets}
\pmdefines{field of sets}

\usepackage{amssymb,amscd}
\usepackage{amsmath}
\usepackage{amsfonts}

% used for TeXing text within eps files
%\usepackage{psfrag}
% need this for including graphics (\includegraphics)
%\usepackage{graphicx}
% for neatly defining theorems and propositions
%\usepackage{amsthm}
% making logically defined graphics
%%%\usepackage{xypic}

% define commands here
\begin{document}
\subsection*{Ring of Sets}
Let $S$ be a set and $2^S$ be the power set of $S$.  A subset $\mathcal{R}$ of $2^S$ is said to be a \emph{ring of sets} of $S$ if it is a lattice under the intersection and union operations.  In other words, $\mathcal{R}$ is a ring of sets if 
\begin{itemize}
\item for any $A,B\in \mathcal{R}$, then $A\cap B\in \mathcal{R}$,
\item for any $A,B\in \mathcal{R}$, then $A\cup B\in \mathcal{R}$.
\end{itemize}

A ring of sets is a distributive lattice.  The word ``ring'' in the name has nothing to do with the ordinary ring found in algebra.  Rather, it is an abelian semigroup with respect to each of the binary set operations.  If $S\in\mathcal{R}$, then $(\mathcal{R},\cap,S)$ becomes an abelian monoid.  Similarly, if $\varnothing\in\mathcal{R}$, then $(\mathcal{R},\cup,\varnothing)$ is an abelian monoid.  If both $S,\varnothing\in\mathcal{R}$, then $(\mathcal{R},\cup,\cap)$ is a commutative semiring, since $\varnothing\cap A=A\cap\varnothing=\varnothing$, and $\cap$ distributes over $\cup$.  Dualizing, we see that $(\mathcal{R},\cap,\cup)$ is also a commutative semiring.  It is perhaps with this connection that the name ``ring of sets'' is so chosen.

Since $S$ is not required to be in $\mathcal{R}$, a ring of sets can in theory be the empty set.  Even if $\mathcal{R}$ may be non-empty, it may be a singleton.  Both cases are not very interesting to study.  To avoid such examples, some authors, particularly measure theorists, define a ring of sets to be a non-empty set with the first condition above replaced by
\begin{itemize}
\item for any $A,B\in \mathcal{R}$, then $A-B\in \mathcal{R}$.
\end{itemize}
This is indeed a stronger condition, as $A\cap B=A-(A-B)\in \mathcal{R}$.  However, we shall stick with the more general definition here.

\subsection*{Field of Sets}
An even stronger condition is to insist that not only is $\mathcal{R}$ non-empty, but that $S\in\mathcal{R}$.  Such a ring of sets is called a field, or algebra of sets.  Formally, given a set $S$, a \emph{field of sets} $\mathcal{F}$ of $S$ satisfies the following criteria
\begin{itemize}
\item $\mathcal{F}$ is a ring of sets of $S$,
\item $S\in\mathcal{F}$, and
\item if $A\in\mathcal{F}$, then the complement $\overline{A}\in\mathcal{F}$.
\end{itemize}
The three conditions above are equivalent to the following three conditions:
\begin{itemize}
\item $\varnothing\in\mathcal{F}$,
\item if $A,B\in \mathcal{F}$, then $A\cup B\in \mathcal{F}$, and
\item if $A\in\mathcal{F}$, then $\overline{A}\in\mathcal{F}$.
\end{itemize}

A field of sets is also known as an \emph{algebra of sets}.

It is easy to see that $\mathcal{F}$ is a distributive complemented lattice, and hence a Boolean lattice.  From the discussion earlier, we also see that $\mathcal{F}$ (of $S$) is a commutative semiring, with $S$ acting as the multiplicative identity and $\varnothing$ both the additive identity and the multiplicative absorbing element.

\textbf{Remark}.  Two remarkable theorems relating to \PMlinkescapetext{representations} of certain lattices as rings or fields of sets are the following: 
\begin{enumerate}
\item a lattice is distributive iff it is \PMlinkname{lattice isomorphic}{LatticeIsomorphism} to a ring of sets (G. Birkhoff and M. Stone); 
\item a lattice is \PMlinkname{Boolean}{BooleanLattice} iff it is lattice \PMlinkescapetext{isomorphic} to a field of sets (M. Stone).
\end{enumerate}

\begin{thebibliography}{7}
\bibitem{prh} P. R. Halmos: {\em Lectures on Boolean Algebras}, Springer-Verlag (1970).
\bibitem{prh1} P. R. Halmos: {\em Measure Theory}, Springer-Verlag (1974).
\bibitem{gg} G. Gr\"atzer: {\em General Lattice Theory}, Birkh\"auser, (1998).
\end{thebibliography}
%%%%%
%%%%%
\end{document}
