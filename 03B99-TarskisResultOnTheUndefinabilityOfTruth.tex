\documentclass[12pt]{article}
\usepackage{pmmeta}
\pmcanonicalname{TarskisResultOnTheUndefinabilityOfTruth}
\pmcreated{2013-03-22 13:49:19}
\pmmodified{2013-03-22 13:49:19}
\pmowner{mathcam}{2727}
\pmmodifier{mathcam}{2727}
\pmtitle{Tarski's result on the undefinability of truth}
\pmrecord{14}{34552}
\pmprivacy{1}
\pmauthor{mathcam}{2727}
\pmtype{Theorem}
\pmcomment{trigger rebuild}
\pmclassification{msc}{03B99}
\pmrelated{IFLogic}

% this is the default PlanetMath preamble.  as your knowledge
% of TeX increases, you will probably want to edit this, but
% it should be fine as is for beginners.

% almost certainly you want these
\usepackage{amssymb}
\usepackage{amsmath}
\usepackage{amsfonts}

% used for TeXing text within eps files
%\usepackage{psfrag}
% need this for including graphics (\includegraphics)
%\usepackage{graphicx}
% for neatly defining theorems and propositions
%\usepackage{amsthm}
% making logically defined graphics
%%%\usepackage{xypic}

% there are many more packages, add them here as you need them

% define commands here
\begin{document}
Assume $\mathbf{L}$ is a logic which is \PMlinkescapetext{closed} under contradictory negation and has the usual truth-functional connectives. Assume also that $\mathbf{L}$ has a notion of \PMlinkescapetext{open} formula with one variable and of substitution. Assume that $T$ is a theory of $\mathbf{L}$ in which we can define surrogates for formulae of $\mathbf{L}$, and in which all true instances of the substitution relation and the truth-functional connective relations are provable. We show that either $T$ is inconsistent or $T$ can't be augmented with a truth predicate $\mathbf{True}$ for which the following T-schema holds

\begin{displaymath}
\mathbf{True}('\phi') \leftrightarrow \phi
\end{displaymath}

Assume that the \PMlinkescapetext{open} formulae with one variable of $\mathbf{L}$ have been indexed by some suitable set that is representable in $T$ (otherwise the predicate $\mathbf{True}$ would be next to useless, since if there's no way to speak of sentences of a logic, there's little hope to define a truth-predicate for it). Denote the $i$:th element in this indexing by $B_i$. Consider now the following open formula with one variable

\begin{displaymath}
 \mathbf{Liar}(x) = \neg \mathbf{True}(B_x(x))
\end{displaymath}

Now, since $\mathbf{Liar}$ is an open formula with one free variable it's indexed by some $i$. Now consider the sentence $\mathbf{Liar}(i)$. From the T-schema we know that

\begin{displaymath}
 \mathbf{True}(\mathbf{Liar}(i)) \leftrightarrow \mathbf{Liar(i)}
\end{displaymath}

and by the definition of $\mathbf{Liar}$ and the fact that $i$ is the \PMlinkescapetext{index} of $\mathbf{Liar}(x)$ we have 

\begin{displaymath}
 \mathbf{True}(\mathbf{Liar}(i)) \leftrightarrow \neg \mathbf{True}(\mathbf{Liar(i)})
\end{displaymath}

which clearly is absurd. Thus there can't be an \PMlinkescapetext{extension} of $T$ with a predicate $\mathbf{Truth}$ for which the T-schema holds.

We have made several assumptions on the logic $\mathbf{L}$ which are crucial in order for this proof to go through. The most important is that $\mathbf{L}$ is closed under contradictory negation. There are logics which allow truth-predicates, but these are not usually closed under contradictory negation (so that it's possible that $\mathbf{True}(\mathbf{Liar}(i))$ is neither true nor false). These logics usually have stronger notions of negation, so that a sentence $\neg P$ says {\em more} than just that $P$ is not true, and the proposition that $P$ is simply not true is not expressible.

An example of a logic for which Tarski's undefinability result does not hold is the so-called Independence Friendly logic, the semantics of which is based on game theory and which allows various generalised quantifiers (the Henkin branching quantifier, etc.) to be used.
%%%%%
%%%%%
\end{document}
