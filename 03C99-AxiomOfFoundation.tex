\documentclass[12pt]{article}
\usepackage{pmmeta}
\pmcanonicalname{AxiomOfFoundation}
\pmcreated{2013-03-22 13:04:31}
\pmmodified{2013-03-22 13:04:31}
\pmowner{Henry}{455}
\pmmodifier{Henry}{455}
\pmtitle{axiom of foundation}
\pmrecord{10}{33485}
\pmprivacy{1}
\pmauthor{Henry}{455}
\pmtype{Definition}
\pmcomment{trigger rebuild}
\pmclassification{msc}{03C99}
\pmsynonym{foundation}{AxiomOfFoundation}
\pmsynonym{regularity}{AxiomOfFoundation}
\pmsynonym{axiom of regularity}{AxiomOfFoundation}
\pmdefines{artinian}
\pmdefines{artinian set}
\pmdefines{artinian sets}

\endmetadata

% this is the default PlanetMath preamble.  as your knowledge
% of TeX increases, you will probably want to edit this, but
% it should be fine as is for beginners.

% almost certainly you want these
\usepackage{amssymb}
\usepackage{amsmath}
\usepackage{amsfonts}

% used for TeXing text within eps files
%\usepackage{psfrag}
% need this for including graphics (\includegraphics)
%\usepackage{graphicx}
% for neatly defining theorems and propositions
%\usepackage{amsthm}
% making logically defined graphics
%%%\usepackage{xypic}

% there are many more packages, add them here as you need them

% define commands here
%\PMlinkescapeword{theory}
\begin{document}
The \emph{axiom of foundation} (also called the \emph{axiom of regularity}) is an axiom of ZF set theory prohibiting circular sets and sets with infinite levels of containment.  Intuitively, it \PMlinkescapetext{states} that every set can be built up from the empty set.  There are several equivalent formulations, for instance:

For any nonempty set $X$ there is some $y\in X$ such that $y\cap X=\emptyset$.

For any set $X$, there is no function $f$ from $\omega$ to the transitive closure of $X$ such that for every $n$, $f(n+1)\in f(n)$.

For any formula $\phi$, if there is any set $x$ such that $\phi(x)$ then there is some $X$ such that $\phi(X)$ but there is no $y\in X$ such that $\phi(y)$.

Sets which satisfy this axiom are called \emph{artinian}.  It is known that, if ZF without this axiom is consistent, then this axiom does not add any inconsistencies.

One important consequence of this property is that no set can contain itself.  For instance, if there were a set $X$ such that $X\in X$ then we could define a function $f(n)=X$ for all $n$, which would then have the property that $f(n+1)\in f(n)$ for all $n$.
%%%%%
%%%%%
\end{document}
