\documentclass[12pt]{article}
\usepackage{pmmeta}
\pmcanonicalname{DefinitionByCases}
\pmcreated{2013-03-22 19:08:13}
\pmmodified{2013-03-22 19:08:13}
\pmowner{CWoo}{3771}
\pmmodifier{CWoo}{3771}
\pmtitle{definition by cases}
\pmrecord{5}{42035}
\pmprivacy{1}
\pmauthor{CWoo}{3771}
\pmtype{Example}
\pmcomment{trigger rebuild}
\pmclassification{msc}{03D20}

\endmetadata

\usepackage{amssymb,amscd}
\usepackage{amsmath}
\usepackage{amsfonts}
\usepackage{mathrsfs}

% used for TeXing text within eps files
%\usepackage{psfrag}
% need this for including graphics (\includegraphics)
%\usepackage{graphicx}
% for neatly defining theorems and propositions
\usepackage{amsthm}
% making logically defined graphics
%%\usepackage{xypic}
\usepackage{pst-plot}

% define commands here
\newcommand*{\abs}[1]{\left\lvert #1\right\rvert}
\newtheorem{prop}{Proposition}
\newtheorem{lem}{Lemma}
\newtheorem{ex}{Example}
\newcommand{\real}{\mathbb{R}}
\newcommand{\pdiff}[2]{\frac{\partial #1}{\partial #2}}
\newcommand{\mpdiff}[3]{\frac{\partial^#1 #2}{\partial #3^#1}}
\begin{document}
\textbf{Definition}  A (total) function $f:\mathbb{N}^k \to \mathbb{N}$ is said to be defined by cases if there are functions $f_1, \ldots, f_m:\mathbb{N}^k \to \mathbb{N}$, and predicates $\Phi_1(\boldsymbol{x}), \ldots, \Phi_m(\boldsymbol{x})$, which are pairwise exclusive $$S(\Phi_i)\cap S(\Phi_j)=\varnothing$$ for $i\ne j$, such that
\begin{displaymath}
f(\boldsymbol{x}):= \left\{
\begin{array}{ll}
f_1(\boldsymbol{x}) & \textrm{if } \Phi_1(\boldsymbol{x}), \\
\cdots \\
f_m(\boldsymbol{x}) & \textrm{if } \Phi_m(\boldsymbol{x}).
\end{array}
\right.
\end{displaymath}
Since $f$ is a total function (domain is all of $\mathbb{N}^k$), we see that $S(\Phi_1) \cup \cdots \cup S(\Phi_m)=\mathbb{N}^k$.

\begin{prop}  As above, if the functions $f_1, \ldots, f_m:\mathbb{N}^k \to \mathbb{N}$, as well as the predicates $\Phi_1(\boldsymbol{x}), \ldots, \Phi_m(\boldsymbol{x})$, are primitive recursive, then so is the function $f:\mathbb{N}^k \to \mathbb{N}$ defined by cases from the $f_i$ and $\Phi_j$. \end{prop}

To see this, we first need the following:
\begin{lem} If functions $f_1, \ldots, f_m:\mathbb{N}^k \to \mathbb{N}$ are primitive recursive, so is $f_1 +\cdots + f_m$. \end{lem}
\begin{proof} By induction on $m$.  The case when $m=1$ is clear.  Suppose the statement is true for $m=n$.  Then $f_1 + \cdots + f_n + f_{n+1} = \operatorname{add}(f_1+\cdots+f_n,f_{n+1})$, which is primitive recursive, since $\operatorname{add}$ is, and that primitive recursiveness is preserved under functional composition.
\end{proof}

\begin{proof}[Proof of Proposition 1] $f$ is just
\begin{displaymath}
f(\boldsymbol{x}):= \left\{
\begin{array}{ll}
f_1(\boldsymbol{x}) & \textrm{if } \boldsymbol{x} \in S(\Phi_1), \\
\cdots \\
f_m(\boldsymbol{x}) & \textrm{if } \boldsymbol{x} \in S(\Phi_m).
\end{array}
\right.
\end{displaymath}
which can be re-written as $$f=\varphi_{S(\Phi_1)} f_1 + \cdots + \varphi_{S(\Phi_m)}f_m,$$ where $\varphi_S$ denotes the characteristic function of set $S$.  By assumption, both $f_i$ and $\varphi_{S(\Phi_i)}$ are primitive recursive, so is their product, and hence the sum of these products.  As a result, $f$ is primitive recursive too.  \end{proof}
%%%%%
%%%%%
\end{document}
