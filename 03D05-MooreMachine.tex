\documentclass[12pt]{article}
\usepackage{pmmeta}
\pmcanonicalname{MooreMachine}
\pmcreated{2013-03-22 18:59:57}
\pmmodified{2013-03-22 18:59:57}
\pmowner{CWoo}{3771}
\pmmodifier{CWoo}{3771}
\pmtitle{Moore machine}
\pmrecord{9}{41867}
\pmprivacy{1}
\pmauthor{CWoo}{3771}
\pmtype{Definition}
\pmcomment{trigger rebuild}
\pmclassification{msc}{03D05}
\pmclassification{msc}{68Q45}
\pmrelated{MealyMachine}
\pmrelated{EquivalentMachines}

\endmetadata

\usepackage{amssymb,amscd}
\usepackage{amsmath}
\usepackage{amsfonts}
\usepackage{mathrsfs}

% used for TeXing text within eps files
%\usepackage{psfrag}
% need this for including graphics (\includegraphics)
\usepackage{graphicx}
% for neatly defining theorems and propositions
\usepackage{amsthm}
% making logically defined graphics
%%\usepackage{xypic}
\usepackage{pst-plot}

% define commands here
\newcommand*{\abs}[1]{\left\lvert #1\right\rvert}
\newtheorem{prop}{Proposition}
\newtheorem{thm}{Theorem}
\newtheorem{ex}{Example}
\newcommand{\real}{\mathbb{R}}
\newcommand{\pdiff}[2]{\frac{\partial #1}{\partial #2}}
\newcommand{\mpdiff}[3]{\frac{\partial^#1 #2}{\partial #3^#1}}
\begin{document}
A \emph{Moore machine} $M$ is a five-tuple $(S,\Sigma,\Delta,\sigma,\lambda)$, where
\begin{enumerate}
\item $S$ is an alphabet whose elements are called \emph{states},
\item $\Sigma$ is an alphabet whose elements are called \emph{input symbols},
\item $\Delta$ is an alphabet whose elements are called \emph{output symbols},
\item $\sigma: S\times \Sigma \to S$ is a function called the \emph{state transition function},
\item $\lambda: S \to \Delta$ is a function called the \emph{output function}.
\end{enumerate}
Given an a pair $(s,a)$ of state $s$ and input symbol $a$, the value $\sigma(s,a)$ is the \emph{next state} of $(s,a)$ and $\lambda(s)$ is the \emph{output} of $(s,a)$.

A Moore machine can be thought of as a Mealy machine, where that the output function does not depend on the input.  Suppose $M=(S,\Sigma,\Delta,\sigma,\lambda)$ is a Moore machine, then $M'=(S,\Sigma, \Delta,\sigma,\lambda')$ where $\lambda':S\times \Sigma \to \Delta$, given by $\lambda'(s,a)=\lambda(s)$, is a Mealy machine.

Like a Mealy machine, a Moore machine also has two visual presentations: via a \emph{flow table}, which describes the transitions to the next states and the outputs, or via a \emph{state diagram}.

For example, consider a Moore machine $M$ where $S=\lbrace s,t\rbrace$, $\Sigma = \lbrace a,b\rbrace$, and $\Delta = \lbrace x,y\rbrace$.  The state transition function $\sigma$ and the output function $\lambda$ are defined by the following table:

\begin{center}
\begin{tabular}{|c||c|c||c|}
\hline
& $a$ & $b$ &  \\
\hline\hline
$s$ & $t$ & $s$ & $y$  \\
\hline
$t$ & $s$ & $t$ & $x$  \\
\hline
\end{tabular}
\end{center}
The first column represents all the states of $M$; the second and third columns are the next states corresponding to the input symbols specified on the top row; and the last column shows the corresponding output symbols.

In addition, one can visualize a Moore machine $M$ by way its state diagram $G_M$, which is just a (labeled) directed graph.  $G_M$ is constructed as follows: the vertices of $G_M$ represent the states of $M$, and each is labeled $s|x$, where $s$ is some state of $M$, and $x=\lambda(s)$.  An edge from vertices $(s,x)$ to $(t,y)$ is constructed iff there is an input $a$ such that $\sigma(s,a)=t$, in which case the edge is labeled $a$.  In the example above, the state diagram $G_M$ of $M$ is the following

\begin{figure}[htp]
\centering
\includegraphics[scale=0.80]{moore.eps}
\end{figure}

Let $M$ be a Moore machine.  Since $M$ is a special type of a Mealy machine, modification of the transition and output functions of $M$ to handle input words could be defined much the same way as if we were treating $M$ as a Mealy machine.  But the downside is is that no output can correspond to the empty input word $\epsilon$.  Because the output function $\lambda$ of $M$ is input-independent, a slightly different approach to modification gets rid of the problem:

First, define the extension of the transition function $\sigma$ of $M$ as if it were a Mealy machine:
$$\sigma(s,\epsilon):=s, \mbox{ and } \sigma(s,ua):=\sigma(\sigma(s,u),a), \mbox{ where }s\in S, u\in \Sigma^*, a\in \Sigma.$$  
Next, define a function $\beta: S\times \Sigma^* \to \Delta$ as follows: 
$$\beta(s,u):=\lambda(\sigma(s,u)), \mbox{ where }s\in S, u\in \Sigma^*.$$  
Call $\beta$ the output function on input \emph{words} for $M$.  Notice that $\beta$ is not an extension of $\lambda$, since $\beta$ is \emph{input-dependent}, whereas $\lambda$ is not.  Furthermore, $\beta(s,\epsilon) = \lambda(s)$.

The approach above shows that the mechanism of a Moore machine for handling inputs/outputs is different from that of a Mealy machine, even though a Moore machine is really just a Mealy machine.

\begin{thebibliography}{8}
\bibitem{sg} S. Ginsburg, {\em An Introduction to Mathematical Machine Theory}, Addision-Wesley, (1962).
\bibitem{ma} M. Arbib, \emph{Algebraic Theory of Machines, Languages, and Semigroups}, Academic Press, (1968).
\bibitem{hs} J. Hartmanis, R.E. Stearns, \emph{Algebraic Structure Theory of Sequential Machines}, Prentice-Hall, (1966).
\end{thebibliography}
%%%%%
%%%%%
\end{document}
