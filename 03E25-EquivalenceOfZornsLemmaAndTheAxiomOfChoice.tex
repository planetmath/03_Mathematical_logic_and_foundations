\documentclass[12pt]{article}
\usepackage{pmmeta}
\pmcanonicalname{EquivalenceOfZornsLemmaAndTheAxiomOfChoice}
\pmcreated{2013-03-22 12:59:04}
\pmmodified{2013-03-22 12:59:04}
\pmowner{Henry}{455}
\pmmodifier{Henry}{455}
\pmtitle{equivalence of Zorn's lemma and the axiom of choice}
\pmrecord{10}{33358}
\pmprivacy{1}
\pmauthor{Henry}{455}
\pmtype{Proof}
\pmcomment{trigger rebuild}
\pmclassification{msc}{03E25}

% this is the default PlanetMath preamble.  as your knowledge
% of TeX increases, you will probably want to edit this, but
% it should be fine as is for beginners.

% almost certainly you want these
\usepackage{amssymb}
\usepackage{amsmath}
\usepackage{amsfonts}

% used for TeXing text within eps files
%\usepackage{psfrag}
% need this for including graphics (\includegraphics)
%\usepackage{graphicx}
% for neatly defining theorems and propositions
%\usepackage{amsthm}
% making logically defined graphics
%%%\usepackage{xypic}

% there are many more packages, add them here as you need them

% define commands here
%\PMlinkescapeword{theory}
\begin{document}
Let $X$ be a set partially ordered by $<$ such that each chain has an upper bound.  Define $p(x)=\{y\in X\mid x< y\}\in P(X)$.
Let $p(X)=\{p(x)\mid x\in X\}$.  If $p(x)=\emptyset$ then it follows that $x$ is maximal.

Suppose no $p(x)=\emptyset$.  Then by the \PMlinkname{axiom of choice}{AxiomOfChoice} there is a choice function $f$ on $p(X)$, and since for each $p(x)$ we have $f(p(x))\in p(x)$, it follows that $x<f(p(x))$.  Define $f_\alpha(p(x))$ for all ordinals $\alpha$ by transfinite induction:

$f_0(p(x))=x$

$f_{\alpha+1}(p(x))=f(p(f_\alpha(p(x))))$

And for a limit ordinal $\alpha$, let $f_\alpha(p(x))$ be an upper bound of $f_i(p(x))$ for $i<\alpha$.

This construction can go on forever, for any ordinal.  Then we can easily construct an injective function from $Ord$ to $X$ by $g(\alpha)=f_\alpha(p(x))$ for an arbitrary $x\in X$.  This must be injective, since $\alpha<\beta$ implies $g(\alpha)<g(\beta)$.  But that requires that $X$ be a proper class, in contradiction to the fact that it is a set.  So there can be no such choice function, and there must be a maximal element of $X$.

For the reverse, assume Zorn's lemma and let $C$ be any set of non-empty sets.  Consider the set of functions $F=\{f\mid \forall a\in\operatorname{dom}(f) (a\in C \wedge f(a)\in a)\}$ partially ordered by inclusion.  Then the union of any chain in $F$ is also a member of $F$ (since the union of a chain of functions is always a function).  By Zorn's lemma, $F$ has a maximal element $f$, and since any function with domain smaller than $C$ can be easily expanded, $\operatorname{dom}(f)=C$, and so $f$ is a choice function for $C$.
%%%%%
%%%%%
\end{document}
