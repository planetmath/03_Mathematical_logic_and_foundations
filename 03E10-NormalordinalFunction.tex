\documentclass[12pt]{article}
\usepackage{pmmeta}
\pmcanonicalname{NormalordinalFunction}
\pmcreated{2013-03-22 15:33:10}
\pmmodified{2013-03-22 15:33:10}
\pmowner{florisje}{7763}
\pmmodifier{florisje}{7763}
\pmtitle{normal (ordinal) function}
\pmrecord{7}{37451}
\pmprivacy{1}
\pmauthor{florisje}{7763}
\pmtype{Definition}
\pmcomment{trigger rebuild}
\pmclassification{msc}{03E10}
%\pmkeywords{ordinals}
%\pmkeywords{ordinal arithmetic}
%\pmkeywords{order preserving}
%\pmkeywords{continuous}
%\pmkeywords{normal function}
%\pmkeywords{normal}
%\pmkeywords{normality}
\pmdefines{continuous (for ordinal functions)}
\pmdefines{order preserving (for ordinal functions)}
\pmdefines{normality}
\pmdefines{normal function}

\endmetadata

% this is the default PlanetMath preamble.  as your knowledge
% of TeX increases, you will probably want to edit this, but
% it should be fine as is for beginners.

% almost certainly you want these
\usepackage{amssymb}
\usepackage{amsmath}
\usepackage{amsfonts}

% used for TeXing text within eps files
%\usepackage{psfrag}
% need this for including graphics (\includegraphics)
%\usepackage{graphicx}
% for neatly defining theorems and propositions
\usepackage{amsthm}
% making logically defined graphics
%%%\usepackage{xypic}

% there are many more packages, add them here as you need them

% define commands here
\DeclareMathOperator{\On}{\mathbf{On}}
\begin{document}
\theoremstyle{definition}

\newtheorem*{def1}{Definition}
\begin{def1}
A function $F\colon\On\to\On$ is \emph{continuous} if and only if for each $u\subset\On$ such that $u\neq\emptyset$ it holds that $F(\sup(u))=\sup\{F(\alpha)|\alpha\in u\}$.
\end{def1}

\newtheorem*{def2}{Definition}
\begin{def2}
A function $F\colon\On\to\On$ is \emph{order preserving} if and only if for each $\alpha,\beta\in\On$ such that $\alpha<\beta$ it follows that $F(\alpha)<F(\beta)$.
\end{def2}

\newtheorem*{def3}{Definition}
\begin{def3}
A function $F\colon\On\to\On$ is a \emph{normal} function if and only if $F$ is continuous and order preserving.
\end{def3}
%%%%%
%%%%%
\end{document}
