\documentclass[12pt]{article}
\usepackage{pmmeta}
\pmcanonicalname{ConsequenceOperatorDeterminedByAClassOfSubsets}
\pmcreated{2013-03-22 16:29:45}
\pmmodified{2013-03-22 16:29:45}
\pmowner{rspuzio}{6075}
\pmmodifier{rspuzio}{6075}
\pmtitle{consequence operator determined by a class of subsets}
\pmrecord{10}{38671}
\pmprivacy{1}
\pmauthor{rspuzio}{6075}
\pmtype{Theorem}
\pmcomment{trigger rebuild}
\pmclassification{msc}{03G25}
\pmclassification{msc}{03G10}
\pmclassification{msc}{03B22}

\endmetadata

% this is the default PlanetMath preamble.  as your knowledge
% of TeX increases, you will probably want to edit this, but
% it should be fine as is for beginners.

% almost certainly you want these
\usepackage{amssymb}
\usepackage{amsmath}
\usepackage{amsfonts}

% used for TeXing text within eps files
%\usepackage{psfrag}
% need this for including graphics (\includegraphics)
%\usepackage{graphicx}
% for neatly defining theorems and propositions
\usepackage{amsthm}
% making logically defined graphics
%%%\usepackage{xypic}

% there are many more packages, add them here as you need them

% define commands here

\newtheorem{theorem}{Theorem}
\begin{document}
\begin{theorem}
Let $L$ be a set and let $K$ be a subset of $\mathcal{P}(L)$.  The the mapping
$C \colon \mathcal{P}(L) \to \mathcal{P}(L)$ defined as $ C(X) = \cap \{ Y \in K
\mid X \subseteq Y\}$ is a consequence operator.
\end{theorem}

\begin{proof}
We need to check that $C$ satisfies the defining properties.

\emph{ Property 1:} 
Since every element of the set $\{ Y \in K \mid X \subseteq Y \}$ 
contains $X$, we have $X \subseteq C(X)$.

\emph{ Property 2:}
For every element $Y$ of $K$ such that $X \subseteq Y$, it also is the case that
$C(X) \subseteq Y$ because an intersection of a family of sets is a subset of
any member of the family.  In other words (or rather, symbols),
 \[ \{ Y \in K \mid X \subseteq Y \} \subseteq \{ Y \in K \mid C(X) 
\subseteq Y \},\]
hence $C(C(X)) \subseteq C(X)$.  By the first property proven above, $C(X)
\subseteq C(C(X))$ so $C(C(X)) = C(X)$.  Thus, $C \circ C = C$.

\emph{Property 3:}
Let $X$ and $Y$ be two subsets of $L$ such that $X \subseteq Y$.  Then if,
for some other subset $Z$ of $L$, we have $Y \subset Z$, it follows that
$X \subset Z$.  Hence,
 \[ \{ Z \in K \mid Y \subseteq Z \} \subseteq \{ Z \in K \mid X 
\subseteq Z \},\]
so $C(X) \subseteq C(Y)$.

\end{proof}
%%%%%
%%%%%
\end{document}
