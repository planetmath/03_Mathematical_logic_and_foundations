\documentclass[12pt]{article}
\usepackage{pmmeta}
\pmcanonicalname{LevelSetsOfSmoothFunctionsOnManifolds}
\pmcreated{2013-03-22 15:20:02}
\pmmodified{2013-03-22 15:20:02}
\pmowner{rspuzio}{6075}
\pmmodifier{rspuzio}{6075}
\pmtitle{level sets of smooth functions on manifolds}
\pmrecord{4}{37148}
\pmprivacy{1}
\pmauthor{rspuzio}{6075}
\pmtype{Definition}
\pmcomment{trigger rebuild}
\pmclassification{msc}{03E20}

% this is the default PlanetMath preamble.  as your knowledge
% of TeX increases, you will probably want to edit this, but
% it should be fine as is for beginners.

% almost certainly you want these
\usepackage{amssymb}
\usepackage{amsmath}
\usepackage{amsfonts}

% used for TeXing text within eps files
%\usepackage{psfrag}
% need this for including graphics (\includegraphics)
%\usepackage{graphicx}
% for neatly defining theorems and propositions
%\usepackage{amsthm}
% making logically defined graphics
%%%\usepackage{xypic}

% there are many more packages, add them here as you need them

% define commands here
\begin{document}
Let $f: \mathbb{R}^n \to \mathbb{R}$ be smooth.  Further suppose that the gradient of $f$ differs from zero at every point of a level set.  Then it follows from the implicit function theorem that that level set is a smooth hypersurface.  Furthermore, at any point of the level set, the gradient of the function at that point is orthogonal to the level set.  

One can generalize this observation to manifolds.  Suppose that $M$ is a smooth manifold and that $f: M \to \mathbb{R}$ is smooth.  Further suppose that the gradient of $f$ differs from zero at every point of a level set.  Then it follows from the implicit function theorem that that level set is a smooth hypersurface.  If one chooses a Riemannian metric on the manifold, the gradient of the function at that point will be orthogonal to the level set.
%%%%%
%%%%%
\end{document}
