\documentclass[12pt]{article}
\usepackage{pmmeta}
\pmcanonicalname{Axiom}
\pmcreated{2013-03-22 12:46:36}
\pmmodified{2013-03-22 12:46:36}
\pmowner{rmilson}{146}
\pmmodifier{rmilson}{146}
\pmtitle{axiom}
\pmrecord{15}{33088}
\pmprivacy{1}
\pmauthor{rmilson}{146}
\pmtype{Definition}
\pmcomment{trigger rebuild}
\pmclassification{msc}{03B99}
\pmclassification{msc}{03A05}
\pmrelated{ZermeloFraenkelAxioms}
\pmdefines{postulate}

% this is the default PlanetMath preamble.  as your knowledge
% of TeX increases, you will probably want to edit this, but
% it should be fine as is for beginners.

% almost certainly you want these
\usepackage{amssymb}
\usepackage{amsmath}
\usepackage{amsfonts}

% used for TeXing text within eps files
%\usepackage{psfrag}
% need this for including graphics (\includegraphics)
%\usepackage{graphicx}
% for neatly defining theorems and propositions
%\usepackage{amsthm}
% making logically defined graphics
%%%\usepackage{xypic} 

% there are many more packages, add them here as you need them

% define commands here
\begin{document}
In a nutshell, the logico-deductive method is a system of inference
where conclusions (new knowledge) follow from premises (old knowledge)
through the application of sound arguments (syllogisms, rules of
inference).  Tautologies excluded, nothing can be deduced if nothing
is assumed.  Axioms and postulates are the basic assumptions
underlying a given body of deductive knowledge.  They are accepted
without demonstration.  All other assertions (theorems, if we are
talking about mathematics) must be proven with the aid of the basic
assumptions.


The logico-deductive method was developed by the ancient Greeks, and
has become the core principle of modern mathematics.  However, the
interpretation of mathematical knowledge has changed from ancient
times to the modern, and consequently the terms \emph{axiom} and
\emph{postulate} hold a slightly different meaning for the present day
mathematician, then they did for Aristotle and Euclid.

The ancient Greeks considered geometry as just one of several
sciences, and held the theorems of geometry on par with scientific
facts.  As such, they developed and used the logico-deductive method
as a means of avoiding error, and for structuring and communicating
knowledge.  Aristotle's \PMlinkexternal{Posterior
  Analytics}{http://classics.mit.edu/Aristotle/posterior.1.i.html}
is a definitive exposition of the classical view.

``Axiom'', in classical terminology, referred to a self-evident assumption
common to many branches of science.  A good example would be the
assertion that
\begin{quote}
  \em When an equal amount is taken from equals, an equal amount results.
\end{quote}

At the foundation of the various sciences lay certain basic hypotheses
that had to be accepted without proof.  Such a hypothesis was termed a
\emph{postulate}.  The postulates of each science were different.
Their validity had to be established by means of real-world
experience. Indeed, Aristotle warns that the content of a science
cannot be successfully communicated, if the learner is in doubt about
the truth of the postulates.

The classical approach is well illustrated by Euclid's elements, where
we see a list of axioms (very basic, self-evident assertions) and
postulates (common-sensical geometric facts drawn from our
experience).
\begin{itemize}
\item[\bf A1] Things which are equal to the same thing are also equal to one
another.  
\item[\bf A2] If equals be added to equals, the wholes are equal.
\item[\bf A3] If equals be subtracted from equals, the remainders are
  equal.
\item[\bf A4] Things which coincide with one another are equal to one
  another.
\item[\bf A5] The whole is greater than the part.
\item[\bf P1] It is possible to
draw a straight line from any point to any other point.
\item[\bf P2] It is possible to produce a finite straight line continuously in a
straight line.
\item[\bf P3] It is possible to describe a circle with any
centre and distance.
\item[\bf P4] It is true that all right angles are
equal to one another.
\item[\bf P5] It is true that, if a straight line
falling on two straight lines make the interior angles on the same
side less than two right angles, the two straight lines, if produced
indefinitely, meet on that side on which are the angles less than the
two right angles.
\end{itemize}
The classical view point is explored in more detail \PMlinkexternal{here}{http://www.mathgym.com.au/history/pythagoras/pythgeom.htm}.


A great lesson learned by mathematics in the last 150 years is that it
is useful to strip the meaning away from the mathematical assertions
(axioms, postulates, propositions, theorems) and definitions. This
abstraction, one might even say formalization, makes mathematical
knowledge more general, capable of multiple different meanings, and
therefore useful in multiple contexts.


In structuralist mathematics we go even further, and develop theories
and axioms (like field theory, group theory, topology, vector spaces)
without \emph{any} particular application in mind.  The distinction
between an ``axiom'' and a ``postulate'' disappears. The postulates of
Euclid are profitably motivated by saying that they lead to a great
wealth of geometric facts.  The truth of these complicated facts rests
on the acceptance of the basic hypotheses.  However by throwing out
postulate 5, we get theories that have meaning in wider contexts,
hyperbolic geometry for example. We must simply be prepared to use
labels like ``line'' and ``parallel'' with greater flexibility.  The
development of hyperbolic geometry taught mathematicians that
postulates should be regarded as purely formal statements, and not as
facts based on experience.

When mathematicians employ the axioms of a field, the intentions are
even more abstract. The propositions of field theory do not concern
any one particular application; the mathematician now works in
complete abstraction.  There are many examples of fields; field
theory gives correct knowledge in all contexts.

It is not correct to say that the axioms of field theory are
``propositions that are regarded as true without proof.''  Rather, the
Field Axioms are a set of constraints.  If any given system of
addition and multiplication tolerates these constraints, then one is
in a position to instantly know a great deal of extra information
about this system. There is a lot of bang for the formalist buck.

Modern mathematics formalizes its foundations to such an
extent that mathematical theories can be regarded as mathematical
objects, and logic itself can be regarded as a branch of mathematics.
Frege, Russell, Poincar\'e, Hilbert, and G\"odel are some of
the key figures in this development.

In the modern understanding, a set of axioms is any collection of
formally stated assertions from which other formally stated assertions
follow by the application of certain well-defined rules.  In this
view, logic becomes just another formal system.  A set of axioms
should be consistent; it should be impossible to derive a
contradiction from the axiom.  A set of axioms should also be
non-redundant; an assertion that can be deduced from other axioms need
not be regarded as an axiom.

It was the early hope of modern logicians that various branches of
mathematics, perhaps all of mathematics, could be derived from a
consistent collection of basic axioms.  An early success of the
formalist program was Hilbert's formalization of Euclidean geometry,
and the related demonstration of the consistency of those axioms.

In a wider context, there was an attempt to base all of mathematics on
Cantor's set theory.  Here the emergence of Russell's paradox, and
similar antinomies of naive set theory raised the possibility that any
such system could turn out to be inconsistent.

The formalist project suffered a decisive setback, when in 1931
G\"odel showed that it is possible, for any sufficiently large set of
axioms (Peano's axioms, for example) to construct a statement whose
truth is independent of that set of axioms.  As a corollary, G\"odel
proved that the consistency of a theory like Peano arithmetic is an
unprovable assertion within the scope of that theory.

It is reasonable to believe in the consistency of Peano arithmetic
because it is satisfied by the system of natural numbers, an infinite
but intuitively accessible formal system.  However, at this date we
have no way of demonstrating the consistency of modern set theory
(Zermelo-Frankel axioms).  The axiom of choice, a key hypothesis
of this theory, remains a very controversial assumption.
Furthermore, using techniques of forcing (Cohen) one can show that the
continuum hypothesis (Cantor) is independent of the Zermelo-Frankel
axioms.  Thus, even this very general set of axioms cannot be regarded
as the definitive foundation for mathematics.
%%%%%
%%%%%
\end{document}
