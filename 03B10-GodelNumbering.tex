\documentclass[12pt]{article}
\usepackage{pmmeta}
\pmcanonicalname{GodelNumbering}
\pmcreated{2013-03-22 12:58:21}
\pmmodified{2013-03-22 12:58:21}
\pmowner{Henry}{455}
\pmmodifier{Henry}{455}
\pmtitle{G\"odel numbering}
\pmrecord{8}{33343}
\pmprivacy{1}
\pmauthor{Henry}{455}
\pmtype{Definition}
\pmcomment{trigger rebuild}
\pmclassification{msc}{03B10}
\pmrelated{BeyondFormalism}
\pmdefines{G\"odel number}

% this is the default PlanetMath preamble.  as your knowledge
% of TeX increases, you will probably want to edit this, but
% it should be fine as is for beginners.

% almost certainly you want these
\usepackage{amssymb}
\usepackage{amsmath}
\usepackage{amsfonts}

% used for TeXing text within eps files
%\usepackage{psfrag}
% need this for including graphics (\includegraphics)
%\usepackage{graphicx}
% for neatly defining theorems and propositions
%\usepackage{amsthm}
% making logically defined graphics
%%%\usepackage{xypic}

% there are many more packages, add them here as you need them

% define commands here
%\PMlinkescapeword{theory}
\begin{document}
A \emph{G\"odel numbering} is any way of assigning numbers to the formulas of a language. This is often useful in allowing sentences of a language to be self-referential.  The number associated with a formula $\phi$ is called its \emph{G\"odel number} and is denoted $\ulcorner\phi\urcorner$.

More formally, if $\mathcal{L}$ is a language and $\mathcal{G}$ is a surjective partial function from the terms of $\mathcal{L}$ to the formulas over $\mathcal{L}$ then $\mathcal{G}$ is a G\"odel numbering.  $\ulcorner\phi\urcorner$ may be any term $t$ such that $\mathcal{G}(t)=\phi$.  Note that $\mathcal{G}$ is not defined within $\mathcal{L}$ (there is no formula or object of $\mathcal{L}$ representing $\mathcal{G}$), however properties of it (such as being in the domain of $\mathcal{G}$, being a subformula, and so on) are.

Athough anything meeting the properties above is a G\"odel numbering, depending on the specific language and usage, any of the following properties may also be desired (and can often be found if more effort is put into the numbering):

\begin{itemize}
\item If $\phi$ is a subformula of $\psi$ then $\ulcorner\phi\urcorner<\ulcorner\psi\urcorner$

\item For every number $n$, there is some $\phi$ such that $\ulcorner\phi\urcorner=n$

\item $\mathcal{G}$ is injective
\end{itemize}
%%%%%
%%%%%
\end{document}
