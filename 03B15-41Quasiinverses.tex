\documentclass[12pt]{article}
\usepackage{pmmeta}
\pmcanonicalname{41Quasiinverses}
\pmcreated{2013-11-17 22:50:48}
\pmmodified{2013-11-17 22:50:48}
\pmowner{PMBookProject}{1000683}
\pmmodifier{rspuzio}{6075}
\pmtitle{4.1 Quasi-inverses}
\pmrecord{2}{87663}
\pmprivacy{1}
\pmauthor{PMBookProject}{6075}
\pmtype{Application}
\pmclassification{msc}{03B15}

\usepackage{xspace}
\usepackage{amssyb}
\usepackage{amsmath}
\usepackage{amsfonts}
\usepackage{amsthm}
\makeatletter
\newcommand{\bfalse}{{0_{\bool}}}
\newcommand{\bool}{\ensuremath{\mathbf{2}}\xspace}
\newcommand{\bproj}[1]{\tproj{}{#1}}
\newcommand{\brck}[1]{\trunc{}{#1}}
\newcommand{\btrue}{{1_{\bool}}}
\newcommand{\ct}{  \mathchoice{\mathbin{\raisebox{0.5ex}{$\displaystyle\centerdot$}}}             {\mathbin{\raisebox{0.5ex}{$\centerdot$}}}             {\mathbin{\raisebox{0.25ex}{$\scriptstyle\,\centerdot\,$}}}             {\mathbin{\raisebox{0.1ex}{$\scriptscriptstyle\,\centerdot\,$}}}}
\newcommand{\defeq}{\vcentcolon\equiv}  
\newcommand{\define}[1]{\textbf{#1}}
\def\@dprd#1{\prod_{(#1)}\,}
\def\@dprd@noparens#1{\prod_{#1}\,}
\def\@dsm#1{\sum_{(#1)}\,}
\def\@dsm@noparens#1{\sum_{#1}\,}
\def\@eatprd\prd{\prd@parens}
\def\@eatsm\sm{\sm@parens}
\newcommand{\eqv}[2]{\ensuremath{#1 \simeq #2}\xspace}
\newcommand{\htpy}{\sim}
\newcommand{\id}[3][]{\ensuremath{#2 =_{#1} #3}\xspace}
\newcommand{\idfunc}[1][]{\ensuremath{\mathsf{id}_{#1}}\xspace}
\newcommand{\idtoeqv}{\ensuremath{\mathsf{idtoeqv}}\xspace}
\newcommand{\isequiv}{\ensuremath{\mathsf{isequiv}}}
\newcommand{\jdeq}{\equiv}      
\def\lam#1{{\lambda}\@lamarg#1:\@endlamarg\@ifnextchar\bgroup{.\,\lam}{.\,}}
\def\@lamarg#1:#2\@endlamarg{\if\relax\detokenize{#2}\relax #1\else\@lamvar{\@lameatcolon#2},#1\@endlamvar\fi}
\def\@lameatcolon#1:{#1}
\def\@lamvar#1,#2\@endlamvar{(#2\,{:}\,#1)}
\newcommand{\opp}[1]{\mathord{{#1}^{-1}}}
\newcommand{\Parens}[1]{\Bigl(#1\Bigr)}
\def\prd#1{\@ifnextchar\bgroup{\prd@parens{#1}}{\@ifnextchar\sm{\prd@parens{#1}\@eatsm}{\prd@noparens{#1}}}}
\def\prd@noparens#1{\mathchoice{\@dprd@noparens{#1}}{\@tprd{#1}}{\@tprd{#1}}{\@tprd{#1}}}
\def\prd@parens#1{\@ifnextchar\bgroup  {\mathchoice{\@dprd{#1}}{\@tprd{#1}}{\@tprd{#1}}{\@tprd{#1}}\prd@parens}  {\@ifnextchar\sm    {\mathchoice{\@dprd{#1}}{\@tprd{#1}}{\@tprd{#1}}{\@tprd{#1}}\@eatsm}    {\mathchoice{\@dprd{#1}}{\@tprd{#1}}{\@tprd{#1}}{\@tprd{#1}}}}}
\newcommand{\proj}[1]{\ensuremath{\mathsf{pr}_{#1}}\xspace}
\newcommand{\qinv}{\ensuremath{\mathsf{qinv}}}
\newcommand{\refl}[1]{\ensuremath{\mathsf{refl}_{#1}}\xspace}
\def\sm#1{\@ifnextchar\bgroup{\sm@parens{#1}}{\@ifnextchar\prd{\sm@parens{#1}\@eatprd}{\sm@noparens{#1}}}}
\def\sm@noparens#1{\mathchoice{\@dsm@noparens{#1}}{\@tsm{#1}}{\@tsm{#1}}{\@tsm{#1}}}
\def\sm@parens#1{\@ifnextchar\bgroup  {\mathchoice{\@dsm{#1}}{\@tsm{#1}}{\@tsm{#1}}{\@tsm{#1}}\sm@parens}  {\@ifnextchar\prd    {\mathchoice{\@dsm{#1}}{\@tsm{#1}}{\@tsm{#1}}{\@tsm{#1}}\@eatprd}    {\mathchoice{\@dsm{#1}}{\@tsm{#1}}{\@tsm{#1}}{\@tsm{#1}}}}}
\newcommand{\Sn}{\mathbb{S}}
\def\@tprd#1{\mathchoice{{\textstyle\prod_{(#1)}}}{\prod_{(#1)}}{\prod_{(#1)}}{\prod_{(#1)}}}
\newcommand{\tproj}[3][]{\mathopen{}\left|#3\right|_{#2}^{#1}\mathclose{}}
\newcommand{\trunc}[2]{\mathopen{}\left\Vert #2\right\Vert_{#1}\mathclose{}}
\def\@tsm#1{\mathchoice{{\textstyle\sum_{(#1)}}}{\sum_{(#1)}}{\sum_{(#1)}}{\sum_{(#1)}}}
\newcommand{\UU}{\ensuremath{\mathcal{U}}\xspace}
\newcommand{\vcentcolon}{:\!\!}
\newcounter{mathcount}
\setcounter{mathcount}{1}
\newtheorem{prelem}{Lemma}
\newenvironment{lem}{\begin{prelem}}{\end{prelem}\addtocounter{mathcount}{1}}
\renewcommand{\theprelem}{4.1.\arabic{mathcount}}
\newtheorem{prethm}{Theorem}
\newenvironment{thm}{\begin{prethm}}{\end{prethm}\addtocounter{mathcount}{1}}
\renewcommand{\theprethm}{4.1.\arabic{mathcount}}
\let\autoref\cref
\let\type\UU
\makeatother

\begin{document}

\index{quasi-inverse|(}%
We have said that $\qinv(f)$ is unsatisfactory because it is not a mere proposition, whereas we would rather that a given function can ``be an equivalence'' in at most one way.
However, we have given no evidence that $\qinv(f)$ is not a mere proposition.
In this section we exhibit a specific counterexample.

\begin{lem}\label{lem:qinv-autohtpy}
  If $f:A\to B$ is such that $\qinv (f)$ is inhabited, then
  \[\eqv{\qinv(f)}{\Parens{\prd{x:A}(x=x)}}.\]
\end{lem}
\begin{proof}
  By assumption, $f$ is an equivalence; that is, we have $e:\isequiv(f)$ and so $(f,e):\eqv A B$.
  By univalence, $\idtoeqv:(A=B) \to (\eqv A B)$ is an equivalence, so we may assume that $(f,e)$ is of the form $\idtoeqv(p)$ for some $p:A=B$.
  Then by path induction, we may assume $p$ is $\refl{A}$, in which case $\idtoeqv(p)$ is $\idfunc[A]$.
  Thus we are reduced to proving $\eqv{\qinv(\idfunc[A])}{(\prd{x:A}(x=x))}$.
  Now by definition we have
  \[ \qinv(\idfunc[A]) \jdeq
  \sm{g:A\to A} \big((g \htpy \idfunc[A]) \times (g \htpy \idfunc[A])\big).
  \]
  By function extensionality, this is equivalent to
  \[ \sm{g:A\to A} \big((g = \idfunc[A]) \times (g = \idfunc[A])\big).
  \]
  And by \PMlinkexternal{Exercise 2.10}{http://planetmath.org/node/87641}, this is equivalent to
  \[ \sm{h:\sm{g:A\to A} (g = \idfunc[A])} (\proj1(h) = \idfunc[A])
  \]
  However, by \PMlinkname{Lemma 3.11.8}{311contractibility#Thmprelem5}, $\sm{g:A\to A} (g = \idfunc[A])$ is contractible with center $\idfunc[A]$; therefore by \PMlinkname{Lemma 3.11.9}{311contractibility#Thmprelem6} this type is equivalent to $\idfunc[A] = \idfunc[A]$.
  And by function extensionality, $\idfunc[A] = \idfunc[A]$ is equivalent to $\prd{x:A} x=x$.
\end{proof}

\noindent
We remark that \PMlinkexternal{Exercise 4.3}{http://planetmath.org/node/87864} asks for a proof of the above lemma which avoids univalence.

Thus, what we need is some $A$ which admits a nontrivial element of $\prd{x:A}(x=x)$.
Thinking of $A$ as a higher groupoid, an inhabitant of $\prd{x:A}(x=x)$ is a natural transformation\index{natural!transformation} from the identity functor of $A$ to itself.
Such transformations are said to form the \define{center of a category},
\index{center!of a category}%
\index{category!center of}%
since the naturality axiom requires that they commute with all morphisms.
Classically, if $A$ is simply a group regarded as a one-object groupoid, then this yields precisely its center in the usual group-theoretic sense.
This provides some motivation for the following.

\begin{lem}\label{lem:autohtpy}
  Suppose we have a type $A$ with $a:A$ and $q:a=a$ such that
  \begin{enumerate}
  \item The type $a=a$ is a set.\label{item:autohtpy1}
  \item For all $x:A$ we have $\brck{a=x}$.\label{item:autohtpy2}
  \item For all $p:a=a$ we have $p\ct q = q \ct p$.\label{item:autohtpy3}
  \end{enumerate}
  Then there exists $f:\prd{x:A} (x=x)$ with $f(a)=q$.
\end{lem}
\begin{proof}
  Let $g:\prd{x:A} \brck{a=x}$ be as given by~\ref{item:autohtpy2}.  First we
  observe that each type $\id[A]xy$ is a set.  For since being a set is a mere
  proposition, we may apply the induction principle of propositional truncation, and assume that $g(x)=\bproj
  p$ and $g(y)=\bproj q$ for $p:a=x$ and $q:a=y$.  In this case, composing with
  $p$ and $\opp{q}$ yields an equivalence $\eqv{(x=y)}{(a=a)}$.  But $(a=a)$ is
  a set by~\ref{item:autohtpy1}, so $(x=y)$ is also a set.

  Now, we would like to define $f$ by assigning to each $x$ the path $\opp{g(x)}
  \ct q \ct g(x)$, but this does not work because $g(x)$ does not inhabit $a=x$
  but rather $\brck{a=x}$, and the type $(x=x)$ may not be a mere proposition,
  so we cannot use induction on propositional truncation.  Instead we can apply
  the technique mentioned in \PMlinkname{\S 3.9}{39theprincipleofuniquechoice}: we characterize
  uniquely the object we wish to construct.  Let us define, for each $x:A$, the
  type
  \[ B(x) \defeq \sm{r:x=x} \prd{s:a=x} (r = \opp s \ct q\ct s).\]
  We claim that $B(x)$ is a mere proposition for each $x:A$.
  Since this claim is itself a mere proposition, we may again apply induction on
  truncation and assume that $g(x) = \bproj p$ for some $p:a=x$.
  Now suppose given $(r,h)$ and $(r',h')$ in $B(x)$; then we have
  \[ h(p) \ct \opp{h'(p)} : r = r'. \]
  It remains to show that $h$ is identified with $h'$ when transported along this equality, which by transport in identity types and function types (\PMlinkname{\S 2.11}{211identitytype},\PMlinkname{\S 2.9}{29pitypesandthefunctionextensionalityaxiom}), reduces to showing
  \[ h(s) = h(p) \ct \opp{h'(p)} \ct h'(s) \]
  for any $s:a=x$.
  But each side of this is an equality between elements of $(x=x)$, so it follows from our above observation that $(x=x)$ is a set.

  Thus, each $B(x)$ is a mere proposition; we claim that $\prd{x:A} B(x)$.
  Given $x:A$, we may now invoke the induction principle of propositional truncation to assume that $g(x) = \bproj p$ for $p:a=x$.
  We define $r \defeq \opp p \ct q \ct p$; to inhabit $B(x)$ it remains to show that for any $s:a=x$ we have
  $r = \opp s \ct q \ct s$.
  Manipulating paths, this reduces to showing that $q\ct (p\ct \opp s) = (p\ct \opp s) \ct q$.
  But this is just an instance of~\ref{item:autohtpy3}.
\end{proof}

\begin{thm}
  There exist types $A$ and $B$ and a function $f:A\to B$ such that $\qinv(f)$ is not a mere proposition.
\end{thm}
\begin{proof}
  It suffices to exhibit a type $X$ such that $\prd{x:X} (x=x)$ is not a mere proposition.
  Define $X\defeq \sm{A:\type} \brck{\bool=A}$, as in the proof of \PMlinkname{Lemma 3.8.5}{38theaxiomofchoice#Thmprelem2}.
  It will suffice to exhibit an $f:\prd{x:X} (x=x)$ which is unequal to $\lam{x} \refl{x}$.

  Let $a \defeq (\bool,\bproj{\refl{\bool}}) : X$, and let $q:a=a$ be the path corresponding to the nonidentity equivalence $e:\eqv\bool\bool$ defined by $e(\bfalse)\defeq\btrue$ and $e(\btrue)\defeq\bfalse$.
  We would like to apply \PMlinkname{Lemma 4.1.2}{41quasiinverses#Thmprelem2} to build an $f$.
  By definition of $X$, equalities in subset types (\PMlinkname{\S 3.5}{35subsetsandpropositionalresizing}), and univalence, we have $\eqv{(a=a)}{(\eqv{\bool}{\bool})}$, which is a set, so~\ref{item:autohtpy1} holds.
  Similarly, by definition of $X$ and equalities in subset types we have~\ref{item:autohtpy2}.
  Finally, \PMlinkexternal{Exercise 2.13}{http://planetmath.org/node/87644} implies that every equivalence $\eqv\bool\bool$ is equal to either $\idfunc[\bool]$ or $e$, so we can show~\ref{item:autohtpy3} by a four-way case analysis.

  Thus, we have $f:\prd{x:X} (x=x)$ such that $f(a) = q$.
  Since $e$ is not equal to $\idfunc[\bool]$, $q$ is not equal to $\refl{a}$, and thus $f$ is not equal to $\lam{x} \refl{x}$.
  Therefore, $\prd{x:X} (x=x)$ is not a mere proposition.
\end{proof}

More generally, \PMlinkname{Lemma 4.1.2}{41quasiinverses#Thmprelem2} implies that any ``Eilenberg--Mac Lane space'' $K(G,1)$, where $G$ is a nontrivial abelian\index{group!abelian} group, will provide a counterexample; see \PMlinkexternal{Chapter 8}{http://planetmath.org/node/87582}.
The type $X$ we used turns out to be equivalent to $K(\mathbb{Z}_2,1)$.
In \PMlinkexternal{Chapter 6}{http://planetmath.org/node/87579} we will see that the circle $\Sn^1 = K(\mathbb{Z},1)$ is another easy-to-describe example.

We now move on to describing better notions of equivalence.

\index{quasi-inverse|)}%

%%%%%%%%%%%%%%%%%%%%%%%%%%%%%%%%%%%%%%

\end{document}
