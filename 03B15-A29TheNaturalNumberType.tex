\documentclass[12pt]{article}
\usepackage{pmmeta}
\pmcanonicalname{A29TheNaturalNumberType}
\pmcreated{2013-11-09 5:40:53}
\pmmodified{2013-11-09 5:40:53}
\pmowner{PMBookProject}{1000683}
\pmmodifier{PMBookProject}{1000683}
\pmtitle{A.2.9 The natural number type}
\pmrecord{1}{}
\pmprivacy{1}
\pmauthor{PMBookProject}{1000683}
\pmtype{Feature}
\pmclassification{msc}{03B15}

\usepackage{xspace}
\usepackage{amssyb}
\usepackage{amsmath}
\usepackage{amsfonts}
\usepackage{amsthm}
\newcommand{\comp}{\textsc{comp}}
\newcommand{\ctx}{\ensuremath{\mathsf{ctx}}}
\newcommand{\elim}{\textsc{elim}}
\newcommand{\form}{\textsc{form}}
\newcommand{\ind}[1]{\mathsf{ind}_{#1}}
\newcommand{\intro}{\textsc{intro}}
\newcommand{\jdeq}{\equiv}      
\newcommand{\jdeqtp}[4]{#1 \vdash #2 \jdeq #3 : #4}
\newcommand{\N}{\ensuremath{\mathbb{N}}\xspace}
\newcommand{\oftp}[3]{#1 \vdash #2 : #3}
\newcommand{\suc}{\mathsf{succ}}
\newcommand{\tmtp}[2]{#1 \mathord{:} #2}
\newcommand{\UU}{\ensuremath{\mathcal{U}}\xspace}
\newcommand{\wfctx}[1]{#1\ \ctx}
\let\autoref\cref

\begin{document}
\index{natural numbers|(}%

We give the rules for natural numbers, following \PMlinkname{\S 1.9}{19thenaturalnumbers}.

\begin{mathparpagebreakable}
  \def\premise{
     \oftp{\Gamma,\tmtp x{\N}}{C}{\UU_i} \\
     \oftp\Gamma{c_0}{C[0/x]} \\
     \oftp{\Gamma,\tmtp{x}\N,\tmtp y C}{c_s}{C[\suc(x)/x]}}
  %
  \inferrule*[right=$\N$-\form]
  {\wfctx\Gamma}
  {\oftp\Gamma{\N}{\UU_i}}
\and
  \inferrule*[right=$\N$-\intro${}_1$]
  {\wfctx\Gamma}
  {\oftp\Gamma{0}{\N}}
\and
  \inferrule*[right=$\N$-\intro${}_2$]
  {\oftp\Gamma{n}{\N}}
  {\oftp\Gamma{\suc(n)}{\N}}
\and
  \inferrule*[right=$\N$-\elim]
  {\premise \\ \oftp\Gamma{n}{\N}}
  {\oftp\Gamma{\ind{\N}(x.C,c_0,x.y.c_s,n)}{C[n/x]}}
\and
  \inferrule*[right=$\N$-\comp${}_1$]
  {\premise}
  {\jdeqtp\Gamma{\ind{\N}(x.C,c_0,x.y.c_s,0)}{c_0}{C[0/x]}}
\and
  \inferrule*[right=$\N$-\comp${}_2$]
  {\premise \\ \oftp\Gamma{n}{\N}}
  {\Gamma\vdash
    {\begin{aligned}[t]
      &\ind{\N}(x.C,c_0,x.y.c_s,\suc(n)) \\
      &\quad \jdeq c_s[n,\ind{\N}(x.C,c_0,x.y.c_s,n)/x,y] : C[\suc(n)/x]
    \end{aligned}}}
\end{mathparpagebreakable}
%
In $\ind{\N}$, $x$ is bound in $C$, and $x$ and $y$ are bound in $c_s$.

Other inductively defined types follow the same general scheme.

\index{natural numbers|)}%


\end{document}
