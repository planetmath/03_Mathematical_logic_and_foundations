\documentclass[12pt]{article}
\usepackage{pmmeta}
\pmcanonicalname{InhabitedSet}
\pmcreated{2013-03-22 14:25:24}
\pmmodified{2013-03-22 14:25:24}
\pmowner{mathwizard}{128}
\pmmodifier{mathwizard}{128}
\pmtitle{inhabited set}
\pmrecord{6}{35931}
\pmprivacy{1}
\pmauthor{mathwizard}{128}
\pmtype{Definition}
\pmcomment{trigger rebuild}
\pmclassification{msc}{03F55}

% this is the default PlanetMath preamble.  as your knowledge
% of TeX increases, you will probably want to edit this, but
% it should be fine as is for beginners.

% almost certainly you want these
\usepackage{amssymb}
\usepackage{amsmath}
\usepackage{amsfonts}

% used for TeXing text within eps files
%\usepackage{psfrag}
% need this for including graphics (\includegraphics)
%\usepackage{graphicx}
% for neatly defining theorems and propositions
%\usepackage{amsthm}
% making logically defined graphics
%%%\usepackage{xypic}

% there are many more packages, add them here as you need them

% define commands here
\begin{document}
A set $A$ is called \emph{inhabited}, if there exists an element $a\in A$. Note that in classical mathematics this is equivalent to $A\neq\emptyset$ (i.e. $A$ being nonempty), yet in intuitionistic mathematics we actually have to find an element $a\in A$. 
For example the set, which contains $1$ if Goldbach's conjecture is true and $0$ if it is false is certainly nonempty, yet by today's state of knowledge we cannot say if $A$ is inhabited, since we do not know an element of $A$.
%%%%%
%%%%%
\end{document}
