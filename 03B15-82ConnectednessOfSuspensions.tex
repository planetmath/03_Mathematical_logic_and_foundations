\documentclass[12pt]{article}
\usepackage{pmmeta}
\pmcanonicalname{82ConnectednessOfSuspensions}
\pmcreated{2013-11-06 4:03:04}
\pmmodified{2013-11-06 4:03:04}
\pmowner{PMBookProject}{1000683}
\pmmodifier{rspuzio}{6075}
\pmtitle{8.2 Connectedness of suspensions}
\pmrecord{1}{}
\pmprivacy{1}
\pmauthor{PMBookProject}{6075}
\pmtype{Feature}
\pmclassification{msc}{03B15}

\endmetadata

\usepackage{xspace}
\usepackage{amssyb}
\usepackage{amsmath}
\usepackage{amsfonts}
\usepackage{amsthm}
\makeatletter
\newcommand{\cocone}[2]{\mathsf{cocone}_{#1}(#2)}
\newcommand{\Ddiag}{\mathscr{D}}
\newcommand{\define}[1]{\textbf{#1}}
\def\@dprd#1{\prod_{(#1)}\,}
\def\@dprd@noparens#1{\prod_{#1}\,}
\def\@dsm#1{\sum_{(#1)}\,}
\def\@dsm@noparens#1{\sum_{#1}\,}
\def\@eatprd\prd{\prd@parens}
\def\@eatsm\sm{\sm@parens}
\newcommand{\eqvsym}{\simeq}    
\newcommand{\function}[4]{\left\{\begin{array}{rcl}#1 &      \longrightarrow & #2 \\ #3 & \longmapsto & #4 \end{array}\right.}
\def\lam#1{{\lambda}\@lamarg#1:\@endlamarg\@ifnextchar\bgroup{.\,\lam}{.\,}}
\def\@lamarg#1:#2\@endlamarg{\if\relax\detokenize{#2}\relax #1\else\@lamvar{\@lameatcolon#2},#1\@endlamvar\fi}
\def\@lameatcolon#1:{#1}
\def\lamu#1{{\lambda}\@lamuarg#1:\@endlamuarg\@ifnextchar\bgroup{.\,\lamu}{.\,}}
\def\@lamuarg#1:#2\@endlamuarg{#1}
\def\@lamvar#1,#2\@endlamvar{(#2\,{:}\,#1)}
\newcommand{\N}{\ensuremath{\mathbb{N}}\xspace}
\def\prd#1{\@ifnextchar\bgroup{\prd@parens{#1}}{\@ifnextchar\sm{\prd@parens{#1}\@eatsm}{\prd@noparens{#1}}}}
\def\prd@noparens#1{\mathchoice{\@dprd@noparens{#1}}{\@tprd{#1}}{\@tprd{#1}}{\@tprd{#1}}}
\def\prd@parens#1{\@ifnextchar\bgroup  {\mathchoice{\@dprd{#1}}{\@tprd{#1}}{\@tprd{#1}}{\@tprd{#1}}\prd@parens}  {\@ifnextchar\sm    {\mathchoice{\@dprd{#1}}{\@tprd{#1}}{\@tprd{#1}}{\@tprd{#1}}\@eatsm}    {\mathchoice{\@dprd{#1}}{\@tprd{#1}}{\@tprd{#1}}{\@tprd{#1}}}}}
\newcommand{\refl}[1]{\ensuremath{\mathsf{refl}_{#1}}\xspace}
\def\sm#1{\@ifnextchar\bgroup{\sm@parens{#1}}{\@ifnextchar\prd{\sm@parens{#1}\@eatprd}{\sm@noparens{#1}}}}
\def\sm@noparens#1{\mathchoice{\@dsm@noparens{#1}}{\@tsm{#1}}{\@tsm{#1}}{\@tsm{#1}}}
\def\sm@parens#1{\@ifnextchar\bgroup  {\mathchoice{\@dsm{#1}}{\@tsm{#1}}{\@tsm{#1}}{\@tsm{#1}}\sm@parens}  {\@ifnextchar\prd    {\mathchoice{\@dsm{#1}}{\@tsm{#1}}{\@tsm{#1}}{\@tsm{#1}}\@eatprd}    {\mathchoice{\@dsm{#1}}{\@tsm{#1}}{\@tsm{#1}}{\@tsm{#1}}}}}
\newcommand{\Sn}{\mathbb{S}}
\def\@tprd#1{\mathchoice{{\textstyle\prod_{(#1)}}}{\prod_{(#1)}}{\prod_{(#1)}}{\prod_{(#1)}}}
\newcommand{\trunc}[2]{\mathopen{}\left\Vert #2\right\Vert_{#1}\mathclose{}}
\def\@tsm#1{\mathchoice{{\textstyle\sum_{(#1)}}}{\sum_{(#1)}}{\sum_{(#1)}}{\sum_{(#1)}}}
\newcommand{\ttt}{\ensuremath{\star}\xspace}
\newcommand{\typelep}[1]{\ensuremath{{(#1)}\text-\mathsf{Type}}\xspace}
\newcommand{\unit}{\ensuremath{\mathbf{1}}\xspace}
\newcounter{mathcount}
\setcounter{mathcount}{1}
\newtheorem{precor}{Corollary}
\newenvironment{cor}{\begin{precor}}{\end{precor}\addtocounter{mathcount}{1}}
\renewcommand{\theprecor}{8.2.\arabic{mathcount}}
\newtheorem{prethm}{Theorem}
\newenvironment{thm}{\begin{prethm}}{\end{prethm}\addtocounter{mathcount}{1}}
\renewcommand{\theprethm}{8.2.\arabic{mathcount}}
\let\autoref\cref
\makeatother

\begin{document}

Recall from \autoref{sec:connectivity} that a type $A$ is called \define{$n$-connected} if $\trunc nA$ is contractible.
The aim of this section is to prove that the operation of suspension from \autoref{sec:suspension} increases connectedness.

\begin{thm} \label{thm:suspension-increases-connectedness}
  If $A$ is $n$-connected then the suspension of $A$ is $(n+1)$-connected.
\end{thm}

\begin{proof}
  We remarked in \autoref{sec:colimits} that the suspension of $A$ is the pushout $\unit\sqcup^A\unit$, so we need to
  prove that the following type is contractible:
  %
  \[\trunc{n+1}{\unit\sqcup^A\unit}.\]
  %
  By \autoref{reflectcommutespushout} we know that
  $\trunc{n+1}{\unit\sqcup^A\unit}$ is a pushout in $\typelep{n+1}$ of the diagram
  \[\xymatrix{\trunc{n+1}A \ar[d] \ar[r] & \trunc{n+1}{\unit} \\
    \trunc{n+1}{\unit} & }.\]
  %
  Given that $\trunc{n+1}{\unit}=\unit$, the type
  $\trunc{n+1}{\unit\sqcup^A\unit}$ is also a pushout of the following diagram in
  $\typelep{n+1}$ (because both diagrams are equal)
  %
  \[\Ddiag=\vcenter{\xymatrix{\trunc{n+1}A \ar[d] \ar[r] & \unit \\
    \unit & }}.\]
  %
  We will now prove that $\unit$ is also a pushout of $\Ddiag$ in
  $\typelep{n+1}$.
  %
  Let $E$ be an $(n+1)$-truncated type; we need to prove that the following map
  is an equivalence
  %
  \[\function{(\unit \to E)}{\cocone{\Ddiag}{E}}{y}
  {(y,y,\lamu{u:{\trunc{n+1}A}} \refl{y(\ttt)})}.\]
  %
  where we recall that $\cocone{\Ddiag}{E}$ is the type
  \[\sm{f:\unit \to E}{g:\unit \to E}(\trunc{n+1}A\to
  (f(\ttt)=_E{}g(\ttt))).\]
  %
  The map $\function{(\unit\to E)}{E}{f}{f(\ttt)}$ is an equivalence, hence
  we also have
  \[\cocone{\Ddiag}{E}=\sm{x:E}{y:E}(\trunc{n+1}A\to(x=_Ey)).\]
  %
  Now $A$ is $n$-connected hence so is $\trunc{n+1}A$ because
  $\trunc n{\trunc{n+1}A}=\trunc nA=\unit$, and $(x=_Ey)$ is $n$-truncated because
  $E$ is $(n+1)$-connected. Hence by \autoref{connectedtotruncated} the
  following map is an equivalence
  %
  \[\function{(x=_Ey)}{(\trunc{n+1}A\to(x=_Ey))}{p}{\lam{z} p}\]
  %
  Hence we have
  %
  \[\cocone{\Ddiag}{E}=\sm{x:E}{y:E}(x=_Ey).\]
  %
  But the following map is an equivalence
  %
  \[\function{E}{\sm{x:E}{y:E}(x=_Ey)}{x}{(x,x,\refl{x})}.\]
  %
  Hence
  %
  \[\cocone{\Ddiag}{E}=E.\]
  %
  Finally we get an equivalence
  %
  \[(\unit \to E)\eqvsym\cocone{\Ddiag}{E}\]
  %
  We can now unfold the definitions in order to get the explicit expression of
  this map, and we see easily that this is exactly the map we had at the
  beginning.

  Hence we proved that $\unit$ is a pushout of $\Ddiag$ in $\typelep{n+1}$. Using
  uniqueness of pushouts we get that $\trunc{n+1}{\unit\sqcup^A\unit}=\unit$
  which proves that the suspension of $A$ is $(n+1)$-connected.
\end{proof}

\begin{cor} \label{cor:sn-connected}
  For all $n:\N$, the sphere $\Sn^n$ is $(n-1)$-connected.
\end{cor}

\begin{proof}
  We prove this by induction on $n$.
  %
  For $n=0$ we have to prove that $\Sn^0$ is merely inhabited, which is clear.
  %
  Let $n:\N$ be such that $\Sn^n$ is $(n-1)$-connected. By definition $\Sn^{n+1}$
  is the suspension of $\Sn^n$, hence by the previous lemma $\Sn^{n+1}$ is
  $n$-connected.
\end{proof}


\end{document}
