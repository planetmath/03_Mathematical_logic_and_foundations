\documentclass[12pt]{article}
\usepackage{pmmeta}
\pmcanonicalname{WellfoundedInduction}
\pmcreated{2013-03-22 12:42:17}
\pmmodified{2013-03-22 12:42:17}
\pmowner{ratboy}{4018}
\pmmodifier{ratboy}{4018}
\pmtitle{well-founded induction}
\pmrecord{22}{32988}
\pmprivacy{1}
\pmauthor{ratboy}{4018}
\pmtype{Theorem}
\pmcomment{trigger rebuild}
\pmclassification{msc}{03B10}
%\pmkeywords{well-founded relation}
%\pmkeywords{induction}
\pmrelated{PrincipleOfFiniteInduction}

% this is the default PlanetMath preamble.  as your knowledge
% of TeX increases, you will probably want to edit this, but
% it should be fine as is for beginners.

% almost certainly you want these
\usepackage{amssymb}
\usepackage{amsmath}
\usepackage{amsfonts}

% used for TeXing text within eps files
%\usepackage{psfrag}
% need this for including graphics (\includegraphics)
%\usepackage{graphicx}
% for neatly defining theorems and propositions
%\usepackage{amsthm}
% making logically defined graphics
\usepackage[arrow,curve,poly,arc,2cell,frame,web]{xypic}
%\usepackage[thmmarks]{ntheorem}

% there are many more packages, add them here as you need them

% define commands here

\newcommand{\br}{[\![}
\newcommand{\rb}{]\!]}
\newcommand{\oq}{\text{``}}
\newcommand{\cq}{\text{''}}


\newcommand{\im}{\mathbf{Im}}
\newcommand{\dom}{\mathbf{Dom}}


\newcommand{\Or}{\vee}
\newcommand{\Implies}{\Rightarrow}
\newcommand{\Iff}{\Leftrightarrow}
\newcommand{\proves}{\vdash}
\renewcommand{\And}{\wedge}
\newcommand{\Sup}{\bigwedge}
\newcommand{\Inf}{\bigvee}
\newcommand{\Z}{\mathbb{Z}}
\newcommand{\F}{\mathbb{F}}
\newcommand{\Q}{\mathbb{Q}}
\newcommand{\R}{\mathbb{R}}
\newcommand{\C}{\mathbb{C}}
\newcommand{\Nat}{\mathbb{N}}
\newcommand{\M}{\mathfrak{M}}
\newcommand{\N}{\mathfrak{N}}
\newcommand{\A}{\mathfrak{A}}
\newcommand{\B}{\mathfrak{B}}
\newcommand{\K}{\mathfrak{K}}
\newcommand{\G}{\mathbb{G}}
\newcommand{\Def}{\overset{\operatorname{def}}{:=}}



\newcommand{\spec}{\text{{\bf Spec}}}
\newcommand{\stab}{\text{{\bf Stab}}}
\newcommand{\ann}{\text{{\bf Ann}}}
\newcommand{\irr}{\text{{\bf Irr}}}
\newcommand{\qt}{\text{{\bf Qt}}}
\newcommand{\st}{\mathcal{Qt}}
\newcommand{\ro}{\mathbf{r.o.}}


\newcommand{\Endo}{\text{{\bf End}}}
\newcommand{\mat}{\text{{\bf Mat}}}
\newcommand{\der}{\text{{\bf Der}}}
\newcommand{\rad}{\text{{\bf Rad}}}
\newcommand{\trd}{\text{{\bf tr.d.}}}
\newcommand{\cl}{\text{{\bf acl}}}
\newcommand{\Int}{\text{{\bf int}}}
\newcommand{\V}{\mathbb{V}}
\newcommand{\D}{\mathbf{D}}

\newcommand{\del}{\partial}
\renewcommand{\O}{\mathcal{O}}
\newcommand{\aut}{\mathbf{Aut}}
\newcommand{\height}{\text{\bf Height}}
\newcommand{\coheight}{\text{\bf Co-height}}

\newcommand{\lcm}{\operatorname{lcm}}

\newcommand{\Gal}{\operatorname{Gal}}
\newcommand{\x}{\mathbf{x}}
\newcommand{\y}{\mathbf{y}}
\newcommand{\inner}[2]{\langle #1|#2\rangle}
\renewcommand{\r}{{r}}
\renewcommand{\t}{{t}}

\newcommand{\restr}{\upharpoonright}
\newcommand{\Matrix}[4]{\left(\begin{array}{cc} #1 & #2 \\ #3 & #4 
\end{array}\right)}
\begin{document}
\PMlinkescapeword{states}

{\bf Definition.} Let $S$ be a non-empty set, and $R$ be a binary relation on $S$.  Then $R$ is said to be a {\bf well-founded} relation if and only if every nonempty subset $X\subseteq S$ has an \PMlinkname{$R$-minimal element}{RMinimalElement}.  When $R$ is well-founded, we also call the underlying set $S$ well-founded.

Note that $R$ is by no means required to be a total order, or even a partial order.  When $R$ is a partial order, then $R$-minimality is the same as minimality (of the partial order).  A classical example of a well-founded set that is not totally ordered is the set $\Nat$ of natural numbers ordered by division, i.e. $aRb$ if and only if $a$ divides $b$, and $a\not=1$.  The $R$-minimal elements of $\Nat$ are the prime numbers.

Let $\Phi$ be a property defined on a well-founded set $S$.  The principle of well-founded induction states that if the following is true :
\begin{enumerate}
\item $\Phi$ is true for all the $R$-minimal elements of $S$
\item for every $a$, if for every $x$ such that $xRa$, we have $\Phi(x)$, then we have $\Phi(a)$
\end{enumerate}
then $\Phi$ is true for every $a\in S$.

As an example of application of this principle, we mention the proof of the fundamental theorem of arithmetic : every natural number has a unique factorization into prime numbers.  The proof goes by well-founded induction in the set $\Nat$ ordered by division.
%%%%%
%%%%%
\end{document}
