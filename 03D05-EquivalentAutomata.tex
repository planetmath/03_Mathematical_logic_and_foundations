\documentclass[12pt]{article}
\usepackage{pmmeta}
\pmcanonicalname{EquivalentAutomata}
\pmcreated{2013-03-22 18:03:21}
\pmmodified{2013-03-22 18:03:21}
\pmowner{CWoo}{3771}
\pmmodifier{CWoo}{3771}
\pmtitle{equivalent automata}
\pmrecord{6}{40585}
\pmprivacy{1}
\pmauthor{CWoo}{3771}
\pmtype{Definition}
\pmcomment{trigger rebuild}
\pmclassification{msc}{03D05}
\pmclassification{msc}{68Q45}

\endmetadata

\usepackage{amssymb,amscd}
\usepackage{amsmath}
\usepackage{amsfonts}
\usepackage{mathrsfs}

% used for TeXing text within eps files
%\usepackage{psfrag}
% need this for including graphics (\includegraphics)
%\usepackage{graphicx}
% for neatly defining theorems and propositions
\usepackage{amsthm}
% making logically defined graphics
%%\usepackage{xypic}
\usepackage{pst-plot}

% define commands here
\newcommand*{\abs}[1]{\left\lvert #1\right\rvert}
\newtheorem{prop}{Proposition}
\newtheorem{thm}{Theorem}
\newtheorem{ex}{Example}
\newcommand{\real}{\mathbb{R}}
\newcommand{\pdiff}[2]{\frac{\partial #1}{\partial #2}}
\newcommand{\mpdiff}[3]{\frac{\partial^#1 #2}{\partial #3^#1}}
\begin{document}
Two automata are said to be \emph{equivalent} if they accept the same language.  Explicitly, if $A_1=(S_1,\Sigma_1,\delta_1,I_1,F_1)$ and $A_2=(S_2,\Sigma_2,\delta_2,I_2,F_2)$ are two automata, then $A_1$ is equivalent to $A_2$ if $L(A_1)=L(A_2)$.  We write $A_1\sim A_2$ when they are equivalent.  It is clear that $\sim$ is an equivalence relation on the class of automata.

First, note that if $A_1\sim A_2$, then every symbol $\alpha\in \Sigma_1$ in a word $a\in L(A_1)$ is a symbol $\alpha$ in $\Sigma_2$.  In other words, every symbol in a word accepted by $A_1$ (or $A_2$) belongs to $\Sigma:=\Sigma_1\cap \Sigma_2$.  As a result, $L(A_1)=L(A_2)\subseteq \Sigma^*$.  If $B_i$ is an automaton obtained from $A_i$ by replacing the alphabet $\Sigma_i$ with $\Sigma$, where $i=1,2$, then $B_i\sim A_i$.  This shows that we may, without loss of generality, assume outright, in the definition of equivalence of $A_1$ and $A_2$, that they have the same underlying alphabet.

The most striking aspect of equivalence of automata is the following: 
\begin{prop} Every non-deterministic automaton is equivalent to a deterministic one. \end{prop}
\begin{proof}  Suppose $A=(S_1,\Sigma,\delta_1,I_1,F_1)$ be a non-deterministic automaton.  We seek a deterministic automaton $B=(S_1,\Sigma, \delta_2,I_2,F_2)$ such that $A\sim B$.  Recall that the difference between $A$ and $B$ lie in the transition functions: $\delta_1$ is a function from $S_1\times \Sigma$ to $P(S_1)$, whereas $\delta_2$ is a function from $S_2\times \Sigma$ to $S_2$, and the fact that $I_2$ is required to be a singleton.  The key to finding $B$ is to realize that $\delta_1$ can be converted into a function from $P(S_1)\times \Sigma$ to $P(S_1)$.

Now, define $S_2:=P(S_1)$, $I_2:=I_1$.  For $T\subseteq S_1$ and $\alpha\in \Sigma$, let $$\delta_2(T,\alpha):=\bigcup_{s\in T} \delta_1(s,\alpha).$$
As usual, we extend $\delta_2$ so it is defined on all of $S_2\times \Sigma^*$.  We want to show that $$\delta_2(\lbrace s\rbrace, a)=\delta_1(s,a)$$ for any $s\in S_1$ and any $a\in \Sigma^*$.  This can be done by induction on the length of $a$:
\begin{itemize}
\item if $a=\lambda$, then $\delta_2(\lbrace s\rbrace,\lambda)=\lbrace s\rbrace = \delta_1(s,\lambda)$ by definition;
\item if $a\in \Sigma$, then $\delta_2(\lbrace s\rbrace, a)=\bigcup_{s\in\lbrace s\rbrace}\delta_1(s,a)=\delta_1(s,a)$, again by definition;
\item if $a=b\alpha$, where $b\in \Sigma^*$ and $\alpha\in \Sigma$, then by the induction step, $\delta_2(\lbrace s\rbrace,b)=\delta_1(s,b)$, so that $\delta_2(\lbrace s\rbrace,a)= \delta_2(\lbrace s\rbrace,b\alpha)=\delta_2(\delta_2(\lbrace s\rbrace,b),\alpha) = \delta_2(\delta_1(s,b),\alpha)=\bigcup_{t\in \delta_1(s,b)} \delta_1(t,\alpha)=\delta_1(\delta_1(s,b),\alpha)=\delta_1(s,b\alpha)=\delta_1(s,a)$.
\end{itemize}
Suppose $a$ is accepted by $A$, so that $\delta_1(s,a)\cap F_1\ne \varnothing$ for some $s\in I_1$.  Then 
\begin{equation}
\delta_2(I_2, a)=\bigcup_{s\in I_2} \delta_1(s,a)=\bigcup_{s\in I_1} \delta_1(s,a),
\end{equation}
which has non-empty intersection with $F_1$.  So, we want $F_2$ to consists of every element of $S_2$ that has non-empty intersection with $F_1$.  Formally, we define $F_2:=\lbrace F\subseteq S_1\mid F\cap F_1\ne \varnothing\rbrace$.  So what we have just shown is that $L(A)\subseteq L(B)$.  

On the other hand, if $a$ is accepted by $B$, then (1) above says that $\bigcup_{s\in I_1} \delta_1(s,a) \in F_2$, or $(\bigcup_{s\in I_1} \delta_1(s,a))\cap F_1\ne \varnothing$, or $\delta_1(s,a)\cap F_1\ne \varnothing$ for some $s\in I_1$, which means $a$ is accepted by $A$, proving the proposition.
\end{proof}
%%%%%
%%%%%
\end{document}
