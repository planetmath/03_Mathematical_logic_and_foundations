\documentclass[12pt]{article}
\usepackage{pmmeta}
\pmcanonicalname{NoncommutativeDynamicModelingDiagrams}
\pmcreated{2013-03-22 18:12:52}
\pmmodified{2013-03-22 18:12:52}
\pmowner{bci1}{20947}
\pmmodifier{bci1}{20947}
\pmtitle{non-commutative dynamic modeling diagrams}
\pmrecord{17}{40796}
\pmprivacy{1}
\pmauthor{bci1}{20947}
\pmtype{Topic}
\pmcomment{trigger rebuild}
\pmclassification{msc}{03C98}
\pmclassification{msc}{03C52}
\pmclassification{msc}{18-00}
\pmclassification{msc}{03D80}
\pmclassification{msc}{03D15}
\pmsynonym{non-abelian structures}{NoncommutativeDynamicModelingDiagrams}
\pmsynonym{non-commutative structures}{NoncommutativeDynamicModelingDiagrams}
\pmsynonym{nonabelian structures}{NoncommutativeDynamicModelingDiagrams}
\pmsynonym{non-Abelian structures}{NoncommutativeDynamicModelingDiagrams}
%\pmkeywords{dynamic models}
%\pmkeywords{complexity}
%\pmkeywords{simple and complex systems}
%\pmkeywords{non-abelian structures}
%\pmkeywords{non-commutative structures}
%\pmkeywords{nonabelian structures}
%\pmkeywords{complex dynamics}
%\pmkeywords{super-complex biodynamics in living systems}
\pmrelated{AxiomsForAnAbelianCategory}
\pmrelated{SystemDefinitions}
\pmrelated{AxiomaticTheoryOfSupercategories}
\pmrelated{AlgebraicCategoryOfLMnLogicAlgebras}
\pmrelated{CategoricalOntology}
\pmrelated{NonCommutingGraphOfAGroup}
\pmrelated{SimilarityAndAnalogousSystemsDynamicAdjointness2}
\pmrelated{SystemDefinitions}
\pmdefines{von Neumann complexity}
\pmdefines{system dynamics}
\pmdefines{model encoding algorithm}
\pmdefines{model decoding}
\pmdefines{homogeneous logic class}
\pmdefines{Rosen complexity}
\pmdefines{heterogeneous logic class}
\pmdefines{modelling diagram}
\pmdefines{complex system}
\pmdefines{super-complex system}

\endmetadata

% this is the default PlanetMath preamble.  as your knowledge
% of TeX increases, you will probably want to edit this, but
% it should be fine as is for beginners.

% almost certainly you want these
\usepackage{amssymb}
\usepackage{amsmath}
\usepackage{amsfonts}

% used for TeXing text within eps files
%\usepackage{psfrag}
% need this for including graphics (\includegraphics)
%\usepackage{graphicx}
% for neatly defining theorems and propositions
%\usepackage{amsthm}
% making logically defined graphics
%%%\usepackage{xypic}

% there are many more packages, add them here as you need them

% define commands here
\usepackage{amsmath, amssymb, amsfonts, amsthm, amscd, latexsym,color,enumerate}
%%\usepackage{xypic}
\xyoption{curve}
\usepackage[mathscr]{eucal}

\setlength{\textwidth}{7in}
%\setlength{\textwidth}{16cm}
\setlength{\textheight}{10.0in}
%\setlength{\textheight}{24cm}

\hoffset=-.75in     %%ps format
%\hoffset=-1.0in     %%hp format
\voffset=-.4in

%the next gives two direction arrows at the top of a 2 x 2 matrix

\newcommand{\directs}[2]{\def\objectstyle{\scriptstyle}  \objectmargin={0pt}
\xy
(0,4)*+{}="a",(0,-2)*+{\rule{0em}{1.5ex}#2}="b",(7,4)*+{\;#1}="c"
\ar@{->} "a";"b" \ar @{->}"a";"c" \endxy }

\theoremstyle{plain}
\newtheorem{lemma}{Lemma}[section]
\newtheorem{proposition}{Proposition}[section]
\newtheorem{theorem}{Theorem}[section]
\newtheorem{corollary}{Corollary}[section]
\newtheorem{conjecture}{Conjecture}[section]

\theoremstyle{definition}
\newtheorem{definition}{Definition}[section]
\newtheorem{example}{Example}[section]
%\theoremstyle{remark}
\newtheorem{remark}{Remark}[section]
\newtheorem*{notation}{Notation}
\newtheorem*{claim}{Claim}


\theoremstyle{plain}
\renewcommand{\thefootnote}{\ensuremath{\fnsymbol{footnote}}}
\numberwithin{equation}{section}
\newcommand{\Ad}{{\rm Ad}}
\newcommand{\Aut}{{\rm Aut}}
\newcommand{\Cl}{{\rm Cl}}
\newcommand{\Co}{{\rm Co}}
\newcommand{\DES}{{\rm DES}}
\newcommand{\Diff}{{\rm Diff}}
\newcommand{\Dom}{{\rm Dom}}
\newcommand{\Hol}{{\rm Hol}}
\newcommand{\Mon}{{\rm Mon}}
\newcommand{\Hom}{{\rm Hom}}
\newcommand{\Ker}{{\rm Ker}}
\newcommand{\Ind}{{\rm Ind}}
\newcommand{\IM}{{\rm Im}}
\newcommand{\Is}{{\rm Is}}
\newcommand{\ID}{{\rm id}}
\newcommand{\GL}{{\rm GL}}
\newcommand{\Iso}{{\rm Iso}}
\newcommand{\Sem}{{\rm Sem}}
\newcommand{\St}{{\rm St}}
\newcommand{\Sym}{{\rm Sym}}
\newcommand{\SU}{{\rm SU}}
\newcommand{\Tor}{{\rm Tor}}
\newcommand{\U}{{\rm U}}

\newcommand{\A}{\mathcal A}
\newcommand{\D}{\mathcal D}
\newcommand{\E}{\mathcal E}
\newcommand{\F}{\mathcal F}
\newcommand{\G}{\mathcal G}
\newcommand{\R}{\mathcal R}
\newcommand{\cS}{\mathcal S}
\newcommand{\cU}{\mathcal U}
\newcommand{\W}{\mathcal W}

\newcommand{\Ce}{\mathsf{C}}
\newcommand{\Q}{\mathsf{Q}}
\newcommand{\grp}{\mathsf{G}}
\newcommand{\dgrp}{\mathsf{D}}

\newcommand{\bA}{\mathbb{A}}
\newcommand{\bB}{\mathbb{B}}
\newcommand{\bC}{\mathbb{C}}
\newcommand{\bD}{\mathbb{D}}
\newcommand{\bE}{\mathbb{E}}
\newcommand{\bF}{\mathbb{F}}
\newcommand{\bG}{\mathbb{G}}
\newcommand{\bK}{\mathbb{K}}
\newcommand{\bM}{\mathbb{M}}
\newcommand{\bN}{\mathbb{N}}
\newcommand{\bO}{\mathbb{O}}
\newcommand{\bP}{\mathbb{P}}
\newcommand{\bR}{\mathbb{R}}
\newcommand{\bV}{\mathbb{V}}
\newcommand{\bZ}{\mathbb{Z}}

\newcommand{\bfE}{\mathbf{E}}
\newcommand{\bfX}{\mathbf{X}}
\newcommand{\bfY}{\mathbf{Y}}
\newcommand{\bfZ}{\mathbf{Z}}

\renewcommand{\O}{\Omega}
\renewcommand{\o}{\omega}
\newcommand{\vp}{\varphi}
\newcommand{\vep}{\varepsilon}

\newcommand{\diag}{{\rm diag}}
\newcommand{\desp}{{\mathbb D^{\rm{es}}}}
\newcommand{\Geod}{{\rm Geod}}
\newcommand{\geod}{{\rm geod}}
\newcommand{\hgr}{{\mathbb H}}
\newcommand{\mgr}{{\mathbb M}}
\newcommand{\ob}{\operatorname{Ob}}
\newcommand{\obg}{{\rm Ob(\mathbb G)}}
\newcommand{\obgp}{{\rm Ob(\mathbb G')}}
\newcommand{\obh}{{\rm Ob(\mathbb H)}}
\newcommand{\Osmooth}{{\Omega^{\infty}(X,*)}}
\newcommand{\ghomotop}{{\rho_2^{\square}}}
\newcommand{\gcalp}{{\mathbb G(\mathcal P)}}

\newcommand{\rf}{{R_{\mathcal F}}}
\newcommand{\glob}{{\rm glob}}
\newcommand{\loc}{{\rm loc}}
\newcommand{\TOP}{{\rm TOP}}

\newcommand{\wti}{\widetilde}
\newcommand{\what}{\widehat}

\renewcommand{\a}{\alpha}
\newcommand{\be}{\beta}
\newcommand{\ga}{\gamma}
\newcommand{\Ga}{\Gamma}
\newcommand{\de}{\delta}
\newcommand{\del}{\partial}
\newcommand{\ka}{\kappa}
\newcommand{\si}{\sigma}
\newcommand{\ta}{\tau}


\newcommand{\lra}{{\longrightarrow}}
\newcommand{\ra}{{\rightarrow}}
\newcommand{\rat}{{\rightarrowtail}}
\newcommand{\oset}[1]{\overset {#1}{\ra}}
\newcommand{\osetl}[1]{\overset {#1}{\lra}}
\newcommand{\hr}{{\hookrightarrow}}


\newcommand{\hdgb}{\boldsymbol{\rho}^\square}
\newcommand{\hdg}{\rho^\square_2}

\newcommand{\med}{\medbreak}
\newcommand{\medn}{\medbreak \noindent}
\newcommand{\bign}{\bigbreak \noindent}

\renewcommand{\leq}{{\leqslant}}
\renewcommand{\geq}{{\geqslant}}

\def\red{\textcolor{red}}
\def\magenta{\textcolor{magenta}}
\def\blue{\textcolor{blue}}
\def\<{\langle}
\def\>{\rangle}
\begin{document}
\subsection{Introduction}
 In an interesting report, Rosen(1987) showed that complex dynamical systems,
such as biological organisms, cannot be adequately modelled through
a \emph{commutative} modelling diagram-- in the sense of digital
computer simulation--whereas the simple (`physical'/ engineering)
dynamical systems can be thus numerically simulated. 

\subsection{Non-commutative vs. commutative dynamic modeling diagrams}
 Furthermore, his modelling commutative diagram for a simple dynamical
system  included both the `encoding' of  the `real' system
$\mathbf{N}$ in ($\mathbf{M}$) as well as the `decoding' of
($\mathbf{M}$) back into $\mathbf{N}$:

$$\xymatrix@C=5pc{[SYSTEM] \ar [r]^{\textbf{Encoding...$\hookrightarrow$}} \ar [d]
_\delta & LOGICS \oplus MATHS.\ar [d]^{\aleph_M }\\ SYSTEM& \ar [l]
^{\\{\textbf{Decoding $\hookleftarrow...$}}}~[MATHS.\Box MODEL]}, $$

 where $\delta$ is the real system dynamics and $\aleph$ is an
algorithm implementing the numerical computation of the
mathematical model ($\mathbf{M}$) on a digital computer. Firstly,
one notes the ominous absence of the \emph{logical model}, \textbf{\emph{L}}, 
from Rosen's diagram published in 1987. Secondly, one also notes 
the obvious presence of logical arguments and indeed (\emph{non-Boolean})
`schemes' related to the entailment of organismic models, such as
\textbf{MR}-systems, in the more recent books that were published last by
Robert Rosen (1994, 2001, 2004).  Further mathematical details are provided in the paper by
Brown, Glazebrook and Baianu (2007). Furthermore, Elsasser (1980) pointed out a fundamental, logical difference between physical systems and biosystems or organisms: whereas the former are readily represented by \emph{homogeneous} logic classes, living organisms exhibit considerable variability and can only be represented
by \emph{heterogeneous} logic classes. One can readily represent
homogeneous logic classes or endow them with `uniform' mathematical 
structures, but heterogeneous ones are far more elusive and may admit
a multiplicity of mathematical representations or possess variable
structure.  This logical criterion may thus be useful for further
distinguishing simple systems from highly complex systems.

 The importance of logic algebras, and indeed of categories of logic algebras, is rarely discussed in modern Ontology even though categorical formulations of specific ontology domains such as biological Ontology and Neural Network ontology are being extensively developed. For a recent review of such categories of logic algebras the reader is referred to the concise presentation by Georgescu (2006); their relevance to network biodynamics was also recently assessed (Baianu, 2004, Baianu and Prisecaru, 2005; Baianu et al, 2006).  

 Super-complex systems, such as those supporting neurophysiological activities, are explained only in terms of non--linear, rather than linear causality. In some way then, these systems are not normally considered as
part of either traditional physics or the complex, chaotic systems physics
that are known to be fully deterministic. However, super-complex (biological) systems have the potential to manifest novel and counter--intuitive behavior such as in the manifestation of `emergence', development/morphogenesis and
biological evolution. The precise meaning of supercomplex systems is formally defined here
in the next section. 

\subsection{Simple and super--complex dynamics: Closed vs. open systems}

 In an early report (Baianu and Marinescu, 1968), the possibility
of formulating a super--categorical unitary theory of systems
(that is, of both simple and complex systems, etc.) was pointed out both in
terms of organizational structure and dynamics. Furthermore, it
was proposed that the formulation of any model or computer simulation of
a complex system-- such as living organism or a society--involves
generating a first--stage logical model (not-necessarily
Boolean!), followed by a mathematical one, \emph{complete
with structure} (Baianu, 1970). Then, it was pointed out that 
such a modeling process involves a diagram containing 
the complex system, (\textbf{CS}) and its dynamics, a corresponding, initial logical model, \textbf{L}, `encoding' the essential dynamic and/or structural properties of \textbf{CS}, and a detailed, structured mathematical
model $\mathbf{M}$; this initial modeling diagram may or may
not be commutative, and the modeling can be iterated through
modifications of \emph{\textbf{L}}, and/or $\mathbf{M}$, until an
acceptable agreement is achieved between the behaviour of the
model and that of the natural, complex system (Baianu and Marinescu, 1968;
Comoroshan and Baianu, 1969). Such an iterative modeling process 
may ultimately converge to appropriate models of the complex system, 
and perhaps a best possible model could be attained as the categorical 
colimit of the directed family of diagrams generated through such a modelling
process. The possible models $\mathbf{L}$, or especially
$\mathbf{M}$, were not considered to be necessarily either
numerical or recursively computable (that is, with an algorithm or
software program) by a digital computer (Baianu, 1971b, 1986-87).
The mathematician John von Neumann regarded and defined \emph{complexity} as \emph{a
measurable property of natural systems below the threshold of
which systems behave `simply', but above which they evolve,
reproduce, self--organize,} etc. It was claimed that any `natural' system fits this profile. But the classical assumption that natural systems are simple, or `mechanistic', is too restrictive since `simple' is applicable only to machines, closed physicochemical systems, computers, or any system that is recursively computable. Rosen (1987) proposed a major refinement of these ideas about complexity by a more exact classification between `simple' and `complex'. Simple systems can be characterized through representations which admit maximal models, and can be therefore re--assimilated via a hierarchy of informational levels. Besides, the duality between dynamical systems and states is also a characteristic of such simple dynamical systems. Complex systems do not admit any maximal model.  On the other hand, an \emph{ultra-complex} system-- as applied to psychological--sociological structures-- can be described in terms of \emph{variable categories} or structures, and thus cannot be reasonably represented by a fixed state space for its entire lifespan. Simulations by limiting dynamical approximations lead to
increasing system `errors'. Just as for simple systems, both \emph{super--complex} and
\emph{ultra-complex} systems admit their own orders of causation,
but the latter two types are different from the first--by inclusion
rather than exclusion-- of the mechanisms that control simple dynamical
systems.

%%%%%
%%%%%
\end{document}
