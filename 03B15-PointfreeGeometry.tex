\documentclass[12pt]{article}
\usepackage{pmmeta}
\pmcanonicalname{PointfreeGeometry}
\pmcreated{2013-03-22 16:33:35}
\pmmodified{2013-03-22 16:33:35}
\pmowner{ggerla}{15808}
\pmmodifier{ggerla}{15808}
\pmtitle{point-free geometry}
\pmrecord{15}{38747}
\pmprivacy{1}
\pmauthor{ggerla}{15808}
\pmtype{Topic}
\pmcomment{trigger rebuild}
\pmclassification{msc}{03B15}
\pmclassification{msc}{03A05}
\pmsynonym{pointless geometry}{PointfreeGeometry}
%\pmkeywords{regions}
%\pmkeywords{Whitehead}
%\pmkeywords{mereology}
%\pmkeywords{mereotopology}
\pmrelated{Geometry}

\endmetadata

% this is the default PlanetMath preamble.  as your knowledge
% of TeX increases, you will probably want to edit this, but
% it should be fine as is for beginners.

% almost certainly you want these
\usepackage{amssymb}
\usepackage{amsmath}
\usepackage{amsfonts}

% used for TeXing text within eps files
%\usepackage{psfrag}
% need this for including graphics (\includegraphics)
%\usepackage{graphicx}
% for neatly defining theorems and propositions
%\usepackage{amsthm}
% making logically defined graphics
%%%\usepackage{xypic}

% there are many more packages, add them here as you need them

% define commands here

\begin{document}
Point-free geometry is based on the idea that in geometry it is not necessary to assume as a primitive the notion of point. Instead, we can start from the notion of a "region" in the space. The points are defined by suitable \textit{abstraction processes}, i.e. order-reversing sequences of regions. Firstly, such a question was analized by A. N. Whitehead in the books "An Inquiry Concerning the Principles of Natural Knowledge" and "The concept of Nature". In these books the "inclusion relation" is the only primitive ("mereology" is the correct collocation). Successively in "Process and Reality" Whitehead proposed a new approach in which the "connection relation" and the one of "oval" is considered which are topological and affine in nature, respectively. Notice that Whitehead's analysis was philosophical but that several authors translated this analysis into systems of axioms for a formal mathematical treatment (for further information see [1]). Successively several approaches to point-free geometry where proposed metrical in nature. The primitives are the inclusion and either the diameter ([4] and [5]) or the diameter and the minimum distance between regions [2]. A totally different approach to point-free geometry is proposed in a very interesting book by H. J. Schmidt. We quote also the deep and extensive investigation of several authors in point-free topology [3].

\textbf{References}

1. G. Gerla, Pointless geometries, in Handbook of Incidence Geometry, F. Buekenhout and W. Kantor (eds) 1994 North-Holland.

2.G. Gerla, Pointless Metric spaces. J. Symb. Logic 55, 207-219, 1990.

3. S. Papert, An abstract theory of topological spaces, Proc. Cambridge Philos. Soc. 60, 197-203, 1964.

4. A. Pultr, Diameters in locales: How bad they can be. Comment. Math. Univ. Carolinae 29, 731-742, 1988.

5. F. Previale, Su una caratterizzazione reticolare del concetto di spazio metrico, Atti d. Acad. Sc. di Torino, Cl Sc. Fis., 100, 766-779, 1966.

6. H. J. Schmidt, Axiomatic Characterization of Physical Geometry, Lecture Notes in Physics, Springer-Verlag, Berlin Heidelberg, 1979.

7. A. N. Whitehead, An Inquiry Concerning the Principles of Natural Knowledge, Camb. Univ. Press, Cambridge, 1919.

8. A. N. Whitehead, The concept of Nature, Camb. Univ. Press, Cambridge, 1920.

9. A. N. Whitehead, Process and Reality, The Macmillan Co., New York 1929.

%%%%%
%%%%%
\end{document}
